\noindent {\color{red} We implement and use an efficient BLS multisignature scheme that has both efficient 
verification and efficient key aggregation (for more details, see section~\ref{sec:bls}). 
In general, multisignatures are susceptible to so-called ``rogue-key attacks'' which can be 
mounted whenever the adversary is allowed to choose his public keys arbitrarily. 
In a typical rogue-key attack, the adversary uses a public key that is a function of an 
honest user's key and this allows him to produce forgeries easily. The BLS multisignatures that we define and use 
in this work make no exception. So, in order to protect against rogue-key attacks, 
we enhance them with proofs-of-posession as defined in~\cite{proofs_of_posession}.} \\

\noindent {\color{red}Alternative constructions for BLS multisignatures as well as other defence mechanisms against rogue key attacks exist and 
we briefly review both below. First, a more general case to BLS mltisignatures exists, namely aggregating BLS signatures for different messages 
(e.g., ~\cite{aggregate_BLS_signatures}). In this case, the rouge-key attacks are not a threat anymore, but multisignature verification 
is computationally more expensive than in our case, requiring $O(n)$ parings for $n$ different messages. Second, alternative 
BLS multisignatures exist (e.g.,~\cite{boneh_compact_multisig}) where both the aggregated public key have their size independent 
of the number of signers and the multisignature verification is as fast as in our variant. However, for our application we prefer the scheme 
detailed in section~\ref{sec:bls}: in spite of requiring a one-time verification of proofs-of-posession, its corresponding aggregate key is a 
simple sum of the individual signers' public keys, while, in ~\cite{boneh_compact_multisig} the key aggregation operation involves more 
expensive scalar multiplications every time the key aggregation is performed.}
