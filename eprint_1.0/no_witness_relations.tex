\label{sec:snarks}

In the following, we construct three related SNARKS, each of them allowing a prover to convince an 
efficient verifier that an alleged aggregated public key has indeed been computed correctly as an aggregate 
of a vector of public keys for which two succinct commitments (one to the vector of x affine coordinates and the other 
to the vector of y affine coordinates, respectively) are publicly known. The differences between the three 
constructions stem from whether a \emph{bitmask} (also called a \emph{bitvector}) with one bit associated to each public key 
(necessary to signal the inclusion or omission of the respective public key w.r.t. the aggregate key) 
is part of the verifier's public input or is part of the witness. For the former case, we describe, in fact, two distinct SNARKs: a 
basic accountable SNARK (the bitmask is represented as a sequence of $\{0,1\}$ field elements) and a packed accountable SNARK (the bitmask is 
partitioned into equal blocks of consecutive binary bits, and, in turn, each block is represented as a field element). For the latter case, 
we describe a counting SNARK. Each of our three SNARKs implements a conditional NP relation bearing the 
same name as the SNARK it implements. Note that the names ``basic accountable" (for short, ``basic''), ``packed accountable" (for short, ``packed'') and ``counting" do 
not refer to the security of the respective SNARK but they summarise properties of the underlying sets of constraints that define 
the SNARKs, and, hence their use case. In particular, we use the basic accountable and the packed accountable SNARKs for building 
accountable light client systems and we use the counting SNARK for building non-accountable light client systems.\\

\noindent In order to compile our desired SNARKs we proceed as follows:
\begin{itemize}
\item We start by defining three conditional NP relations based only on vectors. These relations capture the specific constraints we are interested in. 
We denote these NP relations by $\Rla$ (i.e., basic accountable), 
$\Ra$ (packed accountable) and $\Rvt$ (counting). 
(See sections \ref{sec_la}, \ref{sec_a} and \ref{sec_vt}, respectively, for full details.) 
\item We design three ranged polynomial protocols for the above three relations. (Again, see sections \ref{sec_la}, \ref{sec_a} and \ref{sec_vt}, 
respectively, for full details.). The definition of ranged polynomial protocols originates in~\cite{plonk} and, for convenience, we remind it to the reader 
in appendix~\ref{sec:poly_protocols_appendix}.  
\item We define and use a two-steps PLONK-based compiler that allows us to compile the three ranged polynomial protocols into the desired 
SNARKs for a novel type of conditional NP relations which include trusted inputs and, in particular, trusted polynomial commitments as part of the 
public inputs. In more detail, we compile three SNARKs for three mixed polynomial commitments and vector based relations which we denote by 
$\Rlacom$, $\Racom$ and $\Rvtcom$, respectively. These are the direct counterparts of pure vector based relations 
$\Rla$, $\Ra$ and $\Rvt$. 
(See section~\ref{sec_two_step_compiler} for full details. For completeness, we also include in appendix~\ref{sec:rolled_out} the full rolled-out 
SNARK implementing $\Racom$.) 
\item We include a detailed comparison between the original PLONK and our SNARKs. 
(See section~\ref{sec:plonk_comparison}.) 
\item We conclude this section with an instantiation for committee key scheme for aggregatable signatures which uses, in turn, our SNARKS compiled in section~\ref{sec_two_step_compiler} 
and our instantiation for BLS aggregatable signatures from section~\ref{sec:bls}. (See section~\ref{sec:inst_committee_key} for full details.)
\end{itemize}

\noindent In more detail, as motivated in the introduction  and in  section~\ref{sec:pairings}, 
we define all our conditional NP relations for our three SNARKs over $\mathbb{F}$ which is the base field of the curve 
$\einn$. Moreover, our corresponding SNARKs provers' circuits are naturally defined as well over 
$\mathbb{F}$ as the scalar field of $\eout$. In particular, the vector of public keys, which is part of the public input for all of 
our three relations, and is denoted by $\mathbf{pk} = (\mathit{pk_0}, \ldots, \mathit{pk_{n-2}})$, is a vector of pairs with each component 
in $\mathbb{F}$. Note that this vector has size $n-1$ where $n$ has been defined in 
section \ref{sec:lagrange}. For the basic accountable and the counting conditional NP relations, we denote 
the $n$ components bitmask by $\mathbf{bit} = (\mathit{bit_0}, \ldots, \mathit{bit_{n-1}})$ 
(meaning that each component belongs to the set $\{0,1\} \subset \mathbb{F}$), 
while the packed accountable conditional NP relation is defined using the \emph{compacted bitmask} 
$\mathbf{b'} = (\mathit{b'_{0}}, \ldots, \mathit{b'_{\frac{n}{\block}-1}})$ of $\frac{n}{\block}$ field elements, 
each of which is $\block$ binary bits long ($\block$ has been defined in section \ref{sec:lagrange}). 
Intuitively, each of the binary bits in the bit representation of these field elements signals the 
inclusion (or exclusion) of the index-wise corresponding public keys into the aggregated public key $\mathit{apk}$. Note that, in fact, 
the last bit of field element $\mathit{b'_{\frac{n}{\block}-1}}$ as well as the $n$-th component $\mathit{bit_{n-1}}$ do not correspond to any public key, 
but, as will become clear in the following, they have been included for easier design of constraints. \\ %{\color{blue} TO DO: Talk about how vectors 
%are being padded when implementation is settled on this issue.} \\ 
%\noindent In order to compile our SNARKs and prove their security, we first define three ranged polynomial 
%protocols for the basic accountable, accountable and counting {\color{red} conditional NP relations}, 
%(see sections \ref{sec_la}, \ref{sec_a} and \ref{sec_vt}, respectively, for each of the ranged polynomial protocol and the corresponding relation) 
%and then we show these ranged polynomial protocols indeed fulfil definition \ref{def_ranged_poly_protocol} 
%(see section \ref{sec:poly_protocols_appendix}). Finally, using {\color{red} batched $\mathsf{KZG}$ polynomial commitments as 
%defined in section 3 of PLONK \cite{plonk} and the protocol compiler described in section \ref{sec_two_step_compiler}, we obtain our 
%succinct non-interactive arguments of knowledge, secure in the AGM model.} For completeness, we include in the appendix 
%(see section \ref{sec:rolled_out}) the resulting AGM secure SNARK implementing the conditional accountable 
%relation (section \ref{sec_a}). The remaining two SNARKs implementing the remaining 
%two conditional relations (see sections \ref{sec_la} and \ref{sec_vt}, respectively), can be analogously derived. \\

\noindent Notation-wise, we denote by $H$ the multiplicative subgroup of $\mathbb{F}$ generated 
by $\omega$ as defined in section \ref{sec:lagrange}. We additionally denote by $\mathit{incl}(a_0, \ldots, a_{n-2})$ the inclusion 
predicate that checks if $(a_0, \ldots, a_{n-2}) \in \ginn{1}^{n-1}$. Moreover let $h = (\mathit{h_x}, \mathit{h_y})$ 
be some fixed, publicly known element in $\einn \setminus \ginn{1}$. (See full version of this work for how to handle the special case $\einn = \ginn{1}$.)
%(See section~\ref{sec:special_source_group} for how to deal with the special case $\einn = \ginn{1}$.) 
We denote by $(a_x, a_y)$ the affine representation in $x$ and $y$ coordinates of $a \in \einn$ and by $\oplus$ the point addition 
in affine coordinates on the elliptic curve $\einn$. We denote by $[s]P$ the scalar multiplication by scalar $s \in \mathbb{F}$ of point 
$P \in \einn$. We denote by $\mathbb{B} = \{0,1\} \subset \mathbb{F}$. \\

\noindent Finally, as mentioned in section~\ref{sec:conditional_relations}, the interpretation of adding explicit domains to public 
inputs in the definition of conditional NP relations is that the honest parties (in our case, both the polynomial protocol verifiers and the SNARKs verifiers 
as defined in this section below) parse the public inputs according to the specified domains without any further checks. Any checks or 
computations that the honest parties perform regarding the public inputs are explicitly described as part of the protocols followed by 
the honest parties.

\subsection{Basic Accountable Ranged Polynomial Protocol}
\label{sec_la}

\noindent We start by describing our conditional basic accountable relation $\Rla$ and the 
corresponding $H$-ranged polynomial protocol $\Pla$. Both $n$ and the domains used in the explicit definitions 
of our conditional NP relations depend implicitly on the security parameter $\lambda$, hence $\Rla$ as well implicitly depends on $\lambda$. 
However, for brevity, here and in the rest of the paper we choose to omit the security parameter $\lambda$ whenever we refer to any of the conditional 
NP relations for which we build our SNARKs \\
 
\noindent \textsf{Conditional Basic Accountable Relation $\Rla$}  
\begin{equation*}
\begin{split}
\Rla = & \{(\mathbf{pk} \in ({\mathbb{F}^2})^{n-1}, \mathbf{bit} \in \mathbb{B}^n,
\mathit{apk} \in \mathbb{F}^2; \_) : \mathit{apk} = \sum_{i=0}^{n-2} [\mathit{bit_i}] \cdot \mathit{pk_i} \ | \ \mathbf{pk} \in \ginn{1}^{n-1} \} 
\end{split}
\end{equation*}

\noindent where $\mathbf{pk} = (\mathit{pk_0}, \ldots, \mathit{pk_{n-2}})$ and $\mathbf{bit} = (\mathit{bit_0}, \ldots, \mathit{bit_{n-1}})$. %and $\lambda$ is an 
%implicit parameter in the definition of $\Rla$ but, for brevity, it has been omitted here and in the following.
\noindent Throughout this section we are going to use the following polynomials and polynomial identities: \\

\noindent \textsf{Polynomials as Computed by Honest Parties} 
\begin{align*}
\mathsf{b(X)} &= \sum_{i=0}^{n-1} \mathit{bit_i} \cdot \mathsf{L_i(X)} \\
\mathsf{pkx(X)} & =  \sum_{i=0}^{n-2} \mathit{pkx_i} \cdot \mathsf{L_i(X)} \\
\mathsf{pky(X)} & =  \sum_{i=0}^{n-2} \mathit{pky_i} \cdot \mathsf{L_i(X)} \\
\mathsf{kaccx(X)} & =  \sum_{i=0}^{n-1} \mathit{kaccx_i} \cdot \mathsf{L_i(X)} \\
\mathsf{kaccy(X)} & = \sum_{i=0}^{n-1} \mathit{kaccy_i} \cdot \mathsf{L_i(X)}, 
\end{align*}

\noindent where $(\mathit{pkx_0}, \ldots, \mathit{pkx_{n-2}})$ 
and $(\mathit{pky_0}, \ldots, \mathit{pky_{n-2}})$ are computed such that $\forall i \in \{0, \ldots, n-2\}$, $\mathit{pk_i}$ 
is interpreted as a pair $(\mathit{pkx_i}, \mathit{pky_i})$ with its components in $\mathbb{F}$; we also have 
$(\mathit{kaccx_{0}}, \mathit{kaccy_{0}}) = (\mathit{h_x}, \mathit{h_y})$ and 
$(\mathit{kaccx_{i+1}}, \mathit{kaccy_{i+1}}) =  (\mathit{kaccx_{i}}, \mathit{kaccy_{i}}) \oplus \mathit{bit_i}(\mathit{pkx_{i}}, \mathit{pky_{i}})$, 
$\forall i < n-1$. Note that in the last relation $\mathit{bit_i}$ is not interpreted as a field element anymore but as a binary bit.\\

\noindent \textsf{Polynomial Identities} 
\begin{align*}
id_1(X) &= (X-\omega^{n-1}) \cdot [b(X) \cdot ((kaccx(X)-pkx(X))^2 \cdot (kaccx(X)+ pkx(X) + kaccx(\omega\cdot X)) - \\&- (pky(X) - kaccy(X))^2) +  (1-b(X))\cdot (kaccy(\omega\cdot X) - kaccy(X))] \\
id_2(X) & =  (X-\omega^{n-1})\cdot [b(X) \cdot ((kaccx(X) - pkx(X)) \cdot (kaccy(\omega \cdot X) + kaccy(X)) - \\&- (pky(X) - kaccy(X)) \cdot (kaccx(\omega \cdot X) - kaccx(X))) +  (1-b(X)) \cdot (kaccx(\omega \cdot X) - kaccx(X))] \\
id_3(X) & =  (kaccx(X) - h_x)\cdot L_0(X) + (kaccx(X) - (h\oplus apk)_{x}) \cdot L_{n-1}(X)  \\
id_4(X) & =  (kaccy(X) - h_y)\cdot L_0(X) + (kaccy(X) - (h\oplus apk)_{y}) \cdot L_{n-1}(X) \\
id_5(X) & =  b(X)(1-b(X)).
\end{align*}

\noindent Note that polynomial identity $id_5(X)$ is not needed for defining ranged polynomial protocols for 
$\Rla$, however it is included here to ease presentation and for proofs consistency 
for the ranged polynomial protocols in the following two sections. \\

\noindent \textsf{$H$-ranged Polynomial Protocol for Conditional Packed Accountable Relation $\Rla$} \\

\noindent In the following, we describe $H$-ranged polynomial protocol $\Pla$ for conditional relation 
$\Rla$. Protocol $\Pla$ describes the interaction of three parties, the prover 
$\mathcal{P}_{poly}$, the verifier $\mathcal{V}_{poly}$ and the trusted third party $\mathcal{I}$ in accordance to 
Definition \ref{def_ranged_poly_protocol} from section \ref{sec:poly_protocols_appendix}. \\

\noindent \textsf{Protocol $\Pla$} \\

\noindent $\mathcal{P}_{poly}$ and $\mathcal{V}_{poly}$ know public input 
$\mathbf{bit} \in \mathbb{B}^n$, $\mathbf{pk} \in (\mathbb{F}^2)^{n-1}$ and $\mathit{apk} \in (\mathbb{F})^2$ 
which are interpreted as per their respective domains.
\begin{enumerate}
\item $\mathcal{V}_{poly}$ computes $b(X)$, $pkx(X)$, $pky(X)$.
\item $\mathcal{P}_{poly}$ sends polynomials $kaccx(X)$ and $kaccy(X)$ to $\mathcal{I}$. 
\item $\mathcal{V}_{poly}$ asks $\mathcal{I}$ to check whether the following polynomial relations hold over range $H$ 
$$id_i(X) = 0, \forall i \in [4].$$
\item $\mathcal{V}_{poly}$  accepts if $\mathcal{I}$'s checks verify. 
\end{enumerate}

\noindent We show that protocol $\Pla$ is an $H$-ranged polynomial protocol for 
conditional relation $\Rla$. For this, we first prove that:
\begin{claim} Assume that $\forall i < n-1$ such that $\mathit{bit}_i = 1$, $pk_i = (pkx_i, pky_i) \in \ginn{1}$. 
If polynomial identities $id_i(X) = 0, \forall i \in [5],$ hold over range 
$H$ and and the polynomial $b(X)$ has been constructed via interpolation on $H$ such that $b(\omega^i) = \mathit{bit}_i, \forall i <n$ then \\
$\mathit{bit}_i \in \mathbb{B} = \{0,1\} \subset \mathbb{F}, \forall i <n$ \\
$(kaccx_{0}, kaccy_{0}) = (h_x, h_y)$, \\
$(kaccx_{n-1}, kaccy_{n-1}) = (h_x, h_y) \oplus (apk_x, apk_y)$, \\
$(kaccx_{i+1}, kaccy_{i+1}) =  (kaccx_{i}, kaccy_{i}) \oplus \mathit{bit}_i(pkx_{i}, pky_{i})$, $\forall i < n-1$, 
where in the last relation $\mathit{bit_i}$ should not be interpreted as a field element but as a binary bit.
\label{claim:keys_affine_comm}
\end{claim}

\begin{proof} Everything but the last property in the claim is easy to derive from polynomial identities 
$id_3(X) =0, id_4(X )= 0, id_5(X) = 0$ holding over $H$. \\

\noindent In order to prove the remaining property, we remind the incomplete addition formulae for curve points in affine 
coordinates, over elliptic curve in short Weierstrasse form and state:\\ 

\noindent \textit{Observation:} Suppose that $\mathit{bit} \in \{0,1\}$, $(x_1,y_1)$ is a point on an elliptic curve in 
short Weierstrasse form, and, if $\mathit{bit} = 1$, so is $(x_2,y_2)$. We claim that the following equations: 
\begin{align*}
&\mathit{bit}((x_1 - x_2)^2 (x_1 + x_2 + x_3) - (y_2 - y_1)^2 ) + (1 - \mathit{bit})(y_3 - y_1) =0 \ (\ast)\\
&\mathit{bit}((x_1 - x_2)(y_3 + y_1) - (y_2 - y_1)(x_3 - x_1)) + (1 - \mathit{bit})(x_3 - x_1) =0 \ (\ast\ast)
\end{align*}

\noindent hold if and only if one of the following three conditions hold 

\begin{enumerate}
\item \label{cond1} $\mathit{bit}=1$ and $(x_1,y_1)\oplus(x_2,y_2)=(x_3,y_3)$ and $x_1 \neq x_2$
\item \label{cond2} $\mathit{bit}=0$ and $(x_3,y_3)=(x_1,y_1)$ 
\item  \label{cond3} $\mathit{bit}=1$ and $(x_1,y_1)=(x_2,y_2)$\footnote{Note that under condition~\ref{cond3}, $(x_3,y_3)$ 
can be any point whatsoever, maybe not even on the curve. The same holds true for $(x_2, y_2)$ under the condition~\ref{cond2}.}.
\end{enumerate}

\noindent It is easy to see that each of the conditions~\ref{cond1},\ref{cond2},\ref{cond3} above implies equations $(\ast)$ and $(\ast \ast)$.
\noindent For the implication in the opposite direction, if we assume that $(\ast)$ and $(\ast \ast)$ hold, then \\

\noindent \textit{Case a:} For $\mathit{bit}=0$, the first term of each equation $(\ast)$ and $(\ast \ast)$ vanishes, 
leaving us with $y_3-y_1=0$ and $x_3-x_1=0$ which are equivalent to condition~\ref{cond2}. \\

\noindent \textit{Case b:} For $\mathit{bit}=1$ and $x_1=x_2$, by simple substitution in $(\ast)$ and $(\ast \ast)$, 
we obtain $y_1 = y_2$, i.e., condition~\ref{cond3}.  \\

\noindent \textit{Case c:} For $\mathit{bit}=1$ and $x_1 \neq x_2$, then we can substitute
$$\beta=\frac{y_2-y_1}{x_2-x_1}$$ into equations $(\ast)$ and $(\ast \ast)$, leaving us with
$$x_1+x_2+x_3=\beta^2 \textrm{ and } y_3+y_1=\beta(x_3-x_1).$$
which are the usual formulae for short Weierstrass form addition of affine coordinate points when $x_1 \neq x_2$ 
so this is equivalent to condition~\ref{cond1}. \\

\noindent We apply the above \textit{Observation} by noticing that if $id_1(X)$ and $id_2(X)$ hold over $H$, 
then $(\ast)$ and $(\ast \ast)$ hold with $(x_1, y_1)$ substituted by $(kaccx_i,kaccy_i)$, $(x_2, y_2)$ 
substituted by $(pkx_i, pky_i)$, $(x_3, y_3)$ substituted by $(kaccx_{i+1},kaccy_{i+1})$ and $\mathit{bit}$ 
substituted by $\mathit{bit}_i$ for $0 \leq i \leq n-2$, where $\mathit{bit_i}$ should not be interpreted as a field element but as binary 
bit. Moreover, since $(kaccx_{0}, kaccy_{0}) = (h_x, h_y) \in \einn \setminus \ginn{1}$ 
and if $(pkx_i, pky_i) \in \ginn{1}$ whenever $\mathit{bit}_i = 1$, then $\forall i < n-1$ 
equations $(\ast)$ and $(\ast \ast)$ obtained after the substitution defined above are equivalent to either 
condition~\ref{cond1} or condition~\ref{cond2}, but never condition~\ref{cond3}, so the result of the sum (i.e., $(kaccx_{i+1}, kaccy_{i+1})$, $0\leq i \leq n-2$) is, 
by induction, at each step a well-defined point on the curve and this concludes our proof.
\end{proof}

\begin{corollary} Assume $\forall i < n-1$ 
such that $\mathit{bit}_i = 1$, $pk_i = (pkx_i, pky_i) \in \ginn{1}$. 
If the polynomial identities $id_i(X) = 0, \forall i \in [4],$ hold over range $H$ and 
$\mathit{bit_i} \in \mathbb{B}$, $\forall i < n-1$ and $b(X) = \sum_{i=0}^{n-1} \mathit{bit_i} \cdot L_i(X)$
then:  \\
$(kaccx_{0}, kaccy_{0}) = (h_x, h_y)$, \\
$(kaccx_{n-1}, kaccy_{n-1}) = (h_x, h_y) \oplus (apk_x, apk_y)$, \\
$(kaccx_{i+1}, kaccy_{i+1}) =  (kaccx_{i}, kaccy_{i}) \oplus \mathit{bit_i}(pkx_{i}, pky_{i})$, $\forall i < n-1$, where in the last relation 
$\mathit{bit_i}$ should not be interpreted as a field element but as a binary bit.
\label{corollary:keys_affine_comm}
\end{corollary}

\begin{proof}The proof follows trivially from the more general result stated by Claim~\ref{claim:keys_affine_comm}. 
\end{proof}

\begin{lemma} 
$\Pla$ as described above is an $H$-ranged polynomial 
protocol for conditional relation $\Rla$.
\end{lemma}

\begin{proof}
It is easy to see that perfect completeness holds. Indeed, if $(\mathbf{bit},\mathbf{pk}, \mathit{apk}) \in \Rla$ holds, 
meaning that $\mathbf{bit} \in \mathbb{B}^n$ and $\mathbf{pk} \in \ginn{1}^{n-1}$ and $\mathit{apk} = \sum_{i=0}^{n-2} [\mathit{bit_i}] \cdot \mathit{pk_i}$ hold, 
then it is easy to see that the honest prover $\mathcal{P}_{poly}$ in $\Pla$ will convince the honest verifier $\mathcal{V}_{poly}$ in 
$\Pla$ to accept with probability $1$. \\
Regarding knowledge-soundness, if the verifier $\mathcal{V}_{poly}$ in $\Pla$ accepts, 
then the extractor $\mathcal{E}$ does not have to do anything as the relation $\Rla$ does not have a witness.  
However, we have to prove that if $\mathbf{pk} \in \ginn{1}^{n-1}$ and the verifier in $\Pla$ accepts, 
then $(\mathbf{bit},\mathbf{pk}, \mathit{apk}) \in \Rla$ holds, which given our definition for conditional relation is 
equivalent to proving that $\mathit{apk} = \sum_{i=0}^{n-2} [\mathit{bit_i}] \cdot \mathit{pk_i}$ holds. This is indeed the case due to 
Corollary~\ref{corollary:keys_affine_comm}.

\end{proof}

\subsection{Packed Accountable Ranged Polynomial Protocol}
\label{sec_a}
In the following, we denote by $\mathbb{F}_{|\block|}$ the subset of field elements in $\mathbb{F}$ that can be represented using at most $\block$ bits, i.e., the set $\{0, \ldots, 2^{\block -1} \}$, 
where $\block$ has been defined in section ~\ref{sec:lagrange}. \\
\noindent Our conditional packed accountable relation $\Ra$ and the corresponding $H$-ranged polynomial protocol 
$\Pa$ are defined as follows:\\
 
\noindent \textsf{Conditional Packed Accountable Relation $\Ra$} 
\begin{equation*}
\begin{split}
\Ra = & \{(\mathbf{pk} \in (\mathbb{F}^2)^{n-1},\mathbf{b'} \in \mathbb{F}_{|\block|}^{\frac{n}{\block}},
\mathit{apk} \in \mathbb{F}^2; \mathbf{bit}) : \mathit{apk} = \sum_{i=0}^{n-2} [\mathit{bit_i}] \cdot \mathit{pk_i} \ | \ \mathbf{pk} \in \ginn{1}^{n-1} \ \wedge \\
 & \wedge \mathbf{bit} \in \mathbb{B}^n  \wedge b'_{j} = \sum_{i=0}^{\block -1}2^i \cdot \mathit{bit_{\block \cdot j + i}}, \forall j < \frac{n}{\block} \} 
\end{split}
\end{equation*}
where $\mathbf{b'} = (b'_{0}, \ldots, b'_{{\frac{n}{\block}} -1})$. \\
\noindent We define new polynomials and polynomial identities: \\

\noindent \textsf{New Polynomials as Computed by Honest Parties} 
\begin{align*}
aux(X) &= \sum_{i=0}^{n-1}aux_i \cdot L_i(X) \\
c_{a}(X) & = \sum_{i=0}^{n-1} c_{a,i} \cdot L_i(X) \\
acc_{a}(X) & =  \sum_{i=0}^{n-1} acc_{a,i}  \cdot L_i(X)
\end{align*}
\noindent where $aux_{i} = 1 \in \mathbb{F}$ if $i$ is divisible with $\block$ and $aux_{i} = 0 \in \mathbb{F}$ otherwise, $\forall i < n$ \\
\noindent and $c_{a,i} = 2^k \cdot r^j$, $k = i \mod \block$, $j = i \div \block$, $\forall i < n$ ($r \in \mathbb{F}$ is introduced in protocol $\Pa$) and $acc_{a,i}$ are components of the vector $(0, \mathit{bit}_0 \cdot c_{a,0}, \mathit{bit}_0 \cdot c_{a,0}+ \mathit{bit}_1  \cdot c_{a,1}, \ldots, \sum_{i=0}^{n-2}\mathit{bit}_i \cdot c_{a,i})$, where $\mathit{bit_{0}}, \ldots, \mathit{bit_{n-1}}$ represent the first $n$ 
bits (however, we interpret them as elements in $\mathbb{B}$) of the concatenation of the binary representation of 
$\mathit{b'_{0}}, \ldots, \mathit{b'_{\frac{n}{\block}-1}}$.\footnote{As 
part of a correct public input for relation $\Ra$, each field element in the set 
$\{\mathit{b'_{0}}, \ldots, \mathit{b'_{\frac{n}{\block}-1}} \}$ is at most $\block$ binary bits long. If any such field element has fewer than $\block$ bits, 
then the honest prover will pad it with $0$s starting from the most significant bit up to a total individual length of $\block$ bits.} 
Note that with this definition of vector $(\mathit{bit_{0}}, \ldots, \mathit{bit_{n-1}})$, 
the definition of $b(X)$ remains the same as in section \ref{sec_la}.\\

\noindent \textsf{New Polynomial Identities} 
\begin{align*}
id_6(X) & =  c_{a}(\omega \cdot X) - c_{a}(X)\cdot (2+ (\frac{r}{2^{\block -1}} -2) \cdot aux(\omega \cdot X)) - (1 - r^{\frac{n}{\block}}) \cdot L_{n-1}(X).\\
id_7(X) & =  acc_{a}(\omega \cdot X) - acc_{a}(X) - b(X)\cdot c_{a}(X) + \mathsf{sum} \cdot L_{n-1}(X),
\end{align*}

\noindent where $\mathsf{sum}$ is a field element known to both $\mathcal{P}_{poly}$ and $\mathcal{V}_{poly}$ and will be defined below. \\ 

\noindent \textsf{$H$-ranged Polynomial Protocol for Conditional Packed Accountable Relation $\Ra$} \\

\noindent In the following, we describe $H$-ranged polynomial protocol $\Pa$ for conditional relation 
$\Ra$. \\

\noindent \textsf{Protocol $\Pa$} \\

\noindent $\mathcal{P}_{poly}$ and $\mathcal{V}_{poly}$ know public inputs 
$\mathbf{b'} \in \mathbb{F}_{|\block|}^{\frac{n}{\block}}$ and 
$\mathbf{pk} \in (\mathbb{F}^2)^{n-1} $ and $\mathit{apk} \in \mathbf{F}^2$ which are interpreted as per their respective domains. \\

\begin{enumerate}
\item $\mathcal{V}_{poly}$ computes $pkx(X)$, $pky(X)$ and $aux(X)$.
\item $\mathcal{P}_{poly}$ sends polynomials $b(X)$, $kaccx(X)$ and $kaccy(X)$ to $\mathcal{I}$. 
\item $\mathcal{V}_{poly}$ replies with a random value $r$ chosen from $\mathbb{F}$. 
\item $\mathcal{V}_{poly}$ computes $\mathsf{sum}$ as $\sum_{j=0}^{\frac{n}{\block}-1} \mathit{b'_{j}} \cdot r^j$.\footnote{Note that if 
$b'_{j} = \sum_{k=0}^{\block -1}2^k \cdot \mathit{bit_{\block \cdot j + k }}$, $\forall j < \frac{n}{\block}$ and $\mathit{bit_i} \in \mathbb{B}, \forall i <n$, 
then $\sum_{i=0}^{n-1} 2^{i \mod \block} \cdot r^{i \div \block} \cdot \mathit{bit}_{i} = \sum_{j=0}^{\frac{n}{\block}-1}(\sum_{i=0}^{\block -1}2^k \cdot \mathit{bit_{\block \cdot j + k }}) \cdot r^j= \sum_{j=0}^{\frac{n}{\block}-1} \mathit{b'_{j}} \cdot r^j$.}
\item $\mathcal{P}_{poly}$ sends polynomials $c_{a}(X)$ and $acc_{a}(X)$ to $\mathcal{I}$. 
\item $\mathcal{V}_{poly}$ asks $\mathcal{I}$ to check whether the following polynomial relations hold over range $H$: 
$$id_i(X) = 0, \forall i \in [7].$$
\item $\mathcal{V}_{poly}$ accepts if $\mathcal{I}$'s checks verify. 
\end{enumerate}

\noindent We show that protocol $\Pa$ is an $H$-ranged polynomial protocol 
for conditional relation $\Ra$. First, we prove the following:

\begin{claim} If the polynomial identities $id_6(X) = 0, id_7(X) = 0$ hold over range $H$, then, 
\ewnp, 
we have $c_{a,i} =  2^{i \mod \block} \cdot r^{i \div \block}$, $\forall i < n$ and $\mathsf{sum} = \sum_{i=0}^{n-1}b_i \cdot c_{a,i}$, 
where $b_i = b(\omega^i), \forall i <n$. If, additionally, identity $id_5(X) = 0$ holds over $H$, 
$r$ has been randomly chosen in $\mathbb{F}$, $\mathsf{sum} = \sum_{j=0}^{\frac{n}{\block}-1} b'_{j}r^j$ 
(as computed by $\mathcal{V}_{poly}$) and $\mathit{bit_{i}} \in \mathbb{B}, \forall i < n$ and 
$b'_{j} = \sum_{k=0}^{\block -1}2^k \cdot \mathit{bit_{\block \cdot j + k}}, \forall 0 \leq j \leq \frac{n}{\block} -1$ 
(due to the input $(b'_{0}, \ldots, b'^{\frac{n}{\block} -1})$ 
being interpreted by the verifier $\mathcal{V}_{poly}$ as in $\mathbb{F}_{|\block|}^{\frac{n}{\block}}$), then \ewnp, 
$b_i = \mathit{bit_{i}}, \forall i <n$.
\label{claim:bitvector_comm}
\end{claim}

\begin{proof}
To prove the first part of the claim, assume by contradiction that $c_{a,0} = k  \neq 1$. 
Then, by induction, since $id_6(X) = 0$ on $H$, 
$$c_{a,i} = k \cdot 2^{i \mod \block} \cdot r^{i \div \block}, \forall 0<i<n.$$ 
Additionally, the property
\begin{align*} 
c_{a,0} = c_{a,n-1} \cdot (2+ (\frac{r}{2^{\block -1}} -2) \cdot 1) + (1 - r^{\frac{n}{\block}}) \tag{1}
\end{align*}  
\noindent holds (again, from $id_6(X) = 0$ on $H$). However, substituting $c_{a,0} = k$ 
and $c_{a,n-1} = k \cdot 2^{\block -1} \cdot r^{\frac{n}{\block} -1}$ in $(1)$, we obtain 
$k = k \cdot 2^{\block -1} \cdot r^{\frac{n}{\block} -1} \cdot \frac{r}{2^{\block -1}} +1 - r^{\frac{n}{\block}}$ which is equivalent to 
$k(1 - r^{\frac{n}{\block}}) = 1 - r^{\frac{n}{\block}}$, and, due to Schwartz-Zippel Lemma and the fact that degree $n$ is negligibly 
smaller compared to the size of $\mathbb{F}$, this implies \ewnp $k =1$ thus contradiction,
so the values $c_{a,i}$ have indeed the claimed form. \\
Next, by expanding $id_7(X) = 0$ over $H$, the following holds
\begin{align*}
acc_{a,1} &= acc_{a,0} + b_0 \cdot c_{a,0} \\
acc_{a,2} &= acc_{a,1} + b_1\cdot c_{a,1} \\
\ldots \\
acc_{a,n-1} &= acc_{a,n-2} + b_{n-2} \cdot c_{a,n-2} \\
acc_{a,0} &= acc_{a,n-1} + b_{n-1} \cdot c_{a,n-1} - \mathsf{sum}.
\end{align*}

\noindent By summing together the LHS and, respectively, the RHS of the equalities above and 
reducing the equal terms, we obtain $\mathsf{sum} = \sum_{i=0}^{n-1}b_i\cdot c_{a,i}$. \\

\noindent For the second part of the claim, since $id_5(X) = 0$ holds over $H$ then 
$b_i = b(\omega^i) \in \mathbb{B}, \forall i \leq n-1$. Finally, from 
verifier's computation and from the first part of the claim we have 
\begin{align*}
\sum_{j=0}^{\frac{n}{\block}-1} b'_{j}r^j = \mathsf{sum} =  \sum_{i=0}^{n-1} b_i \cdot c_{a,i} & = \sum_{i=0}^{n-1} b_i \cdot 2^{i \mod \block} \cdot r^{i \div \block} = \\
&= \sum_{j=0}^{\frac{n}{\block} -1} (\sum_{k=0}^{\block -1} 2^k \cdot b_{\block\cdot j +k}) \cdot r^j = \sum_{j=0}^{\frac{n}{\block} -1} b''_{j}r^j \tag{2},
\end{align*}
where $\forall j, b''_{j}$ are field elements equal to the binary representation that uses contiguous blocks 
of $\block$ components from the bitmask $(\mathit{b}_0, \ldots, \mathit{b}_{n-1})$.
Since both the LHS and the RHS of relation (2) represent two ways of computing $\mathsf{sum}$ as an inner product of a vector 
of field elements (on one hand, $(\mathit{b'_{0}}, \ldots, \mathit{b'_{\frac{n}{\block}-1}})$, on the other hand, 
$(\mathit{b''_{0}}, \ldots, \mathit{b''_{\frac{n}{\block}-1}})$ ) with the vector $(1, r, \ldots, r^{\frac{n}{\block}-1})$, 
where $r$ has been chosen at random, by the small exponents test \cite{small_exponents}, we obtain that \ewnp 
$\mathit{b''_{j}} = \mathit{b'_{j}}, \forall 0\leq j\leq \frac{n}{\block}-1$. Finally, if we equate the bit representation in $\mathbb{F}$ 
(i.e., using field elements from $\mathbb{B}$) of field elements $\mathit{b''_{j}}$ and $\mathit{b'_{j}}, \forall 0\leq j\leq \frac{n}{\block}-1$ and remember that, 
by verifier's check or by construction, respectively, each such field element has no more that \block binary bits, we can conclude that \ewnp 
$b_i = \mathit{bit_i}, \forall i <n$.
\end{proof}

\begin{lemma} 
$\Pa$ as described above is an $H$-ranged polynomial protocol for conditional relation $\Ra$.
\end{lemma}

\begin{proof}
It is easy to see that perfect completeness holds. Indeed, if 
$(\mathbf{b'}, \mathbf{pk}, \mathit{apk}, \mathbf{bit}) \in \Ra$, meaning that 
$\mathbf{pk} \in \ginn{1}^{n-1}$ and $\mathbf{bit} \in \mathbb{B}^n$ and $\mathit{apk} = \sum_{i=0}^{n-2} [\mathit{bit_i}] \cdot \mathit{pk_i}$ and 
 $b'_{j} = \sum_{i=0}^{\block -1}2^i \cdot \mathit{bit_{\block \cdot j + i}}, \forall j < \frac{n}{\block}$ 
hold then it is easy to see that the honest prover $\mathcal{P}_{poly}$ in $\Pa$ will convince the honest 
verifier $\mathcal{V}_{poly}$ in $\Pa$ to accept with probability $1$.  \\  
Regarding knowledge-soundness, if the verifier $\mathcal{V}_{poly}$ in $\Pa$ accepts, 
then the extractor $\mathcal{E}$ sets $(\mathit{bit}_0, \ldots, \mathit{bit}_{n-1})$ as the vector of evaluations over $H$ of polynomial $b(X)$ sent by $\mathcal{P}_{poly}$ 
to $\mathcal{I}$. Next, we prove that if $(\mathit{pk_0}, \ldots, \mathit{pk_{n-2}}) \in \ginn{1}^{n-1}$ and the verifier in $\Pa$ accepts, 
then $$((\mathit{b'_{0}}, \ldots, \mathit{b'_{\frac{n}{\block}-1}}), (\mathit{pk_0}, \ldots, \mathit{pk_{n-2}}), \mathit{apk}, (\mathit{bit}_0, \ldots, \mathit{bit}_{n-1})) \in \Ra,$$ 
which is equivalent to proving that  $\mathit{apk} = \sum_{i=0}^{n-2} [\mathit{bit_i}] \cdot \mathit{pk_i}$ and 
 $\mathbf{bit} \in \mathbb{B}^n$ and  $$b'_{j} = \sum_{i=0}^{\block -1}2^i \cdot \mathit{bit_{\block \cdot j + i}}, \forall j < \frac{n}{\block}.$$
According to Claim~\ref{claim:bitvector_comm} and Corollary~\ref{corollary:keys_affine_comm} this indeed holds \ewnp
\end{proof}

\subsection{Counting Ranged Polynomial Protocol}
\label{sec_vt}
\noindent In the following relation, $\mathit{apk}$ is the aggregated public key of at least $s$ and at most $s+1$ public keys. 
Hence we interpret $s$ as a threshold on the number of public keys included in the aggregated public key. Since $\mathit{bit_{n-1}}$ as the last component of the 
bitmask witness does not correspond to any public key and we have to account for the fact that $\mathit{bit_{n-1}}$ may be 
$1 \in \mathbb{F}$, relation $\Rvt$ includes the off-by-one constraint $\sum_{i=0}^{n-1} \mathit{bit_i} = s+1$.\\

\noindent \textsf{Conditional Counting Relation $\Rvt$}  
\begin{equation*}
\begin{split}
 \Rvt = & \{(\mathbf{pk} \in (\mathbb{F}^2)^{n-1}, s \in \mathbb{F}^2, \mathit{apk} \in \mathbb{F}^2; \mathbf{bit}) : 
 \mathit{apk} = \sum_{i=0}^{n-2} [\mathit{bit_i}]  \cdot \mathit{pk_i} \ | \ \mathbf{pk}  \in \ginn{1}^{n-1} \ \wedge \\ 
& \wedge \ \mathbf{bit} \in \mathbb{B}^n \ \wedge \ \sum_{i=0}^{n-1} \mathit{bit_i} = s+1\} 
\end{split}
\end{equation*}

\noindent The new polynomials and polynomial identities required in this section are: \\

\noindent \textsf{New Polynomial as Computed by Honest Parties} 
\begin{align*}
acc_{vt}(X) & = \sum_{i=0}^{n-1} acc_{vt,i} \cdot L_i(X),
\end{align*}
where $acc_{vt,i}$ are the components of the vector $(0, \mathit{bit_0}, \mathit{bit_0} + \mathit{bit_1}, \ldots, \sum_{i=0}^{n-2}\mathit{bit_i})$, $\forall i < n$. \\

\noindent \textsf{New Polynomial Identities} 
\begin{align*}
id_8(X) & = acc_{vt}(\omega \cdot X) - acc_{vt}(X) - b(X) + (s+1) \cdot L_{n-1}(X),   \\
\end{align*}

\noindent \textsf{{$H$-ranged Polynomial Protocol for Conditional Counting Relation $\Rvt$}} \\

\noindent \textsf{Protocol $\Pvt$} \\

\noindent $\mathcal{P}_{poly}$ and $\mathcal{V}_{poly}$ know public input $s \in \mathbb{F}^2$, 
$\mathbf{pk} \in (\mathbb{F}^2)^{n-1}$ and $\mathit{apk} \in \mathbb{F}^2$ which are interpreted as per their respective domains. 

\begin{enumerate}
\item $\mathcal{V}_{poly}$ computes $pkx(X)$, $pky(X)$.
\item $\mathcal{P}_{poly}$ sends polynomials $b(X)$, $kaccx(X)$, $kaccy(X)$, $acc_{vt}(X)$ to $\mathcal{I}$. 
\item $\mathcal{V}_{poly}$ asks $\mathcal{I}$ to check whether the following polynomial relations hold over range $H$:
$$id_i(X) = 0, \forall i \in [5] \textit{ and } id_8(X) = 0.$$ 
\item $\mathcal{V}_{poly}$ accepts if all of $\mathcal{I}$'s checks verify. 
\end{enumerate}

\noindent We show that protocol $\Pvt$ is an $H$-ranged polynomial protocol for conditional relation 
$\Rvt$. 

\begin{lemma} $\Pvt$ as described above is an $H$-ranged polynomial protocol for conditional relation $\Rvt$.
\end{lemma}

\begin{proof}
It is easy to see that perfect completeness holds. Indeed, if $(\mathbf{bit},\mathbf{pk}, \mathit{apk}) \in \Rvt$ holds, 
meaning that $\mathbf{bit} \in \mathbb{B}^n$ and $\mathbf{pk} \in \ginn{1}^{n-1}$ and $\mathit{apk} = \sum_{i=0}^{n-2} [\mathit{bit_i}] \cdot \mathit{pk_i}$ and 
$\sum_{i=0}^{n-1} \mathit{bit_i} = s+1$ hold, then it is easy to see that the honest prover $\mathcal{P}_{poly}$ in $\Pvt$ will convince the honest 
verifier $\mathcal{V}_{poly}$ in $\Pvt$ to accept with probability $1$. \\
Regarding knowledge-soundness, if the verifier $\mathcal{V}_{poly}$ in $\Pvt$ accepts, 
then we construct the extractor $\mathcal{E}$ in the following way. Using the polynomial $b(X)$ which 
was part of the messages from $\mathcal{P}_{poly}$ to $\mathcal{I}$ and evaluating it at the elements of the set 
$H$, $\mathcal{E}$ obtains evaluation vector $\mathbf{bit} = (b(1), \ldots, b(\omega^{n-1}))$ which, 
in the following, we denote as $(\mathit{bit}_0, \ldots, \mathit{bit}_{n-1}) \in \mathbb{F}^n$.\\ 
\noindent Next, we show that if $(\mathit{pk_0}, \ldots, \mathit{pk_{n-2}}) \in \ginn{1}^{n-1}$ holds and the 
verifier in $\Pvt$ accepts, then 
$$((\mathit{pk_0}, \ldots, \mathit{pk_{n-2}}), s, \mathit{apk}, (\mathit{bit_0}, \ldots, \mathit{bit_{n-1}})) \in \Rvt,$$ 
which is equivalent to proving that $\mathit{apk} = \sum_{i=0}^{n-2} [\mathit{bit_i}]  \cdot \mathit{pk_i}$ and 
$\mathbf{bit} \in \mathbb{B}^n$ and  $\sum_{i=0}^{n-1} \mathit{bit_i} = s+1$.
\noindent First, since $id_8(X) = 0$ holds over $H$, we can expand that as follows:
\begin{align*}
acc_{vt,1} &= acc_{vt,0} + \mathit{bit}_{0} \\
acc_{vt, 2} &= acc_{vt,1} + \mathit{bit}_{1} \\
\ldots \\
acc_{vt,n-1} &= acc_{vt,n-2} + \mathit{bit}_{n-2} \\
acc_{vt,0} &= acc_{vt,n-1} + \mathit{bit}_{n-1} - (s+1).
\end{align*}
\noindent By summing together the LHS and, respectively, the RHS of the equalities above and 
reducing the equal terms, we obtain $s+1 = \sum_{i=0}^{n-1}\mathit{bit}_i$. \\ 
Second, since it holds over $H$ that $id_i(X) = 0$, $\forall i \in [5]$ and $b(\omega^i) = \mathit{bit_i}, \forall i<n$ (by the definition 
of $\mathcal{E}$), the properties concluded in Claim~\ref{claim:keys_affine_comm} hold. Combining the two proof steps above, we obtain the desired conclusion.
\end{proof}

\subsection{Two-Steps PLONK-Based Compiler for Hybrid Model SNARKs with Mixed Inputs}
\label{sec_two_step_compiler}

\noindent In the following we present a two-steps PLONK-based compilation technique from 
ranged polynomial protocols for conditional NP relations (formal definition in appendix~\ref{sec:poly_protocols_appendix}) to hybrid model SNARKs as per 
definition~\ref{dfn_snark} such that the conditional NP relations that define the SNARKs we compile in the second step contain both 
(polynomial) commitments and vectors of field elements as public inputs. For completeness and as an example, in appendix~\ref{sec:rolled_out} 
we give the full rolled-out hybrid model SNARK protocol $\Pah$ for relation $\Racom$, where we define $\Pah$ and $\Racom$ in Step 2 of our compiler 
below. By using just the first step of our compiler which is equivalent (modulo some more clarifications necessary for our use cases) 
to the original PLONK compiler, one would not be able to obtain SNARKs with mixed public inputs consisting of both vectors of field 
elements and also (polynomial) commitments. In turn, this type of NP relations with mixed inputs is crucial for designing and proving the security of our 
accountable light clients in section~\ref{new_light_client}.\\

\noindent \textbf{Step 1 (PLONK compiler - from polynomial protocols to SNARKs):} \\
\label{compiler_step_1}

\noindent Our first step applies the PLONK compiler~\cite{plonk}. More precisely, we compile the information theoretical 
ranged polynomial protocols $\Pla$, $\Pa$ and 
$\Pvt$ for relations $\Rla$, $\Ra$ 
and $\Rvt$ respectively (as defined in sections \ref{sec_la}, \ref{sec_a}, \ref{sec_vt}) into 
computationally secure protocols against AGM adversaries. The resulting protocols are, in fact, SNARKs. 
In order to keep in sync with PLONK notation, we denote the resulting SNARK protocols by $\Plastar$, 
$\Pastar$ and $\Pvtstar$, respectively. In fact, we can define this compilation step in 
a general way, for any ranged polynomial protocols for relations (as per definition in appendix~\ref{sec:poly_protocols_appendix}). 
In order to do that we need: 

\begin{itemize}
\item  The batched version of KZG polynomial commitments~\cite{KZG_10} described in section 3 of PLONK~\cite{plonk}.\footnote{In fact, 
one can replace the use of KZG polynomial commitments with any binding polynomial commitment that has knowledge-soundness, including non-homomorphic polynomial commitments, 
such as FRI-based polynomial commitments (e.g., RedShift~\cite{redshift}). If the optimisation gained from PLONK linearisation technique is a goal, 
then, with minimal changes one can use any homomorphic polynomial commitment, e.g., the discrete logarithm based polynomial commitment 
from Halo~\cite{halo}.}
\item A general compilation technique: such a technique has been already defined in lemma 4.7 of PLONK; combined with lemma 4.5 
from PLONK this technique can be applied with minor adaptations (this includes the corresponding technical measures) to the notion of ranged 
polynomial protocols as defined in appendix~\ref{sec:poly_protocols_appendix}.  
\item So far, both the ranged polynomial protocols for relations and the protocols resulted after the first compilation step have been explicitly defined as interactive 
protocols. In order to obtain the non-interactive version of the latter (essentially the N in SNARK) one has to apply the Fiat-Shamir 
transform~\cite{FS_transform}, \cite{FS_transform_with_proof}, \cite{SE_plonk}.
\end{itemize}

\noindent Let $\mathcal{R}$ be a (conditional) NP relation, let $\mathscr{P}_{\mathcal{R}}$ be a ranged polynomial protocol for 
relation $\mathcal{R}$ and let $\mathscr{P}^*_{\mathcal{R}}$ be the SNARK compiled from $\mathscr{P}_{\mathcal{R}}$ using the PLONK compiler 
(as summarised above). Going into more detail, the above compilation technique requires the SNARK prover of  $\mathscr{P}^*_{\mathcal{R}}$ to compute 
polynomial commitments to all polynomials that the prover $\mathcal{P}_{poly}$ in $\mathscr{P}_{\mathcal{R}}$ sent to the ideal party $\mathcal{I}$. Analogously, 
it requires the SNARK verifier of $\mathscr{P}^*_{\mathcal{R}}$ to compute polynomial commitments to all pre-processed polynomials\footnote{This is a one-time computation that is 
reused by the SNARK verifier for all SNARK proofs over the same circuit.} as well polynomial commitments to polynomials the verifier $\mathcal{V}_{poly}$ 
in $\mathscr{P}_{\mathcal{R}}$ sent to the ideal party $\mathcal{I}$. Then, the SNARK prover sends the SNARK verifier openings to 
all the polynomial commitments computed by him as well as the polynomial commitments computed by the SNARK verifier. The SNARK 
prover additionally sends the corresponding batched proofs for polynomial commitment openings. In turn, the SNARK verifier accepts or rejects based 
on the result of the verification of the batched polynomial commitment scheme. \\

\noindent A more efficient compilation technique exists which reduces the number of polynomial commitments and alleged polynomial commitments openings 
(i.e., both group elements and field elements) sent by the SNARK prover to the SNARK verifier; this, in turn, reduces the size of the SNARK proof. 
This technique is called linearisation and is described, at a high level, after Lemma 4.7 in PLONK. The existing description however covers only the 
SNARK prover side and it does not detail the SNARK verifier side so in the following we cover that. \\

\noindent By functionality, the vectors that are handled by the the verifier $\mathcal{V}_{poly}$ are 
of two types: pre-processed vectors and public input vectors. These two types of vectors are used by $\mathcal{V}_{poly}$ 
to obtain, via interpolation over the range on which the respective range polynomial protocol is defined, pre-processed polynomials 
(as used in the definition~\ref{def_ranged_poly_protocol}, e.g., polynomial $aux(X)$ used in section~\ref{sec_a}) and 
public-inputs-derived polynomials (e.g., polynomials $pkx(X)$ and $pky(X)$ used in sections~\ref{sec_la},~\ref{sec_a},~\ref{sec_vt} 
and polynomial $b(X)$ used in section~\ref{sec_la}). The efficient linearisation technique allows the SNARK verifier to reduce the 
number of polynomial commitments it has to compute compared to the general PLONK compiler in the following way. Instead of 
having to compute polynomial commitments to all polynomials $\mathcal{V}_{poly}$ sends to $\mathcal{I}$ (including any corresponding 
pre-processed polynomials), the SNARK verifier computes polynomial evaluations at one or multiple random points (as per the linearisation 
step specific requirements) for all the polynomials that are either easy to evaluate (e.g., polynomial $aux(X)$ used in section~\ref{sec_a}) or 
all the polynomials that are obtained from vectors that do not take up a large amount of memory (e.g., polynomial $b(X)$ used in section~\ref{sec_la}). 
For the rest of the polynomials (e.g., $\mathit{pkx}(X)$ and $\mathit{pky}(X)$), the SNARK verifier computes polynomial commitments as before.\\

\noindent We note we can apply all the techniques mentioned above, including the combined prover-and-verifier-side linearisation 
to compile our three ranged polynomial protocols $\Pla$, $\Pa$ and $\Pvt$ into the corresponding SNARKs $\Plastar$, $\Pastar$ and $\Pvtstar$, respectively. 
To conclude this step, we formally state in appendix~\ref{sec:poly_protocols_appendix}, lemma~\ref{le:compilation_step_1} under which condition and how efficiently 
one can compile ranged polynomial protocols for conditional NP relations (where the public inputs are interpreted as vector of field elements) 
into hybrid model SNARKs by using only the original PLONK compiler. \\

\noindent \textbf{Step 2 (Mixed Vector and Commitments as Input for NP Relations and Associated SNARKs):} \\
\label{compiler_step_2}

\noindent The type of NP relations we have worked with so far as well as the more general PLONK NP relation 
(\cite{plonk}, section 8.2), do not have the result of cryptographic operations as part of their public input but rather the 
public inputs are interpreted by honest parties as vectors of field elements. In the following, we show that the SNARKs 
we have compiled using Step 1 can become, under certain trust assumption, SNARKs for an additional type of NP relation 
that specifically contains polynomial commitments as part of the input. As detailed in section~\ref{new_light_client}, interpreting 
our already compiled SNARKs as SNARKs for this additional type of NP relation is essential for modelling and achieving 
the security properties for our accountable light client systems.\\ 

\noindent In order to define Step 2 of our compiler, we need first to introduce some notation. 
To start with, assume a conditional NP relation $\mathcal{R}_{\mathit{vec}}^c$ (according to the notion introduced 
in section~\ref{sec:conditional_relations}) of the form:
\begin{align*}
\mathcal{R}_{\mathit{vec}}^c = \{&(\mathbf{input_1} \in \mathbf{\mathcal{D}_1}, \mathbf{input_2} \in\mathbf{\mathcal{D}_2}; \mathbf{witness_1}): \\  
&p_1(\mathbf{input_1}, \mathbf{input_2}, \mathbf{witness_1}) = 1 \ | \ c(\mathbf{input_1}) = 1 \ \wedge\ \\
&\wedge \ p_2(\mathbf{input_1}, \mathbf{input_2}, \mathbf{witness_1}) = 1 \},
\end{align*}
\noindent where $\mathbf{input_1}$ is a set of public input vectors that should be parsed (but not checked) 
by the honest parties as belonging to some domain $\mathcal{D}_1$. Analogously, for the set of public input vectors $\mathbf{input_2}$ and their respective domain 
$\mathcal{D}_2$. Finally, $\mathbf{witness_1}$ is a set of witness vectors and $c$ and $p_1$ and $p_2$ are predicates. 
Let $\mathscr{P}_{\mathit{vec}}$ be a ranged polynomial protocol for relation 
$\mathcal{R}_{\mathit{vec}}^c$. Note that since the condition predicate $c$ applies only to a part of the public input for relation $\mathcal{R}_{\mathit{vec}}^c$ 
(i.e., $\mathbf{input_1}$), we can apply lemma~\ref{le:compilation_step_1} and Step 1 of our compiler to polynomial protocol $\mathscr{P}_{\mathit{vec}}$. \\

\noindent Next, we make the following assumptions which we call hybrid model assumptions:
\begin{itemize}
\item (HMA.1.) The verifier $\mathcal{V}_{poly}$ in $\mathscr{P}_{\mathit{vec}}$ computes polynomials 
$\mathit{Q_{1,\mathbf{input_1}}}(X), \ldots, \mathit{Q_{m, \mathbf{input_1}}}(X)$ which depend deterministically on $\mathbf{input_1}$ and sends them to $\mathcal{I}$. 
\item (HMA.2.) The verifier $\mathcal{V}_{poly}$ in $\mathscr{P}_{\mathit{vec}}$ does not use $\mathbf{input_1}$ in any further computation of 
any other polynomials or values its sends to $\mathcal{I}$.
\item (HMA.3.) By evaluating $\mathit{Q_{1,\mathbf{input_1}}}(X), \ldots, \mathit{Q_{m, \mathbf{input_1}}}(X)$ over the range on which the ranged polynomial protocol 
$\mathscr{P}_{\mathit{vec}}$ is defined one obtains (using some efficiently computable and deterministic transformations) the set of vectors $\mathbf{input_1}$. 
\end{itemize} 
Let us denote by $\mathscr{P}^*_{\mathit{vec}}$ the hybrid model SNARK obtained after compiling $\mathscr{P}_{\mathit{vec}}$ using Step 1 of our compiler. 
Due to assumption (HMA.1.) and according to Step 1 of our compiler, the SNARK verifier in 
$\mathscr{P}^*_{\mathit{vec}}$ computes $$\mathit{Com_1} = \mathit{Com}(\mathit{Q_{1,\mathbf{input_1}}}), \ldots, \mathit{Com_m} = \mathit{Com}(\mathit{Q_{m,\mathbf{input_1}}})$$ 
which are KZG polynomial commitments to $\mathit{Q_{1,\mathbf{input_1}}}(X), \ldots, \mathit{Q_{m, \mathbf{input_1}}}(X)$. For brevity, we denote the vector
$(\mathit{Com_1}, \ldots, \mathit{Com_m})$ by $\mathbf{Com}(\mathbf{input_1})$ and we denote 
by $\mathcal{C}$ the set of all $\mathit{KZG}$ polynomial commitments or vectors of such polynomial commitments. \\

\noindent Let us also define the relation: 
\begin{align*}
\mathcal{R}_{\mathit{vec}, \mathit{com}}^c = \{& \mathbf{C} \in \mathcal{C}, \mathbf{input_2} \in \mathbf{\mathcal{D}_2}; \mathbf{witness_1}, \mathbf{witness_2}):  \\
& p_1(\mathbf{witness_2}, \mathbf{input_2}, \mathbf{witness_1}) =1  \ |\ c(\mathbf{witness_2}) = 1  \ \wedge\  \\
& \wedge\ p_2(\mathbf{witness_2}, \mathbf{input_2}, \mathbf{witness_1}) = 1\ \wedge \\
& \wedge\ \mathbf{C} = \mathbf{Com}(\mathbf{witness_2})\}
\end{align*}

\noindent For some $\mathit{srs}$, $mathbf{witness_2}$ part of $\mathit{state_1}$ and 
$\mathcal{R} = \mathcal{R}_{\mathit{vec}, \mathit{com}}^c$, let us finally define $\mathit{SNARK.PartInput}$ as follows:

\begin{align*} 
&\mathit{SNARK.PartInput}(\mathit{srs}, \mathit{state_1},\mathcal{R}_{\mathit{vec}, \mathit{com}}^c) \\  
& \mathit{If \ }  c(\mathbf{input_1}) = 0 \\
& \ \ \ \ \mathit{Return} \\
& \mathit{Else} \\
& \ \ \ \ \textit{Compute via interpolation on the range for } \mathscr{P}_{\mathit{vec}}  \textit{ polynomials } 
\mathit{Q_{1,\mathbf{input_1}}}(X), \ldots, \mathit{Q_{m, \mathbf{input_1}}}(X).\\
& \ \ \ \ \mathbf{C} = (\mathit{Com}(Q_{1,\mathbf{input_1}}(X)), \ldots, \mathit{Com}(Q_{m,\mathbf{input_1}}(X))) \\
& \ \ \ \ \mathit{state_2} =  \mathit{state_1} \cup \{ \mathbf{C} \} \\
& \ \ \ \ \mathit{Return} (\mathit{state_2,  \mathbf{C}})
\end{align*}

\noindent With the above notation, \textbf{our compiler's Step 2 is:} \\
\noindent The alleged hybrid model SNARK $\mathscr{P}_{\mathit{vec}}^{h}$ for relation $\mathcal{R}_{\mathit{vec}, \mathit{com}}^c$ is defined as:
\begin{itemize}
\item $\mathit{SNARK.Setup}$ and $\mathit{SNARK.KeyGen}$ are the same as for relation $\mathcal{R}^{c}_{\mathit{vec}}$.
\item The algorithm $\mathit{SNARK.PartInput}$ for relation $\mathcal{R}^{c}_{\mathit{vec}}$ 
(see lemma~\ref{le:compilation_step_1} in appendix~\ref{sec:poly_protocols_appendix}) 
is replaced with $\mathit{SNARK.PartInput}$ for relation $\mathcal{R}_{\mathit{vec}, \mathit{com}}^c$ as described above.
\item The algorithm $\mathit{SNARK.Prover}$ for relation $\mathcal{R}_{\mathit{vec}, \mathit{com}}^c$ is identical with the algorithm 
$\mathit{SNARK.Prover}$ for relation $\mathcal{R}^{c}_{\mathit{vec}}$ (as compiled using Step 1) with the appropriate 
re-interpretation of the public inputs and witness.
\item The algorithm $\mathit{SNARK.Verifier}$ for relation $\mathcal{R}_{\mathit{vec}, \mathit{com}}^c$ is identical with the algorithm 
$\mathit{SNARK.Verifier}$ for relation $\mathcal{R}^{c}_{\mathit{vec}}$ (as compiled using Step 1) with the appropriate 
re-interpretation of the public inputs and such that $\mathit{SNARK.Verifier}$ for $\mathcal{R}_{\mathit{vec}, \mathit{com}}^c$ does
not compute the polynomial commitments to the polynomials defined by the assumption (HMA.1.).
\end{itemize}
\noindent \begin{lemma} 
\label{sergey_type_relations} 
Let $\mathscr{P}_{\mathit{vec}}$ be a ranged polynomial protocol for relation $\mathcal{R}^c_{\mathit{vec}}$ defined above and let 
$\mathscr{P}_{\mathit{vec}}^{*}$ be the hybrid model SNARK for relation $\mathcal{R}^c_{\mathit{vec}}$ secure in the AGM obtained 
by compiling $\mathscr{P}_{\mathit{vec}}$ using our compiler's Step 1. If the hybrid model assumptions (HMA.1.) - (HMA.3.) hold w.r.t. 
protocol $\mathscr{P}_{\mathit{vec}}$ and $\mathit{State}_{\mathcal{R}_{\mathit{vec}, \mathit{com}}} \neq \emptyset $ then 
$\mathscr{P}_{\mathit{vec}}^{h}$ as compiled using our compiler's Step 2 is a hybrid model SNARK for relation 
$\mathcal{R}_{\mathit{vec}, \mathit{com}}^c$ secure also in the AGM.
\end{lemma}

\begin{proof} Let $\mathcal{E}_{\mathit{KZG}}$ and $\mathcal{E}$ be the extractors from the knowledge-soundness definitions for the 
$\mathit{KZG}$ batch polynomial commitment scheme (as in definition 3.1, section 3 in~\cite{plonk}) and the hybrid model 
SNARK $\mathscr{P}^*_{\mathcal{R}}$ for relation $\mathcal{R}^c_{\mathit{vec}}$ (as per definition~\ref{dfn_snark}), respectively. 
Let $\mathcal{A}$ be an adversary against knowledge soundness in the hybrid model w.r.t. 
$\mathscr{P}_{\mathit{vec}}^{h}$ and relation $\mathcal{R}_{\mathit{vec}, \mathit{com}}^c$ and let $\mathit{aux}_{\mathit{SNARK}} \in \mathcal{D}$ 
and let $\mathit{state_1} \in \mathit{State}_{\mathcal{R}_{\mathit{vec}, \mathit{com}}}$; let 
$(\mathbf{C},\mathit{state_2}) = \mathit{SNARK.PartInput}(\mathit{srs}, \mathit{state_1}, \mathcal{R}_{\mathit{vec}, \mathit{com}}^c)$. 
By the definition of $\mathit{SNARK.PartInput}$ for $\mathscr{P}_{\mathit{vec}}^{h}$, there exists 
$\mathbf{input_1}$ such that $\mathbf{C} = \mathbf{Com}(\mathbf{input_1})$ and $c(\mathbf{input_1}) = 1$. 
We denote by $(\mathbf{input_2}, \pi)$ the output of $\mathcal{A}(\mathit{srs}, \mathit{state_2},  \mathcal{R}_{\mathit{vec}, \mathit{com}}^c)$ 
and let $\mathcal{A}_1$ be the part of $\mathcal{A}$ that sends openings and batched proofs for the polynomial commitments in 
$\mathbf{C}$. \\

\noindent On the one hand, if the verifier $\mathit{SNARK.Verifier}(\mathit{srs}_{\mathit{vk}}, (\mathbf{C},\mathbf{input_2}),\pi,\mathcal{R}_{\mathit{vec}, \mathit{com}}^c)$ 
in $\mathscr{P}_{\mathit{vec}}^{h}$ accepts, then the $\mathit{KZG}$ verifier corresponding to 
$\mathcal{A}_1$ also accepts. When such an event takes place, then, \ewnp $\mathcal{E}_{\mathit{KZG}}$ extracts polynomials 
$Q'_1(X), \ldots, Q'_m(X)$ that represent witnesses for the vector $\mathbf{C}$ of commitments and the alleged openings of $\mathcal{A}_1$. 
Because the $\mathit{KZG}$ polynomial commitment scheme is binding and by the definition of 
$\mathit{SNARK.PartInput}$ for $\mathscr{P}_{\mathit{vec}}^{h}$, we obtain that $Q'_1(X) = Q_1(X), \ldots, Q'_m(X) = Q_m(X).$ 
Since per (HMA.3.), the set $\{Q_1(X), \ldots, Q_m(X)\}$ evaluates to $\mathbf{input_1}$ over the range over which $\mathscr{P}_{\mathit{vec}}$ 
was defined, \ewnp the witness polynomials extracted by $\mathcal{E}_{\mathit{KZG}}$ evaluate to $\mathbf{input_1}$. \\

\noindent On the other hand, if the verifier $\mathit{SNARK.Verifier}(\mathit{srs}_{\mathit{vk}}, (\mathbf{C},\mathbf{input_2}),\pi,\mathcal{R}_{\mathit{vec}, \mathit{com}}^c)$ 
in $\mathscr{P}_{\mathit{vec}}^{h}$ accepts, then the verifier 
$\mathit{SNARK.Verifier}(\mathit{srs}_{\mathit{vk}}, (\mathbf{input_1},\mathbf{input_2}),\pi,\mathcal{R}_{\mathit{vec}}^c)$ 
in $\mathscr{P}_{\mathit{vec}}^{*}$ also accepts. In turn, this acceptance together with the fact that $\mathscr{P}_{\mathit{vec}}^{*}$ 
has knowledge-soundness as per definition~\ref{dfn_snark}, it implies the extractor $\mathcal{E}$ \ewnp extracts $\mathbf{witness_1}$ 
such that $$(\mathbf{input_1}, \mathbf{input_2}, \mathbf{witness_1}) \in \mathcal{R}_{\mathit{vec}}^{c} \ \ \ \ \ (o).$$ 

\noindent By the definition of $\mathit{SNARK.PartInput}$ for $\mathscr{P}_{\mathit{vec}}^{h}$ and the way $\mathbf{input_1}$ was defined, 
it holds that $c(\mathbf{input_1}) = 1$. Due to $(o)$ and by the definition of relation $\mathcal{R}_{\mathit{vec}}^{c}$, 
the predicates: $p_1$($\mathbf{input_1}$, $\mathbf{input_2}$, $\mathbf{witness_1}$) $= 1$ and 
$p_2(\mathbf{input_1}, \mathbf{input_2}, \mathbf{witness_1}) = 1$ hold. If we let 
$\mathbf{witness_2} = \mathbf{input_1}$, then it is clear that 
$$(\mathbf{C} = \mathbf{Com}(\mathbf{input_1}), \mathbf{input_2}, \mathbf{witness_1}, \mathbf{input_1}) \in \mathcal{R}_{\mathit{vec}, \mathit{com}}^c,$$ so 
using $\mathcal{E}_{\mathit{KZG}}$ and $\mathcal{E}$ we can build an extractor for any knowledge-soundness adversary $\mathcal{A}$ for alleged 
hybrid model SNARK $\mathscr{P}_{\mathit{vec}}^{h}$ for relation $\mathcal{R}_{\mathit{vec}, \mathit{com}}^c$, which concludes the proof.
\end{proof}


\noindent To conclude this step and the detailed compiler presentation we note that it is straightforward to apply the technique 
described above to our SNARKs $\Plah$, $\Pah$ and $\Pvth$ 
compiled in Step 2 and obtain relations $\Rlacom$, $\Racom$ 
and $\Rvtcom$ that fulfil lemma~\ref{sergey_type_relations}.\footnote{Note that due to our specific application 
and the proof-of-stake blockchain context in which we make use of our custom SNARKs, 
the assumption/requirement that  $\mathit{State}_{\mathcal{R}_{\mathit{vec}, \mathit{com}}} \neq \emptyset$ for 
$\mathcal{R}_{\mathit{vec}, \mathit{com}} \in \{\Rlacom, \Racom, \Rvtcom\}$ is fulfilled.}
For completeness, we include these three conditional NP relations below. We remind the reader that these NP relations depend on $\lambda$, 
but, for brevity, we have omitted it from the following definitions.  
\begin{align*}
{\Rlacom} = \{(&\mathbf{C} \in \mathcal{C}, \mathbf{bit} \in \mathbb{B}^n, \mathit{apk} \in \mathbb{F}^2; \mathbf{pk}) : 
\mathit{apk} = \sum_{i=0}^{n-2} [\mathit{bit_i}] \cdot \mathit{pk_i} \ | \ \mathbf{pk} \in \ginn{1}^{n-1} \ \wedge \ \\
& \wedge \  \mathbf{C} = \mathbf{Com}(\mathbf{pk}) \} 
\end{align*}
\vspace*{-0.75cm}
\begin{align*}
{\Racom}  = \{(&\mathbf{C} \in \mathcal{C}, \mathbf{b'} \in \mathbb{F}_{|\block|}^{\frac{n}{\block}}, \mathit{apk} \in \mathbb{F}^2;\mathbf{pk}, \mathbf{bit}) : 
\mathit{apk} = \sum_{i=0}^{n-2} [\mathit{bit_i}] \cdot \mathit{pk_i} \ | \ \mathbf{pk} \in \ginn{1}^{n-1} \ \wedge \ \\
& \wedge \mathbf{bit} \in \mathbb{B}^n  \wedge b'_{j} = \sum_{i=0}^{\block -1}2^i \cdot \mathit{bit_{\block \cdot j + i}}, \forall j < \frac{n}{\block}  \wedge \  \mathbf{C} = \mathbf{Com}(\mathbf{pk}) \} 
\end{align*}
\vspace*{-0.75cm}
\begin{align*}
{\Rvtcom} = \{(&\mathbf{C} \in \mathcal{C}, s \in \mathbb{F}^2, apk \in \mathbb{F}^2 ; \mathbf{pk}, \mathbf{bit}) : 
\mathit{apk} = \sum_{i=0}^{n-2} [\mathit{bit_i}] \cdot \mathit{pk_i} \ | \ \mathbf{pk} \in \ginn{1}^{n-1} \ \wedge \ \\
& \wedge \ \mathbf{bit} \in \mathbb{B}^n  \wedge  \sum_{i=0}^{n-1} \mathit{bit_i} = s+1 \ \wedge \mathbf{C} = \mathbf{Com}(\mathbf{pk}) \}
\end{align*}
\vspace*{-0.75cm}

 \subsection{Comparison between PLONK and our SNARKs}
\label{sec:plonk_comparison}

\noindent In the following, we briefly look at the differences between PLONK universal SNARK and the SNARKs designed in this work. We observe 
that while the NP relation that defines PLONK is more general, the relations that define our SNARKs are bespoke as we are only interested in efficiently proving 
public key aggregation. Because our relations are so bespoke, it turns out we do not require the full functionality that PLONK has to offer, and, in particular, our SNARKs  
do not require any permutation argument. \\

\noindent A second difference is that while PLONK's circuit is defined by a number of selector 
polynomials (which are a type of pre-processed polynomials) and a PLONK verifier needs to perform 
a one-time expensive computation of the polynomial commitments to those selector polynomials, our SNARK verifiers 
are able to avoid such a pre-processing phase. Indeed, in the case of $\mathscr{P}^h_{\mathsf{a}}$ (which is the only one of 
our three SNARKs that has a polynomial, namely $\mathit{aux}(X)$, that defines its circuit), our respective SNARK verifier does not need to compute a 
commitment to its only ``selector polynomial'' as, due to its structure, $\mathit{aux}(X)$ can be directly and efficiently evaluated by our SNARK verifier itself. \\ 
%Hence, in this case, we can use a simpler form of the corresponding linearisation polynomial which includes only the evaluation of 
%$aux(X)$\footnote{In fact, we need an evaluation of $aux(\omega \cdot X)$.} instead of $aux(X)$ itself.  \\

%\noindent A third difference is that PLONK relation (see section 8.2 in~\cite{plonk}) has a private witness while one of our three NP relations, 
%i.e., $\mathcal{R}^{\mathit{incl}}_{\mathsf{la}}$, has only public inputs. \\

%\noindent Another difference is that for all our three NP relations $\mathcal{R}^{\mathit{incl}}_{\mathsf{la}}$, $\mathcal{R}^{\mathit{incl}}_{\mathsf{a}}$ 
%and $\mathcal{R}^{\mathit{incl}}_{\mathsf{vt}}$ the public input contains a vector which has essentially the same length $O(n)$ 
%as the degree of the polynomials committed to by the SNARK prover. In particular this implies that after completing the first step of our compiler~\ref{sec_two_step_compiler} 
%(which is equivalent to running the original PLONK compiler with its linearisation optimisation) the SNARK verifier ends up parsing an $O(n)$ long vector 
%of public keys, computing the polynomials $pkx(X)$ and $pky(X)$ associated with the affine coordinates of the public keys and, finally, computing the 
%polynomial commitments to these two polynomials. Since public inputs, in general, and the vector of public keys, in particular, may differ between different 
%calls, the expensive computation described above cannot be reused by the SNARK verifier (as it is the case for the SNARK verifier in PLONK relative to the commitments to the 
%pre-processed polynomials) and, thus, would render our SNARK verifier less practical. However, the second and third steps of our compiler take care of this aspect by offloading 
%the expensive computation to a third trusted party. \\
\noindent A third difference is that using our two-steps compiler, our SNARKs verifiers are able to efficiently handle input vectors of length $O(n)$, 
where the degree of the polynomials committed to by our SNARK provers is also $O(n)$. Our SNARKs verifiers achieve efficiency by offloading the 
expensive polynomial commitment computation involving the public inputs to a trusted third party. \\

\noindent Moreover, while PLONK does not incorporate trusted inputs, one can easily apply the Step 2 of our compiler to PLONK. In particular, one could imagine a situation 
where a PLONK verifier is relying on a trusted party to compute some or all of the polynomial commitments to the circuit's selector polynomials. This is equivalent to our hybrid 
model SNARK definition applied to PLONK. The benefit is that by delegating such a computation, the PLONK verifier becomes more efficient. \\ %at the expense of relying on some trust assumptions. \\
%\noindent Finally, PLONK system does not incorporate by default trusted inputs in the sense we have defined in section~\ref{sec_two_step_compiler} but it can 
%be modified in a simple way to do that. For example, one could imagine a situation where a PLONK verifier is relying on a trusted party to compute some or all of 
%the polynomial commitments to the circuit's selector polynomials. This is the equivalent to our hybrid model applied to PLONK and the benefit is that by delegating this 
%computation the verifier becomes more efficient at the expense of relying on some trust assumptions.}
%{\color{blue}TO DO: Summarise how we can get rid of trust assumptions as well, in our case. For example: In a real world implementation, the trust assumption can, in turn, 
%be replaced by having, for example, the commitments signed by a set of parties whose public keys are known to the verifier and whom the verifier trusts that will not collude.}\\

%{\color{blue} \noindent TO DO Once we have defined the Polkadot instantiation for our light client in section~\ref{sec:instantiation} 
%and we have (an informal) sequential proof for soundness (for example), we can connect to that and mention that trusted inputs have even more of an efficiency 
%impact in case of our SNARKs since they replace computation that should be otherwise performed by our SNARK verifier at every step, while, in case of PLONK, 
%delegating the one-off computation of commitments to selector polynomials, even though beneficial as well, may not have had such an efficiency impact on the 
%verifier for the long term.}

\noindent Finally, looking ahead at our light client system instantiation in section~\ref{sec:LCinstantiation}, due to the inductive structure of the soundness proof (theorem~\ref{th:soundness}), 
the efficiency of using a hybrid model SNARK has an even greater impact for the light client system verifier than that compared to verifying multiple instances of PLONK for the same circuit:
while for the latter the PLONK verifier has to compute commitments to selector polynomial only once anyway, in the case of the former, the commitments to public inputs may differ at very step hence a trusted third party relives a higher computation burden from the light client verifier overall. 



\subsection{An Instantiation for Committee Key Scheme for Aggregatable Signatures}
\label{sec:inst_committee_key}
\noindent Given the SNARKs compiler described in section~\ref{sec_two_step_compiler} and its application to the conditional NP relations 
mentioned at the end of that section, we are ready to present an instantiation for committee key scheme for aggregatable signatures 
(see section~\ref{sec:committee_key} for the definition of this notion) as used in this work (i.e. in section~\ref{sec:LCinstantiation}). 
We instantiate $u$ and $v$ introduced in section~\ref{sec:committee_key} as follows: let $u = n-1$, where $n$ was defined in 
section~\ref{sec:lagrange} and we let $v \in \mathbb{N}, n-1 \leq v$, $v = \mathsf{poly}(\lambda)$, where by $v$ we denote the 
maximum number of validators that the scheme allows.

\begin{construction}(Committee Key Scheme for Aggregatable Signatures)
\label{inst:cks} In our implementation we call committee key scheme for 
aggregatable signatures the following instantiation of definition~\ref{def: committee_key}, where $\mathcal{R} \in \{\Rlacom, \Racom\}$ 
as defined in the end of section~\ref{sec_two_step_compiler}:
\begin{itemize}
\item $\mathit{CKS_{\mathcal{R}}.Setup}(v)$ calls the following algorithms \\
$(\ginn{1}, \sginn{1}, \ginn{2}, \sginn{2}, \gtinn, \epinn, \Hinn, \HPoP)$ from $\mathit{pp}$ where $\mathit{pp} \leftarrow \mathit{AS.Setup}(\mathit{aux_{\mathit{AS}}}= v+1)$ 
which is part of instantiation~\ref{insta:bls} with the additional specification that $\mathit{aux_{\mathit{AS}}} = v+1$ and using the notation detailed in section~\ref{sec:bls}, \\
$\mathit{srs}=([1]_{\indexoneout}, [\tau]_{\indexoneout}, [\tau^2]_{\indexoneout}, \ldots, [\tau^{3v}]_{\indexoneout}, [1]_{\indextwoout}, [\tau]_{\indextwoout}) \leftarrow \mathit{SNARK.Setup}(\mathit{aux_{\mathit{SNARK}}} = (v, 3v))$, 
\vspace{-0.20cm}
\begin{align*}
&(\mathit{rs}_{\mathit{pk}}, \mathit{rs}_{\mathit{vk}}) = \\
& = (([1]_{\indexoneout}, [\tau]_{\indexoneout}, [\tau^2]_{\indexoneout}, \ldots, [\tau^{3v}]_{\indexoneout}), 
([1]_{\indexoneout}, [1]_{\indextwoout}, [\tau]_{\indextwoout})) \leftarrow \mathit{SNARK.KeyGen}(\mathit{srs}, \mathcal{R})
\end{align*}
%where $\mathcal{R} \in \{\Rlacom, \Racom \}$ is one of the accountable relations defined in the end of section~\ref{sec_two_step_compiler}.

\item $\mathit{ck} = ([pkx]_{\indexoneout}, [pky]_{\indexoneout}) \leftarrow \mathit{CKS_{\mathcal{R}}.GenerateCommitteeKey}(\mathit{rs_{pk}}, (\mathit{pk_i})_{i=1}^{n-1})$, 
where \\ $\mathbf{pkx} = (\mathit{pkx_{1}}, \ldots, \mathit{pkx_{n-1}})$ and $\mathbf{pky} = (\mathit{pky_{1}}, \ldots, \mathit{pky_{n-1}})$ 
such that $\forall i \in \{1, \ldots, n-1\}$, $\mathit{pk_i} = (\mathit{pkx_i}, \mathit{pky_i}) \in \mathbb{F}^{2}$ and 
the polynomials $pkx(X) = \sum_{i=0}^{n-2} \mathit{pkx_{i+1}} \cdot L_i(X)$ and $pky(X) = \sum_{i=0}^{n-2} \mathit{pky_{i+1}} \cdot L_i(X)$ and, finally,  
$[pkx]_{\indexoneout} = pkx(\tau) \cdot [1]_{\indexoneout}$ and $[pky]_{\indexoneout} = pky(\tau) \cdot [1]_{\indexoneout}$. Note that 
$\mathit{CKS_{\mathcal{R}}.GenerateCommitteeKey}$ first checks whether $(\mathit{pk_i})_{i=1}^{n-1} \in \ginn{1}^{n-1} (\ast) $; if that is not the case it outputs 
$\bot$; if $(\ast)$ holds, the algorithm $\mathit{CKS_{\mathcal{R}}.GenerateCommitteeKey}$ continues with the computations described above. 

\item $\pi = (\pi_{SNARK}, \mathit{apk}) \leftarrow \mathit{CKS_{\mathcal{R}}.Prove}
(\mathit{rs}_{\mathit{pk}}, \mathit{ck}, (\mathit{pk_i})_{i=1}^{n-1}, (\mathit{bit_i})_{i=1}^{n-1})$ 
where $\mathit{CKS_{\mathcal{R}}.Prove}$ calls \\ 
$\mathit{apk} = \sum_{i=1}^{n-1} \mathit{bit_i} \cdot \mathit{pk_i} \leftarrow \mathit{AS.AggregateKeys}(\mathit{pp}, (\mathit{pk_i})_{i:\mathit{bit_i = 1}})$ 
as defined in instantiation~\ref{insta:bls} and \\
$\pi_{SNARK} \leftarrow \mathit{SNARK.Prove}(\mathit{rs_{pk}}, (x,w), \mathcal{R})$, for $\mathcal{R} \in \{\Rlacom, \Racom \}$ \\ where
$(x = (\mathit{ck}, (\mathit{bit_i})_{i=1}^{n-1}||0, \mathit{apk}), w = (\mathit{pk}_i)_{i=1}^{n-1})$ for $\mathcal{R} = \Rlacom$ and \\
$(x = (\mathit{ck}, \mathbf{b'}, \mathit{apk}), w = ((\mathit{pk}_i)_{i=1}^{n-1}, (\mathit{bit_i})_{i=1}^{n-1}||0)$ for $\mathcal{R} = \Racom$, 
where $\mathbf{b'}$ is the vector of field elements formed from blocks of size $\mathsf{block}$ of bits from vector $(\mathit{bit_i})_{i=1}^{n-1}||0$ 
and $\mathsf{block}$ is the highest power of 2 smaller than the size of a field element in $\mathbb{F}$. 

\item $0/1 \leftarrow \mathit{CKS_{\mathcal{R}}.Verify}(\mathit{pp}, \mathit{rs}_{\mathit{vk}}, \mathit{ck}, m, \mathit{asig}, \pi, \mathbf{bitvector})$ 
parses $\pi$ to retrieve $\pi_{\mathit{SNARK}}$ and $\mathit{apk}$ and it calls $\mathit{AS.Verify(\mathit{pp}, \mathit{apk}, m, \mathit{asig})}$ 
as defined in instantiation~\ref{insta:bls} and it also calls \\
$\mathit{SNARK.Verify}(\mathit{rs_{vk}}, x, \pi_{\mathit{SNARK}}, \mathcal{R})$ (where $\pi_{\mathit{SNARK}}$, $x$ and $\mathcal{R}$ are as defined in the 
paragraph above with the only difference that $(\mathit{bit_i})_{i=1}^{n-1}$ represents the first $n-1$ bits of $\mathbf{bitvector}$, padded with $0$s, if not sufficiently 
many exist in $\mathbf{bitvector}$); it outputs $1$ if both algorithms output $1$ and it outputs $0$ otherwise.
\end{itemize}
\end{construction}

\begin{theorem}Given the hybrid model SNARK scheme secure for relation $\mathcal{R} \in \{ \Rlacom, \Racom\}$ as 
obtained using our two-step compiler in section~\ref{sec_two_step_compiler} and the aggregatable signature scheme $\mathit{AS}$ 
                     as per instantiation~\ref{insta:bls} (which fulfils definition~\ref{def:aggregate_signatures}) with the additional 
                     specification that $\mathit{aux}_{\mathit{AS}} = v+1$ and choosing $v = n-1$, 
if we assume that an efficient adversary (against the soundness of) $\mathit{CKS}_{\mathcal{R}}$ outputs public keys only from the source group $\ginn{1}$,  
then the committee key scheme $\mathit{CKS}_{\mathcal{R}}$ as per instantiation~\ref{inst:cks} is secure with respect to definition~\ref{def: committee_key}.
\end{theorem}

%\begin{theorem} Given the committee key scheme instantiation~\ref{inst:cks} and assuming that the hybrid model SNARK 
%scheme used is secure for either relation $\Rlacom$ or $\Racom$ according to definition~\ref{dfn_snark} and the aggregatable 
%signature scheme used fulfils definition~\ref{def:aggregate_signatures}, then the committee key scheme instantiation~\ref{inst:cks} 
%is secure as per definition~\ref{def: committee_key}. 
%\end{theorem}

\noindent For details of the proof, please see Appendix~\ref{sec:proof_sec_cks}.

