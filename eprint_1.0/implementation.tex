
\noindent We implemented and benchmarked the protocol. The implementation allows us to evaluate the performance of our protocol and serve as prototype for future deployment. The implementation is available at \url{https://github.com/w3f/apk-proofs}. It is written in Rust and uses the Arkworks library. \\

\noindent Table \ref{tab:benchmarks} gives the prover and verifier time for the three schemes (basic accountable, packed accountable and counting, see Section 3) with $v = n-1 = 2^{10}-1$, $v = n-1 = 2^{16}-1$ 
and $v=n-1=2^{20}-1$ signers. The benchmarks were run on commodity hardware, with an i7 and 16GB RAM. We remind the reader that by $v$ we denote the maximum number of  validators in our system and that $n$ was defined in section~\ref{sec:lagrange}.\\

\noindent These signer numbers are approximately the range of the number of validators that we were aiming our implementation at e.g. the Kusama blockchain (\url{https://kusama.network/}) has 1000 validators which is also the number that Polkadot is aiming for, and Ethereum 2 has about 348,000 validators and it has been suggested that there will be no more than $2^{19}$~\cite{ethresearch1}. \\

\noindent At $v = n-1 = 1023$, the prover can generate a proof in any scheme in well under a second, which is short enough to generate a proof for every block in most prominent blockchain protocols. Even for $v= n-1 =2^{20}-1$, the prover time is under 6.4 minutes, which is the time for an Ethereum 2 epoch, the time that validators finalise the chain. For verification time, the basic accountable scheme is slower, considerably so for larger signer numbers. \\

\noindent  Table \ref{tab:operations} gives the number of operations the prover and verifier use. Table \ref{tab:proof-size} gives the proof constituents and also the total proof and input sizes in bits. The basic accountable scheme's verifier performance at large numbers is so slow because it includes $O(n)$ field operations, which dominate the running time, however at 1023 signers it gives the smallest size. The packed accountable scheme, which includes $O(n/\lambda)$ field operations, fairs better on the benchmarks, having similar verification time than the counting scheme which has sublinear verification time, even at $2^{20}-1$ signers. The prover is considerably slower for the latter two schemes because it needs to do additional operations. At larger signer sizes, the proof size for the accountable schemes is dominated by the bitfield.

\begin{table}[h!]
\begin{tabular}{| l | l | l | l | l |l | l |}

\hline

 Scheme & \multicolumn{2}{|c|}{$v = 2^{10}-1$} & \multicolumn{2}{|c|}{$v = 2^{16}-1$} & \multicolumn{2}{|c|}{$v = 2^{20}-1$}	 \\
\cline{2-7}
 &  prover & verifier & prover & verifier &  prover & verifier \\
\hline
Basic Accountable & 587.376ms & 27.388ms & 20.587s & 36.452ms & 258.417s & 146.006ms \\
Packed Accountable & 544.614ms & 16.565ms & 28.398s & 36.908ms & 398.030s & 25.867ms \\
Counting 			& 510.316ms & 15.758ms & 28.461s & 26.089ms & 357.939s & 27.270ms \\
\hline
\end{tabular}
\caption{Proof and verifier times for the different schemes and different numbers of signers}
\label{tab:benchmarks}
\end{table}


\begin{table}[h!]
\begin{tabular}{| l | l| l| l|}
\hline
Scheme & Prover operations  &Verifier operations \\
\hline
Basic Accountable & $12\times FFT(N)+FFT(4N)+9ME(N)$  & $2P+11E+O(n)F$ \\
Packed Accountable & $18\times FFT(N)+FFT(4N)+12ME(N)$  & $2P+16E+O(n/\lambda+log(n))F$ \\
Counting & $13\times FFT(N)+FFT(4N)+11ME(N)$  & $2P+14E+O(log(n))F$ \\
\hline
\end{tabular}
\caption{Expensive prover and verifier operations. $FFT(M)$ is an FFT of size M. $ME(M)$ is a multi-exponentiation (or multi-scalar multiplication) of size $M$. $P$ is a pairing, $E$ is a single exponentiation (scalar multiplication) and $F$ is a field operation.}
\label{tab:operations}
\end{table}

\begin{table}[h!]
\begin{tabular}{| l | l | l | l | l | l |}
\hline
Scheme & Proof & Input & \multicolumn{3}{|c|}{Actual proof + input size in bits} \\
\cline{4-6}
& & & $v = 2^{10}-1$ & $v = 2^{16}-1$ & $v = 2^{20}-1$ \\
\hline
Basic Accountable & $5\mathbb{G}_{1,out}+5\mathbb{F}$ & $2\mathbb{G}_{1,out}+1\mathbb{G}_{1,inn}+n$ bits & 9088 & 73600 & 1056640 \\
Packed Accountable & $8\mathbb{G}_{1,out}+8\mathbb{F}$ & $2\mathbb{G}_{1,out}+1\mathbb{G}_{1,inn}+n$ bits & 12544 & 77056 & 1060096 \\
Counting &  $7\mathbb{G}_{1,out}+7\mathbb{F}$ & $2\mathbb{G}_{1,out}+1\mathbb{G}_{1,inn}$ & 10368  & 10368  & 10368 \\
\hline
\end{tabular}
\caption{Proof and input constituents and total proof and input size for the implementation.}
\label{tab:proof-size}
\end{table}
