\subsection{SNARKs}
\label{sec:snarks_defs}

\noindent All three SNARKs we design in this work have access to a \emph{structured reference string} (srs) of the form 
$(\{[\tau^i]_1\}_{i=0}^{d}$ , $\{[\tau^i]_2\}_{i=0}^{1})$ where $\tau$ is a random (and allegedly secret) value in $\mathbb{F}$ and $d$ 
is bounded by a polynomial in $\lambda$. Such an srs is \emph{universal} and \emph{updatable} \cite{updatable_universal_srs_2018} and, 
as long as at least one of the participants that took part in the MPC generating the srs was honest, the srs cannot be used by 
any coalition of other MPC participants to prove false statements with more than a negligible probability of 
success \cite{updatable_universal_srs_2018,ariel_MPC_SRS_2017}. \\

\noindent Our SNARKs are secure in the \emph{algebraic group model} (AGM) \cite{AGM_model}. 
If $\mathbb{G}$ is a cyclic group of prime order $p$, then, informally, we call an algorithm $\mathcal{A}$ \emph{algebraic} if it
fulfils the following requirement: whenever $\mathcal{A}$ outputs a group element $g \in  \mathbb{G}$, 
it also outputs a representation $\mathbf{a} = (a_1, . . . , a_t) \in \mathbb{Z}_p^t$ such that $g = \sum_{i=1}^{t} a_i \cdot B_i$
where $(B_1, \ldots, B_t )$ are all the $\mathbb{G}$ group elements that were given to $\mathcal{A}$ during its execution so far. 
The AGM lies in between the \emph{generic group model} (GGM) \cite{GGM_model1, GGM_model2} and the standard model and, 
lately, it has been the preferred model for proving security for the most efficient SNARKs 
(e.g., PLONK~\cite{plonk}, Marlin~\cite{marlin} or Groth16~\cite{groth16} with its proof in the AGM model 
presented in \cite{AGM_model, another_extractable_groth16}). \\

\noindent In the following, we introduce a generalisation of the usual SNARK definition which we call a \emph{hybrid model SNARK}. 
As mentioned in the introduction, this is inspired by the notion of online-offline SNARKs~\cite{HP_paper}, however, for 
our use case we need to further refine it as describe below:                                      
\begin{dfn}(Hybrid Model SNARK)
\label{dfn_snark}
A hybrid model \emph{succinct non-interactive argument of knowledge for relation $\mathcal{R}$} is a tuple of PPT algorithms 
$(\mathit{SNARK.Setup}, \mathit{SNARK.KeyGen}, \mathit{SNARK.Prove}, \\ \mathit{SNARK.Verify}, \mathit{SNARK.PartInputs})$ 
such that for implicit security parameter $\lambda$: 

\begin{itemize}
\item $\mathit{srs} \leftarrow \mathit{SNARK.Setup} (\mathit{aux_{\mathit{SNARK}}})$: a setup algorithm that on input auxiliary parameter 
$\mathit{aux_{\mathit{SNARK}}}$ from some domain $\mathcal{D}$ outputs a universal structured reference string tuple $\mathit{srs}$, 

\item $(\mathit{srs_{pk}}, \mathit{srs_{vk}}) \leftarrow \mathit{SNARK.KeyGen}(\mathit{srs}, \mathcal{R})$: a key generation algorithm that on input a
universal structured reference string $\mathit{srs}$ and an NP relation $\mathcal{R}$ outputs a \emph{proving key} and 
a \emph{verification key} pair $(\mathit{srs_{pk}}, \mathit{srs_{vk}})$,

\item $\pi \leftarrow \mathit{SNARK.Prove}(\mathit{srs_{pk}}, (x,w), \mathcal{R})$: a proof generation algorithm that on input a proving key 
$\mathit{srs_{pk}}$ and a pair $(x,w) \in \mathcal{R}$ outputs \emph{proof} $\pi$, 

\item $0/1 \leftarrow \mathit{SNARK.Verify}(\mathit{srs_{vk}}, x, \pi, \mathcal{R})$: a proof verification algorithm that on input a verification key 
$\mathit{srs_{vk}}$, an instance $x$ and a proof $\pi$ outputs a bit that signals acceptance (if output is $1$) or rejection (if output is $0$)
%\item $\pi_{\mathit{Sim}} \leftarrow \mathit{Sim}(R, \sigma, x)$: The simulator $\mathit{Sim}$ takes the relation $R$, the structured reference string 
%$\mathit{srs}$ and the instance $x$ as input and outputs a simulated proof $\pi_{\mathit{Sim}}$,

\item $(x_1, \mathit{state}_2) \leftarrow \mathit{SNARK.PartInputs}(\mathit{srs}, \mathit{state}_1, \mathcal{R})$: a deterministic 
public inputs generation algorithm that takes as input a universal structured reference string $\mathit{srs}$, an NP relation $\mathcal{R}$ and 
some state $\mathit{state}_1$ and outputs some updated state $\mathit{state}_2$ and some partial public input $x_1$,

\end{itemize}
and satisfies completeness, knowledge soundness with respect to $\mathit{SNARK.PartInputs}$ and succinctness as defined below:

\noindent \textbf{Perfect Completeness} holds if an honest prover will always convince an honest verifier: for all  
$(x,w) \in \mathcal{R}$ and for all $\mathit{aux_{\mathit{SNARK}}} \in \mathcal{D}$
\begin{align*}
\mathit{Pr}[\mathit{SNARK.Verify}(\mathit{srs_{vk}}, x, \pi, \mathcal{R}) = 1 \ | \ & 
\mathit{srs} \leftarrow \mathit{SNARK.Setup}(\mathit{aux_{\mathit{SNARK}}}), \\ 
& (\mathit{srs_{pk}}, \mathit{srs_{vk}})\leftarrow \mathit{SNARK.KeyGen}(\mathit{srs}, \mathcal{R}), \\
& \pi \leftarrow \mathit{SNARK.Prove}(\mathit{srs_{pk}}, (x,w), \mathcal{R}) \ ] = 1.
\end{align*}

\noindent \textbf{Notation} In the following, we denote by $\mathit{State_{\mathcal{R}}}$ the set of all states $\mathit{state}_1$ 
such that given some relation $\mathcal{R}$ and any possible $\mathit{srs}$, 
for any output $x_1$ of $\mathit{SNARK.PartInputs}(\mathit{srs}, \mathcal{R}, \mathit{state}_1)$ 
with $\mathit{state_1} \in \mathit{State_{\mathcal{R}}}$, we have that there exists $x_2$ and $w$ with $(x=(x_1, x_2), w) \in \mathcal{R}$; 
we further make the assumption that $\mathit{State_{\mathcal{R}}} \neq \emptyset$.\\

\noindent \textbf{Knowledge-soundness with respect to $\mathit{SNARK.PartInputs}$}
holds if there exists a PPT extractor $\mathcal{E}$ such that for all PPT 
adversaries $\mathcal{A}$, for all $\mathit{aux_{\mathit{SNARK}}} \in \mathcal{D}$ and for all $\mathit{state_1} \in \mathit{State_{\mathcal{R}}}$
\begin{align*}
\mathit{Pr}[&(x = (x_1, x_2), w) \in \mathcal{R} \wedge 1 \leftarrow \mathit{SNARK.Verify}(\mathit{srs_{vk}}, x = (x_1, x_2), \pi, \mathcal{R}) \ | \\
& \mathit{srs} \leftarrow \mathit{SNARK.Setup}(\mathit{aux_{\mathit{SNARK}}}), (\mathit{srs_{pk}}, \mathit{srs_{vk}})\leftarrow \mathit{SNARK.KeyGen}(\mathit{srs}, \mathcal{R}), \\ 
& (x_1, \mathit{state}_2) \leftarrow \mathit{SNARK.PartInput}(\mathit{srs}, \mathit{state}_1, \mathcal{R}), 
(x_2, \pi) \leftarrow \mathcal{A}(\mathit{srs}, \mathit{state}_2, \mathcal{R}),  
w \leftarrow \mathcal{E}^{\mathcal{A}}(srs,\mathit{state_2}, \mathcal{R})]
\end{align*}
is overwhelming in $\lambda$, where by $\mathcal{E}^{\mathcal{A}}$ we denote the extractor $\mathcal{E}$ that has access to all of 
$\mathcal{A}$'s messages during the protocol with the honest verifier. 

\noindent \textbf{Succinctness} holds if the size of the proof $\pi$ is $\mathsf{poly}(\lambda)$ and $\mathit{SNARK.Verify}$ runs in time 
$\mathsf{poly}(\lambda + |x|)$. % $+ \log{|w|}$ has been removed in both cases.
\end{dfn}

\noindent Firstly, note that if one chooses $x_1$, $\mathit{state_1}$ and $\mathit{state_2}$ to be the empty strings in the definition of 
$\mathit{SNARK.PartInput}$ and in relation to the knowledge soundness property, one obtains a more standard SNARK definition.
Secondly, $\mathcal{R}$ is not a component of the vector $\mathit{aux_{\mathit{SNARK}}}$ so even if $\mathit{SNARK.Setup}$ has 
$\mathit{aux_{\mathit{SNARK}}}$ as parameter, it is universal, 
i.e., it can be used to derive proving and verification keys for circuits of any size up to a polynomial in the security parameter $\lambda$,   
independently of any specific NP relation. Moreover, for the SNARKs we design, the size of the key used by 
the honest verifier is much smaller than the size of the honest prover's key. We have made the separation clear between the two keys to be able 
to better capture this special case; however, a potential adversarial prover has access to the complete $\mathit{srs}$ key. 
Thirdly, as mentioned the SNARKs that we design in this work are secure in the $\mathit{AGM}$ model. This means that we limit our adversaries to 
$\mathit{AGM}$ adversaries only and by $\mathcal{E}^{\mathcal{A}}$ we denote the extractor $\mathcal{E}$ that has access to all of 
$\mathcal{A}$'s messages during the protocol with the honest verifier: the messages include the coefficients of the linear combinations of 
group elements used by the $\mathit{AGM}$ adversary at any step in order to output new group elements at the next step in the protocol. 
Moreover, the auxiliary input (i.e., $\mathit{state_2}$) is required to be drawn from a ``benign distribution'' or else extraction may be 
impossible~\cite{extractability_limits_1,extractability_limits_2}. 
%Note also that in some SNARKs related work (e.g., PLONK~\cite{plonk}) the length of the public input is not explicitly taken into account in the definition 
%of succinctness, however we make that explicit in our definition and this is in line with both older and newer work in the 
%area (\cite{groth16, marlin}). %Moreover, similarly to PLONK (even though not explicitly specified there), the number of 
%field operations which are part of our SNARK verification depends as well on the length of the public input.
Finally, in the SNARK definition above we did not include the notion of zero-knowledge since it is not required in the rest of the paper.