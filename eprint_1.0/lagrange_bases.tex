\subsection{Lagrange Bases}
\label{sec:lagrange}

In order to design the SNARKs presented in this work, it is more convenient to represent the polynomials 
we work with over the Lagrange base rather than the monomial base. Formally, for the finite field $\mathbb{F}$ defined in section~\ref{sec:pairings} 
we denote by $H$ a subgroup of the multiplicative group of $\mathbb{F}$ such that $n = |H|$ is a large power of 2. Let $\omega$ be an $n$-th 
root of unity in $\mathbb{F}$ such that $\omega$ is a generator of $H$. Then, we call the following polynomial base $\{L_i(X)\} _{0 \leq i\leq n-1}$ 
a Lagrange base, where $\forall i, 0 \leq i \leq n-1$, $L_i(X)$ is the unique polynomial in $\mathbb{F}_{<n}[X]$ such that 
$L_i(\omega^i) =1$ and $L_i(\omega^j) = 0, \forall j \neq i$.\\

\noindent Independent of the notion of Lagrange bases, but related to $n$ we define $\block$ also a power of 2 such that $\block < n$. 
We use $\block$ when defining one of our conditional NP relations in section \ref{sec:snarks}. In the following we assume 
$n = \mathsf{poly} (\lambda)$ and $\block = \Theta(\lambda)$ and $|\mathbb{F}|= 2^{\Theta(\lambda)}$.%, i.e., $|\mathbb{F}|$ is exponential in $\lambda$.}

 
%{\color{blue} TO DO: Make the statement about $|\mathbb{F}|$ stronger. In the following we assume 
%$n= \mathsf{poly} (\lambda)$ and $|\mathbb{F}|= \lambda^{\omega(1)}$, i.e., $|\mathbb{F}|$ is super-polynomial in $\lambda$. Also, is $\block$ a constant or is $\mathsf{poly} (\lambda)$? }

