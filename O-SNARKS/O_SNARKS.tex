%\documentclass{llncs}
%\pagestyle{plain}

\documentclass[10pt]{article} 
\usepackage{hyperref}   
\usepackage[
    type={CC},
    modifier={by-nc-sa},
    version={3.0},
]{doclicense}

%\usepackage{hyperref}   
\usepackage{authblk}
\usepackage[toc,page]{appendix}
\usepackage[utf8]{inputenc}     
     
%%% page dimensions
\usepackage{geometry} 
\geometry{margin=1.08in} 
%\usepackage{numdef}
   
\usepackage{blindtext}
\usepackage{tabularx}
\usepackage{url}   
%\usepackage[hyphen]{url}
\usepackage{xspace}
\usepackage{graphicx} 
\usepackage{amsmath,amsthm}
\usepackage{amsfonts}
\usepackage{amssymb} %\therefore
\usepackage{cryptocode}
\usepackage{framed} 
% \usepackage[parfill]{parskip} % activate to begin paragraphs with an empty line rather than an indent
    
\usepackage{booktabs} 
\usepackage{array} 
\usepackage{paralist} 
\usepackage{verbatim} 
\usepackage{subfig}     
\usepackage{mathrsfs}
\usepackage[normalem]{ulem}
\usepackage{soul}

%%% headers & footers
\newcommand{\prove}{\ensuremath{\mathsf{Prove}}\xspace}
\newcommand{\group}{\ensuremath{\mathsf{group}}\xspace}
\newcommand{\ck}{\ensuremath{\mathsf{ck}}\xspace}
\newcommand{\key}[1]{\ensuremath{\mathsf{ck}_{#1}}\xspace}
\newcommand{\konefirst}{\ensuremath{\key{1,[:m']}}\xspace}
\newcommand{\konesecond}{\ensuremath{\key{1,[m':]}}\xspace}
\newcommand{\ktwofirst}{\ensuremath{\key{2,[:m']}}\xspace}
\newcommand{\ktwosecond}{\ensuremath{\key{2,[m':]}}\xspace}
\newcommand{\avec}{\ensuremath{\mathbf{\mathsf{a}}}\xspace}
\newcommand{\vvec}{\ensuremath{\mathbf{v}}\xspace}

\newcommand{\Avec}{\ensuremath{\mathbf{A}}\xspace}

\newcommand{\bvec}{\ensuremath{\mathbf{\mathsf{b}}}\xspace}
\newcommand{\zL}{\ensuremath{\mathit{z_L}}\xspace}
\newcommand{\zR}{\ensuremath{\mathit{z_R}}\xspace}

\newcommand{\afirst}{\ensuremath{\avec_{[:m']}}\xspace}
\newcommand{\asecond}{\ensuremath{\avec_{[m':]}}\xspace}
\newcommand{\Afirst}{\ensuremath{\Avec_{[:m']}}\xspace}
\newcommand{\Asecond}{\ensuremath{\Avec_{[m':]}}\xspace}

\newcommand{\vfirst}{\ensuremath{\vvec_{[:m']}}\xspace}
\newcommand{\vsecond}{\ensuremath{\vvec_{[m':]}}\xspace}
\newcommand{\bfirst}{\ensuremath{\bvec_{[:m']}}\xspace}
\newcommand{\bsecond}{\ensuremath{\bvec_{[m':]}}\xspace}

\newcommand{\CL}{\ensuremath{\mathit{C_L}}\xspace}
\newcommand{\CR}{\ensuremath{\mathit{C_R}}\xspace}
\newcommand{\TL}{\ensuremath{\mathit{T_L}}\xspace}
\newcommand{\TR}{\ensuremath{\mathit{T_R}}\xspace}

\newcommand{\negl}{\ensuremath{\mathit{negl}}\xspace}
\newcommand{\einn}{\ensuremath{\mathit{E}_{\mathit{inn}}}\xspace}
\newcommand{\eout}{\ensuremath{\mathit{E}_{\mathit{out}}}\xspace}
\newcommand{\ginn}[1]{\ensuremath{\mathbb{G}_{\mathit{#1,inn}}}\xspace}
\newcommand{\gout}[1]{\ensuremath{\mathbb{G}_{\mathit{#1,out}}}\xspace}
\newcommand{\gtinn}{\ensuremath{\mathbb{G}_{\mathit{T,inn}}}\xspace}
\newcommand{\gtout}{\ensuremath{\mathbb{G}_{\mathit{T,out}}}\xspace}
\newcommand{\sginn}[1]{\ensuremath{g_{\mathit{#1,inn}}}\xspace}
\newcommand{\sgout}[1]{\ensuremath{g_{\mathit{#1,out}}}\xspace}
\newcommand{\sgtinn}{\ensuremath{g_{\mathit{T,inn}}}\xspace}
\newcommand{\sgtout}{\ensuremath{g_{\mathit{T,out}}}\xspace}

\newcommand{\indexoneinn}{\ensuremath{\mathit{1,inn}}\xspace}
\newcommand{\indextwoinn}{\ensuremath{\mathit{2,inn}}\xspace}

\newcommand{\indexoneout}{\ensuremath{\mathit{1,out}}\xspace}
\newcommand{\indextwoout}{\ensuremath{\mathit{2,out}}\xspace}


\newcommand{\epinn}{\ensuremath{\mathit{e}_{\mathit{inn}}}\xspace}
\newcommand{\epout}{\ensuremath{\mathit{e}_{\mathit{out}}}\xspace}
\newcommand{\block}{\ensuremath{\mathsf{block}}\xspace}
\newcommand{\LCseed}{\ensuremath{\mathit{LC.seed}}\xspace}

%\usepackage{fancyhdr} % this should be set after setting up the page geometry
%\pagestyle{fancy} % options: empty , plain , fancy
%\renewcommand{\headrulewidth}{0pt} 
%\lhead{}\chead{}\rhead{}
%\lfoot{}\cfoot{\thepage}\rfoot{}
    
%%% section title appearance
%\usepackage{sectsty}
%\allsectionsfont{\sffamily\mdseries\upshape} 
  
%%% toc (table of contents) appearance
%\usepackage[nottoc,notlof,notlot]{tocbibind} % put the bibliography in the toc
%\usepackage[titles,subfigure]{tocloft} % alter the style of the table of contents
%\renewcommand{\cftsecfont}{\rmfamily\mdseries\upshape}
%\renewcommand{\cftsecpagefont}{\rmfamily\mdseries\upshape} % no bold!

%%% end article customizations
%\newcommand{\code}[1]{\texttt{#1}}
%\newcommand\tstrut{\rule{0pt}{2.6ex}}         % = `top' strut
%\newcommand\bstrut{\rule[-0.9ex]{0pt}{0pt}}   % = `bottom' strut
\newcommand{\bgamma}{\boldsymbol{\gamma}}
\newcommand{\bsigma}{\boldsymbol{\sigma}}

%\newcommand{\al}[2]{{\uline{#1}}\textcolor{cyan}{(#2)}}
%\newcommand{\sergey}[2]{{\uline{#1}}\textcolor{magenta}{(#2)}}

\newcommand{\GoneBLS}{\mathbb{G}_{1, \mathit{BLS}}\xspace}
\newcommand{\GtwoBLS}{\mathbb{G}_{2, \mathit{BLS}}\xspace}
\newcommand{\GTBLS}{\mathbb{G}_{T, \mathit{BLS}}\xspace}

\newcommand{\goneBLS}{\mathit{g}_{1, \mathit{BLS}}\xspace}
\newcommand{\gtwoBLS}{\mathit{g}_{2, \mathit{BLS}}\xspace}
\newcommand{\gTBLS}{\mathit{g}_{T, \mathit{BLS}}\xspace}

\newcommand{\eBLS}{\mathit{e}_{\mathit{BLS}}\xspace}
\newcommand{\HBLS}{\mathit{H}_{\mathit{BLS}}\xspace}
\newcommand{\HPoP}{\mathit{H}_{\mathit{PoP}}\xspace}
\newcommand{\Hinn}{\mathit{H}_{\mathit{inn}}\xspace}
\newcommand{\piinn}{\pi_{\mathit{inn}}\xspace}
\newcommand{\PoP}{\mathit{inn}\xspace}

\newcommand{\Rla}{\mathcal{R}^{\mathit{incl}}_{\mathsf{ba}}\xspace}
\newcommand{\Ra}{\mathcal{R}^{\mathit{incl}}_{\mathsf{pa}}\xspace}
\newcommand{\Rvt}{\mathcal{R}^{\mathit{incl}}_{\mathsf{c}}\xspace}

\newcommand{\Rlacom}{\mathcal{R}^{\mathit{incl}}_{\mathsf{ba},\mathit{com}}\xspace}
\newcommand{\Racom}{\mathcal{R}^{\mathit{incl}}_{\mathsf{pa},\mathit{com}}\xspace}
\newcommand{\Rvtcom}{\mathcal{R}^{\mathit{incl}}_{\mathsf{c},\mathit{com}}\xspace}


\newcommand{\Pla}{\mathscr{P}_{\mathsf{ba}}\xspace}
\newcommand{\Pa}{\mathscr{P}_{\mathsf{pa}}\xspace}
\newcommand{\Pvt}{\mathscr{P}_{\mathsf{c}}\xspace}

\newcommand{\Plastar}{\mathscr{P}_{\mathsf{ba}}^{\ast}\xspace}
\newcommand{\Pastar}{\mathscr{P}_{\mathsf{pa}}^{\ast}\xspace}
\newcommand{\Pvtstar}{\mathscr{P}_{\mathsf{c}}^{\ast}\xspace}

\newcommand{\Plah}{\mathscr{P}_{\mathsf{ba}}^{h}\xspace}
\newcommand{\Pah}{\mathscr{P}_{\mathsf{pa}}^{h}\xspace}
\newcommand{\Pvth}{\mathscr{P}_{\mathsf{c}}^{h}\xspace}
\newcommand{\kate}{\ensuremath{\mathsf{KZG}}\xspace}
\newcommand{\ewnp}{e.w.n.p.\ }

%%\spnewtheorem{construction}[theorem]{Instantiation}{\bfseries}{\itshape}
%%\spnewtheorem{assumption}[theorem]{Assumption}{\bfseries}{\itshape}
%%\spnewtheorem{test_claim}[theorem]{Claim}{\bfseries}{\itshape}
%%\newenvironment{sketch}{{\bfseries}{\itshape}\paragraph{Proof Sketch}}{\hfill$\square$}
%\spnewproof{sketch}{Proof Sketch}{\bfseries}{\itshape}
%\usepackage{amsthm}


\title{O-SNARKs for the Win: \\
 A Note on Security and A Concrete Example} 
\author[1]{Oana Ciobotaru
\thanks{A significant part of the research was conducted while the first author was affiliated to Web3 Foundation Technologies.}}
\author[2]{Alistair Stewart}

\affil[1]{OpenZeppelin}
\affil[2]{Web3 Foundation Technologies}
%\affil[2]{Independent Researcher}
%\date{ \today}

\newtheorem{lemma}{Lemma}
\newtheorem{theorem}[lemma]{Theorem}
\newtheorem{definition}[lemma]{Definition}
\newtheorem{claim}[lemma]{Claim}
\newtheorem{corollary}[lemma]{Corollary}
\newtheorem{remark}[lemma]{Remark}
\newtheorem{construction}[lemma]{Instantiation}
\newtheorem{assumption}[lemma]{Assumption}
\newtheorem{proposition}[lemma]{Proposition}
\newtheorem{test_claim}[lemma]{Claim}

\begin{document}
   \maketitle

\abstract In this work we revisit the notion of O-SNARKs in relation to a wide class of 
oracles which we call AGM respecting. {\color{red}These oracles support operations that are independent from the operations 
performed on the elliptic curve on which a SNARK prover runs.} Our main result shows that many modern SNARKs 
(e.g., PLONK~\cite{plonk}, Marlin~\cite{marlin}, Groth16~\cite{groth16}) are in fact O-SNARKs when combined 
with AGM respecting oracles. On one hand, from a theoretical perspective, AGM respecting oracles is a wide class 
which includes oracles often used when modelling practical applications {\color{red}(e.g., ...)}. On the other hand, via a concrete example, 
we shed more light on the subtleties regarding where and how O-SNARK security is useful for complex real-world applications.
    
\section{Introduction} \label{sec_intro}
Blockchain systems rely on consensus among a number of participants, where the size of this number is important for decentralisation and the foundation of blockchain security. To know that a transaction is valid, one needs to follow the consensus of the blockchain. However, following consensus can become expensive in 
terms of bandwidth, storage and computation. Depending on the consensus type, these challenges can be aggravated when the size of participants' set becomes bigger or when the participants' set changes frequently. Light clients (such as SPV clients in Bitcoin \cite{nakamoto2008bitcoin} or inter-blockchain bridge components that support interoperability) are designed to allow resource constrained users to follow consensus of a blockchain with 
minimal cost. We are interested in blockchains that use Byzantine agreement type consensus protocols, particularly proof of stake systems 
like Polkadot~\cite{polkadot}, Ethereum~\cite{ethereum}
%~\cite{eth2} 
or many other systems~\cite{cosmos,tendermint_paper, celo}. These protocols 
may have a large number of consensus participants, from 1000s to 100000s, and in such PoS protocols, the set of participants often changes regularly. \\

\vspace{-0.2cm}
\noindent Following the consensus protocols in the examples above entails proving that a large subset of a designated set of participants, 
which are called validators, signed the same message (e.g., a block header). Existing approaches have limiting shortcomings as follows:
1) verifying all signatures which has a large communication overhead for large validator sets;
2) verifying a single aggregatable signature, by computing an aggregate public key from the signer's public keys, has the shortcoming that any verifier still needs to know the entire list of public keys and this, again, has expensive communication if the list changes frequently; 
3) verifying a threshold signature which has two shortcomings: first, such a signature does not reveal the set of signers impacting the 
security of PoS systems; second, it requires an interactive setup which becomes expensive if the validator set is large or changes frequently.

\vspace{-0.2cm}
\paragraph{Our Approach: Committee Key Schemes} We introduce a committee key scheme which allows to succinctly prove that a subset of signers 
signed a message using a commitment to a list of all the signers' public keys. Our primitive is an extension of an aggregatable signature scheme and 
it allows us to prove the desired statement, in turn, by proving the correctness of an aggregate public key for the subset of signers. 
In more detail, the committee key scheme defines a committee key which is a commitment to all the signers' public keys. It generates a succinct proof that a particular subset of the list of public keys signed a message. The proof can be verified using the committee key. 
Because of the way the aggregatable signature scheme works, we need to specify the subset of signers; for this purpose we use a bitvector. 
More precisely, if the owner of the $m$th public key in the list of public keys signed the message, then the $m$th bit of this bitvector is $1$ otherwise is $0$.
Using the committee key, the proof and the bitvector, a light client can verify that the corresponding subset of validators to whom the 
public keys belong (as per our use case) signed the message. Although the bitvector has length proportional to the number of validators, it is still orders of magnitude 
more succinct than giving all the public keys or signatures. Public keys or signatures are usually 100s of bits long and as a result this scheme reduces the amount of 
data required by a factor of 100 times or more. We could instantiate our committee key scheme using any universal SNARK scheme and suitable commitment scheme. 
However, to avoid long prover times for large validator sets, we use optimised custom SNARKs. We have implemented this scheme 
(Section~\ref{sec_implementation}) and it gives fast enough proving times for the use cases we consider: a prover with 
commodity hardware can generate these custom SNARK proofs in real time, i.e., as fast as the consensus generates instances of this problem.

\vspace{-0.2cm}
\paragraph{Application: Accountable Light Client Systems} %To understand when and how our scheme described above is useful and compare it to other approaches, we return to what a light client is used for. 
Light clients allow resource constrained devices such as browsers or phones to follow a decentralised consensus protocols. A blockchain is also resource constrained and hence could benefit from a light client system. In this case a light client verifier (e.g, smart contracts on 
Ethereum) allows building trustless bridges protocols between blockchains. % \cite{BridgeSOK}. 
Currently, computation and storage 
costs on existing blockchains are much higher than those in a browser on a modern phone. If such a bridge is responsible for securing assets with high total value, then the corresponding light client system which defines such a light client verifier must be secure as well as efficient. Using the primitives and techniques described in this work, one can design a light client system with the following properties: accountability, asynchronous safety and incrementability reviewed below.  \vspace{-0.2cm}
%\begin{itemize}
%\item 
\noindent\paragraph{Accountability}Our light client system is accountable, i.e., if the light client verifier is misled and the transcript of its communication is given to the network then one can identify a large 
number (e.g., 1/3) of misbehaving consensus participants (e.g., validators in our case). Identifying misbehaving consensus participants 
is challenging in the light client system context when we want to send minimal data to the light client verifier. However, 
identifying misbehaviour is necessary for any proof of stake protocols including Polkadot and Ethereum whose security relies on identifying and punishing misbehaving consensus participants. 
%\item 
\vspace{-0.2cm}
\noindent\paragraph{Asynchronous Safety}Our light client system has asynchronous safety i.e., under the consensus' honesty assumptions, our light client verifier cannot be misled even if it has a restricted view of the 
network, e.g., only connecting to one node, which may be malicious. This is because our light client system inherits the property of asynchronous safety from the Byzantine agreement 
protocol of the blockchain. Such light client systems would not be possible for consensus based on longest chains.  
%\item 
\vspace{-0.2cm}
\noindent\paragraph{Succinctness}Our light client system is incremental - i.e its succinct state is incrementally updated - it is optimised to make these updates efficiently, which is particularly relevant for the bridge application, as opposed to trying to optimise verifying consensus decisions from the blockchain genesis.
%\end{itemize}
\vspace{-0.2cm}
\subsection{Impact on decentralisation} For a blockchain network, having  a large number of validators  contributes greatly to better decentralization. This leads to better security both in terms of less point of physical failure and being able to distribute control over consensus which makes collusion harder. Some protocols have restricted their validator numbers to make light clients or bridges more efficient, e.g., by being able to run a DKG for threshold signatures (e.g., Dfinity~\cite{dfinity}) or obtaining Byzantine agreement with all validators on every block (e.g., Cosmos~\cite{tendermint_paper}). More efficient light clients for blockchains with large validator sets offer both decentralisation and interoperability (bridging) without compromise.

\vspace{-0.25cm}
\subsection{Relevance to Bridge Security} 
%\vspace{-0.2cm}
\noindent In this section we review the impact of our scheme on bridge security. Blockchain bridges are protocols that allow value transfer between blockchains. Bridges have frequently been the target of attacks. We note that \$1.2 billion has been stolen in attacks on insecure bridges during first 8 months of 2022 alone ~\cite{elliptic_harmony,elliptic_nomad}. Of the top 10 crypto thefts of all times, \$1.6bn out of \$3.4bn come from bridge attacks \cite{elliptic_nomad}. These confirm that bridges have frequently been a weaker point, compared to the security of the blockchains themselves and they carry a lot of economical value.

\noindent An ideal bridge would be as secure as the least secure of the two blockchains. The most secure bridges use in-chain light client systems, e.g., Cosmos IBC protocol~\cite{IBC_paper}, to achieve this. Each bridged chain follow the other chain's consensus on-chain. To simplify, we will consider an on-chain light client of chain B on chain A, although B will also have the same for A. If B's consensus and the on-chain logic of A are secure, then adversary cannot convince the logic of A that B decided some event that B's clients do not agree as decided. This translates to the adversary for example not being able to create value on A without having locked any value on B.

\noindent A main reason why bridges might not use this approach is efficiency. Smart contracts and other on-chain logic is an extremely resource constrained environment compared to browsers or phones that light clients might target. One approach for efficiency is to design B's consensus so that the light clients are cheaper, for example by reducing the validator numbers. Cosmos chains currently have 33-175 validators~\cite{CosmosValNYX},\cite{CosmosValHUB}. Many chains have many more, e.g., Ethereum's hundreds of thousands of validators, for more decentralisation and security. Alternatively, the light client can use threshold signature may be used however that means not having the same accountability guarantees and also limits validator numbers in practice, both discussed elsewhere. 

% Notaries are obviously bad but lots of money has been stolen from such bridges [TODO: Expand this].

\noindent Another approach to reducing on-chain complexity is optimism. Entities make a claim on chain A that something happened on chain B and this is accepted if no entity makes an on-chain challenge within a certain time, claiming that this is incorrect and triggering a more expensive procedure. A bridge that uses this approach in Optics for bridging Celo to other chains~\cite{CeloOptics}. A less extreme example of this approach is NEAR's Rainbow bridge~\cite{NEARrainbowB}, where signatures are stored but not checked unless the correctness of a signature is challenged. The optimism approach relies on the censorship resistance of blockchain for security. In practice, blockchains may be censored for a period of time by an attacker with enough resources. An example of this was the result of the first round of Fomo3D on Ethereum~\cite{Fomo3DPM}, a smart contract that would pay a jackpot, a large amount (in the end 10,469 ETH), to the last user to pay the contract when no user does so for 30 seconds. The jackpot grew to such a large amount that it was worth a user buying up all the block space for 30 seconds~\cite{Fomo3DPM}. For a claim and challenge protocol, the challenge is itself quite computationally expensive, so it may be sufficient to increase the cost of computation, the gas price on Ethereum, to make such a challenge unprofitable. Security against this attack requires a large reward for challenges or a long challenge period. For example the rainbow bridge has an 8 hour challenge period~\cite{RainbowBridgeFAQ}. Long challenge periods would mean that bridge operations take a long time with consequences for usability.

\noindent Stakers in proof of stake protocol have an incentive for the chain (chain A) using that protocol to keep working, however they may not have stake in a chain (chain B) bridged to their chain. As a result, they may have no particular incentive in the correct functioning of a light client of chain A on chain B and so not to mislead the light client. In the case when the protocol of chain A has slashing, if an accountable light client on chain B is misled, one can prove to chain A, using information that is publicly available on chain B, of validator of chain A misbehaving in a way that will result in those validators being slashed on chain A. This gives the bridge similar economic security to chain A itself.

Protocols Cardano and Algorand
\vspace{-0.08in}
\subsection{Applicability of Our Scheme}

\noindent Our scheme is applicable to proof of stake blockchains where if something is decided by the chain, then a message is signed by some threshold fraction of a validator set, defined as a set of nodes or their public keys, which changes at well-defined times, those changes being signed by an appropriate threshold of the existing set. As mentioned such chains as Polkadot, the many Cosmos chains, or Ethereum fit this model. Our scheme is not applicable to chains using proof of work or many other proof of X schemes. Nor proof of stake protocols when only random validators or random subsets of validators decide something and the whole set never votes, such as protocols using the longest chain rule without a finality gadget.

\noindent Our scheme might well require a hard fork to be applied to many blockchains, especially those that have not implemented the required cryptography. It should be easily implementable for chains that use BLS signatures for consensus but those using signatures that do not support aggregation (e.g., the many using Ed25519), would need to use SNARKs with much slower prover time (e.g., zkBridge~\cite{zkBridge} for Ed25519)). To naively implement our scheme, we would also want validators to compute and sign the commitment to the next set. We note however that this is not strictly necessary, as the commitment could be computed on chain, maybe in a smart contract, as long as light client proof of the result of this computation can be constructed. This would result in longer proofs that cover validator set changes. For blockchains with expensive on-chain computation, native code support for the cryptography we use e.g., with precompiles for smart contracts might be required. It is planned to make the required changes to Polkadot and implement this scheme for it. We discuss in detail what would be required for a light client of Ethereum in Appendix Section~\ref{sec:ethereum}.
\vspace{-0.1in}
%The following paragraph should be commented out and changed in case of a conference submission.
\paragraph{Structure} The paper is organised as follows. 
In Section~\ref{sec:sketch}, we sketch our proposed protocols and compare 
them to existing work. In Section~\ref{sec_prelims}, we give cryptographic 
preliminaries necessary for later sections. In Section~\ref{sec_apk_proofs}
we describe our custom SNARKs and our committee key scheme. In 
Section~\ref{sec_implementation} we give benchmarks for our custom SNARKs 
implementations. We conclude in Section~\ref{conclusions}. Our paper includes 
an extensive appendix for more details.% as follows. 
%\vspace{-0.01in}
\vspace{-0.25cm}
\section{Our Solution} 
\label{sec:sketch}
\vspace{-0.2cm}

In this section we present a sketch of our solution for both the committee key scheme and the accountable light client system, 
then describe the technical challenges and contributions and finish with an overview of related work.

\vspace{-0.3cm}
\subsection{Sketch of Committee Key Scheme}
\label{sec:lcsketch}
\vspace{-0.1cm}

\noindent Suppose that a prover wants to prove to a verifier that a subset $S$ of some set $T$ of signers {\color{red} with equal stakes} have signed a message. 
One obvious approach would be using BLS aggregatable signatures with the following steps:

\begin{itemize}
\item[a.] Verifier knows all public keys $\{\mathit{pk}_i\}_{i \in T}$ of signers.% in $T$.

\item[b.] Prover sends the verifier an aggregatable signature $\sigma$ and a representation of the subset $S$.

\item[c.] Verifier computes the aggregate public key $\mathit{apk}=\sum_{i \in S} \mathit{pk}_i$ of the public keys of signers in $S$. 
Then it verifies the aggregatable signature $\sigma$ for the aggregate public key $\mathit{apk}$ and it accepts if the verification succeeds.
\end{itemize}

\noindent However, we can represent a subset $S$ of a list of signers compactly using a bitvector $b$: 
the $i$th signer in the list is in $S$ if and only if the $i$th bit of $b$ is $1$. Our committee key scheme describes an alternative approach:

\begin{itemize}
\item[a'.]\label{a'} Verifier knows a commitment $C$ to the list of public keys $(pk_i)_{i \in T}$.

\item[b'.]\label{b'} Prover sends the verifier an aggregatable signature $\sigma$, a bitvector $b$ representing $S$, an aggregate public key 
$\mathit{apk}$ and $\pi$, a succinct proof that $\mathit{apk}=\sum_i b_i \mathit{pk}_i$ i.e., 
that $\mathit{apk}$ is the aggregate public key for the subset of signers in $S$ given by the bitvector $b$; all of the public keys in $S$ are a subset 
of the list of public keys committed to using $C$.

\item[c'.] The verifier using $C$, $\mathit{apk}$ and the bitvector $b$ checks if $\pi$ is valid. 
It then verifies $\sigma$ against $\mathit{apk}$ and accepts if both steps succeed.
\end{itemize}

\noindent With the above committee key scheme, if $C$ and $\pi$ are constant size, 
the communication cost becomes $O(1)+|T|$ bits instead of $|T|$ public keys. {\color{red} So far we have implicitly assumed validators have equal stakes. 
One can generalise our alternative approach introduced above to validators with unequal stakes by including at~\ref{a'}, a'.,  a commitment to all stakes 
and to~\ref{b'}, b'., a claimed total signing stake that can be proved via a scalar product between stakes of the signing validators and 
the bitvector. Moreover, $\mathit{apk}$ is appropriately replaced by the scalar product between signing validators' stakes and their 
respective public keys. The bitvector cannot be removed as it is needed for ensuring accountability of our light client system.}

\subsection{From CKS to Accountable Light Client}
\label{sec_intro_committee}

Below we sketch how a light client verifier uses our committee key scheme. Suppose that a light client verifier wants to know some information $\mathit{info}_n$ about the state of a blockchain at block number $n$ without having to download the entire blockchain. Another entity, a full node, who knows all the data of the blockchain and is following the consensus, should be able to convince the light client verifier using a computational proof that $info_n$ was indeed decided.

We assume that $\mathit{info}_n$ can be proven from a commitment to the state at block number $n$ that is signed by validators, here we assume that this commitment is a block hash $H_n$. To convince the light client verifier that $H_n$  was decided, the full node needs to convince the light client verifier that a threshold number $t$ of validators from the current validator set signed $H_n$, where $t$ depends on the type of consensus. Byzantine fault tolerant based consensus often uses $t$ to be over 2/3 of the total number of validators.

\noindent\paragraph{Keeping Track of the Validator Set:} A light client verifier must be initialised with a committee key $cpk_1$ corresponding to the genesis validator set with keys $\bf{pk}_1$. At the end of each epoch, i.e., the time a validator set needs to be updated, the validators set of epoch $i$, with keys $\bf{pk}_i$ sign a message $(i,cpk_{i+1})$ where $cpk_{i+1}$ is {\color{red} a commitment to the validator set} for the next epoch $\bf{pk}_{i+1}$. The light client verifier keeps track of $cpk_i$ for each epoch. A light client proof must include a committee key scheme proof that a bitvector of validators, with a threshold number of 1s, with keys committed to in $cpk_i$ signed $(i,cpk_{i+1})$. To convince a light client verifier knowing only $cpk_1$ of something in block $n$, all such proofs up to the epoch containing block $n$ must be included. For an incremental light client system, such as one on a bridge, these validator set update proofs only need to be given once an epoch.
\vspace{-0.05in}

\noindent\paragraph{Proving the General Claim $\mathit{info}_n$:} Once the light client verifier is convinced of $cpk_{n-1}$ for the epoch $n-1$ and $t$ of the validators in epoch $n-1$ signed $H_n$, it needs a committee key scheme proof for $cpk_n$ and a bitfield with $t$ ones that $t$ validators signed $H_n$. Finally, such a proof needs the opening of the commitment $H_n$ to $\mathit{info}_n$.
\vspace{-0.08in}

\noindent\paragraph{Accountability:} Now suppose that a full node obtains a light client proof for something that contradicts something it sees as decided 
by the blockchain. For our bridge use case, all light client proofs will be publicly available on another blockchain. We assume that we can express this 
contradiction in terms of a pair of messages that should never be signed by an honest validator, and that any validators doing so can be punished. 
These we call incompatible messages. In this example, such pairs of messages should include validator sets commitment to different 
commitments $(i,cpk_{i+1})$ and $(i,cpk'_{i+1})$, $cpk_{i+1} \neq cpk'_{i+1}$ and similarly distinct $H_n$ and $H'_n$. If a light client proof 
contains a message signed by a committee key which is a commitment to the public keys of a known set of validators which is incompatible with a 
message the same set signed on the blockchain, then the signature in the proof is a valid BLS signature with the claimed set of signers and so the full 
node should be able to report that public keys that signed both messages misbehaved. If an incompatible message was signed by a committee key 
which doesn't correspond to the claimed epoch's validator set, then at some point previously the light client proof must have shown that the committee 
key for some correct validator set signed the wrong committee key for the next set which is a message that is incompatible with the correct committee 
key that they signed on the real chain. Note that the accountability of our light client system instantiation relies on the accountability of the 
underlying consensus protocol. Indeed, our light client is accountable only if signatures on incompatible messages are enough for consensus 
accountability e.g., in Casper FFG~\cite{CasperFFG} and it is not directly applicable to consensus protocols where forensics (such as in~\cite{forensics}) 
are required for accountability, e.g., Polkadot's GRANDPA, Section 4.1~\cite{GRANDPA}. If the consensus protocol is not accountable with signatures, 
then the consensus protocol needs to be modified by adding another layer (e.g. ABC\cite{ABC}, Polkadot's BEEFY~\cite{BEEFY}).
\vspace{-0.05in}.

\noindent\paragraph{Efficiency Gain:} If one follows the obvious approach described above using BLS aggregation and aims to convince the 
light client verifier that $\mathit{info}_n$ is decided, then one needs to send $O(v)$ public keys for each validator set change, where $v$ is 
the upper bound on the size of the validator set. Using our succinct committee key scheme however, one requires only a constant size 
proof and $v$ bits for each validator set change to convince the light client that $\mathit{info}_n$ was decided. Since a public key or 
signature typically takes 100s of bits, our approach achieves much smaller proof sizes.  More details our achieved efficiency are 
available in Section~\ref{sec_implementation}.
\vspace{-0.05in}

\noindent\paragraph{Formalisation:} We give a formal model for the security properties of our accountable light client in Appendix~\ref{sec:LCinstantiation}.

\vspace{-0.1cm}
\subsection{Our Custom SNARKs}
\label{sec_intro_custom_snarks}
%\vspace{-0.1cm}

%\noindent In the following we discuss how we use custom SNARKs with efficient prover time to implement the committee key scheme. 
%While we achieved very fast proving times in our implementation, it came at the cost of not using a general purpose SNARK protocol, 
%leading to a more involved security model and the necessity of additional security proofs. \\
\noindent Here we discuss how we use custom SNARKs with efficient prover time to implement our committee key scheme. 
While we achieved very fast proving time in our SNARKs implementation, this came at the cost of not using a general purpose 
SNARK protocol, in turn leading to a more involved security model and the necessity of additional security proofs.  \\

\vspace{-0.05cm}
%\noindent Our SNARKs have inputs an aggregate public key $\mathit{apk}$, a commitment $C$ to a list of public keys $(pk_i)$, and a bitvector $(b_i)$. 
%It needs to be a proof that  $apk=\sum_i b_i pk_i$. The SNARK verifies witnesses to the partial sums $kacc_j = h + \sum_{i=0}^j b_i pk_i$ 
%where $h$ is chosen to avoid the incompleteness of the addition formulae we use.} We list two of the optimisations we used for our custom SNARK below. 
\noindent The public inputs for our SNARKs are: an aggregate public key $\mathit{apk}$, a commitment $C$ to the list of 
public keys $(pk_i)_{i \in T}$ and a bitvector $(b_i)_{i \in T}$ succinctly representing a subset $S$ of public keys. 
Our SNARKs provers output a proof that $apk=\sum_{i \in T} b_i pk_i$ and that $C$ is the commitment to the list of public keys 
$(pk_i)_{i \in T}$. However the list itself is a witness for the relations defining our SNARKs and so the verifiers do not need it 
and do not have to parse or check anything based on this possibly long list. 
We detail below two further optimisations of our custom SNARKs.

%\begin{itemize}
%\item Our scheme is an instance of commit and prove SNARKs (see section~\ref{sec:commit_prove} for more details) that works as follows. 
%The verifier takes only a commitment to part of the input of the SNARK, for us $C$ is a commitment to the list of public keys, and the list public 
%keys themselves are not used by the verifier. We use the same polynomial commitment scheme for this purpose as is used in th SNARK itself. 
%Hence, we do not need to add the decommitment of $C$ to the SNARK constraint system. Since our constraint system is simple adding a decommitment 
%to use, e.g. a hash for the commitment, would increase the size of the constraint system and lead to several orders of magnitude increase in prover time. 
%The tradeoff is that we cannot use an existing SNARK system or polynomial constraint compiler as a black box, making the proofs in this paper more complicated.}
%\item 
\vspace{-0.05in}
\paragraph{Commit and Prove SNARKs:} Our SNARKs are an instance of commit and prove SNARKs (see Section~\ref{sec:commit_prove}). 
The underlying commitment scheme used for computing the public input commitment $C$ is the same as the (polynomial) commitment scheme used in the rest of our SNARK(s). Hence, we do not need to add a witness for $C$ 
to the SNARK constraint system in the same way we would have to if our commitment scheme were, e.g., to use a hash function.
The constraints for checking a hash inside our custom SNARKs would increase the size of the constraint system so much that it would lead to several orders of magnitude increase in our prover time. 
The trade-off for our SNARKs design (i.e., with a commitment as part of the public input) is that we cannot use an existing SNARK compiler as a black box.

%\item Our constraint system is of a form where the wiring together of different constraints is trivial enough that we can avoid doing a permutation argument 
%or sparse matrix vector product that general proving systems would use to wire together gates. This reduces the proof size and proving time.}
%\item 
\vspace{-0.05in}
\paragraph{Constraint System Simplicity: }Our constraint system is simple enough such that our custom SNARKs do not require a permutation argument or a matrix-vector product argument 
which general proving systems need to bind together gates. In fact, the underlying circuit for our SNARKs can be described as an affine addition gate with a couple of constraints added to avoid the incompleteness of our addition formulae. This simplification leads to smaller proof sizes  and faster proving times.


%\vspace{-0.2cm}

\vspace{-0.2cm}
\subsection{Related Work}

\subsubsection{Naive Approaches and Their Use in Blockchains}
There are a number of approaches commonly used in practice to verifying that a subset of a large set signed a message. 
%However, among these, the approaches that have slow verification limit the size of the validators' set, which in turn limits decentralisation. 
\vspace{-0.2cm}
\noindent \paragraph{Verify All Signatures}  One could verify a signature for each signing validator. This is what participants  do in protocols like Polkadot~\cite{polkadot}, with 297 validators
% \cite{PolkaExplorer} 
(or Kusama with 1000 validators) %\cite{PolkaExplorer}
and Tendermint~\cite{tendermint_paper}, which is frequently used with 100 validators). %\cite{CosmosExplorer}). 
The Tendermint light client system, which is accountable and uses the verification of all individual signatures approach, 
is used in bridges in the IBC protocol\cite{IBC_paper}. This approach becomes prohibitively expensive for a light client verifier when there are 1000s or millions of signatures. 
\vspace{-0.1in}
\noindent \paragraph{Aggregatable Signatures} One could use an aggregatable signature scheme like BLS~\cite{BLS_signatures,boneh_compact_multisig}  and reduce this to verifying one signature, but that requires calculating an aggregate public key. This aggregate key is different for every subset of signers and needs to be calculated from the public keys. This is what Ethereum 
%~\cite{eth2} 
does, which currently has 415,278 validators. %\cite{EthExplorer}. 
However for a light client verifier, it is expensive to keep a list of 100,000s of public keys updated. As such only full nodes of Ethereum use this approach and instead light clients verifiers of Ethereum~\cite{sync_committee} follow signatures of randomly selected subsets of validators of size 512. This means that the resulting light client system is not accountable because these 512 validators are only backed by a small fraction of the total stake.
\vspace{-0.2cm}
\noindent \paragraph{Threshold Signatures} Alternatively a threshold signature scheme may be used, with one public key for the entire set of validators. This approach was adopted by Dfinity~\cite{GrothDKG}. Threshold signature schemes used in practice use secret sharing for the secret key corresponding to the single public key. This gives the schemes two downsides. Firstly, they require a communication-heavy distributed key generation protocol for the setup which is difficult to scale to large numbers of validators. Indeed, despite recent progress~\cite{AggregatableDKG,GrothDKG,LWEDKG}, it is still challenging to implement setup schemes for threshold signatures across a peer-to-peer network with a large number of participants, which is what many blockchain related use cases require. Moreover, such a setup may need repeating whenever the signer set changes. Secondly, for secret sharing based threshold signature schemes, the signature does not depend on the set of signers and so we cannot tell which subset of the validators signed a signature i.e., they are not accountable. Dfinity~\cite{GrothDKG} uses a re-shareable BLS threshold signature, where the threshold public key remains the same even when the validator set changes. Such a signature scheme 
provides the light client verifier with a constant size proof, even over many validator set changes, but means that the proof not only does not identify which of a particular set of validators are misbehaving, but also we cannot say when this misbehaviour happened i.e., which validator set misbehaved. This is because the signature would be the same for any threshold subset of any validator set.

%\vspace{-0.2cm}
\noindent It is worth noting that if a protocol has already implemented aggregatable BLS signatures, our committee key scheme can be used without altering the consensus layer. Indeed it may be easier to alter a protocol that uses individual 
signatures to use aggregatable BLS signatures than to implement threshold signatures from scratch because the latter requires waiting for an interactive setup before making validator set changes.

%\vspace{-0.2cm}
\subsubsection{Using SNARKs to Roll up Consensus}
%\vspace{-0.1cm}

\noindent{Celo~\cite{celo} and Mina~\cite{mina} blockchains have associated light client verifiers which allow their resource constrained users 
to efficiently and securely sync from the beginning of the blockchain to the latest block.}

%\vspace{-0.2cm}
\noindent \paragraph{Plumo~\cite{plumo}} is the most relevant comparison to our scheme. It also tackles the problem we consider, i.e., that of 
proving validator set changes. In more detail, Plumo uses a Groth16 SNARK~\cite{groth16} to prove that enough validators signed 
a statement using BLS signatures from a set of the public keys. In Celo~\cite{celo}, the blockchain that designed and plans to use 
Plumo, validators may change every epoch which is about a day long and the Plumo's SNARK iteratively proves 120 epochs worth of 
validator set changes. Since in Celo there are no more than 100 validators in a validator set at any one time, the respective public 
keys are used in plain as public input for Plumo's SNARK, as opposed to a succinct polynomial commitment in the case of our custom SNARKs. 
All of the above increase the size of Plumo's prover circuit. Since Plumo is designed to help resource constrained light clients sync from scratch, 
it is not an impediment that the Plumo SNARK cannot be efficiently generated, i.e., in real time. In the case of a light client verifier for bridges 
(i.e., the most resource constrained application), we expect it to be in sync at all times and, by design, we care only about one validator 
set change at a time. Our slimmed down and custom SNARK not only can be generated in real time, but, also due to the use of specialised 
commitments schemes for public keys, our validator sets can scale up to much larger sizes as well without impacting the efficiency of our system. 

\begin{comment}
\paragraph{Mina} achieves light clients with $O(1)$ sized light client proofs using recursive SNARKs. 
This requires some nodes have a large computational overhead to produce these proofs. 
%Also because this requires verifying consensus with small circuits, they do not use the consensus paradigm discussed above where a majority of validators sign, and instead use a longest chain rule version of proof of stake~\cite{mina}. 
Their protocol is not accountable because, as with Dfinity above, it is not possible to tell from the proof which validators signed off on a fork, nor when this happened. 
%Another downside is that because the proof only shows the length of a chain (and its block density), similar to a Bitcoin SPV proof, a light client needs to be connected to an 
%honest node to tell if a block is in the longest chain. If the client is connected to a single malicious node, it could be given a proof for a shorter fork and not see any proofs of chains the fork choice rule would preder.
\end{comment}

%\vspace{-0.2cm}
\paragraph{Mina~\cite{mina}} achieves light clients with $O(1)$ sized light client proofs using recursive SNARKs. This requires some nodes have a large computational overhead to produce proofs. Also because this requires verifying consensus with small circuits, they do not use the consensus paradigm 
discussed above where a majority of validators sign, and instead use a longest chain rule version of proof of stake~\cite{mina}. 
Their protocol is not accountable because, as with Dfinity above, it is not possible to tell from the proof which validators signed off on a fork, nor when this happened. 
Another downside is that because the proof only shows the length of a chain (and its block density), similar to a Bitcoin SPV proof, a light client needs to be 
connected to an honest node to tell if a block is in the longest chain. If the client is connected to a single malicious node, it could be given a proof for a 
shorter fork and not see any proofs of chains the fork choice rule would prefer.
\vspace{-0.03cm}
\subsubsection{Commit-and-Prove and Related Approaches}
\label{sec:commit_prove}

\noindent Our custom SNARKs are an instance of the commit-and-prove paradigm~\cite{KilianPhD,CLOS02,CP_proposal,HP_paper,CP_paper} 
which is a generalisation for zero-knowledge proofs/arguments in which the prover proves statements about values that are committed.\\

\begin{comment}
In practice, commit-and-prove systems (for short, CP) can be used to compress a large data structure and then prove something about its 
content (e.g., polynomial commitments~\cite{KZG_10}, vector commitments~\cite{vector_commitment_1}, accumulators~\cite{first_accumulator}). 
CP schemes can also be used to decouple the publishing of commitments to some data from the proof generation: each of these actions may be 
performed by different parties or entities~\cite{zkp_reference}. Finally, commitments can be used to make different proof systems 
interoperable~\cite{CP_paper,interoperability_2}. Our SNARKs have properties from the first two categories, however we could not 
have simply re-used an existing argument system: by designing custom circuits and SNARKs, we ensured improved efficiency for our use cases. 
\end{comment}
\vspace{-0.1in}

{\color{red}In this context, ECLIPSE~\cite{eclipse} presents a compiler that starts off with popular SNARKs (e.g., Sonic~\cite{sonic}, PLONK~\cite{plonk}, 
Marlin~\cite{marlin}) and via a new general compilation method generates CP-variants for these SNARKs. 
Our proposed compiler uses as a first step the standard PLONK compiler.  As a second step we simply 
re-cast the SNARK resulted in the first step as a SNARK for a new relation. 
The security of the re-casting holds under mild conditions that deterministically relate some polynomials 
processed by the verifier in the ranged polynomial protocol (before applying PLONK compiler) to some 
public inputs. To our knowledge, our re-casting conditions are less stringent than the conditions needed 
in~\cite{eclipse}.}

{\color{red}We cannot use the ECLIPSE compilation technique either in full or in part to compile our custom SNARKs 
since the types of NP relations derived after ECLIPSE compilation are simply incompatible with ours. While in the 
case of ECLIPSE, the witnesses for the NP relations before compilation remain witnesses also for the relations 
after compilation, in our case, some part of the public input before compilation becomes witness after the 
re-casting of the SNARK for a new NP relation. Thus, overall, ECLIPSE and the current work solve different problems. 
Finally, our compilation method requires only the PLONK compiler without additional computational 
steps so it is more efficient than the one in~\cite{eclipse}.} 

%The paragraph below was a comment/was commented out when shortening the paper.
\begin{comment}
\noindent Another paradigm related to commit-and-prove is called hash-and-prove~\cite{HP_paper}: for large data structures or simply data that is expensive to be 
handled directly by a computationally constrained verifier, one can hash that data and then create a (succinct) proof for some verifiable computation that uses 
the original, large, dataset. The committee key scheme notion that we define in this work has both similarities to but also differences with regard to this 
paradigm. The similarities are that, both the way we instantiate our committee key (i.e., using a polynomial commitment 
with a trusted universal setup) and the way we instantiate our aggregate public key, can be generalised as some form of (possibly deterministic) 
hash function. One difference is that the setup for the polynomial commitment is the same as that from which the proving and verification key for our committee key scheme are 
computed; thus our version of the hashes and the keys for the committee key scheme are definitely not independent as in the case of hash-and-commit~\cite{HP_paper}. Finally, 
built into our definition of committee key scheme and its security properties, we use a secure aggregatable signature scheme which allows us to design and 
prove the security properties of our accountable light client(s). In fact, to add some intuition to the fact that a committee key scheme is more than 
just a hash-and-prove instance, we mention that our committee key scheme inherits an unforgeability property from its aggregatable scheme sub-component. 
This is one property that as far as we are aware no hash-and-prove scheme has. \\
\vspace{-0.08in}
\end{comment}

\begin{comment}
\noindent When proving the security of our arguments, we use an extension of some of the more commonly employed SNARK definitions which we call a ``a hybrid model SNARK''. This resembles the existing notion of SNARKs with online-offline verifiers as described in~\cite{HP_paper}, where the verifier computation is split into 
two parts: during the offline phase some computation (possibly of commitments) happens; this computation takes some public inputs as parameters and, when not 
performed by the verifier, it may also be performed (in part) by the prover. The online phase is the main computation performed by the verifier. In the case of our hybrid 
model SNARKs, however, the input to the offline counterpart described above (which we call the $\mathit{PartInput}$ algorithm) may even be the witness or 
a part of the witness for the respective relation. For our custom SNARKs, $\mathit{PartInput}$ produces part of the public input used by the verifier; 
since for our use case, $\mathit{PartInput}$ does handle a portion of the witness, this operation cannot be performed by the verifier for that relation. 
Moreover, in our instantiation, $\mathit{PartInput}$ produces computationally binding commitment schemes that are opened by the prover. Both of these properties 
are not explicitly part of our general definition for hybrid model SNARKs, but they are crucial and explicitly assumed and used 
in proving the security for our compiler's second step (see Appendix~\ref{sec_two_step_compiler}).
\end{comment}
\vspace{-0.1in}
%\begin{comment}
%\subsection{BLS multisignatures} \label{ssec:BLS}

%\noindent {\color{red} We implement and use an efficient BLS multisignature scheme that has both efficient 
verification and efficient key aggregation (for more details, see section~\ref{sec:bls}). 
In general, multisignatures are susceptible to so-called ``rogue-key attacks'' which can be 
mounted whenever the adversary is allowed to choose his public keys arbitrarily. 
In a typical rogue-key attack, the adversary uses a public key that is a function of an 
honest user's key and this allows him to produce forgeries easily. The BLS multisignatures that we define and use 
in this work make no exception. So, in order to protect against rogue-key attacks, 
we enhance them with proofs-of-posession as defined in~\cite{proofs_of_posession}.} \\

\noindent {\color{red}Alternative constructions for BLS multisignatures as well as other defence mechanisms against rogue key attacks exist and 
we briefly review both below. First, a more general case to BLS mltisignatures exists, namely aggregating BLS signatures for different messages 
(e.g., ~\cite{aggregate_BLS_signatures}). In this case, the rouge-key attacks are not a threat anymore, but multisignature verification 
is computationally more expensive than in our case, requiring $O(n)$ parings for $n$ different messages. Second, alternative 
BLS multisignatures exist (e.g.,~\cite{boneh_compact_multisig}) where both the aggregated public key have their size independent 
of the number of signers and the multisignature verification is as fast as in our variant. However, for our application we prefer the scheme 
detailed in section~\ref{sec:bls}: in spite of requiring a one-time verification of proofs-of-posession, its corresponding aggregate key is a 
simple sum of the individual signers' public keys, while, in ~\cite{boneh_compact_multisig} the key aggregation operation involves more 
expensive scalar multiplications every time the key aggregation is performed.}


%TODO: This could be explained somewhere other than the intro if preferred {\color{red}@Alistair from Oana: Can we replace or merge the 4 sentences below on multisignatures with/%into the more detailed description I give above?}

%There are several variants of BLS multisignatures. We will consider the easiest one for which the aggregate public key is simply the sum of the public keys of the signers

%This naive scheme is vulnerable to rogue key attacks, which can be prevented using proofs of possession [cn].  In our case, we can assume that the list of public keys comes from a %trusted source, such as the previous set of validators, who checked the proofs of possession and so the verifier themselves does not need to.

%\subsection{Implementation}
%\label{sec:intro_implementation}

%\noindent {\color{red} Our implementation leverages a pair of pairing-friendly elliptic curves which we call the inner curve and the outer curve respectively, such that the base field of the %inner curve matches the scalar field of the outer curve.} \\

%\noindent {\color{red}The first pair of pairing friendly elliptic curves where the inner curve's base field and the outer curve's scalar field are identical 
%but the two curves do not form a cycle has been introduced by ZEXE~\cite{zexe}. The authors call such a pair of elliptic curves a two-chain. 
%The ZEXE two-chain curves are BLS12-377 and CP6-782. In this work we build on the two-chain ZEXE instantiation in the following way: we 
%keep BLS12-377 as the inner curve and, for efficiency reasons which will be detailed later in the paper, we replace the CP6-782 outer curve with 
%BW6-761~\cite{BW6}.\\

%\noindent Both BLS12-377 and BW6-761 are pairing friendly curves and we make use of that as follows. Intuitively we use BLS12-377 
%to sign and verify BLS signatures with the public keys being elements of the first source group of the efficient pairing associated with BLS12-377. Hence, 
%our BLS signature public keys are natively represented over the base field of BLS12-377. Then we use BW6-761 to prove using succinct proofs 
%(i.e., snarks) public key aggregation for public keys signing the same message. Because the base field of BLS12-377 matches the scalar field 
%of BW6-761 and since in a snark system the proof is performed in an arithmetic circuit over the scalar field of the curve, so, 
%in our case the scalar field of BW6-761, any efficiency loss due to curves mismatch is avoided. \\

%\noindent Note that even if our implementation is instantiated with a specific pairing-friendly two-chain as described above, 
%our theoretical results (see Section \ref{sec:snarks}) generalise to any pairing-friendly two-chain and, where possible, we state 
%them as such.} 
%\end{comment}



\section{Preliminaries} \label{sec_prelims}
%\subsection{Conventions}
%\label{sec:conventions}
%\vspace{-0.03in}
\noindent We assume all algorithms receive an implicit security parameter $\lambda$. 
An efficient algorithm is one that runs in uniform probabilistic polynomial time (PPT) in the length of its input and $\lambda$. 
%{\color{blue} Every input to each of our algorithms apart from the message in our BLS (multi)signature 
%has at most polynomial length in the security parameter. If the length of that message would be polynomially bounded we could then just say that 
%that ``efficient algorithm means an algorithm that runs in polynomial time in the security parameter".}
%When we say that $A$ is an efficient adversary we mean that $A$ is a family $\{A_{\lambda}\}_{\lambda \in \mathbb{N}}$ 
%of non-uniform polynomial-size circuits. If the adversary consists of multiple circuit families $A_1, A_2, \ldots$, then we write $A = (A_1, A_2, \ldots)$. 
We assume the correct parameters for the curves, groups, pairings, the group generators, etc. have been generated and shared with all parties before running any algorithm or protocol. 
A function $f(\lambda)$ is negligible in $\lambda$, written as $\mathsf{negl}(\lambda)$, if $1/f(\lambda)$ grows faster than 
any polynomial in $\lambda$ and is overwhelming in $\lambda$ if $1-f(\lambda)=\mathsf{negl}(\lambda)$. By $\mathsf{poly}(\lambda)$ 
we mean some polynomial in $\lambda$ and \ewnp means except with probability $\mathsf{negl}(\lambda)$.
We write $y = A(x; r)$ when algorithm $A$ on input $x$ and randomness $r$, outputs $y$.
We write $y \leftarrow A(x)$ for picking randomness $r$ uniformly at random and setting $y = A(x; r)$. We denote by $|S|$ the cardinality of set $S$. 
%Unless otherwise stated, when we write that an event holds with some probability, we implicitly mean 
%that the probability is computed over the randomness of all randomised algorithms involved.
%We say a function is negligible in $\lambda$ and denote it by $\mathit{negl}(\lambda)$ if that function vanishes faster than the inverse of any polynomial in $\lambda$. 
%We say that a function is overwhelming in $\lambda$ if it has the form $1- $ some function negligible in $\lambda$. 
%We also use the notation \ewnp to mean except with negligible probability, or, equivalently, with overwhelming probability. We denote by $\mathsf{poly}(\lambda)$ an  
%unspecified function which has a polynomial expression in $\lambda$. 
%We generally use boldface font to denote vectors whose components we explicitly make use of in the text and 
%we use italic font to denote the rest of the variables.
%We work over finite fields of large characteristic. 
$\mathbb{F}_{<d}[X]$ is the set of all polynomials of degree less than $d$ over the field $\mathbb{F}$. For any integer 
$n \geq 1$, we denote by $[n]$ the set $\{1, \ldots, n\}$.
\vspace{-0.015in}

%\vspace{-0.05in}
\subsection{Pairings}
\label{sec:pairings}
\begin{comment}
%\vspace{-0.02in}
\noindent If $E$ is an elliptic curve defined over a prime field $\mathbb{F}_{p}$ of large characteristic $p$, 
we denote by $E(\mathbb{F}_{p})$ the abelian group containing all the points $(x, y) \in (\mathbb{F}_{p})^2$ 
on the curve along with the point at infinity. We will work with pairing friendly curves i.e., those with a secure~\cite{secure_pairings,pairings_for_cryptographers} efficiently computable, bilinear, non-degenerate mapping from a prime order subgroup of $E(\mathbb{F}_{p})$ and a subgroup of the curve over the extension field.
We will work with a \emph{pairing-friendly two-chain}, i.e., a pair of pairing friendly elliptic curves $\einn=E(\mathbb{F}_{p})$ (\emph{the inner curve}) and $\eout=E'(\mathbb{F}_{r})$ (\emph{the outer curve}), such that the pairing $\epinn$ on $\einn$ works on subgroups of order $r$. $\mathbb{F}_p$ is the \emph{base field} of $\einn=E(\mathbb{F}_{p})$ and $\mathbb{F}_r$ is its \emph{scalar field}. 
We write $\ginn{1}$, $\ginn{2}$, $\gtinn$, $\gout{1}$, $\gout{2}$, $\gtout$ for cyclic subgroups of $\einn$, $E(\mathbb{F}_{p^l}$),$\mathbb{F}_{p^k}$, $\eout$, $E'(\mathbb{F}_{r^{l'}}$), $\mathbb{F}_{r^{k'}}$ respectively for suitable $l,k,l',k$ with the two pairings $\epinn:\ginn{1} \times \ginn{2} \rightarrow \gtinn$ and by $\epout:\gout{1} \times \gout{2} \rightarrow \gtout$.
We write $\sginn{1}$, $\sginn{2}$, $\sgtinn$, $\sgout{1}$, $\sgout{2}$, $\sgtout$ respectively for randomly chosen generators of these groups. We use additive notation for group operations and write $[x]_{\indexoneinn} = x \cdot \sginn{1}$, $[x]_{\indextwoinn} = x \cdot \sginn{2}$. Concretely, our implementation uses BLS12-377~\cite{zexe} and BW6-761~\cite{BW6} for $\einn$ and $\eout$.
\vspace{-0.05in}
\end{comment}

%\subsection{Secure Signature Aggregation}
\label{sec:multisig_short}
\vspace{-0.03in}
An aggregatable signature scheme (AS) compresses signatures using
different signing keys into one signature. In this work we use an aggregatable 
signature scheme making explicit use of the proofs-of-possession (PoPs)~\cite{proofs_of_posession}.
Overall, for our concrete instantiation, we use aggregatable BLS signatures with an 
efficient aggregation procedure, i.e., by adding together keys and by adding together 
signatures, and we protect against rogue key attacks~\cite{proofs_of_posession} using PoPs. 
This is in contrast to other aggregation procedures that do not require PoPs for security 
but incur a higher computational cost (e.g., due to the use of multi-scalar
multiplication~\cite{boneh_compact_multisig}). For our concrete use case of accountable 
light clients systems, our efficient signature aggregation method results 
in a simple and more efficient SNARK which compensates for the cost of having to work with PoPs. 
\vspace{-0.1in}
\begin{definition}
\label{def:aggregate_signatures}
(Aggregatable Signature Scheme) An aggregatable signature scheme consists of
the following tuple of algorithms ($\mathit{AS.Setup}$, $\mathit{AS.GenKeypair}$, $\mathit{AS.VerifyPoP}$, 
$\mathit{AS.Sign}$, $\mathit{AS.AggKeys}$, $\mathit{AS.AggSig}$, $\mathit{AS.Verify}$) 
such that for implicit security parameter $\lambda$:
\vspace{-0.05in}
\begin{itemize}

\item $\mathit{pp} \leftarrow  \mathit{AS.Setup}(\mathit{aux_{\mathit{AS}}})$: a setup algorithm that, given an 
auxiliary parameter $\mathit{aux_{\mathit{AS}}}$, outputs public protocol parameters $\mathit{pp}$. 

\item $((\mathit{pk},\mathit{\pi_{PoP}}),\mathit{sk}) \leftarrow \mathit{AS.GenKeypair}(\mathit{pp})$:
a key pair generation algorithm that
outputs 
a secret key $\mathit{sk}$,
and the corresponding public key $\mathit{pk}$
together with a proof of possession $\mathit{\pi_{PoP}}$ for the secret key.

\item $0/1 \leftarrow \mathit{AS.VerifyPoP}(\mathit{pp}, \mathit{pk},\mathit{\pi_{PoP}})$:
a public key verification algorithm that,
given a public key $\mathit{pk}$
and a proof of possession $\mathit{\pi_{PoP}}$,
outputs
$1$ if $\mathit{\pi_{PoP}}$ is valid for $\mathit{pk}$ and $0$ otherwise.

\item $\sigma \leftarrow \mathit{AS.Sign}(\mathit{pp}, \mathit{sk}, m)$:
a signing algorithm that,
given a secret key $\mathit{sk}$ and a message $m \in \{0, 1\}^*$, returns a signature $\sigma$.

\item $\mathit{apk} \leftarrow \mathit{AS.AggKeys}(\mathit{pp}, (\mathit{pk_i})_{i=1}^{u})$:
a public key aggregation algorithm that,
given a vector of public keys $(\mathit{pk_i})_{i=1}^u$,
returns
an aggregate public key $\mathit{apk}$.

\item $\mathit{asig} \leftarrow \mathit{AS.AggSig}(\mathit{pp}, (\sigma_i)_{i=1}^u)$:
a signature aggregation algorithm that,
given a vector of signatures $(\sigma_i)_{i=1}^u$,
returns
an aggregate signature $\mathit{asig}$.

\item $0/1 \leftarrow \mathit{AS.Verify}(\mathit{pp}, \mathit{apk}, m, \mathit{asig})$:
a signature verification algorithm that,
given an aggregate public key $\mathit{apk}$, a message $m \in \{0, 1\}^*$, and an aggregate signature $\sigma$,
returns
1 or 0 to indicate validity.
\end{itemize}
\vspace{-0.07in}
\noindent We say AS is an aggregatable signature scheme if it satisfies \emph{perfect completeness} and \emph{unforgeability} 
as standard security definitions (see appendix~\ref{sec:multisig} for full details) 
and, additionally, \emph{perfect completeness} for aggregation defined below.
\end{definition}

%\noindent We require an aggregatable signature scheme as defined above to
%satisfy \emph{perfect completeness}, \emph{unforgeability} and 
%{\color{red} \emph{verifiable aggregation w.r.t.\ malicious signers}} as follows:

%\noindent \textbf{Perfect Completeness} An aggregatable signature scheme
%($\mathit{AS.Setup}$, $\mathit{AS.GenKeypair}$, $\mathit{AS.VerifyPoP}$, $\mathit{AS.Sign}$, $\mathit{AS.AggregateKeys}$, 
%$\mathit{AS.AggregateSignatures}$, $\mathit{AS.Verify}$) has perfect completeness if for any message $m \in \{0,1\}^*$ and any 
%$u\in\mathbb{N}$ it holds that:
%\begin{align*}
%\mathit{Pr} [\mathit{AS.Verify}(\mathit{pp}, \mathit{apk}, m, \mathit{asig})=1 \ & \wedge \ \forall  i \in [u]\ \mathit{AS.VerifyPoP}(\mathit{pp}, \mathit{pk_i},\mathit{\pi_{\mathit{PoP},i}})=1\ |\\
%& \mathit{pp} \leftarrow \mathit{AS.Setup}(\mathit{aux_{\mathit{AS}}}), \\
%& ((pk_{i},\pi_{\mathit{PoP}, i}), sk_{i} ) \leftarrow \mathit{AS.GenKeypair}(\mathit{pp}),\ i=1,\ldots, u\\
%&\mathit{apk} \leftarrow \mathit{AggregateKeys}(\mathit{pp}, (\mathit{pk}_{i})_{i=1}^{u}), \\
%& \sigma_i \leftarrow \mathit{AS.Sign}(\mathit{pp}, \mathit{sk_i}, m),\ i=1,\ldots,u, \\
%& \mathit{asig} \leftarrow \mathit{AS.AggregateSignatures(\mathit{pp}, (\sigma_{i})_{i=1}^{u})}] = 1.
%\end{align*}
%\noindent We note that an aggregatable signature scheme with perfect completeness implies the underlying signature scheme
%has perfect completeness. \\
\vspace{-0.02in}
\noindent \textbf{Perfect Completeness for Aggregation} An aggregatable 
signature scheme AS
has perfect completeness for aggregation if, for every adversary $\mathcal{A}$
\begin{align*}
\mathit{Pr} & [\mathit{AS.Verify}(\mathit{pp}, \mathit{apk}, m, \mathit{asig}) = 1 \ | \ \mathit{pp} \leftarrow \mathit{AS.Setup}(\mathit{aux_{\mathit{AS}}}),  ((\mathit{pk_i})_{i=1}^u, m, (\sigma_i)_{i=1}^{u}) \leftarrow \mathcal{A}(\mathit{\mathit{pp})}, \\ 
%& {\color{red} \forall i \in [u], \mathit{AS.VerifyPoP}(\mathit{pp}, \mathit{pk_i}, \pi_{PoP,i}) = 1, }\\
 &\forall i \in [u], \mathit{AS.Verify}(\mathit{pp}, \mathit{pk_i}, m, \sigma_i) = 1, \mathit{apk} \leftarrow \mathit{AS.AggKeys}(\mathit{pp},  (\mathit{pk}_{i})_{i=1}^{u}), \mathit{asig} \leftarrow \mathit{AS.AggSigs}(\mathit{pp}, (\sigma_i)_{i=1}^u)] = 1.
\end{align*}

%\noindent \textbf{Unforgeable Aggregatable Signature}
%For an aggregatable signature scheme ($\mathit{AS.Setup}$, $\mathit{AS.GenKeypair}$, $\mathit{AS.VerifyPoP}$, $\mathit{AS.Sign}$,
%$\mathit{AS.AggregateKeys}$, $\mathit{AS.AggregateSignatures}$, $\mathit{AS.Verify}$)
%the advantage of an adversary against unforgeability is defined by

%$$\mathit{Adv}^{\mathit{forge}}_{\mathcal{A}}({\lambda}) = \mathit{Pr}[\mathit{Game}^{\mathit{forge}}_{\mathcal{A}}({\lambda}) =1]$$
%\noindent where
%\begin{align*}
%&\mathit{Game}^{\mathit{forge}}_{\mathcal{A}}({\lambda}): \\
%& \mathit{pp} \leftarrow \mathit{AS.Setup}(\mathit{aux_{\mathit{AS}}}) \\
%& ((\mathit{pk}^*,\pi^*_{\mathit{PoP}}), \mathit{sk}^*) \leftarrow \mathit{AS.GenKeypair}(\mathit{pp})\\
%& Q \leftarrow \emptyset \\
%& ((\mathit{pk_i}, \pi_{\mathit{PoP},i})_{i=1}^{u}, m, \mathit{asig}) \leftarrow \mathcal{A}^{\mathit{OSign}}(\mathit{pp}, (\mathit{pk^*},\pi^*_{\mathit{PoP}})) \\
%& \textit{If } \mathit{pk}^* \notin \{\mathit{pk_i}\}_{i=1}^{u} \vee m \in Q, \textit{ then return } 0 \\
%& \textit{For } i \in [u] \\
%& \ \ \ \ \ \textit{ If } \mathit{AS.VerifyPoP}(\mathit{pp}, \mathit{pk_i}, \pi_{\mathit{PoP},i})=0  \textit{ return } 0 \\
%& \mathit{apk} \leftarrow \mathit{AS.AggregateKeys}(\mathit{pp}, (\mathit{pk_i})_{i=1}^{u}) \\
%& \textit{Return } \mathit{AS.Verify}(\mathit{pp}, \mathit{apk}, m, \mathit{asig})
%\end{align*}
%\noindent and
%\begin{align*}
%& \mathit{OSign}(m_j): \\
%& \sigma_j \leftarrow \mathit{AS.Sign}(\mathit{pp}, \mathit{sk}^*, m_j) \\
%&  Q \leftarrow Q \cup \{m_j\} \\
%& \textit{Return} \ \sigma_j
%\end{align*}

%\noindent and $\mathcal{A}^{\mathit{OSign}}$ denotes the adversary $\mathcal{A}$ with access to oracle $\mathit{OSign}$. \\

%\noindent We say an aggregatable signature scheme is unforgeable if for all efficient adversaries
%$\mathcal{A}$ it holds that $\mathit{Adv}^{\mathit{forge}}_{\mathcal{A}}({\lambda}) \leq \mathit{negl}(\lambda)$. 
\vspace{-0.08in}
\subsubsection{An Aggregatable Signature Instantiation}
\label{sec:bls}
\noindent In the following, we instantiate the aggregatable signature definition given above with a scheme inspired by the BLS signature
scheme~\cite{BLS_signatures} and its follow-up variants~\cite{proofs_of_posession,boneh_compact_multisig}.
\vspace{-0.08in}
\begin{construction}(Aggregatable Signatures) 
\label{insta:bls}
For aggregatable signatures, our implementation uses an instantiation of BLS signatures using proofs-of-possession which are $\ginn{2}$ elements, 
where the public keys are in $\ginn{1}$  and the signatures are in $\ginn{2}$. The public key aggregation is a simple sum of the 
public keys and the signature aggregation is a simple sum of the individual signatures. We instantiate $\einn$ with BLS12-377~\cite{zexe}. Full details can be found in Appendix~\ref{sec:multisig}.
 
\begin{comment}
\begin{itemize}
\item $(\ginn{1}, \sginn{1}, \ginn{2}, \sginn{2}, \gtinn, \epinn, \Hinn, \HPoP)$ from $\mathit{pp}$ where 
$\mathit{pp} \leftarrow  \mathit{AS.Setup}(\mathit{aux_{\mathit{AS}}})$, 
where $\ginn{1}$, $\sginn{1}$, $\ginn{2}$, $\sginn{2}$, $\gtinn$, $\epinn$ were defined in Section~\ref{sec:pairings} and 
$\Hinn: \{0,1\}^* \rightarrow \ginn{2}$ and $\HPoP: \{0,1\}^* \rightarrow \ginn{2}$ are two hash functions. 
The auxiliary parameter $\mathit{aux_{\mathit{AS}}}$ is such that there exists $N \in \mathbb{N}$, 
$N$ is the first component of the vector $\mathit{aux_{\mathit{AS}}}$ and there exists a subgroup of size at least $N$ in the multiplicative group of $\mathbb{F}$, where $\mathbb{F}$ 
is the base field of $\einn$, but also the size of the subgroup $\in O(N)$.

\item $(\mathit{pk},\mathit{sk}, \pi_{\PoP}) \leftarrow \mathit{AS.GenKeypair}(\mathit{pp})$, where $\mathit{sk} \xleftarrow{\$} \mathbb{Z}_{r}^{*}$  
and $\mathit{pk} = \mathit{sk} \cdot \sginn{1} \in \ginn{1}$ and $\pi_{\PoP} \leftarrow {\mathit{sk}} \cdot \HPoP(\mathit{pk})$ 
and $r$ was defined in Section~\ref{sec:pairings} as the characteristic of the scalar field of $\einn$.

\item $0/1 \leftarrow \mathit{AS.VerifyPoP}(\mathit{pp}, \mathit{pk}, \pi_{\PoP})$, where $\mathit{AS.VerifyPoP}$ outputs $1$ if 
$$\epinn( \sginn{1}, \pi_{\PoP}) = \epinn(\mathit{pk}, \HPoP(\mathit{pk}))$$ holds and $0$ otherwise. Note that implicitly, as part of running \\
$\mathit{AS.VerifyPoP}$, one checks that $\mathit{pk} \in \ginn{1}$ also holds.

\item $\sigma \leftarrow \mathit{AS.Sign}(\mathit{pp}, \mathit{sk}, m)$: 
where $\sigma = \mathit{sk} \cdot \Hinn(m) \in \ginn{2}$.

\item $\mathit{apk} \leftarrow \mathit{AS.AggregateKeys}(\mathit{pp}, (\mathit{pk_i})_{i=1}^{u})$, where  $\mathit{apk} = \sum_{i=1}^{u} \mathit{pk_i}$. 
Note that $\mathit{AS.AggregateKeys}$ checks whether $((\mathit{pk_i})_{i=1}^{u}) \in \ginn{1}^{u} (\ast)$ and, if that is not the case, it outputs $\bot$; 
if $(\ast)$ holds, the algorithm $\mathit{AS.AggregateKeys}$ continues with the computations described above. 


\item $\mathit{asig} \leftarrow \mathit{AS.AggregateSignatures}(\mathit{pp}, (\sigma_i)_{i=1}^u)$, where $\mathit{asig}$ = $\sum_{i=1}^{u} \sigma_i$.  

\item $0/1 \leftarrow  \mathit{AS.Verify}(\mathit{pp}, \mathit{apk}, m, \mathit{asig})$, where $\mathit{AS.Verify}$ outputs $1$ if $\mathit{apk} \neq \bot$ and
$\mathit{apk} \in \ginn{1}$ and $\epinn(\mathit{apk}, \Hinn(m)) = \epinn(\sginn{1}, \mathit{asig})$; otherwise, it outputs $0$.
\end{itemize}
\end{comment}
\end{construction}
\vspace{-0.1in}
%\subsection{Aggregatable Signature Scheme Definition}
\label{sec:multisig}
%\label{suplementary_aggregatable}
An aggregatable signature scheme compresses signatures issued using possibly 
different signing keys into one signature. In this work we use an aggregatable 
signature scheme making explicit use of the proofs-of-possession (PoPs)~\cite{proofs_of_posession}.
For our concrete instantiation we use aggregatable BLS signatures with an 
efficient aggregation procedure, i.e., by adding together keys and by multiplying together 
signatures, and protect against rogue key attacks~\cite{proofs_of_posession} using PoPs. 
This is in contrast to other aggregation procedures that do not require PoPs for security 
but incur a higher computational cost (e.g., due to the use of multi-scalar multiplication~\cite{boneh_compact_multisig}). 
For our concrete use case of accountable light clients systems, our efficient signature aggregation method results 
in a simple and more efficient custom argument scheme (i.e., SNARK), which, in turn, compensates for the cost of having 
to work with PoPs. 
\begin{definition}
\label{def:aggregate_signatures}
(Aggregatable Signature Scheme) An aggregatable signature scheme consists of
the following tuple of algorithms ($\mathit{AS.Setup}$, $\mathit{AS.GenerateKeypair}$, $\mathit{AS.VerifyPoP}$, 
$\mathit{AS.Sign}$, \\ $\mathit{AS.AggregateKeys}$, $\mathit{AS.AggregateSignatures}$, $\mathit{AS.Verify}$) 
such that for implicit security parameter $\lambda$:
\begin{itemize}

\item $\mathit{pp} \leftarrow  \mathit{AS.Setup}(\mathit{aux_{\mathit{AS}}})$: a setup algorithm that, given an 
auxiliary parameter $\mathit{aux_{\mathit{AS}}}$, outputs public protocol parameters $\mathit{pp}$. 

\item $((\mathit{pk},\mathit{\pi_{PoP}}),\mathit{sk}) \leftarrow \mathit{AS.GenerateKeypair}(\mathit{pp})$:
a key pair generation algorithm that
outputs
a secret key $\mathit{sk}$,
and the corresponding public key $\mathit{pk}$
together with a proof of possession $\mathit{\pi_{PoP}}$ for the secret key.

\item $0/1 \leftarrow \mathit{AS.VerifyPoP}(\mathit{pp}, \mathit{pk},\mathit{\pi_{PoP}})$:
a public key verification algorithm that,
given a public key $\mathit{pk}$
and a proof of possession $\mathit{\pi_{PoP}}$,
outputs
$1$ if $\mathit{\pi_{PoP}}$ is valid for $\mathit{pk}$ and $0$ otherwise.

\item $\sigma \leftarrow \mathit{AS.Sign}(\mathit{pp}, \mathit{sk}, m)$:
a signing algorithm that,
given a secret key $\mathit{sk}$ and a message $m$ in $\{0, 1\}^*$, returns a signature $\sigma$.

\item $\mathit{apk} \leftarrow \mathit{AS.AggregateKeys}(\mathit{pp}, (\mathit{pk_i})_{i=1}^{u})$:
a public key aggregation algorithm that,
given a vector of public keys $(\mathit{pk_i})_{i=1}^u$,
returns
an aggregate public key $\mathit{apk}$.

\item $\mathit{asig} \leftarrow \mathit{AS.AggregateSignatures}(\mathit{pp}, (\sigma_i)_{i=1}^u)$:
a signature aggregation algorithm that,
given a vector of signatures $(\sigma_i)_{i=1}^u$,
returns
an aggregate signature $\mathit{asig}$.

\item $0/1 \leftarrow \mathit{AS.Verify}(\mathit{pp}, \mathit{apk}, m, \mathit{asig})$:
a signature verification algorithm that,
given an aggregate public key $\mathit{apk}$, a message $m \in \{0, 1\}^*$, and an aggregate signature $\sigma$,
returns
1 or 0 to indicate if the signature is valid.
\end{itemize}

\noindent We say ($\mathit{AS.Setup}$, $\mathit{AS.GenerateKeypair}$, $\mathit{AS.VerifyPoP}$, 
$\mathit{AS.Sign}$, $\mathit{AS.AggregateKeys}$, \\ $\mathit{AS.AggregateSignatures}$, 
$\mathit{AS.Verify}$) is an aggregatable signature scheme if it satisfies \emph{perfect completeness}  and 
\emph{perfect completeness for aggregation}  and \emph{unforgeability} as defined below. \\

\noindent \textbf{Perfect Completeness} An aggregatable signature scheme
($\mathit{AS.Setup}$, $\mathit{AS.GenerateKeypair}$, \\ $\mathit{AS.VerifyPoP}$, $\mathit{AS.Sign}$, $\mathit{AS.AggregateKeys}$,
$\mathit{AS.AggregateSignatures}$, $\mathit{AS.Verify}$) has perfect completeness if for any message $m \in \{0,1\}^*$ and any 
$u\in\mathbb{N}$ it holds that:
\begin{align*}
&\mathit{Pr} [\mathit{AS.Verify}(\mathit{pp}, \mathit{apk}, m, \mathit{asig})=1 \  \wedge \ \forall  i \in [u]\ \mathit{AS.VerifyPoP}(\mathit{pp}, \mathit{pk_i},\mathit{\pi_{\mathit{PoP},i}})=1\ |\\
& \mathit{pp} \leftarrow \mathit{AS.Setup}(\mathit{aux_{\mathit{AS}}}), \\
& ((pk_{i},\pi_{\mathit{PoP}, i}), sk_{i} ) \leftarrow \mathit{AS.GenerateKeypair}(\mathit{pp}),\ i=1,\ldots, u\\
&\mathit{apk} \leftarrow \mathit{AggregateKeys}(\mathit{pp}, (\mathit{pk}_{i})_{i=1}^{u}), \\
& \sigma_i \leftarrow \mathit{AS.Sign}(\mathit{pp}, \mathit{sk_i}, m),\ i=1,\ldots, u, \\
& \mathit{asig} \leftarrow \mathit{AS.AggregateSignatures(\mathit{pp}, (\sigma_{i})_{i=1}^{u})}] = 1.
\end{align*}
\noindent We note that an aggregatable signature scheme with perfect completeness implies the underlying signature scheme
has perfect completeness. \\

\noindent \textbf{Perfect Completeness for Aggregation} An aggregatable signature scheme 
($\mathit{AS.Setup}$, \\ $\mathit{AS.GenerateKeypair}$, $\mathit{AS.VerifyPoP}$, $\mathit{AS.Sign}$, 
$\mathit{AS.AggregateKeys}$, $\mathit{AS.AggregateSignatures}$, $\mathit{AS.Verify}$)
has perfect completeness for aggregation if, for every adversary $\mathcal{A}$
\begin{align*}
& \mathit{Pr}[\mathit{AS.Verify}(\mathit{pp}, \mathit{apk}, m, \mathit{asig}) = 1 \ | \ \mathit{pp} \leftarrow \mathit{AS.Setup}(\mathit{aux_{\mathit{AS}}}), \\
& ((\mathit{pk_i})_{i=1}^u, m, (\sigma_i)_{i=1}^{u}) \leftarrow \mathcal{A}(\mathit{\mathit{pp})} \ 
\textit{such that} \ \forall i \in [u], \mathit{AS.Verify}(\mathit{pp}, \mathit{pk_i}, m, \sigma_i) = 1, \\
& \mathit{apk} \leftarrow \mathit{AS.AggregateKeys}(\mathit{pp},  (\mathit{pk}_{i})_{i=1}^{u}), \\
&  \mathit{asig} \leftarrow \mathit{AS.AggregateSignatures}(\mathit{pp}, (\sigma_i)_{i=1}^u)] = 1.
\end{align*}

\noindent \textbf{Unforgeable Aggregatable Signature}
For an aggregatable signature scheme ($\mathit{AS.Setup}$, \\ $\mathit{AS.GenerateKeypair}$, $\mathit{AS.VerifyPoP}$, $\mathit{AS.Sign}$,
$\mathit{AS.AggregateKeys}$, $\mathit{AS.AggregateSignatures}$, $\mathit{AS.Verify}$)
the advantage of an adversary against unforgeability is defined by

$$\mathit{Adv}^{\mathit{forge}}_{\mathcal{A}}({\lambda}) = \mathit{Pr}[\mathit{Game}^{\mathit{forge}}_{\mathcal{A}}({\lambda}) =1]$$
\noindent where
\begin{align*}
&\mathit{Game}^{\mathit{forge}}_{\mathcal{A}}({\lambda}): \\
& \mathit{pp} \leftarrow \mathit{AS.Setup}(\mathit{aux_{\mathit{AS}}}) \\
& ((\mathit{pk}^*,\pi^*_{\mathit{PoP}}), \mathit{sk}^*) \leftarrow \mathit{AS.GenerateKeypair}(\mathit{pp})\\
& Q \leftarrow \emptyset \\
& ((\mathit{pk_i}, \pi_{\mathit{PoP},i})_{i=1}^{u}, m, \mathit{asig}) \leftarrow \mathcal{A}^{\mathit{OSign}}(\mathit{pp}, (\mathit{pk^*},\pi^*_{\mathit{PoP}})) \\
& \textit{If } \mathit{pk}^* \notin \{\mathit{pk_i}\}_{i=1}^{u} \vee m \in Q, \textit{ then return } 0 \\
& \textit{For } i \in [u] \\
& \ \ \ \ \ \textit{ If } \mathit{AS.VerifyPoP}(\mathit{pp}, \mathit{pk_i}, \pi_{\mathit{PoP},i})=0  \textit{ return } 0 \\
& \mathit{apk} \leftarrow \mathit{AS.AggregateKeys}(\mathit{pp}, (\mathit{pk_i})_{i=1}^{u}) \\
& \textit{Return } \mathit{AS.Verify}(\mathit{pp}, \mathit{apk}, m, \mathit{asig})
\end{align*}
\noindent and
\begin{align*}
& \mathit{OSign}(m_j): \\
& \sigma_j \leftarrow \mathit{AS.Sign}(\mathit{pp}, \mathit{sk}^*, m_j) \\
&  Q \leftarrow Q \cup \{m_j\} \\
& \textit{Return} \ \sigma_j
\end{align*}

\noindent and $\mathcal{A}^{\mathit{OSign}}$ denotes the adversary $\mathcal{A}$ with access to oracle $\mathit{OSign}$. \\

\noindent We say an aggregatable signature scheme is unforgeable if for all efficient adversaries
$\mathcal{A}$ it holds that $\mathit{Adv}^{\mathit{forge}}_{\mathcal{A}}({\lambda}) \leq \mathit{negl}(\lambda)$. 
\end{definition}

\subsubsection{An Aggregatable Signature Instantiation}
\label{sec:bls}
\noindent In the following, we instantiate the aggregatable signature definition given above with a scheme inspired by the BLS signature
scheme~\cite{BLS_signatures} and its follow-up variants~\cite{proofs_of_posession,boneh_compact_multisig}.

\begin{construction}(Aggregatable Signatures) 
\label{insta:bls}
In our implementation we call aggregatable signatures the following 
instantiation of aggregatable signatures definition. Note that in our implementation we instantiate $\einn$ with BLS12-377~\cite{zexe}.
\begin{itemize}
\item $(\ginn{1}, \sginn{1}, \ginn{2}, \sginn{2}, \gtinn, \epinn, \Hinn, \HPoP)$ from $\mathit{pp}$ where 
$\mathit{pp} \leftarrow  \mathit{AS.Setup}(\mathit{aux_{\mathit{AS}}})$, 
where $\ginn{1}$, $\sginn{1}$, $\ginn{2}$, $\sginn{2}$, $\gtinn$, $\epinn$ were defined in Section~\ref{sec:pairings} and 
$\Hinn: \{0,1\}^* \rightarrow \ginn{2}$ and $\HPoP: \{0,1\}^* \rightarrow \ginn{2}$ are two hash functions. 
The auxiliary parameter $\mathit{aux_{\mathit{AS}}}$ is such that there exists $N \in \mathbb{N}$, 
$N$ is the first component of the vector $\mathit{aux_{\mathit{AS}}}$ and there exists a subgroup of size at least $N$ in the multiplicative group of $\mathbb{F}$, where $\mathbb{F}$ 
is the base field of $\einn$, but also the size of the subgroup $\in O(N)$.

\item $(\mathit{pk},\mathit{sk}, \pi_{\PoP}) \leftarrow \mathit{AS.GenerateKeypair}(\mathit{pp})$, where $\mathit{sk} \xleftarrow{\$} \mathbb{Z}_{r}^{*}$  
and $\mathit{pk} = \mathit{sk} \cdot \sginn{1} \in \ginn{1}$ and $\pi_{\PoP} \leftarrow {\mathit{sk}} \cdot \HPoP(\mathit{pk})$ 
and $r$ was defined in Section~\ref{sec:pairings} as the characteristic of the scalar field of $\einn$.

\item $0/1 \leftarrow \mathit{AS.VerifyPoP}(\mathit{pp}, \mathit{pk}, \pi_{\PoP})$, where $\mathit{AS.VerifyPoP}$ outputs $1$ if 
$$\epinn( \sginn{1}, \pi_{\PoP}) = \epinn(\mathit{pk}, \HPoP(\mathit{pk}))$$ holds and $0$ otherwise. Note that implicitly, as part of running \\
$\mathit{AS.VerifyPoP}$, one checks that $\mathit{pk} \in \ginn{1}$ also holds.

\item $\sigma \leftarrow \mathit{AS.Sign}(\mathit{pp}, \mathit{sk}, m)$: 
where $\sigma = \mathit{sk} \cdot \Hinn(m) \in \ginn{2}$.

\item $\mathit{apk} \leftarrow \mathit{AS.AggregateKeys}(\mathit{pp}, (\mathit{pk_i})_{i=1}^{u})$, where  $\mathit{apk} = \sum_{i=1}^{u} \mathit{pk_i}$. 
Note that $\mathit{AS.AggregateKeys}$ checks whether $((\mathit{pk_i})_{i=1}^{u}) \in \ginn{1}^{u} (\ast)$ and, if that is not the case, it outputs $\bot$; 
if $(\ast)$ holds, the algorithm $\mathit{AS.AggregateKeys}$ continues with the computations described above. 


\item $\mathit{asig} \leftarrow \mathit{AS.AggregateSignatures}(\mathit{pp}, (\sigma_i)_{i=1}^u)$, where $\mathit{asig}$ = $\sum_{i=1}^{u} \sigma_i$.  

\item $0/1 \leftarrow  \mathit{AS.Verify}(\mathit{pp}, \mathit{apk}, m, \mathit{asig})$, where $\mathit{AS.Verify}$ outputs $1$ if $\mathit{apk} \neq \bot$ and
$\mathit{apk} \in \ginn{1}$ and $\epinn(\mathit{apk}, \Hinn(m)) = \epinn(\sginn{1}, \mathit{asig})$; otherwise, it outputs $0$.
\end{itemize}
\end{construction}

\subsection{Conditional NP Relations}
\label{sec:conditional_relations}
%\vspace{-0.01in}
\noindent By $\mathcal{R} =\{(x;w): p(x,w) = 1 \}$ we denote the binary relation such that $(x,w)$ 
fulfil predicate $p(x,w) = 1$. We say $\mathcal{R}$ is an NP relation if predicate $p$ can be checked in polynomial 
time in the length of both inputs $x$ and $w$ and $\mathcal{L}(\mathcal{R})= \{x \ | \ \exists w \textit{ s.t. } (x,w) \in \mathcal{R} \}$ 
is an NP language w.r.t. predicate $p$. In such a case we call $x$ an \emph{instance} and $w$ a \emph{witness}.  \\

%\vspace{-0.07in}
\noindent  In order to model a specific property of our NP relations, we introduce further notation which we call \emph{conditional NP relation}, we denote it by 
$$\mathcal{R}^c = \{(x;w) : (p_1(x,w) =1 \ | \ c(x,w) =1) \ \wedge \ p_2(x,w) = 1 \}$$ and we interpret it as the NP relation containing the pairs of inputs and witnesses 
$(x,w)$ such that $c(x,w) =1$, $p_1(x,w) = 1$ and $p_2(x,w) =1$ hold. However, in order to prove that $(x,w) \in \mathcal{R}^c$ we assume/take it as a given that 
$c(x,w) =1$ and we are left to prove only that $p_1(x,w) = 1$ and $p_2(x,w) =1$ hold. The reason we separate predicate $c(x,w)$ from predicate $p_1(x,w)$ in the definition 
of $\mathcal{R}^c$ is that predicate $c(x,w)$ may be inefficient to prove inside a proof system (e.g., in our case, inside a SNARK); using the above separation, one can delegate 
(in some particular situations) the verification of $c(x,w)$ to a trusted party outside the proof system.\\

%\vspace{-0.07in}
\noindent We explicitly include in the definition of any NP relation $\mathcal{R}$ or $\mathcal{R}^c$ the corresponding domain for each type 
of public input. The interpretation of such domains is that each type of public input is parsed by the honest parties (e.g., a SNARK verifier for an NP relation 
$\mathcal{R}$ or $\mathcal{R}^c$) as per the definition of the respective domain, without additional checks. We assume that all our relations have been 
generated using implicit security parameter $\lambda$. Finally, when we make a statement about an NP relation we implicitly 
mean the statement is about a conditional relation $\mathcal{R}^c$, where $c$ may be the predicate that always outputs $1$. 

%\noindent {\color{red} In order to model a specific property of our NP relations, we introduce further notation which we call \emph{conditional NP relation} 
%and we denote it by $\mathcal{R}^c = \{(x;w) \  | \ c(x,w): p(x,w) \}$ which means that $x$ and $w$ fulfil predicate $p$ as long as $x$ and $w$ fulfil additional 
%predicate $c$. As a special case, we will encounter the situation that the witness for each public input is, in fact, the empty string; this is denoted by 
%$\mathcal{R}^c= \{(x;) \ | \ c(x) : p(x)\}$. The reason we separate predicate $c(x,w)$ from predicate $p(x,w)$ in the definition of $\mathcal{R}^c$ 
%is that predicate $c(x,w)$ may be expensive to prove and/or verify; by separating them, one can delegate the verification of $c(x,w)$ 
%to a trusted party with enough computational power. In turn, such a trusted party can be implemented accordingly in a 
%real-world system.}
%\vspace{-0.09in}
\section{O-SNARKs} \label{sec:short_snarks_defs}
\subsection{O-SNARKs}
\label{sec:short_snarks_defs}
In the following, we remind the reader the definition of an O-SNARK from~\cite{O_SNARK} and we prove that {\color{red}XXX}.
In order to do that, we start with a couple building block definitions: i.e., for algebraic adversaries (see, for example~\cite{AGM_model}) 
and our new definition of AGM respecting oracles.

\begin{definition}[Algebraic Adversaries]
\label{def:algebraic_adv}
\end{definition}

\begin{definition}[AGM Respecting Oracles]
\label{def:agm_oracles}
\end{definition}

\begin{definition}[O-SNARKs]
\label{def:osnarks}
\end{definition}

\begin{theorem}[O-SNARKS with AGM Respecting Oracles]
\label{the:when_osnarks} 
Every $Z$ auxiliary input SNARK $\Pi$ secure in the AGM model is an 
O-SNARK for $\mathbb{O}$ if $\mathbb{O}$ is AGM respecting and 
$Z$ is defined as the probability distribution of all the public parameters that define 
$\mathbb{O}$ together with all the polynomial number $Q$ of queries and answers that 
the adversarial prover in the O-SNARK makes to the oracle $\mathbb{O}$.
\end{theorem}

\begin{proof}
\end{proof}



%\subsection{SNARKs}
\label{sec:snarks_defs}

\noindent All three SNARKs we design in this work have access to a \emph{structured reference string} (srs) of the form 
$(\{[\tau^i]_1\}_{i=0}^{d}$ , $\{[\tau^i]_2\}_{i=0}^{1})$ where $\tau$ is a random (and allegedly secret) value in $\mathbb{F}$ and $d$ 
is bounded by a polynomial in $\lambda$. Such an srs is \emph{universal} and \emph{updatable} \cite{updatable_universal_srs_2018} and, 
as long as at least one of the participants that took part in the MPC generating the srs was honest, the srs cannot be used by 
any coalition of other MPC participants to prove false statements with more than a negligible probability of 
success \cite{updatable_universal_srs_2018,ariel_MPC_SRS_2017}. \\

\noindent Our SNARKs are secure in the \emph{algebraic group model} (AGM) \cite{AGM_model}. 
If $\mathbb{G}$ is a cyclic group of prime order $p$, then, informally, we call an algorithm $\mathcal{A}$ \emph{algebraic} if it
fulfils the following requirement: whenever $\mathcal{A}$ outputs a group element $g \in  \mathbb{G}$, 
it also outputs a representation $\mathbf{a} = (a_1, . . . , a_t) \in \mathbb{Z}_p^t$ such that $g = \sum_{i=1}^{t} a_i \cdot B_i$
where $(B_1, \ldots, B_t )$ are all the $\mathbb{G}$ group elements that were given to $\mathcal{A}$ during its execution so far. 
The AGM lies in between the \emph{generic group model} (GGM) \cite{GGM_model1, GGM_model2} and the standard model and, 
lately, it has been the preferred model for proving security for the most efficient SNARKs 
(e.g., PLONK~\cite{plonk}, Marlin~\cite{marlin} or Groth16~\cite{groth16} with its proof in the AGM model 
presented in \cite{AGM_model, another_extractable_groth16}). \\

\noindent In the following, we introduce a generalisation of the usual SNARK definition which we call a \emph{hybrid model SNARK}. 
As mentioned in the introduction, this is inspired by the notion of online-offline SNARKs~\cite{HP_paper}, however, for 
our use case we need to further refine it as describe below:                                      
\begin{dfn}(Hybrid Model SNARK)
\label{dfn_snark}
A hybrid model \emph{succinct non-interactive argument of knowledge for relation $\mathcal{R}$} is a tuple of PPT algorithms 
$(\mathit{SNARK.Setup}, \mathit{SNARK.KeyGen}, \mathit{SNARK.Prove}, \\ \mathit{SNARK.Verify}, \mathit{SNARK.PartInputs})$ 
such that for implicit security parameter $\lambda$: 

\begin{itemize}
\item $\mathit{srs} \leftarrow \mathit{SNARK.Setup} (\mathit{aux_{\mathit{SNARK}}})$: a setup algorithm that on input auxiliary parameter 
$\mathit{aux_{\mathit{SNARK}}}$ from some domain $\mathcal{D}$ outputs a universal structured reference string tuple $\mathit{srs}$, 

\item $(\mathit{srs_{pk}}, \mathit{srs_{vk}}) \leftarrow \mathit{SNARK.KeyGen}(\mathit{srs}, \mathcal{R})$: a key generation algorithm that on input a
universal structured reference string $\mathit{srs}$ and an NP relation $\mathcal{R}$ outputs a \emph{proving key} and 
a \emph{verification key} pair $(\mathit{srs_{pk}}, \mathit{srs_{vk}})$,

\item $\pi \leftarrow \mathit{SNARK.Prove}(\mathit{srs_{pk}}, (x,w), \mathcal{R})$: a proof generation algorithm that on input a proving key 
$\mathit{srs_{pk}}$ and a pair $(x,w) \in \mathcal{R}$ outputs \emph{proof} $\pi$, 

\item $0/1 \leftarrow \mathit{SNARK.Verify}(\mathit{srs_{vk}}, x, \pi, \mathcal{R})$: a proof verification algorithm that on input a verification key 
$\mathit{srs_{vk}}$, an instance $x$ and a proof $\pi$ outputs a bit that signals acceptance (if output is $1$) or rejection (if output is $0$)
%\item $\pi_{\mathit{Sim}} \leftarrow \mathit{Sim}(R, \sigma, x)$: The simulator $\mathit{Sim}$ takes the relation $R$, the structured reference string 
%$\mathit{srs}$ and the instance $x$ as input and outputs a simulated proof $\pi_{\mathit{Sim}}$,

\item $(x_1, \mathit{state}_2) \leftarrow \mathit{SNARK.PartInputs}(\mathit{srs}, \mathit{state}_1, \mathcal{R})$: a deterministic 
public inputs generation algorithm that takes as input a universal structured reference string $\mathit{srs}$, an NP relation $\mathcal{R}$ and 
some state $\mathit{state}_1$ and outputs some updated state $\mathit{state}_2$ and some partial public input $x_1$,

\end{itemize}
and satisfies completeness, knowledge soundness with respect to $\mathit{SNARK.PartInputs}$ and succinctness as defined below:

\noindent \textbf{Perfect Completeness} holds if an honest prover will always convince an honest verifier: for all  
$(x,w) \in \mathcal{R}$ and for all $\mathit{aux_{\mathit{SNARK}}} \in \mathcal{D}$
\begin{align*}
\mathit{Pr}[\mathit{SNARK.Verify}(\mathit{srs_{vk}}, x, \pi, \mathcal{R}) = 1 \ | \ & 
\mathit{srs} \leftarrow \mathit{SNARK.Setup}(\mathit{aux_{\mathit{SNARK}}}), \\ 
& (\mathit{srs_{pk}}, \mathit{srs_{vk}})\leftarrow \mathit{SNARK.KeyGen}(\mathit{srs}, \mathcal{R}), \\
& \pi \leftarrow \mathit{SNARK.Prove}(\mathit{srs_{pk}}, (x,w), \mathcal{R}) \ ] = 1.
\end{align*}

\noindent \textbf{Notation} In the following, we denote by $\mathit{State_{\mathcal{R}}}$ the set of all states $\mathit{state}_1$ 
such that given some relation $\mathcal{R}$ and any possible $\mathit{srs}$, 
for any output $x_1$ of $\mathit{SNARK.PartInputs}(\mathit{srs}, \mathcal{R}, \mathit{state}_1)$ 
with $\mathit{state_1} \in \mathit{State_{\mathcal{R}}}$, we have that there exists $x_2$ and $w$ with $(x=(x_1, x_2), w) \in \mathcal{R}$; 
we further make the assumption that $\mathit{State_{\mathcal{R}}} \neq \emptyset$.\\

\noindent \textbf{Knowledge-soundness with respect to $\mathit{SNARK.PartInputs}$}
holds if there exists a PPT extractor $\mathcal{E}$ such that for all PPT 
adversaries $\mathcal{A}$, for all $\mathit{aux_{\mathit{SNARK}}} \in \mathcal{D}$ and for all $\mathit{state_1} \in \mathit{State_{\mathcal{R}}}$
\begin{align*}
\mathit{Pr}[&(x = (x_1, x_2), w) \in \mathcal{R} \wedge 1 \leftarrow \mathit{SNARK.Verify}(\mathit{srs_{vk}}, x = (x_1, x_2), \pi, \mathcal{R}) \ | \\
& \mathit{srs} \leftarrow \mathit{SNARK.Setup}(\mathit{aux_{\mathit{SNARK}}}), (\mathit{srs_{pk}}, \mathit{srs_{vk}})\leftarrow \mathit{SNARK.KeyGen}(\mathit{srs}, \mathcal{R}), \\ 
& (x_1, \mathit{state}_2) \leftarrow \mathit{SNARK.PartInput}(\mathit{srs}, \mathit{state}_1, \mathcal{R}), 
(x_2, \pi) \leftarrow \mathcal{A}(\mathit{srs}, \mathit{state}_2, \mathcal{R}),  
w \leftarrow \mathcal{E}^{\mathcal{A}}(srs,\mathit{state_2}, \mathcal{R})]
\end{align*}
is overwhelming in $\lambda$, where by $\mathcal{E}^{\mathcal{A}}$ we denote the extractor $\mathcal{E}$ that has access to all of 
$\mathcal{A}$'s messages during the protocol with the honest verifier. %(the messages include the coefficients of the linear combinations of 
%group elements used by $\mathcal{A}$ at any step in order to output new group elements at the next step in the protocol). 

%\noindent \textbf{Knowledge-soundness} holds if there exists a PPT extractor $\mathcal{E}$ such that for all PPT AGM adversaries $\mathcal{A}$
%\begin{align*}
%\mathit{Pr}[(x, w) \in \mathcal{R} \wedge \mathit{SNARK.Verify}(\mathit{srs_{vk}}, x, \pi, \mathcal{R}) = 1 \ | \ &
%\mathit{srs} \leftarrow \mathit{SNARK.Setup}(\lambda) \\ 
%& (\mathit{srs_{pk}}, \mathit{srs_{vk}})\leftarrow \mathit{SNARK.KeyGen}(\mathit{srs}, \mathcal{R}), \\
%& (x, \pi) \leftarrow \mathcal{A}(\mathit{srs}, \mathcal{R}), w \leftarrow \mathcal{E}^{\mathcal{A}}(srs)]
%\end{align*}
%is overwhelming in $\lambda$, where by $\mathcal{E}^{\mathcal{A}}$ we denote the extractor $\mathcal{E}$ that has access to all of 
%$\mathcal{A}$'s messages during the protocol with the honest verifier (the messages include the coefficients of the linear combinations of 
%group elements used by $\mathcal{A}$ at any step in order to output new group elements at the next step in the protocol).

\noindent \textbf{Succinctness} holds if the size of the proof $\pi$ is $\mathsf{poly}(\lambda)$ and $\mathit{SNARK.Verify}$ runs in time 
$\mathsf{poly}(\lambda + |x|)$. % $+ \log{|w|}$ has been removed in both cases.
\end{dfn}

\noindent Firstly, note that if one chooses $x_1$, $\mathit{state_1}$ and $\mathit{state_2}$ to be the empty strings in the definition of 
$\mathit{SNARK.PartInput}$ and in relation to the knowledge soundness property, one obtains a more standard SNARK definition.
Secondly, $\mathcal{R}$ is not a component of the vector $\mathit{aux_{\mathit{SNARK}}}$ so even if $\mathit{SNARK.Setup}$ has 
$\mathit{aux_{\mathit{SNARK}}}$ as parameter, it is universal, 
i.e., it can be used to derive proving and verification keys for circuits of any size up to a polynomial in the security parameter $\lambda$,   
independently of any specific NP relation. Moreover, for the SNARKs we design, the size of the key used by 
the honest verifier is much smaller than the size of the honest prover's key. We have made the separation clear between the two keys to be able 
to better capture this special case; however, a potential adversarial prover has access to the complete $\mathit{srs}$ key. 
Thirdly, as mentioned the SNARKs that we design in this work are secure in the $\mathit{AGM}$ model. This means that we limit our adversaries to 
$\mathit{AGM}$ adversaries only and by $\mathcal{E}^{\mathcal{A}}$ we denote the extractor $\mathcal{E}$ that has access to all of 
$\mathcal{A}$'s messages during the protocol with the honest verifier: the messages include the coefficients of the linear combinations of 
group elements used by the $\mathit{AGM}$ adversary at any step in order to output new group elements at the next step in the protocol. 
Moreover, the auxiliary input (i.e., $\mathit{state_2}$) is required to be drawn from a ``benign distribution'' or else extraction may be 
impossible~\cite{extractability_limits_1,extractability_limits_2}. 
%Note also that in some SNARKs related work (e.g., PLONK~\cite{plonk}) the length of the public input is not explicitly taken into account in the definition 
%of succinctness, however we make that explicit in our definition and this is in line with both older and newer work in the 
%area (\cite{groth16, marlin}). %Moreover, similarly to PLONK (even though not explicitly specified there), the number of 
%field operations which are part of our SNARK verification depends as well on the length of the public input.
Finally, in the SNARK definition above we did not include the notion of zero-knowledge since it is not required in the rest of the paper.
%\subsection{Ranged Polynomial Protocols and Polynomial Commitments}
In order to prove the security of the SNARKs designed in this work we use a SNARK compiler inspired by the one provided in lemma 4.7 from 
PLONK~\cite{plonk}. In more detail, for each of our three conditional NP relations we describe a ranged polynomial protocol and then we use our compiler to obtain three SNARKs 
secure in the AGM. We remind the definition of ranged polynomial protocols in section~\ref{sec:poly_protocols_appendix} in the appendix. Moreover, we also make use of 
KZG polynomial commitments \cite{KZG_10}, in particular their batched version and their security definitions as described in section 3 from PLONK. For brevity, 
and since we do not make any alterations to the definition of batched KZG commitments, we do not repeat it in this initial version of our work but invite the reader 
to review them, if necessary, by following the reference provided. 

%In order to prove the security of the snarks designed in this work we use the snark compiler proposed in lemma 4.7 from 
%PLONK~\cite{plonk}. In more detail, for each of our snarks, we describe an $H$-ranged polynomial protocol for a relation $\mathcal{R}_i$, 
%where the relations $\mathcal{R}_i, i \in \{1,2,3\}$ are chosen to model certain statements we are interested in with respect to a set of 
%BLS public keys and their simple aggregate. Each $H$-ranged polynomial protocol for a relation $\mathcal{R}_i$ follows the definitions 4.1 and 4.3 
%with the additional clarifications given in the beginning of section 4.1 of PLONK. In our case, $H$ is an appropriately chosen multiplicative 
%subgroup as defined in section \ref{sec:lagrange} such that the fast Fourier transforms (FFTs) performed by the snarks provers are efficient. \\


%\subsection{Lagrange Bases}
\label{sec:lagrange}

In order to design the SNARKs presented in this work, it is more convenient to represent the polynomials 
we work with over the Lagrange base rather than the monomial base. Formally, for the finite field $\mathbb{F}$ defined in section~\ref{sec:pairings} 
we denote by $H$ a subgroup of the multiplicative group of $\mathbb{F}$ such that $n = |H|$ is a large power of 2. Let $\omega$ be an $n$-th 
root of unity in $\mathbb{F}$ such that $\omega$ is a generator of $H$. Then, we call the following polynomial base $\{L_i(X)\} _{0 \leq i\leq n-1}$ 
a Lagrange base, where $\forall i, 0 \leq i \leq n-1$, $L_i(X)$ is the unique polynomial in $\mathbb{F}_{<n}[X]$ such that 
$L_i(\omega^i) =1$ and $L_i(\omega^j) = 0, \forall j \neq i$.\\

\noindent Independent of the notion of Lagrange bases, but related to $n$ we define $\block$ also a power of 2 such that $\block < n$. 
We use $\block$ when defining one of our conditional NP relations in section \ref{sec:snarks}. In the following we assume 
$n = \mathsf{poly} (\lambda)$ and $\block = \Theta(\lambda)$ and $|\mathbb{F}|= 2^{\Theta(\lambda)}$.%, i.e., $|\mathbb{F}|$ is exponential in $\lambda$.}

 
%{\color{blue} TO DO: Make the statement about $|\mathbb{F}|$ stronger. In the following we assume 
%$n= \mathsf{poly} (\lambda)$ and $|\mathbb{F}|= \lambda^{\omega(1)}$, i.e., $|\mathbb{F}|$ is super-polynomial in $\lambda$. Also, is $\block$ a constant or is $\mathsf{poly} (\lambda)$? }



\section{A Fully Succinct Public Keys Aggregation Argument} \label{sec_apk_proofs}
In the following, we construct a \emph{counting SNARK} which allows a prover to convince an efficient verifier that an alleged 
aggregated public key has been computed correctly as an aggregate of a vector of public keys for which 
two succinct commitments (to the $x$ and $y$ affine coordinates of points) are publicly known. Additionally, our counting 
SNARK ensures that the alleged aggregated public key, in turn, is a scalar product between the entire set of public keys 
committed as mentioned above and a \emph{bitvector} with one bit associated to each public key 
(necessary to signal the inclusion or omission of the respective public key w.r.t. the aggregate key).  
We finally mention how to transform the counting SNARK into a SNARK for building non-accountable light client systems. \\

\noindent To compile our desired SNARK we proceed as follows: in 
Section~\ref{sec_vt} we define vector-based conditional NP relations $\Rvt$ (i.e., counting) and we design a 
ranged polynomial protocol for this relation. The ranged polynomial protocol notion originates in~\cite{plonk}; 
we review it in Section~\ref{supplementary_poly_protocols_appendix} including a refinement introduced in~\cite{LC_paper}.  
In Section~\ref{sec:interesting} we use the two-steps PLONK-based compiler introduced in~\cite{LC_paper} which is necessary to compile the counting 
ranged polynomial protocol into an O-SNARK for an NP relation based on mixed vector and pair of polynomial commitments; 
we denote this relation by $\Rvtcom$. \\

\noindent To start with, we define our vector based counting NP relation over $\mathbb{F}$, i.e., the base field of a pairing friendly 
elliptic curve $\einn$. Our SNARK prover's circuit is defined as well over $\mathbb{F}$ as the scalar field of a second pairing friendly elliptic 
curve $\eout$. The vector of public keys, which is part of the public input for $\Rvt$ and is denoted by 
$\mathbf{pk} = (\mathit{pk_0}, \ldots, \mathit{pk_{n-2}})$, is a vector of pairs with each component in $\mathbb{F}$. 
This vector has size $n-1$ (where $n$ is the order of a group $H$ defined below). For $\Rvt$ we denote 
the $n$ components bitvector by $\mathbf{bit} = (\mathit{bit_0}, \ldots, \mathit{bit_{n-1}})$ 
(meaning that each component belongs to the set $\{0,1\} \subset \mathbb{F}$). 
Each bit signals the inclusion (or exclusion) of the index-wise corresponding public keys 
into the aggregated public key $\mathit{apk}$. The $n$-th component $\mathit{bit_{n-1}}$ does not correspond to any public key, 
but, as will become clear in the following, has been included for easier design of constraints. \\ 

\vspace{-0.009in}
\noindent We denote by $H$ the multiplicative subgroup of $\mathbb{F}$ generated 
by an $n$-th root of unity $\omega$. We denote by \\ $\mathit{incl}(a_0, \ldots, a_{n-2})$ the inclusion 
predicate that checks if $(a_0, \ldots, a_{n-2}) \in \ginn{1}^{n-1}$, where $\ginn{1}$ is the first source group 
of pairing function associated with $\einn$. Moreover let $h = (\mathit{h_x}, \mathit{h_y})$ be some fixed, 
publicly known element in $\einn \setminus \ginn{1}$. We denote by $(a_x, a_y)$ the affine representation in 
$x$ and $y$ coordinates of $a \in \einn$ and by $\oplus$ the point addition in affine coordinates on the 
elliptic curve $\einn$. We denote by $\mathbb{B} = \{0,1\} \subset \mathbb{F}$. \\
\vspace{-0.15in}

\subsection{Counting Ranged Polynomial Protocol}
\label{sec_vt}

\noindent Next, we introduce the following Lagrange interpolation polynomials of degree at most $n-1 = |H|$ over cyclic group $H$: 
$b(X)$ - interpolates the bits of bitvector $\mathit{bit}$; $pkx(X)$, $pky(X)$ - interpolate all public keys' 
$x$ and $y$ coordinates, respectively; $kaccx(X)$, $kaccy(X)$ - interpolate $x$ and $y$ coordinates, respectively, 
of the iterative partial aggregate sum of the actual signing validators' public keys. We also define six polynomial identities 
$id_1(X)=0, \ldots, id_6(X)=0$ supporting the following intuition: $id_1(X)=0$, $id_2(X)=0$ over $H$ 
ensure the $x$ and, respectively, the $y$ coordinates of the iterative partial aggregate sums of actual signing validators public keys 
(up to each index $i \leq n-2$) follow formulas $(\ast)$, $(\ast\ast)$ from Observation 1 (see Appendix~\ref{sec:delayed}) which gives all possible cases of 
complete curve point addition when the second curve point is multiplied by a bit; $id_3(X)=0$, $id_4(X)=0$ over $H$ 
ensure first partial aggregate sum is $h$ and the total aggregate sum is $h + \mathit{apk}$; this is necessary in order to ensure the 
addition of the public keys (i.e., elliptic curve points) never falls into condition (3) defined in Observation 1 (see Appendix~\ref{sec:delayed}). 
This recursively implies the partial aggregate sum at every step is a well defined curve point, hence, it is a suitable input for the next step; 
$id_5(X)=0$ over $H$ ensures $b(X)$ evaluates to bits over $H$. $id_6(X)= 0$ over $H$ ensures that the sum of bits in the 
bitvector $\mathit{bit}$ is $s+1$. \\

\noindent Intuitively and overall the identities $id_1(X)=0$ to $id_6(X)=0$ over $H$ 
ensure $\mathit{apk}$ is the aggregated public key of at least $s$ and at most $s+1$ public keys. 
Hence, we interpret $s$ as a threshold on the number of public keys included in the aggregated public key. 
Since $\mathit{bit_{n-1}}$ as the last component of the bitmask witness does not correspond to any public 
key and we have to account for the fact that $\mathit{bit_{n-1}}$ may be 
$1 \in \mathbb{F}$, relation $\Rvt$ includes the off-by-one constraint $\sum_{i=0}^{n-1} \mathit{bit_i} = s+1$.\\

\noindent \textsf{Conditional Counting Relation $\Rvt$}  
\begin{equation*}
\begin{split}
 \Rvt = & \{(\mathbf{pk} \in (\mathbb{F}^2)^{n-1}, s \in \mathbb{F}, \mathit{apk} \in \mathbb{F}^2; \mathbf{bit}) : 
 \mathit{apk} = \sum_{i=0}^{n-2} [\mathit{bit_i}]  \cdot \mathit{pk_i} \ | \ \mathbf{pk}  \in \ginn{1}^{n-1} \ \wedge \\ 
& \wedge \ \mathbf{bit} \in \mathbb{B}^n \ \wedge \ \sum_{i=0}^{n-1} \mathit{bit_i} = s+1\} 
\end{split}
\end{equation*}

{\color{red} EXPLAIN THE CONDITIONAL PART IN THIS RELATION.} \\
\noindent The polynomials and polynomial identities computed and used are: \\

\noindent \textsf{Polynomials as Computed by Honest Parties} 

\begin{align*}
&\mathsf{b(X)} = \sum_{i=0}^{n-1} \mathit{bit_i} \cdot \mathsf{L_i(X)}; \mathsf{pkx(X)} =  \sum_{i=0}^{n-2} \mathit{pkx_i} \cdot \mathsf{L_i(X)}; 
\mathsf{pky(X)} =  \sum_{i=0}^{n-2} \mathit{pky_i} \cdot \mathsf{L_i(X)} \\
&\mathsf{kaccx(X)}  =  \sum_{i=0}^{n-1} \mathit{kaccx_i} \cdot \mathsf{L_i(X)}; \mathsf{kaccy(X)}  = \sum_{i=0}^{n-1} \mathit{kaccy_i} \cdot \mathsf{L_i(X)}; 
acc_{vt}(X)  = \sum_{i=0}^{n-1} acc_{vt,i} \cdot L_i(X),
\end{align*}
\noindent where $(\mathit{pkx_0}, \ldots, \mathit{pkx_{n-2}})$ 
and $(\mathit{pky_0}, \ldots, \mathit{pky_{n-2}})$ are computed such that $\forall i \in \{0, \ldots, n-2\}$, $\mathit{pk_i}$ 
is interpreted as a pair $(\mathit{pkx_i}, \mathit{pky_i})$ with its components in $\mathbb{F}$; we also have 
$(\mathit{kaccx_{0}}, \mathit{kaccy_{0}}) = (\mathit{h_x}, \mathit{h_y})$ and 
$(\mathit{kaccx_{i+1}}, \mathit{kaccy_{i+1}}) =  (\mathit{kaccx_{i}}, \mathit{kaccy_{i}}) \oplus \mathit{bit_i}(\mathit{pkx_{i}}, \mathit{pky_{i}})$, 
$\forall i < n-1$. Finally, $acc_{vt,i}$ are the components of the vector 
$(0, \mathit{bit_0}, \mathit{bit_0} + \mathit{bit_1}, \ldots, \sum_{i=0}^{n-2}\mathit{bit_i})$, $\forall i < n$. \\

\noindent \textsf{Polynomial Identities} 

\begin{align*}
& id_1(X) = (X-\omega^{n-1}) \cdot [b(X) \cdot ((kaccx(X)-pkx(X))^2 \cdot (kaccx(X)+ pkx(X) +  kaccx(\omega\cdot X)) - \\ 
& \ \ \ \ \ \ \ \ -(pky(X) - kaccy(X))^2) +  (1-b(X)) \cdot (kaccy(\omega\cdot X) - kaccy(X))]. \\
& id_2(X)  =  (X-\omega^{n-1})\cdot [b(X) \cdot ((kaccx(X) - pkx(X)) \cdot (kaccy(\omega \cdot X) + kaccy(X)) - \\
& \ \ \ \ \ \ \ \ - (pky(X) - kaccy(X)) \cdot (kaccx(\omega \cdot X) - kaccx(X))) + (1-b(X)) \cdot (kaccx(\omega \cdot X) - kaccx(X))]. \\
& id_3(X)  =  (kaccx(X) - h_x)\cdot L_0(X) + (kaccx(X) - (h\oplus apk)_{x}) \cdot L_{n-1}(X). \\ 
& id_4(X) =  (kaccy(X) - h_y)\cdot L_0(X) + (kaccy(X)  - (h\oplus apk)_{y}) \cdot L_{n-1}(X). \\
& id_5(X) =  b(X)(1-b(X)). \\
& id_6(X)  = acc_{vt}(\omega \cdot X) - acc_{vt}(X) - b(X) + (s+1) \cdot L_{n-1}(X).   \\
\end{align*}

\noindent \textsf{{$H$-ranged Polynomial Protocol for Conditional Counting Relation $\Rvt$}} \\

\noindent \textsf{Protocol $\Pvt$} \\

\noindent $\mathcal{P}_{poly}$ and $\mathcal{V}_{poly}$ know public input $s \in \mathbb{F}^2$, 
$\mathbf{pk} \in (\mathbb{F}^2)^{n-1}$ and $\mathit{apk} \in \mathbb{F}^2$ which are interpreted as per their respective domains. 

\begin{enumerate}
\item $\mathcal{V}_{poly}$ computes $pkx(X)$, $pky(X)$.
\item $\mathcal{P}_{poly}$ sends polynomials $b(X)$, $kaccx(X)$, $kaccy(X)$, $acc_{vt}(X)$ to $\mathcal{I}$. 
\item $\mathcal{V}_{poly}$ asks $\mathcal{I}$ to check whether the following polynomial relations hold over range $H$:
$$id_i(X) = 0, \forall i \in [6].$$ 
\item $\mathcal{V}_{poly}$ accepts if all of $\mathcal{I}$'s checks verify. 
\end{enumerate}

\noindent We show that protocol $\Pvt$ is an $H$-ranged polynomial protocol for conditional relation 
$\Rvt$. 

\begin{lemma} $\Pvt$ as described above is an $H$-ranged polynomial protocol for conditional relation $\Rvt$.
\end{lemma}

\begin{proof}
It is easy to see that perfect completeness holds. Indeed, if $(\mathbf{bit},\mathbf{pk}, \mathit{apk}) \in \Rvt$ holds, 
meaning that $\mathbf{bit} \in \mathbb{B}^n$ and $\mathbf{pk} \in \ginn{1}^{n-1}$ and $\mathit{apk} = \sum_{i=0}^{n-2} [\mathit{bit_i}] \cdot \mathit{pk_i}$ and 
$\sum_{i=0}^{n-1} \mathit{bit_i} = s+1$ hold, then it is easy to see that the honest prover $\mathcal{P}_{poly}$ in $\Pvt$ will convince the honest 
verifier $\mathcal{V}_{poly}$ in $\Pvt$ to accept with probability $1$. \\
Regarding knowledge-soundness, if the verifier $\mathcal{V}_{poly}$ in $\Pvt$ accepts, 
then we construct the extractor $\mathcal{E}$ in the following way. Using the polynomial $b(X)$ which 
was part of the messages from $\mathcal{P}_{poly}$ to $\mathcal{I}$ and evaluating it at the elements of the set 
$H$, $\mathcal{E}$ obtains evaluation vector $\mathbf{bit} = (b(1), \ldots, b(\omega^{n-1}))$ which, 
in the following, we denote as $(\mathit{bit}_0, \ldots, \mathit{bit}_{n-1}) \in \mathbb{F}^n$.\\ 
\noindent Next, we show that if $(\mathit{pk_0}, \ldots, \mathit{pk_{n-2}}) \in \ginn{1}^{n-1}$ holds and the 
verifier in $\Pvt$ accepts, then 
$$((\mathit{pk_0}, \ldots, \mathit{pk_{n-2}}), s, \mathit{apk}, (\mathit{bit_0}, \ldots, \mathit{bit_{n-1}})) \in \Rvt,$$ 
which is equivalent to proving that $\mathit{apk} = \sum_{i=0}^{n-2} [\mathit{bit_i}]  \cdot \mathit{pk_i}$ and 
$\mathbf{bit} \in \mathbb{B}^n$ and  $\sum_{i=0}^{n-1} \mathit{bit_i} = s+1$.
\noindent First, since $id_6(X) = 0$ holds over $H$, we can expand that as follows:
\begin{align*}
acc_{vt,1} &= acc_{vt,0} + \mathit{bit}_{0} \\
acc_{vt, 2} &= acc_{vt,1} + \mathit{bit}_{1} \\
\ldots \\
acc_{vt,n-1} &= acc_{vt,n-2} + \mathit{bit}_{n-2} \\
acc_{vt,0} &= acc_{vt,n-1} + \mathit{bit}_{n-1} - (s+1).
\end{align*}
\noindent By summing together the LHS and, respectively, the RHS of the equalities above and 
reducing the equal terms, we obtain $s+1 = \sum_{i=0}^{n-1}\mathit{bit}_i$. \\ 
Second, since it holds over $H$ that $id_i(X) = 0$, $\forall i \in [5]$ and $b(\omega^i) = \mathit{bit_i}, \forall i<n$ (by the definition 
of $\mathcal{E}$), the properties concluded in Claim~\ref{claim:keys_affine_comm} from Appendix~\ref{sec:delayed} hold. 
Combining the two proof steps above, we obtain the desired conclusion.
\end{proof}

\subsection{From a Polynomial Protocol to an O-SNARK}
\label{sec:interesting}
\noindent One can use the standard PLONK compiler (see Lemma 4.7~\cite{plonk}) to compile the 
counting ranged polynomial protocol for relation $\Rvt$ introduced in Section~\ref{sec_vt} 
into a SNARK $\Pi_{\Rvt}$ for the same relation. More precisely, let $\mathbb{O}$ be an 
AGM respecting oracle as per Definition~\ref{def:agm_oracles} and let $\mathcal{Z}_{\mathbb{O}}$ be defined as in the statement of Theorem~\ref{the:when_osnarks}. 
Then according to the PLONK compiler, $\mathcal{Z}_{\mathbb{O}}$ auxiliary input $\Pi_{\Rvt}$ is a SNARK.
Next, invoking Theorem~\ref{the:when_osnarks}, one can prove that $\Pi_{\Rvt}$ is an O-SNARK for $\mathbb{O}$. \\

\noindent In fact, what we have described so far corresponds to the first step of the compiler for the two-step compiler 
introduced in~\cite{LC_paper}.\footnote{As a reminder, the first compilation step in~\cite{LC_paper} uses a standard PLONK compiler.}
{\color{red} DOES this really hold?} Next, using a similar technique as the one employed in the proof of Theorem~\ref{the:when_osnarks}, one can show that 
$\mathcal{Z}_{\mathbb{O}}$ auxiliary input $\Pi_{\Rvtcom}$ is an O-SNARK for relation $\Rvtcom$, where

%$$\Rvtcom = {\color{red}Fill \ in. \ }$$

%\begin{align*}
%& {\Rlacom} = \{(\mathbf{C} \in \mathcal{C}, \mathbf{bit} \in \mathbb{B}^n, \mathit{apk} \in \mathbb{F}^2; \mathbf{pk}) : 
%\mathit{apk} = \sum_{i=0}^{n-2} [\mathit{bit_i}] \cdot \mathit{pk_i} \ | \\ & | \ \mathbf{pk} \in \ginn{1}^{n-1} \ \wedge \  \mathbf{C} = \mathbf{Com}(\mathbf{pk}) \} 
%\end{align*}

\begin{align*}
 {\Rvtcom} = \{&(\mathbf{C} \in \mathcal{C}, \mathit{apk} \in \mathbb{F}^2, s \in \mathbb{F}; \mathbf{pk}, \mathbf{bit}) : 
                       \mathit{apk} = \sum_{i=0}^{n-2} [\mathit{bit_i}] \cdot \mathit{pk_i} \  | \ \mathbf{pk} \in \ginn{1}^{n-1} \ \wedge \  \\
                       & \wedge \ \mathbf{bit} \in \mathbb{B}^n \ \wedge \ \sum_{i=0}^{n-1} \mathit{bit_i} = s+1 \ \wedge \ \mathbf{C} = \mathbf{Com}(\mathbf{pk}) \} 
\end{align*}
%\begin{equation*}
%\begin{split}
% \Rvt = & \{(\mathbf{pk} \in (\mathbb{F}^2)^{n-1}, s \in \mathbb{F}, \mathit{apk} \in \mathbb{F}^2; \mathbf{bit}) : 
% \mathit{apk} = \sum_{i=0}^{n-2} [\mathit{bit_i}]  \cdot \mathit{pk_i} \ | \ \mathbf{pk}  \in \ginn{1}^{n-1} \ \wedge \\ 
%& \wedge \ \mathbf{bit} \in \mathbb{B}^n \ \wedge \ \sum_{i=0}^{n-1} \mathit{bit_i} = s+1\} 
%\end{split}
%\end{equation*}
\noindent where $\mathit{Com}$ denotes the KZG polynomial commitment, 
$\mathcal{C}$ is the set of all $\mathit{KZG}$ poly commitments or vectors of such poly commitments and $\mathbf{C}$ is an alleged tuple of such commitments.


\noindent In turn, by using $\Pi_{\Rvtcom}$ as an $\mathcal{Z}_{\mathbb{O}}$ auxiliary input O-SNARK and an aggregatable signature 
scheme as defined also in~\cite{LC_paper}, one can construct a non-accountable light client scheme which is a very close analog 
to the accountable light client schemes introduced in~\cite{LC_paper}. While the perfect completeness for the non-accountable light client scheme 
can requires perfect completeness of the O-SNARK (which, in turn, is identical to the perfect completeness for a SNARK scheme), 
the soundness of the non-accountable scheme makes direct use of the knowledge-soundness 
property of the $\mathcal{Z}_{\mathbb{O}}$ auxiliary input O-SNARK for $\Pi_{\Rvtcom}$ which is specific for O-SNARKs. 
For more details on why is crucial we use an O-SNARK for the soundness security proof, please read Section~\ref{sec:discussion}.

\subsection{Discussion}
\label{sec:discussion}

{\color{red} Fill in.}

{\color{red} Comparison between the soundness proofs with and without O-SNARKs.}



\section{Implementation} \label{sec_implementation}
%\vspace{-0.2cm}
\noindent We implemented and benchmarked our custom SNARKs. The implementation allows us to evaluate the performance of our SNARKs and serve as prototype for future deployment. The implementation 
is open source and publicly available at \url{https://github.com/CCS23-anonymous/light-client}. It is written in Rust and uses the Arkworks library. \\

%\vspace{-0.1cm}
\noindent Table \ref{tab:benchmarks} gives the prover and verifier time for the two SNARK schemes (basic accountable and  packed accountable, see Section~\ref{sec:snarks}) with $v = n-1 = 2^{10}-1$, $v = n-1 = 2^{16}-1$ 
and $v=n-1=2^{20}-1$ signers. The benchmarks were run on a 3.6GHz 16-core AMD Ryzen 9 5950X. Here $v$ is the maximum number of signers and 
$n> v$ is the size of a multiplicative subgroup of the field (see Section~\ref{sec:lagrange}).\\

%\begin{center}
\begin{table*}[h!]
\hfill
\begin{tabular}{| l | l | l | l | l |l | l |}
\hline
 Scheme & \multicolumn{2}{|c|}{$v = 2^{10}-1$} & \multicolumn{2}{|c|}{$v = 2^{16}-1$} & \multicolumn{2}{|c|}{$v = 2^{20}-1$}     \\
\cline{2-7}
 &  prover & verifier & prover & verifier &  prover & verifier \\
\hline

Basic Accountable & 112ms & 11.6ms & 2.96s & 15.3ms & 42.1s & 89.7ms \\
Packed Accountable & 157ms & 12.5ms & 4.1s & 12.6ms & 58.0s & 14.2ms \\
%Counting            & 761ms & 27ms & 31s & 30ms & 692s & 107ms \\
\hline
\end{tabular}
%\setlength{\belowcaptionskip}{-0.5cm}
\caption{Proof and verifier times for the different schemes and numbers of signers}
\label{tab:benchmarks}
\end{table*}
%\end{center}

%\vspace{-0.2cm}
\noindent These signer set sizes are approximately the range of the number of validators that we aimed our implementation at e.g., the Kusama blockchain (\url{https://kusama.network/}). This network has 1000 validators which is also the number that Polkadot is aiming for, while Ethereum 2 has about 348,000 validators and it has been suggested that there will be no more than $2^{19}$~\cite{ethresearch1}. \\

\noindent At $v = n-1 = 1023$, the prover can generate a proof in any scheme in well under a second, which is short enough to generate a proof for every block in most prominent blockchains. Even for $v= n-1 =2^{20}-1$, the prover time is under 1 minute, 
when the time for an Ethereum 2 epoch is 6 minutes, i.e., the period that validators sign messages for finality of the chain. For verification time, the basic accountable scheme is slower, considerably so for larger sets of signers. \\

\noindent  Table \ref{tab:operations} gives the number of operations the prover and verifier use. Table \ref{tab:proof-size} gives the proof constituents and also the total proof and input sizes in bits. The basic accountable scheme's verifier performance at large numbers is slower because it includes $O(n)$ field operations, which dominate the running time, however at 1023 signers it gives the smallest size. The packed accountable scheme, which includes $O(n/\lambda)$ field operations, fairs better w.r.t. the verification benchmarks for large signer sets. The prover is considerably slower for the latter scheme because it needs to do additional operations. At larger signer sizes, the proof size  is dominated by the bitfield.

%\vspace{-0.1in}
\begin{table*}[h!]
\hfill
\begin{tabular}{| l | l| l| l|}
\hline
Scheme & Prover operations  &Verifier operations \\
\hline
Basic Accountable & $12\times FFT(n)+FFT(4n)+9ME(n)$  & $2P+11E+O(n)F$ \\
Packed Accountable & $18\times FFT(n)+FFT(4n)+12ME(n)$  & $2P+16E+O(n/\lambda+log(n))F$ \\
%Counting & $13\times FFT(n)+FFT(4n)+11ME(n)$  & $2P+14E+O(log(n))F$ \\
\hline
\end{tabular}
%\setlength{\belowcaptionskip}{-1cm}
\caption{Expensive prover and verifier operations. $FFT(M)$ is an FFT of size M. $ME(M)$ is a  multi-scalar multiplication of size $M$. $P$ is a pairing, $E$ is a single scalar multiplication and $F$ is a field operation.}
\label{tab:operations}
%\end{table*}
\vspace{-0.1in}
%\begin{table*}[h!]
\begin{tabular}{| l | l | l | l | l | l |}
\hline
Scheme & Proof & Input & \multicolumn{3}{|c|}{Actual proof + input size in bits} \\
\cline{4-6}
& & & $v = 2^{10}-1$ & $v = 2^{16}-1$ & $v = 2^{20}-1$ \\
\hline
Basic Accountable & $5\mathbb{G}_{1,out}+5\mathbb{F}$ & $2\mathbb{G}_{1,out}+1\mathbb{G}_{1,inn}+n$ bits & 9088 & 73600 & 1056640 \\
Packed Accountable & $8\mathbb{G}_{1,out}+8\mathbb{F}$ & $2\mathbb{G}_{1,out}+1\mathbb{G}_{1,inn}+n$ bits & 12544 & 77056 & 1060096 \\
%Counting &  $7\mathbb{G}_{1,out}+7\mathbb{F}$ & $2\mathbb{G}_{1,out}+1\mathbb{G}_{1,inn}$ & 10368  & 10368  & 10368 \\
\hline
\end{tabular}
\caption{Proof/input constituents and total proof/input size for implementation.}
\label{tab:proof-size}
\end{table*}


\section{Conclusions} \label{conclusions}
In this work we have revisited the notion of O-SNARKs in relation with a wide class of oracles which we call AGM 
respecting. Our result shows that many modern SNARKs (i.e., PLONK~\cite{plonk}, Marlin~\cite{marlin}, Groth16~\cite{groth16}, etc.) 
are in fact O-SNARKs when combined with AGM respecting oracles. Additionally, we shed more light via a concrete 
example where and how O-SNARK security is useful for practical and more complex real-world applications. 



\bibliography{bibliography} 
\bibliographystyle{ieeetr}
\appendix
\section{Ranged Polynomial Protocols for NP Relations}
\label{supplementary_poly_protocols_appendix}

\noindent In the following, we keep the convention that all algorithms receive an implicit security parameter $\lambda$. The definition below 
is a natural extension of the notions of polynomial protocols and polynomial protocols for relations from Section 4 of PLONK~\cite{plonk} to 
polynomial protocols over ranges for conditional NP relations with additional refinements required by our specific use case; these refinements are 
incorporated into steps (4), (5) and (6) as follows: 

\begin{definition}(Polynomial Protocols over Ranges for Conditional NP Relations)
\label{def_ranged_poly_protocol}
Assume three parties, a prover $\mathcal{P}_{poly}$, a verifier $\mathcal{V}_{poly}$ and a trusted party $\mathcal{I}$. 
Let $\mathcal{R}^c$ be a conditional NP relation (with $c$ being a predicate) and let $x$ be a public 
input both of which have been given to $\mathcal{P}_{poly}$ and $\mathcal{V}_{poly}$ by an $\mathit{InitGen}$ efficient algorithm. 
For positive integers $d$, $D$, $t$, $l$, $u$, $e$ and for set 
$S \subset \mathbb{F}$, an $S$-ranged $(d, D, t, l, u, e)$-polynomial protocol $\mathscr{P}_{\mathcal{R}^c}$ for relation $\mathcal{R}^c$ is a multi-round 
protocol between $\mathcal{P}_{poly}$, $\mathcal{V}_{poly}$ and $\mathcal{I}$ such that:

\begin{enumerate}
\item The protocol $\mathscr{P}_{\mathcal{R}^c}$ definition includes a set of pre-processed polynomials $g_1(X), \ldots, g_l(X) \in \mathbb{F}_{<d}[X]$. 

\item The messages of $\mathcal{P}_{poly}$ are sent to $\mathcal{I}$ and are of the form $f(X)$ for $f(X) \in \mathbb{F}_{<d}[X]$. 

If $\mathcal{P}_{poly}$ sends a message not of this form, the protocol is aborted.
\item The messages from $\mathcal{V}_{poly}$ to $\mathcal{P}_{poly}$ are random coins.

\item
$\mathcal{V}_{poly}$ may perform arithmetic computations using input $x$ and the random coins used in the 
communication with $\mathcal{P}_{poly}$. Let $(\mathit{res_1}, \ldots, \mathit{res_u})$ be the results of those computations 
which $\mathcal{V}_{poly}$ sends to $\mathcal{I}$. 

\item 
Using vectors which are part of input $x$ and/or other ad-hoc vectors which $\mathcal{V}_{poly}$ deems useful, $\mathcal{V}_{poly}$ 
may compute interpolation polynomials $s_1(X), \ldots, s_e(X)$ over domain $S$ such that $s_1(X), \ldots, s_e(X) \in \mathbb{F}_{<d}[X]$. 
$\mathcal{V}_{poly}$ sends $s_1(X), \ldots, s_e(X)$ to $\mathcal{I}$. 

\item 
At the end of the protocol, suppose $f_1(X), \ldots, f_t(X)$ are the polynomials that were sent from $\mathcal{P}_{poly}$ to 
$\mathcal{I}$. $\mathcal{V}_{poly}$ may ask $\mathcal{I}$ if certain polynomial identities hold between 
$$\{f_1(X), \ldots , f_t(X), g_1(X), \ldots, g_l(X), s_1(X), \ldots, s_e(X) \}$$ over set $S$ 
(i.e., if by evaluating all the polynomials that define the identity at each of the field elements from $S$ 
we obtain a true statement). Each such identity is of the form 

$$F(X) := G(X, h_1(v_1(X)), \dots , h_M(v_M(X))) \equiv 0,$$
for some $h_i(X) \in \{f_1(X), \ldots , f_t(X), g_1(X), \ldots , g_l(X), s_1(X), \ldots, s_e(X) \}$, \\ $G(X, X_1, \ldots, X_M) \in \mathbb{F}[X, X_1, \ldots , X_M]$, 
$v_1(X), \ldots , v_M(X) \in  \mathbb{F}_{<d}[X]$ such that $F(X) \in \mathbb{F}_{<D}[X]$ for every choice of 
$f_1(X), \ldots, f_t(X)$ made by $P_{poly}$ when following the protocol correctly. Note that some of the coefficients in the identities above may be 
from the set $\{\mathit{res_1}, \ldots, \mathit{res_u}\}$.
\item After receiving the answers from $I$ regarding the polynomial identities, 
$\mathcal{V}_{poly}$ outputs $\mathsf{acc}$ if all identities hold over set $S$, 
and outputs $\mathsf{rej}$ otherwise.
\end{enumerate}

\noindent Additionally, the following properties hold: \\

\noindent \textbf{Perfect Completeness:} If $\mathcal{P}_{poly}$ follows the protocol correctly and uses a witness $\omega$ with 
$(x, \omega) \in \mathcal{R}^c$, $\mathcal{V}_{poly}$ accepts with probability one.

\noindent \textbf{Knowledge Soundness:} There exists an efficient algorithm $E$, that given access to the messages of $\mathcal{P}_{poly}$ 
to $\mathcal{I}$ it outputs $\omega$ such that, for any strategy of $\mathcal{P}_{poly}$, the probability of $\mathcal{V}_{poly}$ 
outputting $\mathsf{acc}$ at the end of the protocol and, simultaneously, $(x, \omega) \in \mathcal{R}^c$ is overwhelming in $\lambda$.

\end{definition}

\noindent Our definition for polynomial protocols over ranges does not include a zero-knowledge property as it is not required in our current work. \\

\noindent Given the definition for polynomial protocols over ranges for conditional relations as detailed above, we are now ready to state the following result.
The proof follows with only minor changes from that of Lemmas 4.5. and 4.7. from~\cite{plonk}. 

\begin{lemma}(Compilation of Ranged Polynomial Protocols for Conditional NP Relations into Hybrid Model SNARKs using PLONK) 
\label{le:compilation_step_1}
Let $\mathscr{P}_{\mathcal{R}^c}$ be a public coin $S$-ranged $(d, D, t, l, u, e)$-polynomial protocol for relation $\mathcal{R}^c$ where only 
one identity is checked by $\mathcal{V}_{poly}$ and predicate $c$ from the definition of ${\mathcal{R}^c}$ needs to be fulfilled only by a part $x_1$ 
of the public input of the relation ${\mathcal{R}^c}$. Then one can construct a hybrid model SNARK protocol $\mathscr{P}^*_{\mathcal{R}^c}$ for relation 
$\mathscr{P}_{\mathcal{R}^c}$ with $\mathit{SNARK.PartInput}$ as defined below 
and with $\mathscr{P}^*_{\mathcal{R}^c}$ secure in the AGM under the $2d$-DLOG 
assumption\footnote{Definition 2.1. in PLONK~\cite{plonk} formally describes the $2d$-DLOG assumption.} such that:
\begin{enumerate}
\item The prover $\mathbf{P}$ in $\mathscr{P}^*_{\mathcal{R}^c}$ requires $\mathsf{e}(\mathscr{P}_{\mathcal{R}^c})$ $\gout{1}$-exponentiations where 
$\mathsf{e}(\mathscr{P}_{\mathcal{R}^c})$ is define analogously as in PLONK (see preamble of Section 4.2.), however it additionally takes into account 
polynomials $s_1(X), \ldots, s_e(X)$. 
\item The total prover communication consists of $t + t^*(\mathscr{P}_{\mathcal{R}^c}) + 1$ $\gout{1}$-elements and M $\mathbb{F}$-elements, where 
$t^*(\mathscr{P}_{\mathcal{R}^c})$ is defined identically as in PLONK (see preamble of Section 4.2.).
\item The verifier $\mathbf{V}$ in $\mathscr{P}^*_{\mathcal{R}^c}$ requires $t + t^*(\mathscr{P}_{\mathcal{R}^c})+1$ $\gout{1}$-exponentiations, 
two pairings and one evaluation of the polynomial $G$, and, additionally, the verifier in $\mathscr{P}^*_{\mathcal{R}^c}$ computes $e$ 
polynomial commitments to polynomials in the set $\{s_1(X), \ldots, s_e(X)\}$. 
\item For $x_1$ part of  $\mathit{state_1}$, the algorithm for computing partial inputs is defined as 
\begin{align*}
&\mathit{SNARK.PartInput}(\mathit{srs}, \mathit{state_1}, \mathcal{R}^c) \\
&\mathit{If \ } c(x_1) = 0 \\
&\ \ \ \ \mathit{Return} \\
&\mathit{Else } \\
&\ \ \ \ \mathit{Return} (\mathit{state_1}, x_1)
\end{align*}
\end{enumerate}
\end{lemma}
%\section{Hybrid Model SNARKS}
\label{sec:snarks_defs}
\noindent When proving the security of our arguments, we use an extension of some of the more commonly employed SNARK definitions which we call a ``a hybrid model SNARK''. This resembles the existing notion of SNARKs with online-offline verifiers as described in~\cite{HP_paper}, where the 
verifier computation is split into two parts: during the offline phase some computation (possibly of commitments) happens; this computation takes some public inputs as parameters and, when not performed by the verifier, it may also be performed (in part) by the prover. The online phase is the main computation performed by the verifier. In the case of our hybrid model SNARKs, however, the input to the offline counterpart described above (which we call the $\mathit{PartInput}$ algorithm) may even be the witness or 
a part of the witness for the respective relation. For our custom SNARKs, $\mathit{PartInput}$ produces part of the public input used by the verifier; 
since for our use case, $\mathit{PartInput}$ does handle a portion of the witness, this operation cannot be performed by the verifier for that relation. 
Moreover, in our instantiation, $\mathit{PartInput}$ produces computationally binding commitment schemes that are opened by the prover. Both of these properties 
are not explicitly part of our general definition for hybrid model SNARKs, but they are crucial and explicitly assumed and used 
in proving the security for our compiler's second step (see Appendix~\ref{sec_two_step_compiler}). Intuitively, our commitments 
are the counterpart CP-SNARK subcomponent of a that computes a commitment (to part of the witness) linking different CP-SNARKs together. 
We do not need such a strong property of linking SNARKs; the commitments are used in our case for the efficiency they bring to the prover/ overall system.\\

\noindent The two SNARKs we design in this work have access to a \emph{structured reference string} (srs) of the form 
$$(\{[\tau^i]_1\}_{i=0}^{d}, \{[\tau^i]_2\}_{i=0}^{1})$$ where $\tau$ is a random (and allegedly secret) value in $\mathbb{F}$ and $d$ 
is bounded by a polynomial in $\lambda$. Such an srs is universal and updatable~\cite{updatable_universal_srs_2018}. 
We introduce a generalisation of the usual SNARK definition which we call a \emph{hybrid model SNARK} inspired by online-offline SNARKs~\cite{HP_paper}. We further refine it as described below:     

\begin{definition}(Hybrid Model SNARK)
\label{dfn_snark}
A hybrid model \emph{succinct non-interactive argument of knowledge for relation $\mathcal{R}$} is a tuple of PPT algorithms 
$(\mathit{SNARK.Setup}, \mathit{SNARK.KeyGen}, \mathit{SNARK.Prove},  \mathit{SNARK.Verify},  \\ \mathit{SNARK.PartInputs})$ 
such that for implicit security parameter $\lambda$: 

\begin{itemize}
\item $\mathit{srs} \leftarrow \mathit{SNARK.Setup} (\mathit{aux_{\mathit{SNARK}}})$: a setup algorithm that on input auxiliary parameter 
$\mathit{aux_{\mathit{SNARK}}}$ from some domain $\mathcal{D}$ outputs a universal structured reference string tuple $\mathit{srs}$, 

\item $(\mathit{srs_{pk}}, \mathit{srs_{vk}}) \leftarrow \mathit{SNARK.KeyGen}(\mathit{srs}, \mathcal{R})$: a key generation algorithm that on input a
universal structured reference string $\mathit{srs}$ and an NP relation $\mathcal{R}$ outputs a \emph{proving key} and 
a \emph{verification key} pair $(\mathit{srs_{pk}}, \mathit{srs_{vk}})$,

\item $\pi \leftarrow \mathit{SNARK.Prove}(\mathit{srs_{pk}}, (x,w), \mathcal{R})$: a proof generation algorithm that on input a proving key 
$\mathit{srs_{pk}}$ and a pair $(x,w) \in \mathcal{R}$ outputs \emph{proof} $\pi$, 

\item $0/1 \leftarrow \mathit{SNARK.Verify}(\mathit{srs_{vk}}, x, \pi, \mathcal{R})$: a proof verification algorithm that on input a verification key 
$\mathit{srs_{vk}}$, an instance $x$ and a proof $\pi$ outputs a bit that signals acceptance (if output is $1$) or rejection (if output is $0$),

\item $(x_1, \mathit{state}_2) \leftarrow \mathit{SNARK.PartInputs}(\mathit{srs}, \mathit{state}_1, \mathcal{R})$: a deterministic 
public inputs generation algorithm that takes as input a universal structured reference string $\mathit{srs}$, an NP relation $\mathcal{R}$ and 
 state $\mathit{state}_1$ and outputs updated state $\mathit{state}_2$ and partial public input $x_1$,

\end{itemize}
and satisfies completeness, knowledge soundness w.r.t. $\mathit{SNARK.PartInputs}$ and succinctness as defined below:

\noindent \textbf{Perfect Completeness} holds if an honest prover will always convince an honest verifier: for all  
$(x,w) \in \mathcal{R}$ and for all $\mathit{aux_{\mathit{SNARK}}} \in \mathcal{D}$
\begin{align*}
& \mathit{Pr}[\mathit{SNARK.Verify}(\mathit{srs_{vk}}, x, \pi, \mathcal{R}) = 1 \ | \  
\mathit{srs} \leftarrow \mathit{SNARK.Setup}(\mathit{aux_{\mathit{SNARK}}}), \\ 
& (\mathit{srs_{pk}}, \mathit{srs_{vk}})\leftarrow \mathit{SNARK.KeyGen}(\mathit{srs}, \mathcal{R}), \pi \leftarrow \mathit{SNARK.Prove}(\mathit{srs_{pk}}, (x,w), \mathcal{R}) \ ] = 1.
\end{align*}

\noindent \textbf{Notation} We denote by $\mathit{State_{\mathcal{R}}}$ the set of all states $\mathit{state}_1$ 
such that given some relation $\mathcal{R}$ and any possible $\mathit{srs}$, 
for any output $x_1$ of  $\mathit{SNARK.PartInputs}(\mathit{srs}, \mathit{state}_1, \mathcal{R})$ 
with $\mathit{state_1} \in \mathit{State_{\mathcal{R}}}$, there exists $x_2$ and $w$ with $(x=(x_1, x_2), w) \in \mathcal{R}$; 
we further assume $\mathit{State_{\mathcal{R}}} \neq \emptyset$.\\
\vspace{-0.08in}

\noindent \textbf{Knowledge-soundness with respect to $\mathit{SNARK.PartInputs}$}
holds if there exists a PPT extractor $\mathcal{E}$ such that for all PPT 
adversaries $\mathcal{A}$, for all $\mathit{aux_{\mathit{SNARK}}} \in \mathcal{D}$ and for all $\mathit{state_1} \in \mathit{State_{\mathcal{R}}}$
\begin{align*}
&\mathit{Pr}[(x = (x_1, x_2), w) \in \mathcal{R} \wedge 1 \leftarrow \mathit{SNARK.Verify}(\mathit{srs_{vk}}, x =  (x_1, x_2), \pi, \mathcal{R}) \ | \\
&\mathit{srs} \leftarrow \mathit{SNARK.Setup}(\mathit{aux_{\mathit{SNARK}}}), (\mathit{srs_{pk}}, \mathit{srs_{vk}})\leftarrow  \mathit{SNARK.KeyGen}(\mathit{srs}, \mathcal{R}), \\ 
& (x_1, \mathit{state}_2) \leftarrow \mathit{SNARK.PartInput}(\mathit{srs}, \mathit{state}_1, \mathcal{R}), (x_2, \pi) \leftarrow \mathcal{A}(\mathit{srs}, \mathit{state}_2, \mathcal{R}), \\
& w \leftarrow \mathcal{E}^{\mathcal{A}}(srs,\mathcal{R})]
\end{align*}
\noindent is overwhelming in $\lambda$, where by $\mathcal{E}^{\mathcal{A}}$ we denote the extractor $\mathcal{E}$ that has access to all of 
$\mathcal{A}$'s messages during the protocol with the honest verifier. \\ 
\noindent \textbf{Succinctness} holds if the size of the proof $\pi$ is $\mathsf{poly}(\lambda)$ and $\mathit{SNARK.Verify}$ runs in time 
$\mathsf{poly}(\lambda + |x|)$. % $+ \log{|w|}$ has been removed in both cases.
\end{definition}
\vspace{-0.04in}
\noindent Firstly, if $x_1$, $\mathit{state_1}$ and $\mathit{state_2}$ are the empty strings, we obtain the standard SNARK definition.
Secondly, $\mathcal{R}$ is not a component of the vector $\mathit{aux_{\mathit{SNARK}}}$ so even if $\mathit{SNARK.Setup}$ has 
$\mathit{aux_{\mathit{SNARK}}}$ as parameter, it is universal, 
i.e., it can be used to derive proving and verification keys for circuits of any size up to a polynomial in the security parameter $\lambda$,   
independently of any specific NP relation. Thirdly, for the SNARKs we design, the size of the key used by the honest verifier is much smaller than the size of the honest prover's key. To capture this special case we made the separation clear between the two keys; however, a potential adversarial prover has access to the complete $\mathit{srs}$ key. 
Moreover, our SNARKs are secure in the $\mathit{AGM}$ model~\cite{AGM_model}, i.e., security is w.r.t. $\mathit{AGM}$ 
adversaries only and by $\mathcal{E}^{\mathcal{A}}$ we denote the 
extractor $\mathcal{E}$ that has access to all of 
$\mathcal{A}$'s messages during the protocol with the honest 
verifier including the coefficients of the linear combinations of 
group elements used by the adversary at any protocol step for outputting new group elements at the next step. Finally, the auxiliary input (i.e., $\mathit{state_1}$) is required to 
be drawn from a ``benign distribution'' or else extraction may be 
impossible~\cite{extractability_limits_1,extractability_limits_2}. \\

\noindent We did not include the notion of zero-knowledge since it is not required.
\vspace{-0.1in}
\section{Delayed Proofs for Section~\ref{sec_apk_proofs}}
\label{sec:delayed}

The following statements and proofs are taken over from~\cite{LC_paper} and included below for convenience. They are delayed from Section~\ref{sec_apk_proofs}. \\


\noindent First, we remind the reader the incomplete addition formulae for curve points in affine coordinates, over elliptic curve in short Weierstrasse form and state:\\ 

\noindent \textit{Observation 1:} \label{obs} Suppose that $\mathit{bit} \in \{0,1\}$, $(x_1,y_1)$ is a point on an elliptic curve in 
short Weierstrasse form, and, if $\mathit{bit} = 1$, so is $(x_2,y_2)$. We claim that the following equations: 
\begin{align*}
&\mathit{bit}((x_1 - x_2)^2 (x_1 + x_2 + x_3) - (y_2 - y_1)^2 ) + (1 - \mathit{bit})(y_3 - y_1)  =0 \ (\ast)\\
&\mathit{bit}((x_1 - x_2)(y_3 + y_1) - (y_2 - y_1)(x_3 - x_1)) + (1 - \mathit{bit})(x_3 - x_1)  =0 \ (\ast\ast)
\end{align*}

\noindent hold if and only if one of the following three conditions hold 

\begin{enumerate}
\item \label{cond1} $\mathit{bit}=1$ and $(x_1,y_1)\oplus(x_2,y_2)=(x_3,y_3)$ and $x_1 \neq x_2$
\item \label{cond2} $\mathit{bit}=0$ and $(x_3,y_3)=(x_1,y_1)$ 
\item  \label{cond3} $\mathit{bit}=1$ and $(x_1,y_1)=(x_2,y_2)$\footnote{Note that under condition~\ref{cond3}, $(x_3,y_3)$ 
can be any point whatsoever, maybe not even on the curve. The same holds true for $(x_2, y_2)$ under the condition~\ref{cond2}.}.
\end{enumerate}

\noindent It is easy to see that each of the conditions~\ref{cond1},\ref{cond2},\ref{cond3} above implies equations $(\ast)$ and $(\ast \ast)$.
\noindent For the implication in the opposite direction, if we assume that $(\ast)$ and $(\ast \ast)$ hold, then \\
\vspace{-0.1in}

\noindent \textit{Case a:} For $\mathit{bit}=0$, the first term of each equation $(\ast)$ and $(\ast \ast)$ vanishes, 
leaving us with $y_3-y_1=0$ and $x_3-x_1=0$ which are equivalent to condition~\ref{cond2}. \\
\vspace{-0.1in}

\noindent \textit{Case b:} For $\mathit{bit}=1$ and $x_1=x_2$, by simple substitution in $(\ast)$ and $(\ast \ast)$, 
we obtain $y_1 = y_2$, i.e., condition~\ref{cond3}.  \\
\vspace{-0.1in}

\noindent \textit{Case c:} For $\mathit{bit}=1$ and $x_1 \neq x_2$, then we can substitute
$\beta=\frac{y_2-y_1}{x_2-x_1}$ into equations $(\ast)$ and $(\ast \ast)$, leaving us with
$$x_1+x_2+x_3=\beta^2 \textrm{ and } y_3+y_1=\beta(x_3-x_1).$$
which are the usual formulae for short Weierstrass form addition of affine coordinate points when $x_1 \neq x_2$ 
so this is equivalent to Condition~\ref{cond1}. \\
\vspace{-0.1in}


\begin{test_claim} Assume that $\forall i < n-1$ such that $\mathit{bit}_i = 1$, $pk_i = (pkx_i, pky_i) \in \ginn{1}$. 
If polynomial identities $id_i(X) = 0, \forall i \in [5],$ hold over range 
$H$ and the polynomial $b(X)$ has been constructed via interpolation on $H$ such that 
$b(\omega^i) = \mathit{bit}_i, \forall i <n$ then the following four properties hold: \\
$\mathit{bit}_i \in \mathbb{B} = \{0,1\} \subset \mathbb{F}, \forall i <n \ \ \ \ \ \ (1)$, \\
$(kaccx_{0}, kaccy_{0}) = (h_x, h_y) \ \ \ \ \ \ (2) $, \\
$(kaccx_{n-1}, kaccy_{n-1}) = (h_x, h_y) \oplus (apk_x, apk_y) \ \ \ \ \ \ (3) $,  \\
$(kaccx_{i+1}, kaccy_{i+1}) =  (kaccx_{i}, kaccy_{i}) \oplus \mathit{bit}_i(pkx_{i}, pky_{i})$, $\forall i < n-1 \ \ \ \ \ \ (4) $.
%where in the last relation $\mathit{bit_i}$ should not be interpreted as a field element but as a binary bit
\label{claim:keys_affine_comm}
\end{test_claim}
\vspace{-0.08in}

\begin{proof} The properties $(1)$, $(2)$ and $(3)$ in the claim are easy to derive from the polynomial identities 
$id_3(X) =0, id_4(X )= 0, id_5(X) = 0$ holding over $H$. To prove property $(4)$, we apply 
the above \textit{Observation} 1 by noticing that if $id_1(X)$ and $id_2(X)$ hold over $H$, 
then $(\ast)$ and $(\ast \ast)$ hold with $(x_1, y_1)$ substituted by $(kaccx_i,kaccy_i)$, $(x_2, y_2)$ 
substituted by $(pkx_i, pky_i)$, $(x_3, y_3)$ substituted by $(kaccx_{i+1},kaccy_{i+1})$ and 
$\mathit{bit}$ substituted by $\mathit{bit}_i$ for $0 \leq i \leq n-2$. %, where $\mathit{bit_i}$ should not be interpreted as a field element but as binary  bit
Moreover, since $$(kaccx_{0}, kaccy_{0}) = (h_x, h_y) \in \einn \setminus \ginn{1}$$ 
and if $(pkx_i, pky_i) \in \ginn{1}$ whenever $\mathit{bit}_i = 1$, then $\forall i < n-1$ 
equations $(\ast)$ and $(\ast \ast)$ obtained after the substitution defined above are equivalent to either 
condition~\ref{cond1} or condition~\ref{cond2}, but never condition~\ref{cond3}, so the result of the sum (i.e., $(kaccx_{i+1}, kaccy_{i+1})$, $0\leq i \leq n-2$) is, 
by induction, at each step a well-defined point on the curve.% and this concludes our proof.
\end{proof}
\vspace{-0.1in}

\begin{comment}
\begin{corollary} Assume $\forall i < n-1$ 
such that $\mathit{bit}_i = 1$, $pk_i = (pkx_i, pky_i) \in \ginn{1}$. 
If the polynomial identities $id_i(X) = 0, \forall i \in [4],$ hold over range $H$ and 
$\mathit{bit_i} \in \mathbb{B}$, $\forall i < n-1$ and $b(X) = \sum_{i=0}^{n-1} \mathit{bit_i} \cdot L_i(X)$
then:  \\
$(kaccx_{0}, kaccy_{0}) = (h_x, h_y)$, \\
$(kaccx_{n-1}, kaccy_{n-1}) = (h_x, h_y) \oplus (apk_x, apk_y)$, \\
$(kaccx_{i+1}, kaccy_{i+1}) =  (kaccx_{i}, kaccy_{i}) \oplus \mathit{bit_i}(pkx_{i}, pky_{i})$, $\forall i < n-1$.
%where in the last relation $\mathit{bit_i}$ should not be interpreted as a field element but as a binary bit.
\label{corollary:keys_affine_comm}
\end{corollary}
\vspace{-0.2in}

\begin{proof}The proof follows trivially from the general result stated by Claim~\ref{claim:keys_affine_comm}. 
\end{proof}
\vspace{-0.1in}
\end{comment}

\end{document}        
