\section{O-SNARKs}
\label{sec:short_snarks_defs}
In the following, we remind the reader the definition of an O-SNARK from~\cite{O_SNARK} and we prove that 
SNARKs such as PLONK~\cite{plonk}, Marlin~\cite{marlin} or Groth16~\cite{groth16} on which a flurry of 
both under development or in production practical applications are based, fulfil this definition for a commonly occurying 
and well-defined class of oracles. In order to do that, we start with a couple of building block definitions: we give a 
reminder of algebraic algorithms (see, for example~\cite{AGM_model}) and introduce our new notion of AGM respecting oracles. \\

\noindent Let $\mathbb{G}$ be a cyclic group of prime order $p$. Informally, an algorithm $A$ is algebraic with respect to $\mathbb{G}$ if it fulfils the following requirement: 
whenever $A$ outputs a group element $h \in \mathbb{G}$, it also outputs a representation $(s_1, \ldots, s_t) \in \mathbb{Z}^{t}_p$ 
such that $h = \sum_ {i =1}^t s_i \cdot g_i$, where $(g_1,...,g_t)$ is the list of all group elements that were given to $A$ during its execution so far. \\

\noindent It is easy to see that the property above is equivalent to $A$ outputting  representation $(v_1, \ldots, v_l) \in \mathbb{Z}^{l}_p$ 
such that $h = \sum_ {i =1}^l v_i \cdot g_i$, where $(g_1,...,g_l)$ is the list of all group elements that were given to $A$ as input. \\


\noindent Let $\mathbb{O} = \{ \mathcal{O} \}_{i \in \mathcal{I}}$ be a family of oracles. We denote by $\mathcal{O} \leftarrow \mathbb{O}$ 
the process of sampling an oracle $\mathcal{O}$ from the family $\mathbb{O}$ according to some (possibly probabilistic) process.
$\mathbb{O}$ can be a random oracle family or a signing oracle corresponding to a signature scheme. For any oracle family 
$\mathbb{O}$, we define an O-SNARK $\Pi$ for $\mathbb{O}$ as follows~\cite{O_SNARK}. \\

\begin{definition}[O-SNARKs]
\label{def:osnarks}
We say $\Pi$ is a $\mathcal{Z}$-auxiliary input O-SNARK for the oracle family $\mathbb{O}$, 
if $\Pi$ satisfies the properties of completeness and 
succinctness as in Definition above {\color{red} ADD THAT} 
and the following property of adaptive proof of knowledge for $\mathbb{O}$:
Consider the following experiment for security parameter $\lambda \in \mathbb{N}$, 
time bound $T \in \mathbb{N}$, adversary $A$, extractor $E_A$, 
auxiliary input generator $\mathcal{Z}$ and oracle family $\mathbb{O}$:\\

\begin{align*}
O-AddaptivePoK&(\lambda, T, A, E_A, \mathcal{Z}, \mathbb{O}): \\
& \mathit{aux} \leftarrow \mathcal{Z}(1^{\lambda}); \mathcal{O} \leftarrow \mathbb{O}; \mathsf{crs} \leftarrow \mathsf{Gen}(1^{\lambda},T); \\
& (y,\pi) \leftarrow A^{\mathcal{O}}(\mathsf{crs},\mathit{aux}); w \leftarrow E_A(\mathsf{crs}, \mathit{aux}, \mathsf{qt}); \\
& if \ \mathsf{Ver}(\mathsf{crs}, y, \pi) = 1 \wedge (y,w) \in R \ return  \ 1 \\
& else \ return \ 0
\end{align*}
\noindent where $\mathsf{qt} = \{q_i, \mathcal{O}(q_i)\}$ is the transcript of all the oracle queries and answers made and received by $A$ during its execution. 

$\Pi$ satisfies adaptive proof of knowledge with respect to oracle family  $\mathbb{O}$ and auxiliary input from $\mathcal{Z}$ 
if for every non-uniform oracle prover $A^{\mathcal{O}}$ of sizes $s(\lambda)= \mathsf{poly}(\lambda)$ making at most 
$Q(\lambda) = \mathsf{poly}(\lambda)$ queries there exists a non-uniform extractor
$E_A$ of size $t(\lambda)=\mathsf{poly}(\lambda)$ and a negligible function $\mathsf{negl}(\lambda)$ 
such that for every time bound $T$, 
$\mathsf{Pr}[$O-AddaptivePoK($\lambda$, $T$, $A$, $E_A$, $\mathcal{Z}$, $\mathcal{O}$) $ \implies  1]\ \leq \mathsf{negl}(\lambda)$. 
\end{definition} 

\begin{definition}[AGM Respecting Oracles]
\label{def:agm_oracles}
Let $\mathbb{G}$ be a cyclic group of prime order $p$ and let $\mathcal{A}$ be the set of all algebraic adversaries with 
respect to $\mathbb{G}$. We say that a family of oracles $\mathbb{O}$ is AGM respecting relative to $\mathbb{G}$ if 
the probability of any oracle $\mathcal{O} \in \mathbb{O}$ to perform a group operation from $\mathbb{G}$ 
(i.e., output an element of $\mathbb{G}$) is $0$. 
\end{definition}

\begin{theorem}[O-SNARKS for AGM Respecting Oracles]
\label{the:when_osnarks} 
Let $\mathbb{G}$ be a cyclic group of prime order $p$ and let $\mathbb{O}$ be an AGM respecting oracle 
relative to $\mathbb{G}$ as per Definition~\ref{def:agm_oracles}. We denote by $\mathcal{Z}_{\mathbb{O}}$ the 
probability distribution of all the public parameters that define a member of $\mathbb{O}$ together with some 
polynomial number $Q$ of queries and answers to $\mathbb{O}$. More formally: 
$\mathcal{Z}_{\mathbb{O}} = (\mathit{pp}, \{\mathcal{O}(q_i), q_i\}_{i=1}^Q)$, $Q = \mathsf{poly}(\lambda)$, $(\mathit{pp}, \mathcal{O}) \leftarrow \mathbb{O}$. 
Then every $\mathcal{Z}_{\mathbb{O}}$-auxiliary input SNARK $\Pi$ secure in the AGM model with respect to $\mathbb{G}$ is an O-SNARK for $\mathbb{O}$.
\end{theorem}

\begin{proof}{\color{red} TO DO}
\end{proof}


