In the following, we construct a \emph{counting SNARK} which allows a prover to convince an efficient verifier that an alleged 
aggregated public key has been computed correctly as an aggregate of a vector of public keys for which 
two succinct commitments (to the $x$ and $y$ affine coordinates of points) are publicly known. Additionally, our counting 
SNARK ensures that the alleged aggregated public key, in turn, is a scalar product between the entire set of public keys 
committed as mentioned above and a \emph{bitvector} with one bit associated to each public key 
(necessary to signal the inclusion or omission of the respective public key w.r.t. the aggregate key).  
We finally mention how to transform the counting SNARK into a SNARK for building non-accountable light client systems. \\

\noindent To compile our desired SNARK we proceed as follows: in 
Section~\ref{sec_vt} we define vector-based conditional NP relations $\Rvt$ (i.e., counting) and we design a 
ranged polynomial protocol for this relation. The ranged polynomial protocol notion originates in~\cite{plonk}; 
we review it in Section~\ref{supplementary_poly_protocols_appendix} including a refinement introduced in~\cite{LC_paper}.  
Then we use the two-steps PLONK-based compiler introduced in~\cite{LC_paper} which is necessary to compile the counting 
ranged polynomial protocol into an O-SNARK for an NP relation based on mixed vector and pair of polynomial commitments; 
we denote this relation by $\Rvtcom$. \\

\noindent To start with, we define our vector based counting NP relation over $\mathbb{F}$, i.e., the base field of a pairing friendly 
elliptic curve $\einn$. Our SNARK prover's circuit is defined as well over $\mathbb{F}$ as the scalar field of a second pairing friendly elliptic 
curve $\eout$. The vector of public keys, which is part of the public input for $\Rvt$ and is denoted by 
$\mathbf{pk} = (\mathit{pk_0}, \ldots, \mathit{pk_{n-2}})$, is a vector of pairs with each component in $\mathbb{F}$. 
This vector has size $n-1$ (where $n$ is the order of a group $H$ defined below). For $\Rvt$ we denote 
the $n$ components bitvector by $\mathbf{bit} = (\mathit{bit_0}, \ldots, \mathit{bit_{n-1}})$ 
(meaning that each component belongs to the set $\{0,1\} \subset \mathbb{F}$). 
Each bit signals the inclusion (or exclusion) of the index-wise corresponding public keys 
into the aggregated public key $\mathit{apk}$. The $n$-th component $\mathit{bit_{n-1}}$ does not correspond to any public key, 
but, as will become clear in the following, has been included for easier design of constraints. \\ 

\vspace{-0.009in}
\noindent We denote by $H$ the multiplicative subgroup of $\mathbb{F}$ generated 
by an $n$-th root of unity $\omega$. We denote by \\ $\mathit{incl}(a_0, \ldots, a_{n-2})$ the inclusion 
predicate that checks if $(a_0, \ldots, a_{n-2}) \in \ginn{1}^{n-1}$, where $\ginn{1}$ is the first source group 
of pairing function associated with $\einn$. Moreover let $h = (\mathit{h_x}, \mathit{h_y})$ be some fixed, 
publicly known element in $\einn \setminus \ginn{1}$. We denote by $(a_x, a_y)$ the affine representation in 
$x$ and $y$ coordinates of $a \in \einn$ and by $\oplus$ the point addition in affine coordinates on the 
elliptic curve $\einn$. We denote by $\mathbb{B} = \{0,1\} \subset \mathbb{F}$. \\
\vspace{-0.15in}

\subsection{Counting Ranged Polynomial Protocol}
\label{sec_vt}

\noindent Next, we introduce the following Lagrange interpolation polynomials of degree at most $n-1 = |H|$ over cyclic group $H$: 
$b(X)$ - interpolates the bits of bitvector $\mathit{bit}$; $pkx(X)$, $pky(X)$ - interpolate all public keys' 
$x$ and $y$ coordinates, respectively; $kaccx(X)$, $kaccy(X)$ - interpolate $x$ and $y$ coordinates, respectively, 
of the iterative partial aggregate sum of the actual signing validators' public keys. We also define six polynomial identities 
$id_1(X)=0, \ldots, id_6(X)=0$ supporting the following intuition: $id_1(X)=0$, $id_2(X)=0$ over $H$ 
ensure the $x$ and, respectively, the $y$ coordinates of the iterative partial aggregate sums of actual signing validators public keys 
(up to each index $i \leq n-2$) follow formulas $(\ast)$, $(\ast\ast)$ from Observation 1 below which gives all possible cases of 
complete curve point addition when the second curve point is multiplied by a bit; $id_3(X)=0$, $id_4(X)=0$ over $H$ 
ensure first partial aggregate sum is $h$ and the total aggregate sum is $h + \mathit{apk}$; this is necessary in order to ensure the 
addition of the public keys (i.e., elliptic curve points) never falls into condition (3) defined in Observation 1 below. This recursively implies the 
partial aggregate sum at every step is a well defined curve point, hence, it is a suitable input for the next step; 
$id_5(X)=0$ over $H$ ensures $b(X)$ evaluates to bits over $H$. $id_6(X)= 0$ over $H$ ensures that the sum of bits in the 
bitvector $\mathit{bit}$ is $s+1$. \\

\noindent Intuitively and overall the identities $id_1(X)=0$ to $id_6(X)=0$ over $H$ 
ensure $\mathit{apk}$ is the aggregated public key of at least $s$ and at most $s+1$ public keys. 
Hence, we interpret $s$ as a threshold on the number of public keys included in the aggregated public key. 
Since $\mathit{bit_{n-1}}$ as the last component of the bitmask witness does not correspond to any public 
key and we have to account for the fact that $\mathit{bit_{n-1}}$ may be 
$1 \in \mathbb{F}$, relation $\Rvt$ includes the off-by-one constraint $\sum_{i=0}^{n-1} \mathit{bit_i} = s+1$.\\

\noindent \textsf{Conditional Counting Relation $\Rvt$}  
\begin{equation*}
\begin{split}
 \Rvt = & \{(\mathbf{pk} \in (\mathbb{F}^2)^{n-1}, s \in \mathbb{F}^2, \mathit{apk} \in \mathbb{F}^2; \mathbf{bit}) : 
 \mathit{apk} = \sum_{i=0}^{n-2} [\mathit{bit_i}]  \cdot \mathit{pk_i} \ | \ \mathbf{pk}  \in \ginn{1}^{n-1} \ \wedge \\ 
& \wedge \ \mathbf{bit} \in \mathbb{B}^n \ \wedge \ \sum_{i=0}^{n-1} \mathit{bit_i} = s+1\} 
\end{split}
\end{equation*}

\noindent The polynomials and polynomial identities computed and used are: \\

\noindent \textsf{Polynomials as Computed by Honest Parties} 

\begin{align*}
&\mathsf{b(X)} = \sum_{i=0}^{n-1} \mathit{bit_i} \cdot \mathsf{L_i(X)}; \mathsf{pkx(X)} =  \sum_{i=0}^{n-2} \mathit{pkx_i} \cdot \mathsf{L_i(X)}; 
\mathsf{pky(X)} =  \sum_{i=0}^{n-2} \mathit{pky_i} \cdot \mathsf{L_i(X)} \\
&\mathsf{kaccx(X)}  =  \sum_{i=0}^{n-1} \mathit{kaccx_i} \cdot \mathsf{L_i(X)}; \mathsf{kaccy(X)}  = \sum_{i=0}^{n-1} \mathit{kaccy_i} \cdot \mathsf{L_i(X)}; 
acc_{vt}(X)  = \sum_{i=0}^{n-1} acc_{vt,i} \cdot L_i(X),
\end{align*}
\noindent where $(\mathit{pkx_0}, \ldots, \mathit{pkx_{n-2}})$ 
and $(\mathit{pky_0}, \ldots, \mathit{pky_{n-2}})$ are computed such that $\forall i \in \{0, \ldots, n-2\}$, $\mathit{pk_i}$ 
is interpreted as a pair $(\mathit{pkx_i}, \mathit{pky_i})$ with its components in $\mathbb{F}$; we also have 
$(\mathit{kaccx_{0}}, \mathit{kaccy_{0}}) = (\mathit{h_x}, \mathit{h_y})$ and 
$(\mathit{kaccx_{i+1}}, \mathit{kaccy_{i+1}}) =  (\mathit{kaccx_{i}}, \mathit{kaccy_{i}}) \oplus \mathit{bit_i}(\mathit{pkx_{i}}, \mathit{pky_{i}})$, 
$\forall i < n-1$. Finally, $acc_{vt,i}$ are the components of the vector 
$(0, \mathit{bit_0}, \mathit{bit_0} + \mathit{bit_1}, \ldots, \sum_{i=0}^{n-2}\mathit{bit_i})$, $\forall i < n$. \\

\noindent \textsf{Polynomial Identities} 

\begin{align*}
& id_1(X) = (X-\omega^{n-1}) \cdot [b(X) \cdot ((kaccx(X)-pkx(X))^2 \cdot (kaccx(X)+ pkx(X) +  kaccx(\omega\cdot X)) - \\ 
& \ \ \ \ \ \ \ \ -(pky(X) - kaccy(X))^2) +  (1-b(X)) \cdot (kaccy(\omega\cdot X) - kaccy(X))]. \\
& id_2(X)  =  (X-\omega^{n-1})\cdot [b(X) \cdot ((kaccx(X) - pkx(X)) \cdot (kaccy(\omega \cdot X) + kaccy(X)) - \\
& \ \ \ \ \ \ \ \ - (pky(X) - kaccy(X)) \cdot (kaccx(\omega \cdot X) - kaccx(X))) + (1-b(X)) \cdot (kaccx(\omega \cdot X) - kaccx(X))]. \\
& id_3(X)  =  (kaccx(X) - h_x)\cdot L_0(X) + (kaccx(X) - (h\oplus apk)_{x}) \cdot L_{n-1}(X). \\ 
& id_4(X) =  (kaccy(X) - h_y)\cdot L_0(X) + (kaccy(X)  - (h\oplus apk)_{y}) \cdot L_{n-1}(X). \\
& id_5(X) =  b(X)(1-b(X)). \\
& id_6(X)  = acc_{vt}(\omega \cdot X) - acc_{vt}(X) - b(X) + (s+1) \cdot L_{n-1}(X).   \\
\end{align*}

\noindent \textsf{{$H$-ranged Polynomial Protocol for Conditional Counting Relation $\Rvt$}} \\

\noindent \textsf{Protocol $\Pvt$} \\

\noindent $\mathcal{P}_{poly}$ and $\mathcal{V}_{poly}$ know public input $s \in \mathbb{F}^2$, 
$\mathbf{pk} \in (\mathbb{F}^2)^{n-1}$ and $\mathit{apk} \in \mathbb{F}^2$ which are interpreted as per their respective domains. 

\begin{enumerate}
\item $\mathcal{V}_{poly}$ computes $pkx(X)$, $pky(X)$.
\item $\mathcal{P}_{poly}$ sends polynomials $b(X)$, $kaccx(X)$, $kaccy(X)$, $acc_{vt}(X)$ to $\mathcal{I}$. 
\item $\mathcal{V}_{poly}$ asks $\mathcal{I}$ to check whether the following polynomial relations hold over range $H$:
$$id_i(X) = 0, \forall i \in [6].$$ 
\item $\mathcal{V}_{poly}$ accepts if all of $\mathcal{I}$'s checks verify. 
\end{enumerate}

\noindent We show that protocol $\Pvt$ is an $H$-ranged polynomial protocol for conditional relation 
$\Rvt$. 

\begin{lemma} $\Pvt$ as described above is an $H$-ranged polynomial protocol for conditional relation $\Rvt$.
\end{lemma}

\begin{proof}
It is easy to see that perfect completeness holds. Indeed, if $(\mathbf{bit},\mathbf{pk}, \mathit{apk}) \in \Rvt$ holds, 
meaning that $\mathbf{bit} \in \mathbb{B}^n$ and $\mathbf{pk} \in \ginn{1}^{n-1}$ and $\mathit{apk} = \sum_{i=0}^{n-2} [\mathit{bit_i}] \cdot \mathit{pk_i}$ and 
$\sum_{i=0}^{n-1} \mathit{bit_i} = s+1$ hold, then it is easy to see that the honest prover $\mathcal{P}_{poly}$ in $\Pvt$ will convince the honest 
verifier $\mathcal{V}_{poly}$ in $\Pvt$ to accept with probability $1$. \\
Regarding knowledge-soundness, if the verifier $\mathcal{V}_{poly}$ in $\Pvt$ accepts, 
then we construct the extractor $\mathcal{E}$ in the following way. Using the polynomial $b(X)$ which 
was part of the messages from $\mathcal{P}_{poly}$ to $\mathcal{I}$ and evaluating it at the elements of the set 
$H$, $\mathcal{E}$ obtains evaluation vector $\mathbf{bit} = (b(1), \ldots, b(\omega^{n-1}))$ which, 
in the following, we denote as $(\mathit{bit}_0, \ldots, \mathit{bit}_{n-1}) \in \mathbb{F}^n$.\\ 
\noindent Next, we show that if $(\mathit{pk_0}, \ldots, \mathit{pk_{n-2}}) \in \ginn{1}^{n-1}$ holds and the 
verifier in $\Pvt$ accepts, then 
$$((\mathit{pk_0}, \ldots, \mathit{pk_{n-2}}), s, \mathit{apk}, (\mathit{bit_0}, \ldots, \mathit{bit_{n-1}})) \in \Rvt,$$ 
which is equivalent to proving that $\mathit{apk} = \sum_{i=0}^{n-2} [\mathit{bit_i}]  \cdot \mathit{pk_i}$ and 
$\mathbf{bit} \in \mathbb{B}^n$ and  $\sum_{i=0}^{n-1} \mathit{bit_i} = s+1$.
\noindent First, since $id_6(X) = 0$ holds over $H$, we can expand that as follows:
\begin{align*}
acc_{vt,1} &= acc_{vt,0} + \mathit{bit}_{0} \\
acc_{vt, 2} &= acc_{vt,1} + \mathit{bit}_{1} \\
\ldots \\
acc_{vt,n-1} &= acc_{vt,n-2} + \mathit{bit}_{n-2} \\
acc_{vt,0} &= acc_{vt,n-1} + \mathit{bit}_{n-1} - (s+1).
\end{align*}
\noindent By summing together the LHS and, respectively, the RHS of the equalities above and 
reducing the equal terms, we obtain $s+1 = \sum_{i=0}^{n-1}\mathit{bit}_i$. \\ 
Second, since it holds over $H$ that $id_i(X) = 0$, $\forall i \in [5]$ and $b(\omega^i) = \mathit{bit_i}, \forall i<n$ (by the definition 
of $\mathcal{E}$), the properties concluded in Claim~\ref{claim:keys_affine_comm} hold. Combining the two proof steps above, we obtain the desired conclusion.
\end{proof}


