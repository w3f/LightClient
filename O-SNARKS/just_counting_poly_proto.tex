%\subsection{Counting Ranged Polynomial Protocol}
\label{sec_vt}
\noindent In the following relation, $\mathit{apk}$ is the aggregated public key of at least $s$ and at most $s+1$ public keys. 
Hence we interpret $s$ as a threshold on the number of public keys included in the aggregated public key. Since $\mathit{bit_{n-1}}$ as the last component of the 
bitmask witness does not correspond to any public key and we have to account for the fact that $\mathit{bit_{n-1}}$ may be 
$1 \in \mathbb{F}$, relation $\Rvt$ includes the off-by-one constraint $\sum_{i=0}^{n-1} \mathit{bit_i} = s+1$.\\

\noindent \textsf{Conditional Counting Relation $\Rvt$}  
\begin{equation*}
\begin{split}
 \Rvt = & \{(\mathbf{pk} \in (\mathbb{F}^2)^{n-1}, s \in \mathbb{F}^2, \mathit{apk} \in \mathbb{F}^2; \mathbf{bit}) : 
 \mathit{apk} = \sum_{i=0}^{n-2} [\mathit{bit_i}]  \cdot \mathit{pk_i} \ | \ \mathbf{pk}  \in \ginn{1}^{n-1} \ \wedge \\ 
& \wedge \ \mathbf{bit} \in \mathbb{B}^n \ \wedge \ \sum_{i=0}^{n-1} \mathit{bit_i} = s+1\} 
\end{split}
\end{equation*}

\noindent The new polynomials and polynomial identities required in this section are: \\

\noindent \textsf{New Polynomial as Computed by Honest Parties} 
\begin{align*}
acc_{vt}(X) & = \sum_{i=0}^{n-1} acc_{vt,i} \cdot L_i(X),
\end{align*}
where $acc_{vt,i}$ are the components of the vector $(0, \mathit{bit_0}, \mathit{bit_0} + \mathit{bit_1}, \ldots, \sum_{i=0}^{n-2}\mathit{bit_i})$, $\forall i < n$. \\

\noindent \textsf{New Polynomial Identities} 
\begin{align*}
id_8(X) & = acc_{vt}(\omega \cdot X) - acc_{vt}(X) - b(X) + (s+1) \cdot L_{n-1}(X),   \\
\end{align*}

\noindent \textsf{{$H$-ranged Polynomial Protocol for Conditional Counting Relation $\Rvt$}} \\

\noindent \textsf{Protocol $\Pvt$} \\

\noindent $\mathcal{P}_{poly}$ and $\mathcal{V}_{poly}$ know public input $s \in \mathbb{F}^2$, 
$\mathbf{pk} \in (\mathbb{F}^2)^{n-1}$ and $\mathit{apk} \in \mathbb{F}^2$ which are interpreted as per their respective domains. 

\begin{enumerate}
\item $\mathcal{V}_{poly}$ computes $pkx(X)$, $pky(X)$.
\item $\mathcal{P}_{poly}$ sends polynomials $b(X)$, $kaccx(X)$, $kaccy(X)$, $acc_{vt}(X)$ to $\mathcal{I}$. 
\item $\mathcal{V}_{poly}$ asks $\mathcal{I}$ to check whether the following polynomial relations hold over range $H$:
$$id_i(X) = 0, \forall i \in [5] \textit{ and } id_8(X) = 0.$$ 
\item $\mathcal{V}_{poly}$ accepts if all of $\mathcal{I}$'s checks verify. 
\end{enumerate}

\noindent We show that protocol $\Pvt$ is an $H$-ranged polynomial protocol for conditional relation 
$\Rvt$. 

\begin{lemma} $\Pvt$ as described above is an $H$-ranged polynomial protocol for conditional relation $\Rvt$.
\end{lemma}

\begin{proof}
It is easy to see that perfect completeness holds. Indeed, if $(\mathbf{bit},\mathbf{pk}, \mathit{apk}) \in \Rvt$ holds, 
meaning that $\mathbf{bit} \in \mathbb{B}^n$ and $\mathbf{pk} \in \ginn{1}^{n-1}$ and $\mathit{apk} = \sum_{i=0}^{n-2} [\mathit{bit_i}] \cdot \mathit{pk_i}$ and 
$\sum_{i=0}^{n-1} \mathit{bit_i} = s+1$ hold, then it is easy to see that the honest prover $\mathcal{P}_{poly}$ in $\Pvt$ will convince the honest 
verifier $\mathcal{V}_{poly}$ in $\Pvt$ to accept with probability $1$. \\
Regarding knowledge-soundness, if the verifier $\mathcal{V}_{poly}$ in $\Pvt$ accepts, 
then we construct the extractor $\mathcal{E}$ in the following way. Using the polynomial $b(X)$ which 
was part of the messages from $\mathcal{P}_{poly}$ to $\mathcal{I}$ and evaluating it at the elements of the set 
$H$, $\mathcal{E}$ obtains evaluation vector $\mathbf{bit} = (b(1), \ldots, b(\omega^{n-1}))$ which, 
in the following, we denote as $(\mathit{bit}_0, \ldots, \mathit{bit}_{n-1}) \in \mathbb{F}^n$.\\ 
\noindent Next, we show that if $(\mathit{pk_0}, \ldots, \mathit{pk_{n-2}}) \in \ginn{1}^{n-1}$ holds and the 
verifier in $\Pvt$ accepts, then 
$$((\mathit{pk_0}, \ldots, \mathit{pk_{n-2}}), s, \mathit{apk}, (\mathit{bit_0}, \ldots, \mathit{bit_{n-1}})) \in \Rvt,$$ 
which is equivalent to proving that $\mathit{apk} = \sum_{i=0}^{n-2} [\mathit{bit_i}]  \cdot \mathit{pk_i}$ and 
$\mathbf{bit} \in \mathbb{B}^n$ and  $\sum_{i=0}^{n-1} \mathit{bit_i} = s+1$.
\noindent First, since $id_8(X) = 0$ holds over $H$, we can expand that as follows:
\begin{align*}
acc_{vt,1} &= acc_{vt,0} + \mathit{bit}_{0} \\
acc_{vt, 2} &= acc_{vt,1} + \mathit{bit}_{1} \\
\ldots \\
acc_{vt,n-1} &= acc_{vt,n-2} + \mathit{bit}_{n-2} \\
acc_{vt,0} &= acc_{vt,n-1} + \mathit{bit}_{n-1} - (s+1).
\end{align*}
\noindent By summing together the LHS and, respectively, the RHS of the equalities above and 
reducing the equal terms, we obtain $s+1 = \sum_{i=0}^{n-1}\mathit{bit}_i$. \\ 
Second, since it holds over $H$ that $id_i(X) = 0$, $\forall i \in [5]$ and $b(\omega^i) = \mathit{bit_i}, \forall i<n$ (by the definition 
of $\mathcal{E}$), the properties concluded in Claim~\ref{claim:keys_affine_comm} hold. Combining the two proof steps above, we obtain the desired conclusion.
\end{proof}
