\documentclass{article}
\usepackage{spconf,amsmath,graphicx}
\usepackage{lipsum}

\usepackage{url, xspace, graphicx}
\usepackage{amsmath,amsthm, amsfonts}
\usepackage{hyperref, cryptocode, framed}
\usepackage{booktabs, array, paralist}
\usepackage{verbatim}
\usepackage{subfig}
\usepackage{mathrsfs}
\usepackage[normalem]{ulem}
\usepackage{soul}


\newcommand{\prove}{\ensuremath{\mathsf{Prove}}\xspace}
\newcommand{\group}{\ensuremath{\mathsf{group}}\xspace}
\newcommand{\ck}{\ensuremath{\mathsf{ck}}\xspace}
\newcommand{\key}[1]{\ensuremath{\mathsf{ck}_{#1}}\xspace}
\newcommand{\konefirst}{\ensuremath{\key{1,[:m']}}\xspace}
\newcommand{\konesecond}{\ensuremath{\key{1,[m':]}}\xspace}
\newcommand{\ktwofirst}{\ensuremath{\key{2,[:m']}}\xspace}
\newcommand{\ktwosecond}{\ensuremath{\key{2,[m':]}}\xspace}
\newcommand{\avec}{\ensuremath{\mathbf{\mathsf{a}}}\xspace}
\newcommand{\vvec}{\ensuremath{\mathbf{v}}\xspace}

\newcommand{\Avec}{\ensuremath{\mathbf{A}}\xspace}

\newcommand{\bvec}{\ensuremath{\mathbf{\mathsf{b}}}\xspace}
\newcommand{\zL}{\ensuremath{\mathit{z_L}}\xspace}
\newcommand{\zR}{\ensuremath{\mathit{z_R}}\xspace}

\newcommand{\afirst}{\ensuremath{\avec_{[:m']}}\xspace}
\newcommand{\asecond}{\ensuremath{\avec_{[m':]}}\xspace}
\newcommand{\Afirst}{\ensuremath{\Avec_{[:m']}}\xspace}
\newcommand{\Asecond}{\ensuremath{\Avec_{[m':]}}\xspace}

\newcommand{\vfirst}{\ensuremath{\vvec_{[:m']}}\xspace}
\newcommand{\vsecond}{\ensuremath{\vvec_{[m':]}}\xspace}
\newcommand{\bfirst}{\ensuremath{\bvec_{[:m']}}\xspace}
\newcommand{\bsecond}{\ensuremath{\bvec_{[m':]}}\xspace}

\newcommand{\CL}{\ensuremath{\mathit{C_L}}\xspace}
\newcommand{\CR}{\ensuremath{\mathit{C_R}}\xspace}
\newcommand{\TL}{\ensuremath{\mathit{T_L}}\xspace}
\newcommand{\TR}{\ensuremath{\mathit{T_R}}\xspace}

\newcommand{\negl}{\ensuremath{\mathit{negl}}\xspace}
\newcommand{\einn}{\ensuremath{\mathit{E}_{\mathit{inn}}}\xspace}
\newcommand{\eout}{\ensuremath{\mathit{E}_{\mathit{out}}}\xspace}
\newcommand{\ginn}[1]{\ensuremath{\mathbb{G}_{\mathit{#1,inn}}}\xspace}
\newcommand{\gout}[1]{\ensuremath{\mathbb{G}_{\mathit{#1,out}}}\xspace}
\newcommand{\gtinn}{\ensuremath{\mathbb{G}_{\mathit{T,inn}}}\xspace}
\newcommand{\gtout}{\ensuremath{\mathbb{G}_{\mathit{T,out}}}\xspace}
\newcommand{\sginn}[1]{\ensuremath{g_{\mathit{#1,inn}}}\xspace}
\newcommand{\sgout}[1]{\ensuremath{g_{\mathit{#1,out}}}\xspace}
\newcommand{\sgtinn}{\ensuremath{g_{\mathit{T,inn}}}\xspace}
\newcommand{\sgtout}{\ensuremath{g_{\mathit{T,out}}}\xspace}

\newcommand{\indexoneinn}{\ensuremath{\mathit{1,inn}}\xspace}
\newcommand{\indextwoinn}{\ensuremath{\mathit{2,inn}}\xspace}

\newcommand{\indexoneout}{\ensuremath{\mathit{1,out}}\xspace}
\newcommand{\indextwoout}{\ensuremath{\mathit{2,out}}\xspace}


\newcommand{\epinn}{\ensuremath{\mathit{e}_{\mathit{inn}}}\xspace}
\newcommand{\epout}{\ensuremath{\mathit{e}_{\mathit{out}}}\xspace}
\newcommand{\block}{\ensuremath{\mathsf{block}}\xspace}
\newcommand{\LCseed}{\ensuremath{\mathit{LC.seed}}\xspace}
\newcommand{\bgamma}{\boldsymbol{\gamma}}
\newcommand{\bsigma}{\boldsymbol{\sigma}}
\newcommand{\GoneBLS}{\mathbb{G}_{1, \mathit{BLS}}\xspace}
\newcommand{\GtwoBLS}{\mathbb{G}_{2, \mathit{BLS}}\xspace}
\newcommand{\GTBLS}{\mathbb{G}_{T, \mathit{BLS}}\xspace}

\newcommand{\goneBLS}{\mathit{g}_{1, \mathit{BLS}}\xspace}
\newcommand{\gtwoBLS}{\mathit{g}_{2, \mathit{BLS}}\xspace}
\newcommand{\gTBLS}{\mathit{g}_{T, \mathit{BLS}}\xspace}

\newcommand{\eBLS}{\mathit{e}_{\mathit{BLS}}\xspace}
\newcommand{\HBLS}{\mathit{H}_{\mathit{BLS}}\xspace}
\newcommand{\HPoP}{\mathit{H}_{\mathit{PoP}}\xspace}
\newcommand{\Hinn}{\mathit{H}_{\mathit{inn}}\xspace}
\newcommand{\piinn}{\pi_{\mathit{inn}}\xspace}
\newcommand{\PoP}{\mathit{inn}\xspace}

\newcommand{\Rla}{\mathcal{R}^{\mathit{incl}}_{\mathsf{ba}}\xspace}
\newcommand{\Ra}{\mathcal{R}^{\mathit{incl}}_{\mathsf{pa}}\xspace}
\newcommand{\Rvt}{\mathcal{R}^{\mathit{incl}}_{\mathsf{c}}\xspace}

\newcommand{\Rlacom}{\mathcal{R}^{\mathit{incl}}_{\mathsf{ba},\mathit{com}}\xspace}
\newcommand{\Racom}{\mathcal{R}^{\mathit{incl}}_{\mathsf{pa},\mathit{com}}\xspace}
\newcommand{\Rvtcom}{\mathcal{R}^{\mathit{incl}}_{\mathsf{c},\mathit{com}}\xspace}


\newcommand{\Pla}{\mathscr{P}_{\mathsf{ba}}\xspace}
\newcommand{\Pa}{\mathscr{P}_{\mathsf{pa}}\xspace}
\newcommand{\Pvt}{\mathscr{P}_{\mathsf{c}}\xspace}

\newcommand{\Plastar}{\mathscr{P}_{\mathsf{ba}}^{\ast}\xspace}
\newcommand{\Pastar}{\mathscr{P}_{\mathsf{pa}}^{\ast}\xspace}
\newcommand{\Pvtstar}{\mathscr{P}_{\mathsf{c}}^{\ast}\xspace}

\newcommand{\Plah}{\mathscr{P}_{\mathsf{ba}}^{h}\xspace}
\newcommand{\Pah}{\mathscr{P}_{\mathsf{pa}}^{h}\xspace}
\newcommand{\Pvth}{\mathscr{P}_{\mathsf{c}}^{h}\xspace}
\newcommand{\kate}{\ensuremath{\mathsf{KZG}}\xspace}
\newcommand{\ewnp}{e.w.n.p.\ }

\newtheorem{claim}[theorem]{Claim}
\newtheorem{remark}[theorem]{Remark}
\newtheorem{construction}[theorem]{Instantiation}
\newtheorem{assumption}[theorem]{Assumption}
\newtheorem{test_claim}[theorem]{Claim}



\title{Accountable Light Client Systems for PoS Blockchains}

%\name{Author Name\thanks{Author's Email}}
%\address{Author Affiliation}

\begin{document}
\maketitle

\begin{abstract} A major challenge for blockchain interoperability is having an on-chain light client protocol that is both efficient and secure.
We present a protocol that provides short proofs about the state of a decentralised consensus protocol while being able to detect misbehaving parties.
To do this naively, a verifier would need to maintain an updated list of all participants' public keys which makes the corresponding proofs long.
Existing solutions either are not able to detect misbehaving parties (i.e. lack accountability) or are not efficient. We define and design a committee key scheme with short proofs that does not include any of the individual participants' public keys in plain which makes it very efficient. Our committee key scheme, in turn, uses a custom designed SNARK which has a fast prover time. Our committee key scheme can be used in an accountable light client system as the main cryptographic core for building bridges between proof-of-stake blockchains. By allowing a large number of participants, our scheme allows decentralization and interoperability without compromise. Finally, we implement a prototype of our custom SNARK for which we provide benchmarks.
\end{abstract}

\section{Introduction} \label{sec_intro}
Blockchain systems rely on consensus among a number of participants, where the size of this number is important for decentralisation and the foundation of blockchain security. To know that a transaction is valid, one needs to follow the consensus of the blockchain. However, following consensus can become expensive in 
terms of bandwidth, storage and computation. Depending on the consensus type, these challenges can be aggravated when the size of participants' set becomes bigger or when the participants' set changes frequently. Light clients (such as SPV clients in Bitcoin \cite{nakamoto2008bitcoin} or inter-blockchain bridge components that support interoperability) are designed to allow resource constrained users to follow consensus of a blockchain with 
minimal cost. We are interested in blockchains that use Byzantine agreement type consensus protocols, particularly proof of stake systems 
like Polkadot~\cite{polkadot}, Ethereum~\cite{ethereum}
%~\cite{eth2} 
or many other systems~\cite{cosmos,tendermint_paper, celo}. These protocols 
may have a large number of consensus participants, from 1000s to 100000s, and in such PoS protocols, the set of participants often changes regularly. \\

\vspace{-0.2cm}
\noindent Following the consensus protocols in the examples above entails proving that a large subset of a designated set of participants, 
which are called validators, signed the same message (e.g., a block header). Existing approaches have limiting shortcomings as follows:
1) verifying all signatures which has a large communication overhead for large validator sets;
2) verifying a single aggregatable signature, by computing an aggregate public key from the signer's public keys, has the shortcoming that any verifier still needs to know the entire list of public keys and this, again, has expensive communication if the list changes frequently; 
3) verifying a threshold signature which has two shortcomings: first, such a signature does not reveal the set of signers impacting the 
security of PoS systems; second, it requires an interactive setup which becomes expensive if the validator set is large or changes frequently.

\vspace{-0.2cm}
\paragraph{Our Approach: Committee Key Schemes} We introduce a committee key scheme which allows to succinctly prove that a subset of signers 
signed a message using a commitment to a list of all the signers' public keys. Our primitive is an extension of an aggregatable signature scheme and 
it allows us to prove the desired statement, in turn, by proving the correctness of an aggregate public key for the subset of signers. 
In more detail, the committee key scheme defines a committee key which is a commitment to all the signers' public keys. It generates a succinct proof that a particular subset of the list of public keys signed a message. The proof can be verified using the committee key. 
Because of the way the aggregatable signature scheme works, we need to specify the subset of signers; for this purpose we use a bitvector. 
More precisely, if the owner of the $m$th public key in the list of public keys signed the message, then the $m$th bit of this bitvector is $1$ otherwise is $0$.
Using the committee key, the proof and the bitvector, a light client can verify that the corresponding subset of validators to whom the 
public keys belong (as per our use case) signed the message. Although the bitvector has length proportional to the number of validators, it is still orders of magnitude 
more succinct than giving all the public keys or signatures. Public keys or signatures are usually 100s of bits long and as a result this scheme reduces the amount of 
data required by a factor of 100 times or more. We could instantiate our committee key scheme using any universal SNARK scheme and suitable commitment scheme. 
However, to avoid long prover times for large validator sets, we use optimised custom SNARKs. We have implemented this scheme 
(Section~\ref{sec_implementation}) and it gives fast enough proving times for the use cases we consider: a prover with 
commodity hardware can generate these custom SNARK proofs in real time, i.e., as fast as the consensus generates instances of this problem.

\vspace{-0.2cm}
\paragraph{Application: Accountable Light Client Systems} %To understand when and how our scheme described above is useful and compare it to other approaches, we return to what a light client is used for. 
Light clients allow resource constrained devices such as browsers or phones to follow a decentralised consensus protocols. A blockchain is also resource constrained and hence could benefit from a light client system. In this case a light client verifier (e.g, smart contracts on 
Ethereum) allows building trustless bridges protocols between blockchains. % \cite{BridgeSOK}. 
Currently, computation and storage 
costs on existing blockchains are much higher than those in a browser on a modern phone. If such a bridge is responsible for securing assets with high total value, then the corresponding light client system which defines such a light client verifier must be secure as well as efficient. Using the primitives and techniques described in this work, one can design a light client system with the following properties: accountability, asynchronous safety and incrementability reviewed below.  \vspace{-0.2cm}
%\begin{itemize}
%\item 
\noindent\paragraph{Accountability}Our light client system is accountable, i.e., if the light client verifier is misled and the transcript of its communication is given to the network then one can identify a large 
number (e.g., 1/3) of misbehaving consensus participants (e.g., validators in our case). Identifying misbehaving consensus participants 
is challenging in the light client system context when we want to send minimal data to the light client verifier. However, 
identifying misbehaviour is necessary for any proof of stake protocols including Polkadot and Ethereum whose security relies on identifying and punishing misbehaving consensus participants. 
%\item 
\vspace{-0.2cm}
\noindent\paragraph{Asynchronous Safety}Our light client system has asynchronous safety i.e., under the consensus' honesty assumptions, our light client verifier cannot be misled even if it has a restricted view of the 
network, e.g., only connecting to one node, which may be malicious. This is because our light client system inherits the property of asynchronous safety from the Byzantine agreement 
protocol of the blockchain. Such light client systems would not be possible for consensus based on longest chains.  
%\item 
\vspace{-0.2cm}
\noindent\paragraph{Succinctness}Our light client system is incremental - i.e its succinct state is incrementally updated - it is optimised to make these updates efficiently, which is particularly relevant for the bridge application, as opposed to trying to optimise verifying consensus decisions from the blockchain genesis.
%\end{itemize}
\vspace{-0.2cm}
\subsection{Impact on decentralisation} For a blockchain network, having  a large number of validators  contributes greatly to better decentralization. This leads to better security both in terms of less point of physical failure and being able to distribute control over consensus which makes collusion harder. Some protocols have restricted their validator numbers to make light clients or bridges more efficient, e.g., by being able to run a DKG for threshold signatures (e.g., Dfinity~\cite{dfinity}) or obtaining Byzantine agreement with all validators on every block (e.g., Cosmos~\cite{tendermint_paper}). More efficient light clients for blockchains with large validator sets offer both decentralisation and interoperability (bridging) without compromise.

\vspace{-0.25cm}
\subsection{Relevance to Bridge Security} 
%\vspace{-0.2cm}
\noindent In this section we review the impact of our scheme on bridge security. Blockchain bridges are protocols that allow value transfer between blockchains. Bridges have frequently been the target of attacks. We note that \$1.2 billion has been stolen in attacks on insecure bridges during first 8 months of 2022 alone ~\cite{elliptic_harmony,elliptic_nomad}. Of the top 10 crypto thefts of all times, \$1.6bn out of \$3.4bn come from bridge attacks \cite{elliptic_nomad}. These confirm that bridges have frequently been a weaker point, compared to the security of the blockchains themselves and they carry a lot of economical value.

\noindent An ideal bridge would be as secure as the least secure of the two blockchains. The most secure bridges use in-chain light client systems, e.g., Cosmos IBC protocol~\cite{IBC_paper}, to achieve this. Each bridged chain follow the other chain's consensus on-chain. To simplify, we will consider an on-chain light client of chain B on chain A, although B will also have the same for A. If B's consensus and the on-chain logic of A are secure, then adversary cannot convince the logic of A that B decided some event that B's clients do not agree as decided. This translates to the adversary for example not being able to create value on A without having locked any value on B.

\noindent A main reason why bridges might not use this approach is efficiency. Smart contracts and other on-chain logic is an extremely resource constrained environment compared to browsers or phones that light clients might target. One approach for efficiency is to design B's consensus so that the light clients are cheaper, for example by reducing the validator numbers. Cosmos chains currently have 33-175 validators~\cite{CosmosValNYX},\cite{CosmosValHUB}. Many chains have many more, e.g., Ethereum's hundreds of thousands of validators, for more decentralisation and security. Alternatively, the light client can use threshold signature may be used however that means not having the same accountability guarantees and also limits validator numbers in practice, both discussed elsewhere. 

% Notaries are obviously bad but lots of money has been stolen from such bridges [TODO: Expand this].

\noindent Another approach to reducing on-chain complexity is optimism. Entities make a claim on chain A that something happened on chain B and this is accepted if no entity makes an on-chain challenge within a certain time, claiming that this is incorrect and triggering a more expensive procedure. A bridge that uses this approach in Optics for bridging Celo to other chains~\cite{CeloOptics}. A less extreme example of this approach is NEAR's Rainbow bridge~\cite{NEARrainbowB}, where signatures are stored but not checked unless the correctness of a signature is challenged. The optimism approach relies on the censorship resistance of blockchain for security. In practice, blockchains may be censored for a period of time by an attacker with enough resources. An example of this was the result of the first round of Fomo3D on Ethereum~\cite{Fomo3DPM}, a smart contract that would pay a jackpot, a large amount (in the end 10,469 ETH), to the last user to pay the contract when no user does so for 30 seconds. The jackpot grew to such a large amount that it was worth a user buying up all the block space for 30 seconds~\cite{Fomo3DPM}. For a claim and challenge protocol, the challenge is itself quite computationally expensive, so it may be sufficient to increase the cost of computation, the gas price on Ethereum, to make such a challenge unprofitable. Security against this attack requires a large reward for challenges or a long challenge period. For example the rainbow bridge has an 8 hour challenge period~\cite{RainbowBridgeFAQ}. Long challenge periods would mean that bridge operations take a long time with consequences for usability.

\noindent Stakers in proof of stake protocol have an incentive for the chain (chain A) using that protocol to keep working, however they may not have stake in a chain (chain B) bridged to their chain. As a result, they may have no particular incentive in the correct functioning of a light client of chain A on chain B and so not to mislead the light client. In the case when the protocol of chain A has slashing, if an accountable light client on chain B is misled, one can prove to chain A, using information that is publicly available on chain B, of validator of chain A misbehaving in a way that will result in those validators being slashed on chain A. This gives the bridge similar economic security to chain A itself.

Protocols Cardano and Algorand
\vspace{-0.08in}
\subsection{Applicability of Our Scheme}

\noindent Our scheme is applicable to proof of stake blockchains where if something is decided by the chain, then a message is signed by some threshold fraction of a validator set, defined as a set of nodes or their public keys, which changes at well-defined times, those changes being signed by an appropriate threshold of the existing set. As mentioned such chains as Polkadot, the many Cosmos chains, or Ethereum fit this model. Our scheme is not applicable to chains using proof of work or many other proof of X schemes. Nor proof of stake protocols when only random validators or random subsets of validators decide something and the whole set never votes, such as protocols using the longest chain rule without a finality gadget.

\noindent Our scheme might well require a hard fork to be applied to many blockchains, especially those that have not implemented the required cryptography. It should be easily implementable for chains that use BLS signatures for consensus but those using signatures that do not support aggregation (e.g., the many using Ed25519), would need to use SNARKs with much slower prover time (e.g., zkBridge~\cite{zkBridge} for Ed25519)). To naively implement our scheme, we would also want validators to compute and sign the commitment to the next set. We note however that this is not strictly necessary, as the commitment could be computed on chain, maybe in a smart contract, as long as light client proof of the result of this computation can be constructed. This would result in longer proofs that cover validator set changes. For blockchains with expensive on-chain computation, native code support for the cryptography we use e.g., with precompiles for smart contracts might be required. It is planned to make the required changes to Polkadot and implement this scheme for it. We discuss in detail what would be required for a light client of Ethereum in Appendix Section~\ref{sec:ethereum}.
\vspace{-0.1in}
%The following paragraph should be commented out and changed in case of a conference submission.
\paragraph{Structure} The paper is organised as follows. 
In Section~\ref{sec:sketch}, we sketch our proposed protocols and compare 
them to existing work. In Section~\ref{sec_prelims}, we give cryptographic 
preliminaries necessary for later sections. In Section~\ref{sec_apk_proofs}
we describe our custom SNARKs and our committee key scheme. In 
Section~\ref{sec_implementation} we give benchmarks for our custom SNARKs 
implementations. We conclude in Section~\ref{conclusions}. Our paper includes 
an extensive appendix for more details.% as follows. 
%\vspace{-0.01in}
\vspace{-0.25cm}
\section{Our Solution} 
\label{sec:sketch}
\vspace{-0.2cm}

In this section we present a sketch of our solution for both the committee key scheme and the accountable light client system, 
then describe the technical challenges and contributions and finish with an overview of related work.

\vspace{-0.3cm}
\subsection{Sketch of Committee Key Scheme}
\label{sec:lcsketch}
\vspace{-0.1cm}

\noindent Suppose that a prover wants to prove to a verifier that a subset $S$ of some set $T$ of signers {\color{red} with equal stakes} have signed a message. 
One obvious approach would be using BLS aggregatable signatures with the following steps:

\begin{itemize}
\item[a.] Verifier knows all public keys $\{\mathit{pk}_i\}_{i \in T}$ of signers.% in $T$.

\item[b.] Prover sends the verifier an aggregatable signature $\sigma$ and a representation of the subset $S$.

\item[c.] Verifier computes the aggregate public key $\mathit{apk}=\sum_{i \in S} \mathit{pk}_i$ of the public keys of signers in $S$. 
Then it verifies the aggregatable signature $\sigma$ for the aggregate public key $\mathit{apk}$ and it accepts if the verification succeeds.
\end{itemize}

\noindent However, we can represent a subset $S$ of a list of signers compactly using a bitvector $b$: 
the $i$th signer in the list is in $S$ if and only if the $i$th bit of $b$ is $1$. Our committee key scheme describes an alternative approach:

\begin{itemize}
\item[a'.]\label{a'} Verifier knows a commitment $C$ to the list of public keys $(pk_i)_{i \in T}$.

\item[b'.]\label{b'} Prover sends the verifier an aggregatable signature $\sigma$, a bitvector $b$ representing $S$, an aggregate public key 
$\mathit{apk}$ and $\pi$, a succinct proof that $\mathit{apk}=\sum_i b_i \mathit{pk}_i$ i.e., 
that $\mathit{apk}$ is the aggregate public key for the subset of signers in $S$ given by the bitvector $b$; all of the public keys in $S$ are a subset 
of the list of public keys committed to using $C$.

\item[c'.] The verifier using $C$, $\mathit{apk}$ and the bitvector $b$ checks if $\pi$ is valid. 
It then verifies $\sigma$ against $\mathit{apk}$ and accepts if both steps succeed.
\end{itemize}

\noindent With the above committee key scheme, if $C$ and $\pi$ are constant size, 
the communication cost becomes $O(1)+|T|$ bits instead of $|T|$ public keys. {\color{red} So far we have implicitly assumed validators have equal stakes. 
One can generalise our alternative approach introduced above to validators with unequal stakes by including at~\ref{a'}, a'.,  a commitment to all stakes 
and to~\ref{b'}, b'., a claimed total signing stake that can be proved via a scalar product between stakes of the signing validators and 
the bitvector. Moreover, $\mathit{apk}$ is appropriately replaced by the scalar product between signing validators' stakes and their 
respective public keys. The bitvector cannot be removed as it is needed for ensuring accountability of our light client system.}

\subsection{From CKS to Accountable Light Client}
\label{sec_intro_committee}

Below we sketch how a light client verifier uses our committee key scheme. Suppose that a light client verifier wants to know some information $\mathit{info}_n$ about the state of a blockchain at block number $n$ without having to download the entire blockchain. Another entity, a full node, who knows all the data of the blockchain and is following the consensus, should be able to convince the light client verifier using a computational proof that $info_n$ was indeed decided.

We assume that $\mathit{info}_n$ can be proven from a commitment to the state at block number $n$ that is signed by validators, here we assume that this commitment is a block hash $H_n$. To convince the light client verifier that $H_n$  was decided, the full node needs to convince the light client verifier that a threshold number $t$ of validators from the current validator set signed $H_n$, where $t$ depends on the type of consensus. Byzantine fault tolerant based consensus often uses $t$ to be over 2/3 of the total number of validators.

\noindent\paragraph{Keeping Track of the Validator Set:} A light client verifier must be initialised with a committee key $cpk_1$ corresponding to the genesis validator set with keys $\bf{pk}_1$. At the end of each epoch, i.e., the time a validator set needs to be updated, the validators set of epoch $i$, with keys $\bf{pk}_i$ sign a message $(i,cpk_{i+1})$ where $cpk_{i+1}$ is {\color{red} a commitment to the validator set} for the next epoch $\bf{pk}_{i+1}$. The light client verifier keeps track of $cpk_i$ for each epoch. A light client proof must include a committee key scheme proof that a bitvector of validators, with a threshold number of 1s, with keys committed to in $cpk_i$ signed $(i,cpk_{i+1})$. To convince a light client verifier knowing only $cpk_1$ of something in block $n$, all such proofs up to the epoch containing block $n$ must be included. For an incremental light client system, such as one on a bridge, these validator set update proofs only need to be given once an epoch.
\vspace{-0.05in}

\noindent\paragraph{Proving the General Claim $\mathit{info}_n$:} Once the light client verifier is convinced of $cpk_{n-1}$ for the epoch $n-1$ and $t$ of the validators in epoch $n-1$ signed $H_n$, it needs a committee key scheme proof for $cpk_n$ and a bitfield with $t$ ones that $t$ validators signed $H_n$. Finally, such a proof needs the opening of the commitment $H_n$ to $\mathit{info}_n$.
\vspace{-0.08in}

\noindent\paragraph{Accountability:} Now suppose that a full node obtains a light client proof for something that contradicts something it sees as decided 
by the blockchain. For our bridge use case, all light client proofs will be publicly available on another blockchain. We assume that we can express this 
contradiction in terms of a pair of messages that should never be signed by an honest validator, and that any validators doing so can be punished. 
These we call incompatible messages. In this example, such pairs of messages should include validator sets commitment to different 
commitments $(i,cpk_{i+1})$ and $(i,cpk'_{i+1})$, $cpk_{i+1} \neq cpk'_{i+1}$ and similarly distinct $H_n$ and $H'_n$. If a light client proof 
contains a message signed by a committee key which is a commitment to the public keys of a known set of validators which is incompatible with a 
message the same set signed on the blockchain, then the signature in the proof is a valid BLS signature with the claimed set of signers and so the full 
node should be able to report that public keys that signed both messages misbehaved. If an incompatible message was signed by a committee key 
which doesn't correspond to the claimed epoch's validator set, then at some point previously the light client proof must have shown that the committee 
key for some correct validator set signed the wrong committee key for the next set which is a message that is incompatible with the correct committee 
key that they signed on the real chain. Note that the accountability of our light client system instantiation relies on the accountability of the 
underlying consensus protocol. Indeed, our light client is accountable only if signatures on incompatible messages are enough for consensus 
accountability e.g., in Casper FFG~\cite{CasperFFG} and it is not directly applicable to consensus protocols where forensics (such as in~\cite{forensics}) 
are required for accountability, e.g., Polkadot's GRANDPA, Section 4.1~\cite{GRANDPA}. If the consensus protocol is not accountable with signatures, 
then the consensus protocol needs to be modified by adding another layer (e.g. ABC\cite{ABC}, Polkadot's BEEFY~\cite{BEEFY}).
\vspace{-0.05in}.

\noindent\paragraph{Efficiency Gain:} If one follows the obvious approach described above using BLS aggregation and aims to convince the 
light client verifier that $\mathit{info}_n$ is decided, then one needs to send $O(v)$ public keys for each validator set change, where $v$ is 
the upper bound on the size of the validator set. Using our succinct committee key scheme however, one requires only a constant size 
proof and $v$ bits for each validator set change to convince the light client that $\mathit{info}_n$ was decided. Since a public key or 
signature typically takes 100s of bits, our approach achieves much smaller proof sizes.  More details our achieved efficiency are 
available in Section~\ref{sec_implementation}.
\vspace{-0.05in}

\noindent\paragraph{Formalisation:} We give a formal model for the security properties of our accountable light client in Appendix~\ref{sec:LCinstantiation}.

\vspace{-0.1cm}
\subsection{Our Custom SNARKs}
\label{sec_intro_custom_snarks}
%\vspace{-0.1cm}

%\noindent In the following we discuss how we use custom SNARKs with efficient prover time to implement the committee key scheme. 
%While we achieved very fast proving times in our implementation, it came at the cost of not using a general purpose SNARK protocol, 
%leading to a more involved security model and the necessity of additional security proofs. \\
\noindent Here we discuss how we use custom SNARKs with efficient prover time to implement our committee key scheme. 
While we achieved very fast proving time in our SNARKs implementation, this came at the cost of not using a general purpose 
SNARK protocol, in turn leading to a more involved security model and the necessity of additional security proofs.  \\

\vspace{-0.05cm}
%\noindent Our SNARKs have inputs an aggregate public key $\mathit{apk}$, a commitment $C$ to a list of public keys $(pk_i)$, and a bitvector $(b_i)$. 
%It needs to be a proof that  $apk=\sum_i b_i pk_i$. The SNARK verifies witnesses to the partial sums $kacc_j = h + \sum_{i=0}^j b_i pk_i$ 
%where $h$ is chosen to avoid the incompleteness of the addition formulae we use.} We list two of the optimisations we used for our custom SNARK below. 
\noindent The public inputs for our SNARKs are: an aggregate public key $\mathit{apk}$, a commitment $C$ to the list of 
public keys $(pk_i)_{i \in T}$ and a bitvector $(b_i)_{i \in T}$ succinctly representing a subset $S$ of public keys. 
Our SNARKs provers output a proof that $apk=\sum_{i \in T} b_i pk_i$ and that $C$ is the commitment to the list of public keys 
$(pk_i)_{i \in T}$. However the list itself is a witness for the relations defining our SNARKs and so the verifiers do not need it 
and do not have to parse or check anything based on this possibly long list. 
We detail below two further optimisations of our custom SNARKs.

%\begin{itemize}
%\item Our scheme is an instance of commit and prove SNARKs (see section~\ref{sec:commit_prove} for more details) that works as follows. 
%The verifier takes only a commitment to part of the input of the SNARK, for us $C$ is a commitment to the list of public keys, and the list public 
%keys themselves are not used by the verifier. We use the same polynomial commitment scheme for this purpose as is used in th SNARK itself. 
%Hence, we do not need to add the decommitment of $C$ to the SNARK constraint system. Since our constraint system is simple adding a decommitment 
%to use, e.g. a hash for the commitment, would increase the size of the constraint system and lead to several orders of magnitude increase in prover time. 
%The tradeoff is that we cannot use an existing SNARK system or polynomial constraint compiler as a black box, making the proofs in this paper more complicated.}
%\item 
\vspace{-0.05in}
\paragraph{Commit and Prove SNARKs:} Our SNARKs are an instance of commit and prove SNARKs (see Section~\ref{sec:commit_prove}). 
The underlying commitment scheme used for computing the public input commitment $C$ is the same as the (polynomial) commitment scheme used in the rest of our SNARK(s). Hence, we do not need to add a witness for $C$ 
to the SNARK constraint system in the same way we would have to if our commitment scheme were, e.g., to use a hash function.
The constraints for checking a hash inside our custom SNARKs would increase the size of the constraint system so much that it would lead to several orders of magnitude increase in our prover time. 
The trade-off for our SNARKs design (i.e., with a commitment as part of the public input) is that we cannot use an existing SNARK compiler as a black box.

%\item Our constraint system is of a form where the wiring together of different constraints is trivial enough that we can avoid doing a permutation argument 
%or sparse matrix vector product that general proving systems would use to wire together gates. This reduces the proof size and proving time.}
%\item 
\vspace{-0.05in}
\paragraph{Constraint System Simplicity: }Our constraint system is simple enough such that our custom SNARKs do not require a permutation argument or a matrix-vector product argument 
which general proving systems need to bind together gates. In fact, the underlying circuit for our SNARKs can be described as an affine addition gate with a couple of constraints added to avoid the incompleteness of our addition formulae. This simplification leads to smaller proof sizes  and faster proving times.


%\vspace{-0.2cm}

\vspace{-0.2cm}
\subsection{Related Work}

\subsubsection{Naive Approaches and Their Use in Blockchains}
There are a number of approaches commonly used in practice to verifying that a subset of a large set signed a message. 
%However, among these, the approaches that have slow verification limit the size of the validators' set, which in turn limits decentralisation. 
\vspace{-0.2cm}
\noindent \paragraph{Verify All Signatures}  One could verify a signature for each signing validator. This is what participants  do in protocols like Polkadot~\cite{polkadot}, with 297 validators
% \cite{PolkaExplorer} 
(or Kusama with 1000 validators) %\cite{PolkaExplorer}
and Tendermint~\cite{tendermint_paper}, which is frequently used with 100 validators). %\cite{CosmosExplorer}). 
The Tendermint light client system, which is accountable and uses the verification of all individual signatures approach, 
is used in bridges in the IBC protocol\cite{IBC_paper}. This approach becomes prohibitively expensive for a light client verifier when there are 1000s or millions of signatures. 
\vspace{-0.1in}
\noindent \paragraph{Aggregatable Signatures} One could use an aggregatable signature scheme like BLS~\cite{BLS_signatures,boneh_compact_multisig}  and reduce this to verifying one signature, but that requires calculating an aggregate public key. This aggregate key is different for every subset of signers and needs to be calculated from the public keys. This is what Ethereum 
%~\cite{eth2} 
does, which currently has 415,278 validators. %\cite{EthExplorer}. 
However for a light client verifier, it is expensive to keep a list of 100,000s of public keys updated. As such only full nodes of Ethereum use this approach and instead light clients verifiers of Ethereum~\cite{sync_committee} follow signatures of randomly selected subsets of validators of size 512. This means that the resulting light client system is not accountable because these 512 validators are only backed by a small fraction of the total stake.
\vspace{-0.2cm}
\noindent \paragraph{Threshold Signatures} Alternatively a threshold signature scheme may be used, with one public key for the entire set of validators. This approach was adopted by Dfinity~\cite{GrothDKG}. Threshold signature schemes used in practice use secret sharing for the secret key corresponding to the single public key. This gives the schemes two downsides. Firstly, they require a communication-heavy distributed key generation protocol for the setup which is difficult to scale to large numbers of validators. Indeed, despite recent progress~\cite{AggregatableDKG,GrothDKG,LWEDKG}, it is still challenging to implement setup schemes for threshold signatures across a peer-to-peer network with a large number of participants, which is what many blockchain related use cases require. Moreover, such a setup may need repeating whenever the signer set changes. Secondly, for secret sharing based threshold signature schemes, the signature does not depend on the set of signers and so we cannot tell which subset of the validators signed a signature i.e., they are not accountable. Dfinity~\cite{GrothDKG} uses a re-shareable BLS threshold signature, where the threshold public key remains the same even when the validator set changes. Such a signature scheme 
provides the light client verifier with a constant size proof, even over many validator set changes, but means that the proof not only does not identify which of a particular set of validators are misbehaving, but also we cannot say when this misbehaviour happened i.e., which validator set misbehaved. This is because the signature would be the same for any threshold subset of any validator set.

%\vspace{-0.2cm}
\noindent It is worth noting that if a protocol has already implemented aggregatable BLS signatures, our committee key scheme can be used without altering the consensus layer. Indeed it may be easier to alter a protocol that uses individual 
signatures to use aggregatable BLS signatures than to implement threshold signatures from scratch because the latter requires waiting for an interactive setup before making validator set changes.

%\vspace{-0.2cm}
\subsubsection{Using SNARKs to Roll up Consensus}
%\vspace{-0.1cm}

\noindent{Celo~\cite{celo} and Mina~\cite{mina} blockchains have associated light client verifiers which allow their resource constrained users 
to efficiently and securely sync from the beginning of the blockchain to the latest block.}

%\vspace{-0.2cm}
\noindent \paragraph{Plumo~\cite{plumo}} is the most relevant comparison to our scheme. It also tackles the problem we consider, i.e., that of 
proving validator set changes. In more detail, Plumo uses a Groth16 SNARK~\cite{groth16} to prove that enough validators signed 
a statement using BLS signatures from a set of the public keys. In Celo~\cite{celo}, the blockchain that designed and plans to use 
Plumo, validators may change every epoch which is about a day long and the Plumo's SNARK iteratively proves 120 epochs worth of 
validator set changes. Since in Celo there are no more than 100 validators in a validator set at any one time, the respective public 
keys are used in plain as public input for Plumo's SNARK, as opposed to a succinct polynomial commitment in the case of our custom SNARKs. 
All of the above increase the size of Plumo's prover circuit. Since Plumo is designed to help resource constrained light clients sync from scratch, 
it is not an impediment that the Plumo SNARK cannot be efficiently generated, i.e., in real time. In the case of a light client verifier for bridges 
(i.e., the most resource constrained application), we expect it to be in sync at all times and, by design, we care only about one validator 
set change at a time. Our slimmed down and custom SNARK not only can be generated in real time, but, also due to the use of specialised 
commitments schemes for public keys, our validator sets can scale up to much larger sizes as well without impacting the efficiency of our system. 

\begin{comment}
\paragraph{Mina} achieves light clients with $O(1)$ sized light client proofs using recursive SNARKs. 
This requires some nodes have a large computational overhead to produce these proofs. 
%Also because this requires verifying consensus with small circuits, they do not use the consensus paradigm discussed above where a majority of validators sign, and instead use a longest chain rule version of proof of stake~\cite{mina}. 
Their protocol is not accountable because, as with Dfinity above, it is not possible to tell from the proof which validators signed off on a fork, nor when this happened. 
%Another downside is that because the proof only shows the length of a chain (and its block density), similar to a Bitcoin SPV proof, a light client needs to be connected to an 
%honest node to tell if a block is in the longest chain. If the client is connected to a single malicious node, it could be given a proof for a shorter fork and not see any proofs of chains the fork choice rule would preder.
\end{comment}

%\vspace{-0.2cm}
\paragraph{Mina~\cite{mina}} achieves light clients with $O(1)$ sized light client proofs using recursive SNARKs. This requires some nodes have a large computational overhead to produce proofs. Also because this requires verifying consensus with small circuits, they do not use the consensus paradigm 
discussed above where a majority of validators sign, and instead use a longest chain rule version of proof of stake~\cite{mina}. 
Their protocol is not accountable because, as with Dfinity above, it is not possible to tell from the proof which validators signed off on a fork, nor when this happened. 
Another downside is that because the proof only shows the length of a chain (and its block density), similar to a Bitcoin SPV proof, a light client needs to be 
connected to an honest node to tell if a block is in the longest chain. If the client is connected to a single malicious node, it could be given a proof for a 
shorter fork and not see any proofs of chains the fork choice rule would prefer.
\vspace{-0.03cm}
\subsubsection{Commit-and-Prove and Related Approaches}
\label{sec:commit_prove}

\noindent Our custom SNARKs are an instance of the commit-and-prove paradigm~\cite{KilianPhD,CLOS02,CP_proposal,HP_paper,CP_paper} 
which is a generalisation for zero-knowledge proofs/arguments in which the prover proves statements about values that are committed.\\

\begin{comment}
In practice, commit-and-prove systems (for short, CP) can be used to compress a large data structure and then prove something about its 
content (e.g., polynomial commitments~\cite{KZG_10}, vector commitments~\cite{vector_commitment_1}, accumulators~\cite{first_accumulator}). 
CP schemes can also be used to decouple the publishing of commitments to some data from the proof generation: each of these actions may be 
performed by different parties or entities~\cite{zkp_reference}. Finally, commitments can be used to make different proof systems 
interoperable~\cite{CP_paper,interoperability_2}. Our SNARKs have properties from the first two categories, however we could not 
have simply re-used an existing argument system: by designing custom circuits and SNARKs, we ensured improved efficiency for our use cases. 
\end{comment}
\vspace{-0.1in}

{\color{red}In this context, ECLIPSE~\cite{eclipse} presents a compiler that starts off with popular SNARKs (e.g., Sonic~\cite{sonic}, PLONK~\cite{plonk}, 
Marlin~\cite{marlin}) and via a new general compilation method generates CP-variants for these SNARKs. 
Our proposed compiler uses as a first step the standard PLONK compiler.  As a second step we simply 
re-cast the SNARK resulted in the first step as a SNARK for a new relation. 
The security of the re-casting holds under mild conditions that deterministically relate some polynomials 
processed by the verifier in the ranged polynomial protocol (before applying PLONK compiler) to some 
public inputs. To our knowledge, our re-casting conditions are less stringent than the conditions needed 
in~\cite{eclipse}.}

{\color{red}We cannot use the ECLIPSE compilation technique either in full or in part to compile our custom SNARKs 
since the types of NP relations derived after ECLIPSE compilation are simply incompatible with ours. While in the 
case of ECLIPSE, the witnesses for the NP relations before compilation remain witnesses also for the relations 
after compilation, in our case, some part of the public input before compilation becomes witness after the 
re-casting of the SNARK for a new NP relation. Thus, overall, ECLIPSE and the current work solve different problems. 
Finally, our compilation method requires only the PLONK compiler without additional computational 
steps so it is more efficient than the one in~\cite{eclipse}.} 

%The paragraph below was a comment/was commented out when shortening the paper.
\begin{comment}
\noindent Another paradigm related to commit-and-prove is called hash-and-prove~\cite{HP_paper}: for large data structures or simply data that is expensive to be 
handled directly by a computationally constrained verifier, one can hash that data and then create a (succinct) proof for some verifiable computation that uses 
the original, large, dataset. The committee key scheme notion that we define in this work has both similarities to but also differences with regard to this 
paradigm. The similarities are that, both the way we instantiate our committee key (i.e., using a polynomial commitment 
with a trusted universal setup) and the way we instantiate our aggregate public key, can be generalised as some form of (possibly deterministic) 
hash function. One difference is that the setup for the polynomial commitment is the same as that from which the proving and verification key for our committee key scheme are 
computed; thus our version of the hashes and the keys for the committee key scheme are definitely not independent as in the case of hash-and-commit~\cite{HP_paper}. Finally, 
built into our definition of committee key scheme and its security properties, we use a secure aggregatable signature scheme which allows us to design and 
prove the security properties of our accountable light client(s). In fact, to add some intuition to the fact that a committee key scheme is more than 
just a hash-and-prove instance, we mention that our committee key scheme inherits an unforgeability property from its aggregatable scheme sub-component. 
This is one property that as far as we are aware no hash-and-prove scheme has. \\
\vspace{-0.08in}
\end{comment}

\begin{comment}
\noindent When proving the security of our arguments, we use an extension of some of the more commonly employed SNARK definitions which we call a ``a hybrid model SNARK''. This resembles the existing notion of SNARKs with online-offline verifiers as described in~\cite{HP_paper}, where the verifier computation is split into 
two parts: during the offline phase some computation (possibly of commitments) happens; this computation takes some public inputs as parameters and, when not 
performed by the verifier, it may also be performed (in part) by the prover. The online phase is the main computation performed by the verifier. In the case of our hybrid 
model SNARKs, however, the input to the offline counterpart described above (which we call the $\mathit{PartInput}$ algorithm) may even be the witness or 
a part of the witness for the respective relation. For our custom SNARKs, $\mathit{PartInput}$ produces part of the public input used by the verifier; 
since for our use case, $\mathit{PartInput}$ does handle a portion of the witness, this operation cannot be performed by the verifier for that relation. 
Moreover, in our instantiation, $\mathit{PartInput}$ produces computationally binding commitment schemes that are opened by the prover. Both of these properties 
are not explicitly part of our general definition for hybrid model SNARKs, but they are crucial and explicitly assumed and used 
in proving the security for our compiler's second step (see Appendix~\ref{sec_two_step_compiler}).
\end{comment}
\vspace{-0.1in}
%\begin{comment}
%\subsection{BLS multisignatures} \label{ssec:BLS}

%\noindent {\color{red} We implement and use an efficient BLS multisignature scheme that has both efficient 
verification and efficient key aggregation (for more details, see section~\ref{sec:bls}). 
In general, multisignatures are susceptible to so-called ``rogue-key attacks'' which can be 
mounted whenever the adversary is allowed to choose his public keys arbitrarily. 
In a typical rogue-key attack, the adversary uses a public key that is a function of an 
honest user's key and this allows him to produce forgeries easily. The BLS multisignatures that we define and use 
in this work make no exception. So, in order to protect against rogue-key attacks, 
we enhance them with proofs-of-posession as defined in~\cite{proofs_of_posession}.} \\

\noindent {\color{red}Alternative constructions for BLS multisignatures as well as other defence mechanisms against rogue key attacks exist and 
we briefly review both below. First, a more general case to BLS mltisignatures exists, namely aggregating BLS signatures for different messages 
(e.g., ~\cite{aggregate_BLS_signatures}). In this case, the rouge-key attacks are not a threat anymore, but multisignature verification 
is computationally more expensive than in our case, requiring $O(n)$ parings for $n$ different messages. Second, alternative 
BLS multisignatures exist (e.g.,~\cite{boneh_compact_multisig}) where both the aggregated public key have their size independent 
of the number of signers and the multisignature verification is as fast as in our variant. However, for our application we prefer the scheme 
detailed in section~\ref{sec:bls}: in spite of requiring a one-time verification of proofs-of-posession, its corresponding aggregate key is a 
simple sum of the individual signers' public keys, while, in ~\cite{boneh_compact_multisig} the key aggregation operation involves more 
expensive scalar multiplications every time the key aggregation is performed.}


%TODO: This could be explained somewhere other than the intro if preferred {\color{red}@Alistair from Oana: Can we replace or merge the 4 sentences below on multisignatures with/%into the more detailed description I give above?}

%There are several variants of BLS multisignatures. We will consider the easiest one for which the aggregate public key is simply the sum of the public keys of the signers

%This naive scheme is vulnerable to rogue key attacks, which can be prevented using proofs of possession [cn].  In our case, we can assume that the list of public keys comes from a %trusted source, such as the previous set of validators, who checked the proofs of possession and so the verifier themselves does not need to.

%\subsection{Implementation}
%\label{sec:intro_implementation}

%\noindent {\color{red} Our implementation leverages a pair of pairing-friendly elliptic curves which we call the inner curve and the outer curve respectively, such that the base field of the %inner curve matches the scalar field of the outer curve.} \\

%\noindent {\color{red}The first pair of pairing friendly elliptic curves where the inner curve's base field and the outer curve's scalar field are identical 
%but the two curves do not form a cycle has been introduced by ZEXE~\cite{zexe}. The authors call such a pair of elliptic curves a two-chain. 
%The ZEXE two-chain curves are BLS12-377 and CP6-782. In this work we build on the two-chain ZEXE instantiation in the following way: we 
%keep BLS12-377 as the inner curve and, for efficiency reasons which will be detailed later in the paper, we replace the CP6-782 outer curve with 
%BW6-761~\cite{BW6}.\\

%\noindent Both BLS12-377 and BW6-761 are pairing friendly curves and we make use of that as follows. Intuitively we use BLS12-377 
%to sign and verify BLS signatures with the public keys being elements of the first source group of the efficient pairing associated with BLS12-377. Hence, 
%our BLS signature public keys are natively represented over the base field of BLS12-377. Then we use BW6-761 to prove using succinct proofs 
%(i.e., snarks) public key aggregation for public keys signing the same message. Because the base field of BLS12-377 matches the scalar field 
%of BW6-761 and since in a snark system the proof is performed in an arithmetic circuit over the scalar field of the curve, so, 
%in our case the scalar field of BW6-761, any efficiency loss due to curves mismatch is avoided. \\

%\noindent Note that even if our implementation is instantiated with a specific pairing-friendly two-chain as described above, 
%our theoretical results (see Section \ref{sec:snarks}) generalise to any pairing-friendly two-chain and, where possible, we state 
%them as such.} 
%\end{comment}



\section{Implementation} \label{sec_implementation}
%%\vspace{-0.2cm}
\noindent We implemented and benchmarked the protocol. The implementation allows us to evaluate the performance of our protocol and serves as prototype for future deployment. The implementation 
is open source and publicly available at \url{https://github.com/CCS23-anonymous/light-client}. It is written in Rust and uses the Arkworks library. \\

%\vspace{-0.1cm}
\noindent Table \ref{tab:benchmarks} gives the prover and verifier time for the two SNARK schemes (basic accountable and packed accountable, see Section~\ref{sec:snarks}) with $v = n-1 = 2^{10}-1$, $v = n-1 = 2^{16}-1$ 
and $v=n-1=2^{20}-1$ signers. The benchmarks were run on commodity hardware, with an 2.2 GHz i7 and 16GB RAM. We remind the reader that by $v$ we denote the maximum number of  validators in our system and that $n$ was defined in Section~\ref{sec:lagrange}.\\

%\begin{center}
\begin{table}[h!]
\hfill
\begin{tabular}{| l | l | l | l | l |l | l |}
\hline
 Scheme & \multicolumn{2}{|c|}{$v = 2^{10}-1$} & \multicolumn{2}{|c|}{$v = 2^{16}-1$} & \multicolumn{2}{|c|}{$v = 2^{20}-1$}     \\
\cline{2-7}
 &  prover & verifier & prover & verifier &  prover & verifier \\
\hline

Basic Accountable & 564ms & 26ms & 22s & 46ms & 332s & 231ms \\
Packed Accountable & 830ms & 29ms & 31s & 33ms & 447s & 57ms \\
%Counting            & 761ms & 27ms & 31s & 30ms & 692s & 107ms \\

\hline
\end{tabular}
%\setlength{\belowcaptionskip}{-0.5cm}
\caption{Proof and verifier times for the different schemes and numbers of signers}
\label{tab:benchmarks}
\end{table}
%\end{center}

%\vspace{-0.2cm}
\noindent These signer numbers are approximately the range of the number of validators that we aimed our implementation at e.g. the Kusama blockchain (\url{https://kusama.network/}) has 1000 validators which is also the number that Polkadot is aiming for, and Ethereum 2 has about 348,000 validators and it has been suggested that there will be no more than $2^{19}$~\cite{ethresearch1}. \\

\noindent At $v = n-1 = 1023$, the prover can generate a proof in any scheme in well under a second, which is short enough to generate a proof for every block in most prominent blockchains. Even for $v= n-1 =2^{20}-1$, the prover time is under 6.4 minutes, the time for an Ethereum 2 epoch, the time that validators finalise the chain. For verification time, the basic accountable scheme is slower, considerably so for larger signer numbers. \\

\noindent  Table \ref{tab:operations} gives the number of operations the prover and verifier use. Table \ref{tab:proof-size} gives the proof constituents and also the total proof and input sizes in bits. The basic accountable scheme's verifier performance at large numbers is so slow because it includes $O(n)$ field operations, which dominate the running time, however at 1023 signers it gives the smallest size. The packed accountable scheme, which includes $O(n/\lambda)$ field operations, fairs better on the benchmarks, having similar verification time than the counting scheme which has sublinear verification time, even at $2^{20}-1$ signers. 
%The prover is considerably slower for the latter two schemes because it needs to do additional operations.
The prover is considerably slower for the second scheme because it needs to do additional operations. 
At larger signer sizes, the proof size for the accountable schemes is dominated by the bitfield.

%\vspace{-0.1in}
\begin{table}[h!]
\hfill
\begin{tabular}{| l | l| l| l|}
\hline
Scheme & Prover operations  &Verifier operations \\
\hline
Basic Accountable & $12\times FFT(N)+FFT(4N)+9ME(N)$  & $2P+11E+O(n)F$ \\
Packed Accountable & $18\times FFT(N)+FFT(4N)+12ME(N)$  & $2P+16E+O(n/\lambda+log(n))F$ \\
%Counting & $13\times FFT(N)+FFT(4N)+11ME(N)$  & $2P+14E+O(log(n))F$ \\
\hline
\end{tabular}
%\setlength{\belowcaptionskip}{-1cm}
\caption{Expensive prover and verifier operations. $FFT(M)$ is an FFT of size M. $ME(M)$ is a  multi-scalar multiplication of size $M$. $P$ is a pairing, $E$ is a single scalar multiplication and $F$ is a field operation.}
\label{tab:operations}
\end{table}
%\vspace{-0.1in}
\begin{table}[h!]
\begin{tabular}{| l | l | l | l | l | l |}
\hline
Scheme & Proof & Input & \multicolumn{3}{|c|}{Actual proof + input size in bits} \\
\cline{4-6}
& & & $v = 2^{10}-1$ & $v = 2^{16}-1$ & $v = 2^{20}-1$ \\
\hline
Basic Accountable & $5\mathbb{G}_{1,out}+5\mathbb{F}$ & $2\mathbb{G}_{1,out}+1\mathbb{G}_{1,inn}+n$ bits & 9088 & 73600 & 1056640 \\
Packed Accountable & $8\mathbb{G}_{1,out}+8\mathbb{F}$ & $2\mathbb{G}_{1,out}+1\mathbb{G}_{1,inn}+n$ bits & 12544 & 77056 & 1060096 \\
%Counting &  $7\mathbb{G}_{1,out}+7\mathbb{F}$ & $2\mathbb{G}_{1,out}+1\mathbb{G}_{1,inn}$ & 10368  & 10368  & 10368 \\
\hline
\end{tabular}
\caption{Proof/input constituents and total proof/input size for implementation.}
\label{tab:proof-size}
\end{table}

%\vspace{-0.2cm}
\noindent We implemented and benchmarked our custom SNARKs. The implementation allows us to evaluate the performance of our SNARKs and serve as prototype for future deployment. The implementation 
is open source and publicly available at \url{https://github.com/CCS23-anonymous/light-client}. It is written in Rust and uses the Arkworks library. \\

%\vspace{-0.1cm}
\noindent Table \ref{tab:benchmarks} gives the prover and verifier time for the two SNARK schemes (basic accountable and  packed accountable, see Section~\ref{sec:snarks}) with $v = n-1 = 2^{10}-1$, $v = n-1 = 2^{16}-1$ 
and $v=n-1=2^{20}-1$ signers. The benchmarks were run on a 3.6GHz 16-core AMD Ryzen 9 5950X. Here $v$ is the maximum number of signers and 
$n> v$ is the size of a multiplicative subgroup of the field (see Section~\ref{sec:lagrange}).\\

%\begin{center}
\begin{table*}[h!]
\hfill
\begin{tabular}{| l | l | l | l | l |l | l |}
\hline
 Scheme & \multicolumn{2}{|c|}{$v = 2^{10}-1$} & \multicolumn{2}{|c|}{$v = 2^{16}-1$} & \multicolumn{2}{|c|}{$v = 2^{20}-1$}     \\
\cline{2-7}
 &  prover & verifier & prover & verifier &  prover & verifier \\
\hline

Basic Accountable & 112ms & 11.6ms & 2.96s & 15.3ms & 42.1s & 89.7ms \\
Packed Accountable & 157ms & 12.5ms & 4.1s & 12.6ms & 58.0s & 14.2ms \\
%Counting            & 761ms & 27ms & 31s & 30ms & 692s & 107ms \\
\hline
\end{tabular}
%\setlength{\belowcaptionskip}{-0.5cm}
\caption{Proof and verifier times for the different schemes and numbers of signers}
\label{tab:benchmarks}
\end{table*}
%\end{center}

%\vspace{-0.2cm}
\noindent These signer set sizes are approximately the range of the number of validators that we aimed our implementation at e.g., the Kusama blockchain (\url{https://kusama.network/}). This network has 1000 validators which is also the number that Polkadot is aiming for, while Ethereum 2 has about 348,000 validators and it has been suggested that there will be no more than $2^{19}$~\cite{ethresearch1}. \\

\noindent At $v = n-1 = 1023$, the prover can generate a proof in any scheme in well under a second, which is short enough to generate a proof for every block in most prominent blockchains. Even for $v= n-1 =2^{20}-1$, the prover time is under 1 minute, 
when the time for an Ethereum 2 epoch is 6 minutes, i.e., the period that validators sign messages for finality of the chain. For verification time, the basic accountable scheme is slower, considerably so for larger sets of signers. \\

\noindent  Table \ref{tab:operations} gives the number of operations the prover and verifier use. Table \ref{tab:proof-size} gives the proof constituents and also the total proof and input sizes in bits. The basic accountable scheme's verifier performance at large numbers is slower because it includes $O(n)$ field operations, which dominate the running time, however at 1023 signers it gives the smallest size. The packed accountable scheme, which includes $O(n/\lambda)$ field operations, fairs better w.r.t. the verification benchmarks for large signer sets. The prover is considerably slower for the latter scheme because it needs to do additional operations. At larger signer sizes, the proof size  is dominated by the bitfield.

%\vspace{-0.1in}
\begin{table*}[h!]
\hfill
\begin{tabular}{| l | l| l| l|}
\hline
Scheme & Prover operations  &Verifier operations \\
\hline
Basic Accountable & $12\times FFT(n)+FFT(4n)+9ME(n)$  & $2P+11E+O(n)F$ \\
Packed Accountable & $18\times FFT(n)+FFT(4n)+12ME(n)$  & $2P+16E+O(n/\lambda+log(n))F$ \\
%Counting & $13\times FFT(n)+FFT(4n)+11ME(n)$  & $2P+14E+O(log(n))F$ \\
\hline
\end{tabular}
%\setlength{\belowcaptionskip}{-1cm}
\caption{Expensive prover and verifier operations. $FFT(M)$ is an FFT of size M. $ME(M)$ is a  multi-scalar multiplication of size $M$. $P$ is a pairing, $E$ is a single scalar multiplication and $F$ is a field operation.}
\label{tab:operations}
%\end{table*}
\vspace{-0.1in}
%\begin{table*}[h!]
\begin{tabular}{| l | l | l | l | l | l |}
\hline
Scheme & Proof & Input & \multicolumn{3}{|c|}{Actual proof + input size in bits} \\
\cline{4-6}
& & & $v = 2^{10}-1$ & $v = 2^{16}-1$ & $v = 2^{20}-1$ \\
\hline
Basic Accountable & $5\mathbb{G}_{1,out}+5\mathbb{F}$ & $2\mathbb{G}_{1,out}+1\mathbb{G}_{1,inn}+n$ bits & 9088 & 73600 & 1056640 \\
Packed Accountable & $8\mathbb{G}_{1,out}+8\mathbb{F}$ & $2\mathbb{G}_{1,out}+1\mathbb{G}_{1,inn}+n$ bits & 12544 & 77056 & 1060096 \\
%Counting &  $7\mathbb{G}_{1,out}+7\mathbb{F}$ & $2\mathbb{G}_{1,out}+1\mathbb{G}_{1,inn}$ & 10368  & 10368  & 10368 \\
\hline
\end{tabular}
\caption{Proof/input constituents and total proof/input size for implementation.}
\label{tab:proof-size}
\end{table*}


\section{Preliminaries} \label{sec_prelims}
%\subsection{Conventions}
%\label{sec:conventions}
%\vspace{-0.03in}
\noindent We assume all algorithms receive an implicit security parameter $\lambda$. 
An efficient algorithm is one that runs in uniform probabilistic polynomial time (PPT) in the length of its input and $\lambda$. 
%{\color{blue} Every input to each of our algorithms apart from the message in our BLS (multi)signature 
%has at most polynomial length in the security parameter. If the length of that message would be polynomially bounded we could then just say that 
%that ``efficient algorithm means an algorithm that runs in polynomial time in the security parameter".}
%When we say that $A$ is an efficient adversary we mean that $A$ is a family $\{A_{\lambda}\}_{\lambda \in \mathbb{N}}$ 
%of non-uniform polynomial-size circuits. If the adversary consists of multiple circuit families $A_1, A_2, \ldots$, then we write $A = (A_1, A_2, \ldots)$. 
We assume the correct parameters for the curves, groups, pairings, the group generators, etc. have been generated and shared with all parties before running any algorithm or protocol. 
A function $f(\lambda)$ is negligible in $\lambda$, written as $\mathsf{negl}(\lambda)$, if $1/f(\lambda)$ grows faster than 
any polynomial in $\lambda$ and is overwhelming in $\lambda$ if $1-f(\lambda)=\mathsf{negl}(\lambda)$. By $\mathsf{poly}(\lambda)$ 
we mean some polynomial in $\lambda$ and \ewnp means except with probability $\mathsf{negl}(\lambda)$.
We write $y = A(x; r)$ when algorithm $A$ on input $x$ and randomness $r$, outputs $y$.
We write $y \leftarrow A(x)$ for picking randomness $r$ uniformly at random and setting $y = A(x; r)$. We denote by $|S|$ the cardinality of set $S$. 
%Unless otherwise stated, when we write that an event holds with some probability, we implicitly mean 
%that the probability is computed over the randomness of all randomised algorithms involved.
%We say a function is negligible in $\lambda$ and denote it by $\mathit{negl}(\lambda)$ if that function vanishes faster than the inverse of any polynomial in $\lambda$. 
%We say that a function is overwhelming in $\lambda$ if it has the form $1- $ some function negligible in $\lambda$. 
%We also use the notation \ewnp to mean except with negligible probability, or, equivalently, with overwhelming probability. We denote by $\mathsf{poly}(\lambda)$ an  
%unspecified function which has a polynomial expression in $\lambda$. 
%We generally use boldface font to denote vectors whose components we explicitly make use of in the text and 
%we use italic font to denote the rest of the variables.
%We work over finite fields of large characteristic. 
$\mathbb{F}_{<d}[X]$ is the set of all polynomials of degree less than $d$ over the field $\mathbb{F}$. For any integer 
$n \geq 1$, we denote by $[n]$ the set $\{1, \ldots, n\}$.
\vspace{-0.015in}

\vspace{-0.05in}
\subsection{Pairings}
\label{sec:pairings}
\begin{comment}
%\vspace{-0.02in}
\noindent If $E$ is an elliptic curve defined over a prime field $\mathbb{F}_{p}$ of large characteristic $p$, 
we denote by $E(\mathbb{F}_{p})$ the abelian group containing all the points $(x, y) \in (\mathbb{F}_{p})^2$ 
on the curve along with the point at infinity. We will work with pairing friendly curves i.e., those with a secure~\cite{secure_pairings,pairings_for_cryptographers} efficiently computable, bilinear, non-degenerate mapping from a prime order subgroup of $E(\mathbb{F}_{p})$ and a subgroup of the curve over the extension field.
We will work with a \emph{pairing-friendly two-chain}, i.e., a pair of pairing friendly elliptic curves $\einn=E(\mathbb{F}_{p})$ (\emph{the inner curve}) and $\eout=E'(\mathbb{F}_{r})$ (\emph{the outer curve}), such that the pairing $\epinn$ on $\einn$ works on subgroups of order $r$. $\mathbb{F}_p$ is the \emph{base field} of $\einn=E(\mathbb{F}_{p})$ and $\mathbb{F}_r$ is its \emph{scalar field}. 
We write $\ginn{1}$, $\ginn{2}$, $\gtinn$, $\gout{1}$, $\gout{2}$, $\gtout$ for cyclic subgroups of $\einn$, $E(\mathbb{F}_{p^l}$),$\mathbb{F}_{p^k}$, $\eout$, $E'(\mathbb{F}_{r^{l'}}$), $\mathbb{F}_{r^{k'}}$ respectively for suitable $l,k,l',k$ with the two pairings $\epinn:\ginn{1} \times \ginn{2} \rightarrow \gtinn$ and by $\epout:\gout{1} \times \gout{2} \rightarrow \gtout$.
We write $\sginn{1}$, $\sginn{2}$, $\sgtinn$, $\sgout{1}$, $\sgout{2}$, $\sgtout$ respectively for randomly chosen generators of these groups. We use additive notation for group operations and write $[x]_{\indexoneinn} = x \cdot \sginn{1}$, $[x]_{\indextwoinn} = x \cdot \sginn{2}$. Concretely, our implementation uses BLS12-377~\cite{zexe} and BW6-761~\cite{BW6} for $\einn$ and $\eout$.
\vspace{-0.05in}
\end{comment}

\subsection{Secure Signature Aggregation}
\label{sec:multisig_short}
\vspace{-0.03in}
An aggregatable signature scheme (AS) compresses signatures using
different signing keys into one signature. In this work we use an aggregatable 
signature scheme making explicit use of the proofs-of-possession (PoPs)~\cite{proofs_of_posession}.
Overall, for our concrete instantiation, we use aggregatable BLS signatures with an 
efficient aggregation procedure, i.e., by adding together keys and by adding together 
signatures, and we protect against rogue key attacks~\cite{proofs_of_posession} using PoPs. 
This is in contrast to other aggregation procedures that do not require PoPs for security 
but incur a higher computational cost (e.g., due to the use of multi-scalar
multiplication~\cite{boneh_compact_multisig}). For our concrete use case of accountable 
light clients systems, our efficient signature aggregation method results 
in a simple and more efficient SNARK which compensates for the cost of having to work with PoPs. 
\vspace{-0.1in}
\begin{definition}
\label{def:aggregate_signatures}
(Aggregatable Signature Scheme) An aggregatable signature scheme consists of
the following tuple of algorithms ($\mathit{AS.Setup}$, $\mathit{AS.GenKeypair}$, $\mathit{AS.VerifyPoP}$, 
$\mathit{AS.Sign}$, $\mathit{AS.AggKeys}$, $\mathit{AS.AggSig}$, $\mathit{AS.Verify}$) 
such that for implicit security parameter $\lambda$:
\vspace{-0.05in}
\begin{itemize}

\item $\mathit{pp} \leftarrow  \mathit{AS.Setup}(\mathit{aux_{\mathit{AS}}})$: a setup algorithm that, given an 
auxiliary parameter $\mathit{aux_{\mathit{AS}}}$, outputs public protocol parameters $\mathit{pp}$. 

\item $((\mathit{pk},\mathit{\pi_{PoP}}),\mathit{sk}) \leftarrow \mathit{AS.GenKeypair}(\mathit{pp})$:
a key pair generation algorithm that
outputs 
a secret key $\mathit{sk}$,
and the corresponding public key $\mathit{pk}$
together with a proof of possession $\mathit{\pi_{PoP}}$ for the secret key.

\item $0/1 \leftarrow \mathit{AS.VerifyPoP}(\mathit{pp}, \mathit{pk},\mathit{\pi_{PoP}})$:
a public key verification algorithm that,
given a public key $\mathit{pk}$
and a proof of possession $\mathit{\pi_{PoP}}$,
outputs
$1$ if $\mathit{\pi_{PoP}}$ is valid for $\mathit{pk}$ and $0$ otherwise.

\item $\sigma \leftarrow \mathit{AS.Sign}(\mathit{pp}, \mathit{sk}, m)$:
a signing algorithm that,
given a secret key $\mathit{sk}$ and a message $m \in \{0, 1\}^*$, returns a signature $\sigma$.

\item $\mathit{apk} \leftarrow \mathit{AS.AggKeys}(\mathit{pp}, (\mathit{pk_i})_{i=1}^{u})$:
a public key aggregation algorithm that,
given a vector of public keys $(\mathit{pk_i})_{i=1}^u$,
returns
an aggregate public key $\mathit{apk}$.

\item $\mathit{asig} \leftarrow \mathit{AS.AggSig}(\mathit{pp}, (\sigma_i)_{i=1}^u)$:
a signature aggregation algorithm that,
given a vector of signatures $(\sigma_i)_{i=1}^u$,
returns
an aggregate signature $\mathit{asig}$.

\item $0/1 \leftarrow \mathit{AS.Verify}(\mathit{pp}, \mathit{apk}, m, \mathit{asig})$:
a signature verification algorithm that,
given an aggregate public key $\mathit{apk}$, a message $m \in \{0, 1\}^*$, and an aggregate signature $\sigma$,
returns
1 or 0 to indicate validity.
\end{itemize}
\vspace{-0.07in}
\noindent We say AS is an aggregatable signature scheme if it satisfies \emph{perfect completeness} and \emph{unforgeability} 
as standard security definitions (see appendix~\ref{sec:multisig} for full details) 
and, additionally, \emph{perfect completeness} for aggregation defined below.
\end{definition}

%\noindent We require an aggregatable signature scheme as defined above to
%satisfy \emph{perfect completeness}, \emph{unforgeability} and 
%{\color{red} \emph{verifiable aggregation w.r.t.\ malicious signers}} as follows:

%\noindent \textbf{Perfect Completeness} An aggregatable signature scheme
%($\mathit{AS.Setup}$, $\mathit{AS.GenKeypair}$, $\mathit{AS.VerifyPoP}$, $\mathit{AS.Sign}$, $\mathit{AS.AggregateKeys}$, 
%$\mathit{AS.AggregateSignatures}$, $\mathit{AS.Verify}$) has perfect completeness if for any message $m \in \{0,1\}^*$ and any 
%$u\in\mathbb{N}$ it holds that:
%\begin{align*}
%\mathit{Pr} [\mathit{AS.Verify}(\mathit{pp}, \mathit{apk}, m, \mathit{asig})=1 \ & \wedge \ \forall  i \in [u]\ \mathit{AS.VerifyPoP}(\mathit{pp}, \mathit{pk_i},\mathit{\pi_{\mathit{PoP},i}})=1\ |\\
%& \mathit{pp} \leftarrow \mathit{AS.Setup}(\mathit{aux_{\mathit{AS}}}), \\
%& ((pk_{i},\pi_{\mathit{PoP}, i}), sk_{i} ) \leftarrow \mathit{AS.GenKeypair}(\mathit{pp}),\ i=1,\ldots, u\\
%&\mathit{apk} \leftarrow \mathit{AggregateKeys}(\mathit{pp}, (\mathit{pk}_{i})_{i=1}^{u}), \\
%& \sigma_i \leftarrow \mathit{AS.Sign}(\mathit{pp}, \mathit{sk_i}, m),\ i=1,\ldots,u, \\
%& \mathit{asig} \leftarrow \mathit{AS.AggregateSignatures(\mathit{pp}, (\sigma_{i})_{i=1}^{u})}] = 1.
%\end{align*}
%\noindent We note that an aggregatable signature scheme with perfect completeness implies the underlying signature scheme
%has perfect completeness. \\
\vspace{-0.02in}
\noindent \textbf{Perfect Completeness for Aggregation} An aggregatable 
signature scheme AS
has perfect completeness for aggregation if, for every adversary $\mathcal{A}$
\begin{align*}
\mathit{Pr} & [\mathit{AS.Verify}(\mathit{pp}, \mathit{apk}, m, \mathit{asig}) = 1 \ | \ \mathit{pp} \leftarrow \mathit{AS.Setup}(\mathit{aux_{\mathit{AS}}}),  ((\mathit{pk_i})_{i=1}^u, m, (\sigma_i)_{i=1}^{u}) \leftarrow \mathcal{A}(\mathit{\mathit{pp})}, \\ 
%& {\color{red} \forall i \in [u], \mathit{AS.VerifyPoP}(\mathit{pp}, \mathit{pk_i}, \pi_{PoP,i}) = 1, }\\
 &\forall i \in [u], \mathit{AS.Verify}(\mathit{pp}, \mathit{pk_i}, m, \sigma_i) = 1, \mathit{apk} \leftarrow \mathit{AS.AggKeys}(\mathit{pp},  (\mathit{pk}_{i})_{i=1}^{u}), \mathit{asig} \leftarrow \mathit{AS.AggSigs}(\mathit{pp}, (\sigma_i)_{i=1}^u)] = 1.
\end{align*}

%\noindent \textbf{Unforgeable Aggregatable Signature}
%For an aggregatable signature scheme ($\mathit{AS.Setup}$, $\mathit{AS.GenKeypair}$, $\mathit{AS.VerifyPoP}$, $\mathit{AS.Sign}$,
%$\mathit{AS.AggregateKeys}$, $\mathit{AS.AggregateSignatures}$, $\mathit{AS.Verify}$)
%the advantage of an adversary against unforgeability is defined by

%$$\mathit{Adv}^{\mathit{forge}}_{\mathcal{A}}({\lambda}) = \mathit{Pr}[\mathit{Game}^{\mathit{forge}}_{\mathcal{A}}({\lambda}) =1]$$
%\noindent where
%\begin{align*}
%&\mathit{Game}^{\mathit{forge}}_{\mathcal{A}}({\lambda}): \\
%& \mathit{pp} \leftarrow \mathit{AS.Setup}(\mathit{aux_{\mathit{AS}}}) \\
%& ((\mathit{pk}^*,\pi^*_{\mathit{PoP}}), \mathit{sk}^*) \leftarrow \mathit{AS.GenKeypair}(\mathit{pp})\\
%& Q \leftarrow \emptyset \\
%& ((\mathit{pk_i}, \pi_{\mathit{PoP},i})_{i=1}^{u}, m, \mathit{asig}) \leftarrow \mathcal{A}^{\mathit{OSign}}(\mathit{pp}, (\mathit{pk^*},\pi^*_{\mathit{PoP}})) \\
%& \textit{If } \mathit{pk}^* \notin \{\mathit{pk_i}\}_{i=1}^{u} \vee m \in Q, \textit{ then return } 0 \\
%& \textit{For } i \in [u] \\
%& \ \ \ \ \ \textit{ If } \mathit{AS.VerifyPoP}(\mathit{pp}, \mathit{pk_i}, \pi_{\mathit{PoP},i})=0  \textit{ return } 0 \\
%& \mathit{apk} \leftarrow \mathit{AS.AggregateKeys}(\mathit{pp}, (\mathit{pk_i})_{i=1}^{u}) \\
%& \textit{Return } \mathit{AS.Verify}(\mathit{pp}, \mathit{apk}, m, \mathit{asig})
%\end{align*}
%\noindent and
%\begin{align*}
%& \mathit{OSign}(m_j): \\
%& \sigma_j \leftarrow \mathit{AS.Sign}(\mathit{pp}, \mathit{sk}^*, m_j) \\
%&  Q \leftarrow Q \cup \{m_j\} \\
%& \textit{Return} \ \sigma_j
%\end{align*}

%\noindent and $\mathcal{A}^{\mathit{OSign}}$ denotes the adversary $\mathcal{A}$ with access to oracle $\mathit{OSign}$. \\

%\noindent We say an aggregatable signature scheme is unforgeable if for all efficient adversaries
%$\mathcal{A}$ it holds that $\mathit{Adv}^{\mathit{forge}}_{\mathcal{A}}({\lambda}) \leq \mathit{negl}(\lambda)$. 
\vspace{-0.08in}
\subsubsection{An Aggregatable Signature Instantiation}
\label{sec:bls}
\noindent In the following, we instantiate the aggregatable signature definition given above with a scheme inspired by the BLS signature
scheme~\cite{BLS_signatures} and its follow-up variants~\cite{proofs_of_posession,boneh_compact_multisig}.
\vspace{-0.08in}
\begin{construction}(Aggregatable Signatures) 
\label{insta:bls}
For aggregatable signatures, our implementation uses an instantiation of BLS signatures using proofs-of-possession which are $\ginn{2}$ elements, 
where the public keys are in $\ginn{1}$  and the signatures are in $\ginn{2}$. The public key aggregation is a simple sum of the 
public keys and the signature aggregation is a simple sum of the individual signatures. We instantiate $\einn$ with BLS12-377~\cite{zexe}. Full details can be found in Appendix~\ref{sec:multisig}.
 
\begin{comment}
\begin{itemize}
\item $(\ginn{1}, \sginn{1}, \ginn{2}, \sginn{2}, \gtinn, \epinn, \Hinn, \HPoP)$ from $\mathit{pp}$ where 
$\mathit{pp} \leftarrow  \mathit{AS.Setup}(\mathit{aux_{\mathit{AS}}})$, 
where $\ginn{1}$, $\sginn{1}$, $\ginn{2}$, $\sginn{2}$, $\gtinn$, $\epinn$ were defined in Section~\ref{sec:pairings} and 
$\Hinn: \{0,1\}^* \rightarrow \ginn{2}$ and $\HPoP: \{0,1\}^* \rightarrow \ginn{2}$ are two hash functions. 
The auxiliary parameter $\mathit{aux_{\mathit{AS}}}$ is such that there exists $N \in \mathbb{N}$, 
$N$ is the first component of the vector $\mathit{aux_{\mathit{AS}}}$ and there exists a subgroup of size at least $N$ in the multiplicative group of $\mathbb{F}$, where $\mathbb{F}$ 
is the base field of $\einn$, but also the size of the subgroup $\in O(N)$.

\item $(\mathit{pk},\mathit{sk}, \pi_{\PoP}) \leftarrow \mathit{AS.GenKeypair}(\mathit{pp})$, where $\mathit{sk} \xleftarrow{\$} \mathbb{Z}_{r}^{*}$  
and $\mathit{pk} = \mathit{sk} \cdot \sginn{1} \in \ginn{1}$ and $\pi_{\PoP} \leftarrow {\mathit{sk}} \cdot \HPoP(\mathit{pk})$ 
and $r$ was defined in Section~\ref{sec:pairings} as the characteristic of the scalar field of $\einn$.

\item $0/1 \leftarrow \mathit{AS.VerifyPoP}(\mathit{pp}, \mathit{pk}, \pi_{\PoP})$, where $\mathit{AS.VerifyPoP}$ outputs $1$ if 
$$\epinn( \sginn{1}, \pi_{\PoP}) = \epinn(\mathit{pk}, \HPoP(\mathit{pk}))$$ holds and $0$ otherwise. Note that implicitly, as part of running \\
$\mathit{AS.VerifyPoP}$, one checks that $\mathit{pk} \in \ginn{1}$ also holds.

\item $\sigma \leftarrow \mathit{AS.Sign}(\mathit{pp}, \mathit{sk}, m)$: 
where $\sigma = \mathit{sk} \cdot \Hinn(m) \in \ginn{2}$.

\item $\mathit{apk} \leftarrow \mathit{AS.AggregateKeys}(\mathit{pp}, (\mathit{pk_i})_{i=1}^{u})$, where  $\mathit{apk} = \sum_{i=1}^{u} \mathit{pk_i}$. 
Note that $\mathit{AS.AggregateKeys}$ checks whether $((\mathit{pk_i})_{i=1}^{u}) \in \ginn{1}^{u} (\ast)$ and, if that is not the case, it outputs $\bot$; 
if $(\ast)$ holds, the algorithm $\mathit{AS.AggregateKeys}$ continues with the computations described above. 


\item $\mathit{asig} \leftarrow \mathit{AS.AggregateSignatures}(\mathit{pp}, (\sigma_i)_{i=1}^u)$, where $\mathit{asig}$ = $\sum_{i=1}^{u} \sigma_i$.  

\item $0/1 \leftarrow  \mathit{AS.Verify}(\mathit{pp}, \mathit{apk}, m, \mathit{asig})$, where $\mathit{AS.Verify}$ outputs $1$ if $\mathit{apk} \neq \bot$ and
$\mathit{apk} \in \ginn{1}$ and $\epinn(\mathit{apk}, \Hinn(m)) = \epinn(\sginn{1}, \mathit{asig})$; otherwise, it outputs $0$.
\end{itemize}
\end{comment}
\end{construction}
\vspace{-0.1in}
%\subsection{Aggregatable Signature Scheme Definition}
\label{sec:multisig}
%\label{suplementary_aggregatable}
An aggregatable signature scheme compresses signatures issued using possibly 
different signing keys into one signature. In this work we use an aggregatable 
signature scheme making explicit use of the proofs-of-possession (PoPs)~\cite{proofs_of_posession}.
For our concrete instantiation we use aggregatable BLS signatures with an 
efficient aggregation procedure, i.e., by adding together keys and by multiplying together 
signatures, and protect against rogue key attacks~\cite{proofs_of_posession} using PoPs. 
This is in contrast to other aggregation procedures that do not require PoPs for security 
but incur a higher computational cost (e.g., due to the use of multi-scalar multiplication~\cite{boneh_compact_multisig}). 
For our concrete use case of accountable light clients systems, our efficient signature aggregation method results 
in a simple and more efficient custom argument scheme (i.e., SNARK), which, in turn, compensates for the cost of having 
to work with PoPs. 
\begin{definition}
\label{def:aggregate_signatures}
(Aggregatable Signature Scheme) An aggregatable signature scheme consists of
the following tuple of algorithms ($\mathit{AS.Setup}$, $\mathit{AS.GenerateKeypair}$, $\mathit{AS.VerifyPoP}$, 
$\mathit{AS.Sign}$, \\ $\mathit{AS.AggregateKeys}$, $\mathit{AS.AggregateSignatures}$, $\mathit{AS.Verify}$) 
such that for implicit security parameter $\lambda$:
\begin{itemize}

\item $\mathit{pp} \leftarrow  \mathit{AS.Setup}(\mathit{aux_{\mathit{AS}}})$: a setup algorithm that, given an 
auxiliary parameter $\mathit{aux_{\mathit{AS}}}$, outputs public protocol parameters $\mathit{pp}$. 

\item $((\mathit{pk},\mathit{\pi_{PoP}}),\mathit{sk}) \leftarrow \mathit{AS.GenerateKeypair}(\mathit{pp})$:
a key pair generation algorithm that
outputs
a secret key $\mathit{sk}$,
and the corresponding public key $\mathit{pk}$
together with a proof of possession $\mathit{\pi_{PoP}}$ for the secret key.

\item $0/1 \leftarrow \mathit{AS.VerifyPoP}(\mathit{pp}, \mathit{pk},\mathit{\pi_{PoP}})$:
a public key verification algorithm that,
given a public key $\mathit{pk}$
and a proof of possession $\mathit{\pi_{PoP}}$,
outputs
$1$ if $\mathit{\pi_{PoP}}$ is valid for $\mathit{pk}$ and $0$ otherwise.

\item $\sigma \leftarrow \mathit{AS.Sign}(\mathit{pp}, \mathit{sk}, m)$:
a signing algorithm that,
given a secret key $\mathit{sk}$ and a message $m$ in $\{0, 1\}^*$, returns a signature $\sigma$.

\item $\mathit{apk} \leftarrow \mathit{AS.AggregateKeys}(\mathit{pp}, (\mathit{pk_i})_{i=1}^{u})$:
a public key aggregation algorithm that,
given a vector of public keys $(\mathit{pk_i})_{i=1}^u$,
returns
an aggregate public key $\mathit{apk}$.

\item $\mathit{asig} \leftarrow \mathit{AS.AggregateSignatures}(\mathit{pp}, (\sigma_i)_{i=1}^u)$:
a signature aggregation algorithm that,
given a vector of signatures $(\sigma_i)_{i=1}^u$,
returns
an aggregate signature $\mathit{asig}$.

\item $0/1 \leftarrow \mathit{AS.Verify}(\mathit{pp}, \mathit{apk}, m, \mathit{asig})$:
a signature verification algorithm that,
given an aggregate public key $\mathit{apk}$, a message $m \in \{0, 1\}^*$, and an aggregate signature $\sigma$,
returns
1 or 0 to indicate if the signature is valid.
\end{itemize}

\noindent We say ($\mathit{AS.Setup}$, $\mathit{AS.GenerateKeypair}$, $\mathit{AS.VerifyPoP}$, 
$\mathit{AS.Sign}$, $\mathit{AS.AggregateKeys}$, \\ $\mathit{AS.AggregateSignatures}$, 
$\mathit{AS.Verify}$) is an aggregatable signature scheme if it satisfies \emph{perfect completeness}  and 
\emph{perfect completeness for aggregation}  and \emph{unforgeability} as defined below. \\

\noindent \textbf{Perfect Completeness} An aggregatable signature scheme
($\mathit{AS.Setup}$, $\mathit{AS.GenerateKeypair}$, \\ $\mathit{AS.VerifyPoP}$, $\mathit{AS.Sign}$, $\mathit{AS.AggregateKeys}$,
$\mathit{AS.AggregateSignatures}$, $\mathit{AS.Verify}$) has perfect completeness if for any message $m \in \{0,1\}^*$ and any 
$u\in\mathbb{N}$ it holds that:
\begin{align*}
&\mathit{Pr} [\mathit{AS.Verify}(\mathit{pp}, \mathit{apk}, m, \mathit{asig})=1 \  \wedge \ \forall  i \in [u]\ \mathit{AS.VerifyPoP}(\mathit{pp}, \mathit{pk_i},\mathit{\pi_{\mathit{PoP},i}})=1\ |\\
& \mathit{pp} \leftarrow \mathit{AS.Setup}(\mathit{aux_{\mathit{AS}}}), \\
& ((pk_{i},\pi_{\mathit{PoP}, i}), sk_{i} ) \leftarrow \mathit{AS.GenerateKeypair}(\mathit{pp}),\ i=1,\ldots, u\\
&\mathit{apk} \leftarrow \mathit{AggregateKeys}(\mathit{pp}, (\mathit{pk}_{i})_{i=1}^{u}), \\
& \sigma_i \leftarrow \mathit{AS.Sign}(\mathit{pp}, \mathit{sk_i}, m),\ i=1,\ldots, u, \\
& \mathit{asig} \leftarrow \mathit{AS.AggregateSignatures(\mathit{pp}, (\sigma_{i})_{i=1}^{u})}] = 1.
\end{align*}
\noindent We note that an aggregatable signature scheme with perfect completeness implies the underlying signature scheme
has perfect completeness. \\

\noindent \textbf{Perfect Completeness for Aggregation} An aggregatable signature scheme 
($\mathit{AS.Setup}$, \\ $\mathit{AS.GenerateKeypair}$, $\mathit{AS.VerifyPoP}$, $\mathit{AS.Sign}$, 
$\mathit{AS.AggregateKeys}$, $\mathit{AS.AggregateSignatures}$, $\mathit{AS.Verify}$)
has perfect completeness for aggregation if, for every adversary $\mathcal{A}$
\begin{align*}
& \mathit{Pr}[\mathit{AS.Verify}(\mathit{pp}, \mathit{apk}, m, \mathit{asig}) = 1 \ | \ \mathit{pp} \leftarrow \mathit{AS.Setup}(\mathit{aux_{\mathit{AS}}}), \\
& ((\mathit{pk_i})_{i=1}^u, m, (\sigma_i)_{i=1}^{u}) \leftarrow \mathcal{A}(\mathit{\mathit{pp})} \ 
\textit{such that} \ \forall i \in [u], \mathit{AS.Verify}(\mathit{pp}, \mathit{pk_i}, m, \sigma_i) = 1, \\
& \mathit{apk} \leftarrow \mathit{AS.AggregateKeys}(\mathit{pp},  (\mathit{pk}_{i})_{i=1}^{u}), \\
&  \mathit{asig} \leftarrow \mathit{AS.AggregateSignatures}(\mathit{pp}, (\sigma_i)_{i=1}^u)] = 1.
\end{align*}

\noindent \textbf{Unforgeable Aggregatable Signature}
For an aggregatable signature scheme ($\mathit{AS.Setup}$, \\ $\mathit{AS.GenerateKeypair}$, $\mathit{AS.VerifyPoP}$, $\mathit{AS.Sign}$,
$\mathit{AS.AggregateKeys}$, $\mathit{AS.AggregateSignatures}$, $\mathit{AS.Verify}$)
the advantage of an adversary against unforgeability is defined by

$$\mathit{Adv}^{\mathit{forge}}_{\mathcal{A}}({\lambda}) = \mathit{Pr}[\mathit{Game}^{\mathit{forge}}_{\mathcal{A}}({\lambda}) =1]$$
\noindent where
\begin{align*}
&\mathit{Game}^{\mathit{forge}}_{\mathcal{A}}({\lambda}): \\
& \mathit{pp} \leftarrow \mathit{AS.Setup}(\mathit{aux_{\mathit{AS}}}) \\
& ((\mathit{pk}^*,\pi^*_{\mathit{PoP}}), \mathit{sk}^*) \leftarrow \mathit{AS.GenerateKeypair}(\mathit{pp})\\
& Q \leftarrow \emptyset \\
& ((\mathit{pk_i}, \pi_{\mathit{PoP},i})_{i=1}^{u}, m, \mathit{asig}) \leftarrow \mathcal{A}^{\mathit{OSign}}(\mathit{pp}, (\mathit{pk^*},\pi^*_{\mathit{PoP}})) \\
& \textit{If } \mathit{pk}^* \notin \{\mathit{pk_i}\}_{i=1}^{u} \vee m \in Q, \textit{ then return } 0 \\
& \textit{For } i \in [u] \\
& \ \ \ \ \ \textit{ If } \mathit{AS.VerifyPoP}(\mathit{pp}, \mathit{pk_i}, \pi_{\mathit{PoP},i})=0  \textit{ return } 0 \\
& \mathit{apk} \leftarrow \mathit{AS.AggregateKeys}(\mathit{pp}, (\mathit{pk_i})_{i=1}^{u}) \\
& \textit{Return } \mathit{AS.Verify}(\mathit{pp}, \mathit{apk}, m, \mathit{asig})
\end{align*}
\noindent and
\begin{align*}
& \mathit{OSign}(m_j): \\
& \sigma_j \leftarrow \mathit{AS.Sign}(\mathit{pp}, \mathit{sk}^*, m_j) \\
&  Q \leftarrow Q \cup \{m_j\} \\
& \textit{Return} \ \sigma_j
\end{align*}

\noindent and $\mathcal{A}^{\mathit{OSign}}$ denotes the adversary $\mathcal{A}$ with access to oracle $\mathit{OSign}$. \\

\noindent We say an aggregatable signature scheme is unforgeable if for all efficient adversaries
$\mathcal{A}$ it holds that $\mathit{Adv}^{\mathit{forge}}_{\mathcal{A}}({\lambda}) \leq \mathit{negl}(\lambda)$. 
\end{definition}

\subsubsection{An Aggregatable Signature Instantiation}
\label{sec:bls}
\noindent In the following, we instantiate the aggregatable signature definition given above with a scheme inspired by the BLS signature
scheme~\cite{BLS_signatures} and its follow-up variants~\cite{proofs_of_posession,boneh_compact_multisig}.

\begin{construction}(Aggregatable Signatures) 
\label{insta:bls}
In our implementation we call aggregatable signatures the following 
instantiation of aggregatable signatures definition. Note that in our implementation we instantiate $\einn$ with BLS12-377~\cite{zexe}.
\begin{itemize}
\item $(\ginn{1}, \sginn{1}, \ginn{2}, \sginn{2}, \gtinn, \epinn, \Hinn, \HPoP)$ from $\mathit{pp}$ where 
$\mathit{pp} \leftarrow  \mathit{AS.Setup}(\mathit{aux_{\mathit{AS}}})$, 
where $\ginn{1}$, $\sginn{1}$, $\ginn{2}$, $\sginn{2}$, $\gtinn$, $\epinn$ were defined in Section~\ref{sec:pairings} and 
$\Hinn: \{0,1\}^* \rightarrow \ginn{2}$ and $\HPoP: \{0,1\}^* \rightarrow \ginn{2}$ are two hash functions. 
The auxiliary parameter $\mathit{aux_{\mathit{AS}}}$ is such that there exists $N \in \mathbb{N}$, 
$N$ is the first component of the vector $\mathit{aux_{\mathit{AS}}}$ and there exists a subgroup of size at least $N$ in the multiplicative group of $\mathbb{F}$, where $\mathbb{F}$ 
is the base field of $\einn$, but also the size of the subgroup $\in O(N)$.

\item $(\mathit{pk},\mathit{sk}, \pi_{\PoP}) \leftarrow \mathit{AS.GenerateKeypair}(\mathit{pp})$, where $\mathit{sk} \xleftarrow{\$} \mathbb{Z}_{r}^{*}$  
and $\mathit{pk} = \mathit{sk} \cdot \sginn{1} \in \ginn{1}$ and $\pi_{\PoP} \leftarrow {\mathit{sk}} \cdot \HPoP(\mathit{pk})$ 
and $r$ was defined in Section~\ref{sec:pairings} as the characteristic of the scalar field of $\einn$.

\item $0/1 \leftarrow \mathit{AS.VerifyPoP}(\mathit{pp}, \mathit{pk}, \pi_{\PoP})$, where $\mathit{AS.VerifyPoP}$ outputs $1$ if 
$$\epinn( \sginn{1}, \pi_{\PoP}) = \epinn(\mathit{pk}, \HPoP(\mathit{pk}))$$ holds and $0$ otherwise. Note that implicitly, as part of running \\
$\mathit{AS.VerifyPoP}$, one checks that $\mathit{pk} \in \ginn{1}$ also holds.

\item $\sigma \leftarrow \mathit{AS.Sign}(\mathit{pp}, \mathit{sk}, m)$: 
where $\sigma = \mathit{sk} \cdot \Hinn(m) \in \ginn{2}$.

\item $\mathit{apk} \leftarrow \mathit{AS.AggregateKeys}(\mathit{pp}, (\mathit{pk_i})_{i=1}^{u})$, where  $\mathit{apk} = \sum_{i=1}^{u} \mathit{pk_i}$. 
Note that $\mathit{AS.AggregateKeys}$ checks whether $((\mathit{pk_i})_{i=1}^{u}) \in \ginn{1}^{u} (\ast)$ and, if that is not the case, it outputs $\bot$; 
if $(\ast)$ holds, the algorithm $\mathit{AS.AggregateKeys}$ continues with the computations described above. 


\item $\mathit{asig} \leftarrow \mathit{AS.AggregateSignatures}(\mathit{pp}, (\sigma_i)_{i=1}^u)$, where $\mathit{asig}$ = $\sum_{i=1}^{u} \sigma_i$.  

\item $0/1 \leftarrow  \mathit{AS.Verify}(\mathit{pp}, \mathit{apk}, m, \mathit{asig})$, where $\mathit{AS.Verify}$ outputs $1$ if $\mathit{apk} \neq \bot$ and
$\mathit{apk} \in \ginn{1}$ and $\epinn(\mathit{apk}, \Hinn(m)) = \epinn(\sginn{1}, \mathit{asig})$; otherwise, it outputs $0$.
\end{itemize}
\end{construction}
\subsection{Committee Key Scheme}
\label{sec:committee_key}
%\vspace{-0.03in}
Below we introduce the notion of committee key scheme for aggregatable signatures ($\mathit{CKS}$). 
This notion, for an appropriate instantiation (Section~\ref{sec:inst_committee_key}),  
builds upon aggregatable signature schemes (Section~\ref{sec:multisig_short}) allowing a prover to convince a verifier that an alleged aggregated signature 
for subset of signers out of an all possible set of signers represented by a bitvector and a proof together with a key summarising an all possible set of signers' 
public keys (in the following called \emph{committee key}) are valid. The notion of $\mathit{CKS}$ and its instantiation can be used, in turn, for an accountable 
light client scheme instantiation ($\mathit{LCI}$) as sketched in Section~\ref{sec:lcsketch}. Before providing formal definition for $\mathit{CKS}$, 
we include some intuition for its chosen security properties.  
\begin{itemize}
\item \emph{Perfect completeness}: If all of $\mathit{CKS.Verify}$ inputs except for an aggregatable signature have been generated honestly and if the signature is accepted 
by $\mathit{AS.Verify}$ (Definition~\ref{def:aggregate_signatures}), then $\mathit{CKS.Verify}$ on the signature and honest inputs accepts; this is the counter-notion to perfect completeness for aggregation 
(Definition~\ref{def:aggregate_signatures}); it is used for proving \emph{$\mathit{LCI}$ perfect completeness}. 
\item \emph{Soundness}: An adversary cannot output a $\mathit{CKS}$ verifying proof and alleged aggregated signature pair without the aggregated signature being accepted 
by $\mathit{AS.Verify}$; this property is crucial for proving \emph{$\mathit{LCI}$ accountability completeness}. 
\item \emph{Unforgeability}: It is similar to the underlying aggregatable signature scheme unforgeability 
and it is a direct consequence of the signature's scheme unforgeability and $\mathit{CKS}$ soundness; 
it shows $\mathit{CKS}$ has further security properties beyond those of an argument system. $\mathit{CKS}$ unforgeability 
is used for proving \emph{$\mathit{LCI}$ accountable soundness.}
\end{itemize}
\vspace{-0.1in}
\begin{definition}
\label{def: committee_key} (Committee Key Scheme for Aggregatable Signatures) Let $\mathit{AS}$ be an aggregatable signature scheme that fulfils 
Definition~\ref{def:aggregate_signatures}.  A committee key scheme for aggregatable signatures consists of the following tuple of algorithms 
($\mathit{CKS.Setup}$, $\mathit{CKS.GenCommitteeKey}$, $\mathit{CKS.Prove}$, $\mathit{CKS.Verify}$) 
such that for implicit security parameter $\lambda$: 

\begin{itemize}
\item $(\mathit{pp}, \mathit{rs}_{\mathit{vk}}, \mathit{rs}_{\mathit{pk}}) \leftarrow \mathit{CKS.Setup}(v)$: a setup algorithm that, 
given an upper bound $v \in \mathbb{N}$, $v = \mathsf{poly}(\lambda)$ outputs some public parameters $\mathit{pp}$ and 
proving and verification keys $\mathit{rs}_{\mathit{pk}}$ and $\mathit{rs}_{\mathit{vk}}$, respectively,  
where $\mathit{pp} \leftarrow \mathit{AS.Setup}(\mathit{aux_{\mathit{AS}}})$, for some 
$\mathit{aux_{\mathit{AS}}}$ chosen by the aggregated signature $\mathit{AS}$.

\item $\mathit{ck} \leftarrow \mathit{CKS.GenCommitteeKey}(\mathit{rs}_{\mathit{pk}}, (\mathit{pk_i})_{i=1}^u)$: a committee key generation algorithm that, 
given a proving key $\mathit{rs}_{\mathit{pk}}$ and a list of public keys, 
outputs a committee key $\mathit{ck}$, where $u \leq v$.

\item $(\pi, \mathit{ck}) \leftarrow \mathit{CKS.Prove}(\mathit{rs}_{\mathit{pk}}, (\mathit{pk_i})_{i=1}^u, (\mathit{bit_i})_{i=1}^u)$: a proving algorithm that, 
given a proving key $\mathit{rs}_{\mathit{pk}}$, a list of public keys and a bitvector $(\mathit{bit_i})_{i=1}^u \in \{0,1\}^u$,  
outputs a proof $\pi$, where $u \leq v$; moreover, $\mathit{ck}$ is generated using \\ $\mathit{CKS.GenCommitteeKey}(\mathit{rs}_{\mathit{pk}}, (\mathit{pk_i})_{i=1}^u)$.
 
\item $0/1 \leftarrow \mathit{CKS.Verify}(\mathit{pp}, \mathit{rs}_{\mathit{vk}}, \mathit{ck}, m, \mathit{asig}, \pi, \mathbf{bitvector})$: a verification algorithm that, 
given public parameters $\mathit{pp}$, a verification key $\mathit{rs}_{\mathit{vk}}$, a committee key $\mathit{ck}$, a message $m$, a 
signature $\mathit{asig}$, a proof $\pi$ and a vector \\ $\mathbf{bitvector} \in \{0,1\}^*$, 
outputs $1$ if the verification succeeds and $0$ otherwise. 
\end{itemize}

\noindent We say ($\mathit{CKS.Setup}$, $\mathit{CKS.GenCommitteeKey}$, $\mathit{CKS.Prove}$, $\mathit{CKS.Verify}$) 
is a committee key scheme for aggregatable signatures if it satisfies \emph{perfect completeness} and 
\emph{soundness} as defined below.

\noindent \textbf{Perfect Completeness} A committee key scheme for aggregatable signatures 
($\mathit{CKS.Setup}$, $\mathit{CKS.GenCommitteeKey}$, $\mathit{CKS.Prove}$, $\mathit{CKS.Verify}$)
has perfect completeness if for any message $m \in \{0,1\}^*$, 
for any vector of public keys $(\mathit{pk_i})_{i=1}^{u}$ generated using $\mathit{AS.GenKeypair}(\mathit{pp})$, 
for any bitmask $(\mathit{bit_i})_{i=1}^{u} \in \{0,1\}^u$,  for any aggregated signature $\mathit{asig}$, 
it holds that: 
\vspace{-0.1cm}
\begin{align*}
\mathit{Pr}[&\mathit{AS.Verify}(\mathit{pp}, \mathit{apk}, m, \mathit{asig}) = 1 \implies 
\mathit{CKS.Verify}(\mathit{pp}, \mathit{rs}_{\mathit{vk}}, \mathit{ck}, m, \mathit{asig}, \pi, (\mathit{bit}_{i})_{i=1}^u) =1 | \\
& (\mathit{pp}, \mathit{rs}_{\mathit{vk}}, \mathit{rs}_{\mathit{pk}}) \leftarrow 
\mathit{CKS.Setup}(v), 
%\mathit{ck} \leftarrow \mathit{CKS.GenCommitteeKey}(\mathit{rs}_{\mathit{pk}}, (\mathit{pk}_{i})_{i=1}^u)), \\
(\pi, \mathit{ck}) \leftarrow \mathit{CKS.Prove}(\mathit{rs}_{\mathit{pk}}, (\mathit{pk}_{i})_{i=1}^u, (\mathit{bit_i})_{i=1}^u), \\
& \mathit{apk} \leftarrow \mathit{AS.AggKeys}(\mathit{pp}, (\mathit{pk}_{i})_{i:\mathit{bit_i}=1})]=1.
\end{align*} 
\vspace{-0.08cm}
\noindent \textbf{Soundness} A $\mathit{CKS}$ for aggregatable signatures \\
($\mathit{CKS.Setup}$, $\mathit{CKS.GenCommitteeKey}$, $\mathit{CKS.Prove}$, $\mathit{CKS.Verify}$)
has soundness if for every efficient adversary $\mathcal{A}$ it holds that: 
\begin{align*}
\mathit{Pr}[&\mathit{CKS.Verify}(\mathit{pp}, \mathit{rs}_{\mathit{vk}}, \mathit{ck}, m,  \mathit{asig}, \pi, (\mathit{bit}_{i})_{i=1}^u) =1 \
 \implies \\
 &\implies \ \mathit{AS.Verify}(\mathit{pp}, \mathit{apk}, m, \mathit{asig}) = 1  \  |  \ 
 (\mathit{pp}, \mathit{rs}_{\mathit{vk}},\mathit{rs}_{\mathit{pk}}) \leftarrow \mathit{CKS.Setup}(v),\\
& (\mathit{pk}_{i})_{i=1}^u, (\mathit{bit}_{i})_{i=1}^u, \mathit{asig}, \pi, m  \leftarrow \mathcal{A}(\mathit{pp}, \mathit{rs}_{\mathit{vk}}, \mathit{rs}_{\mathit{pk}}), \\
& \mathit{ck} \leftarrow \mathit{CKS.GenCommitteeKey}(\mathit{rs}_{\mathit{pk}}, (\mathit{pk}_{i})_{i=1}^u), \\
& \mathit{apk} \leftarrow \mathit{AS.AggKeys}( \mathit{pp},( \mathit{pk}_{i})_{i:\mathit{bit_i}=1})] = 1 - \negl(\lambda).
\end{align*}
\end{definition}

\noindent Next, we define \emph{unforgeability} which ensures that an adversary 
cannot forge a verifying aggregatable signature with a corresponding bitmask and a committee key 
that includes an honestly generated public key.\\
\vspace{-0.05in}

\noindent \textbf{Unforgeability}  For a committee key scheme for aggregatable signatures \\
($\mathit{CKS.Setup}$, $\mathit{CKS.GenCommitteeKey}$, $\mathit{CKS.Prove}$, 
$\mathit{CKS.Verify}$) the advantage of an 
adversary $\mathcal{A}$ against unforgeability is defined by \\
$\mathit{Adv}_{\mathcal{A}}^{\mathit{forgecomkey}}(\lambda) = \Pr[\mathit{Game}^{\mathit{forgecomkey}}_{\mathcal{A}}(\lambda) = 1]$, 
where
\begin{comment}
\begin{align*}
& \mathit{Game}^{\mathit{forgecomkey}}_{\mathcal{A}}({\lambda}): \\
& (\mathit{pp}, \mathit{rs}_{\mathit{vk}},\mathit{rs}_{\mathit{pk}}) \leftarrow \mathit{CKS.Setup}(v), \\
&((\mathit{pk}^*,\pi^*_{\mathit{PoP}}), \mathit{sk}^*) \leftarrow \mathit{AS.GenKeypair}(\mathit{pp}), Q \leftarrow \emptyset \\
& ((\mathit{pk_i}, \pi_{\mathit{PoP},i})_{i=1}^{u}, (\mathit{bit_i})_{i=1}^u, \mathit{asig}, \pi, m) \leftarrow \mathcal{A}^{\mathit{OSign}}(\mathit{pp},\mathit{rs}_{\mathit{vk}}, \\ &\mathit{rs}_{\mathit{pk}}, (\mathit{pk^*},\pi^*_{\mathit{PoP}})) \\
&\textit{If } ( i  \in [u], \mathit{pk}^* = \mathit{pk_i} \wedge \mathit{bit_i}=1) \vee m \in Q, \textit{ then return } 0 \\
& \textit{For } i \in [u] 
\end{align*}
\end{comment}
%\begin{comment}
\begin{align*}
&\mathit{Game}^{\mathit{forgecomkey}}_{\mathcal{A}}({\lambda}): \\
%& \mathit{pp} \leftarrow \mathit{AS.Setup}(\lambda) \\
& (\mathit{pp}, \mathit{rs}_{\mathit{vk}},\mathit{rs}_{\mathit{pk}}) \leftarrow \mathit{CKS.Setup}(v), \\
&((\mathit{pk}^*,\pi^*_{\mathit{PoP}}), \mathit{sk}^*) \leftarrow \mathit{AS.GenKeypair}(\mathit{pp}), Q \leftarrow \emptyset \\
& ((\mathit{pk_i}, \pi_{\mathit{PoP},i})_{i=1}^{u}, (\mathit{bit_i})_{i=1}^u, \mathit{asig}, \pi, m) \leftarrow \mathcal{A}^{\mathit{OSign}}(\mathit{pp},\mathit{rs}_{\mathit{vk}},\mathit{rs}_{\mathit{pk}}, (\mathit{pk^*},\pi^*_{\mathit{PoP}})) \\
%& \textit{If } (\forall i: \mathit{pk}^* \neq \mathit{pk_i} \wedge \mathit{bit_i}=0) \vee m \in Q, \textit{ then return } 0 \\
&\textit{If } (\nexists i  \in [u], \mathit{pk}^* = \mathit{pk_i} \wedge \mathit{bit_i}=1) \vee m \in Q, \textit{ then return } 0 \\
& \textit{For } i \in [u] \\
& \ \ \ \ \ \textit{ If } \mathit{AS.VerifyPoP}(\mathit{pp}, \mathit{pk_i}, \pi_{\mathit{PoP,i}})=0  \textit{ return } 0 \\
& \mathit{ck} \leftarrow \mathit{CKS.GenCommitteeKey}(\mathit{rs}_{\mathit{pk}}, (\mathit{pk_i})_{i=1}^u) \\
& \textit{Return } \mathit{CKS.Verify}(\mathit{pp}, \mathit{rs}_{\mathit{vk}},  \mathit{ck}, m, \mathit{asig}, \pi, (\mathit{bit_i})_{i=1}^u)
\end{align*}
%\end{comment}
and 
\begin{align*}
& \mathit{OSign}(m_j): \\
& \sigma_j \leftarrow \mathit{AS.Sign}(\mathit{pp}, \mathit{sk}^*, m_j); Q \leftarrow Q \cup \{m_j\}; \textit{Return} \ \sigma_j
\end{align*}

\noindent A committee key scheme for aggregatable signatures is unforgeable if for all efficient adversaries $\mathcal{A}$ it holds that 
$\mathit{Adv}_{\mathcal{A}}^{\mathit{forgecomkey}}(\lambda) \leq \negl(\lambda)$.
%\vspace{-0.05in}
\begin{corollary} Let $\mathit{AS}$ be an aggregatable signature scheme that fulfils  
definition~\ref{def:aggregate_signatures}. If $\mathit{CKS}$ is a committee key scheme for aggregatable signatures that fulfils Definition~\ref{def: committee_key}, 
then $\mathit{CKS}$ is unforgeable, as defined above.  
\end{corollary}

\begin{proof}[Proof Sketch] Assume by contradiction there exists an efficient adversary $\mathcal{A}$ such that 
$\mathit{Adv}_{\mathcal{A}}^{\mathit{forgecomkey}}(\lambda)$ is non-negligible. Using $\mathcal{A}$ and the 
soundness property of a committee key scheme, one can construct in a straightforward manner an efficient 
adversary $\mathcal{A'}$ such that $ \mathit{Adv}_{\mathcal{A'}}^{\mathit{forge}}(\lambda) \geq \mathit{Adv}_{\mathcal{A}}^{\mathit{forgecomkey}}(\lambda)  - \mathit{negl}(\lambda).$ 
This, in turn, implies that $\mathit{Adv}_{\mathcal{A'}}^{\mathit{forge}}$ is non-negligible which contradicts the unforgeability property of aggregatable 
signature scheme $\mathit{AS}$. Thus, our assumption is false and our statement holds.
\end{proof}
\vspace{-0.15in}

\subsection{Conditional NP Relations}
\label{sec:conditional_relations}
%\vspace{-0.01in}
\noindent By $\mathcal{R} =\{(x;w): p(x,w) = 1 \}$ we denote the binary relation such that $(x,w)$ 
fulfil predicate $p(x,w) = 1$. We say $\mathcal{R}$ is an NP relation if predicate $p$ can be checked in polynomial 
time in the length of both inputs $x$ and $w$ and $\mathcal{L}(\mathcal{R})= \{x \ | \ \exists w \textit{ s.t. } (x,w) \in \mathcal{R} \}$ 
is an NP language w.r.t. predicate $p$. In such a case we call $x$ an \emph{instance} and $w$ a \emph{witness}.  \\

%\vspace{-0.07in}
\noindent  In order to model a specific property of our NP relations, we introduce further notation which we call \emph{conditional NP relation}, we denote it by 
$$\mathcal{R}^c = \{(x;w) : (p_1(x,w) =1 \ | \ c(x,w) =1) \ \wedge \ p_2(x,w) = 1 \}$$ and we interpret it as the NP relation containing the pairs of inputs and witnesses 
$(x,w)$ such that $c(x,w) =1$, $p_1(x,w) = 1$ and $p_2(x,w) =1$ hold. However, in order to prove that $(x,w) \in \mathcal{R}^c$ we assume/take it as a given that 
$c(x,w) =1$ and we are left to prove only that $p_1(x,w) = 1$ and $p_2(x,w) =1$ hold. The reason we separate predicate $c(x,w)$ from predicate $p_1(x,w)$ in the definition 
of $\mathcal{R}^c$ is that predicate $c(x,w)$ may be inefficient to prove inside a proof system (e.g., in our case, inside a SNARK); using the above separation, one can delegate 
(in some particular situations) the verification of $c(x,w)$ to a trusted party outside the proof system.\\

%\vspace{-0.07in}
\noindent We explicitly include in the definition of any NP relation $\mathcal{R}$ or $\mathcal{R}^c$ the corresponding domain for each type 
of public input. The interpretation of such domains is that each type of public input is parsed by the honest parties (e.g., a SNARK verifier for an NP relation 
$\mathcal{R}$ or $\mathcal{R}^c$) as per the definition of the respective domain, without additional checks. We assume that all our relations have been 
generated using implicit security parameter $\lambda$. Finally, when we make a statement about an NP relation we implicitly 
mean the statement is about a conditional relation $\mathcal{R}^c$, where $c$ may be the predicate that always outputs $1$. 

%\noindent {\color{red} In order to model a specific property of our NP relations, we introduce further notation which we call \emph{conditional NP relation} 
%and we denote it by $\mathcal{R}^c = \{(x;w) \  | \ c(x,w): p(x,w) \}$ which means that $x$ and $w$ fulfil predicate $p$ as long as $x$ and $w$ fulfil additional 
%predicate $c$. As a special case, we will encounter the situation that the witness for each public input is, in fact, the empty string; this is denoted by 
%$\mathcal{R}^c= \{(x;) \ | \ c(x) : p(x)\}$. The reason we separate predicate $c(x,w)$ from predicate $p(x,w)$ in the definition of $\mathcal{R}^c$ 
%is that predicate $c(x,w)$ may be expensive to prove and/or verify; by separating them, one can delegate the verification of $c(x,w)$ 
%to a trusted party with enough computational power. In turn, such a trusted party can be implemented accordingly in a 
%real-world system.}
%\vspace{-0.09in}
\subsection{O-SNARKs}
\label{sec:short_snarks_defs}
In the following, we remind the reader the definition of an O-SNARK from~\cite{O_SNARK} and we prove that {\color{red}XXX}.
In order to do that, we start with a couple building block definitions: i.e., for algebraic adversaries (see, for example~\cite{AGM_model}) 
and our new definition of AGM respecting oracles.

\begin{definition}[Algebraic Adversaries]
\label{def:algebraic_adv}
\end{definition}

\begin{definition}[AGM Respecting Oracles]
\label{def:agm_oracles}
\end{definition}

\begin{definition}[O-SNARKs]
\label{def:osnarks}
\end{definition}

\begin{theorem}[O-SNARKS with AGM Respecting Oracles]
\label{the:when_osnarks} 
Every $Z$ auxiliary input SNARK $\Pi$ secure in the AGM model is an 
O-SNARK for $\mathbb{O}$ if $\mathbb{O}$ is AGM respecting and 
$Z$ is defined as the probability distribution of all the public parameters that define 
$\mathbb{O}$ together with all the polynomial number $Q$ of queries and answers that 
the adversarial prover in the O-SNARK makes to the oracle $\mathbb{O}$.
\end{theorem}

\begin{proof}
\end{proof}



\subsection{Ranged Polynomial Protocols and Polynomial Commitments}
In order to prove the security of the SNARKs designed in this work we use a SNARK compiler inspired by the one provided in lemma 4.7 from 
PLONK~\cite{plonk}. In more detail, for each of our three conditional NP relations we describe a ranged polynomial protocol and then we use our compiler to obtain three SNARKs 
secure in the AGM. We remind the definition of ranged polynomial protocols in section~\ref{sec:poly_protocols_appendix} in the appendix. Moreover, we also make use of 
KZG polynomial commitments \cite{KZG_10}, in particular their batched version and their security definitions as described in section 3 from PLONK. For brevity, 
and since we do not make any alterations to the definition of batched KZG commitments, we do not repeat it in this initial version of our work but invite the reader 
to review them, if necessary, by following the reference provided. 

%In order to prove the security of the snarks designed in this work we use the snark compiler proposed in lemma 4.7 from 
%PLONK~\cite{plonk}. In more detail, for each of our snarks, we describe an $H$-ranged polynomial protocol for a relation $\mathcal{R}_i$, 
%where the relations $\mathcal{R}_i, i \in \{1,2,3\}$ are chosen to model certain statements we are interested in with respect to a set of 
%BLS public keys and their simple aggregate. Each $H$-ranged polynomial protocol for a relation $\mathcal{R}_i$ follows the definitions 4.1 and 4.3 
%with the additional clarifications given in the beginning of section 4.1 of PLONK. In our case, $H$ is an appropriately chosen multiplicative 
%subgroup as defined in section \ref{sec:lagrange} such that the fast Fourier transforms (FFTs) performed by the snarks provers are efficient. \\


\subsection{Lagrange Bases}
\label{sec:lagrange}

In order to design the SNARKs presented in this work, it is more convenient to represent the polynomials 
we work with over the Lagrange base rather than the monomial base. Formally, for the finite field $\mathbb{F}$ defined in section~\ref{sec:pairings} 
we denote by $H$ a subgroup of the multiplicative group of $\mathbb{F}$ such that $n = |H|$ is a large power of 2. Let $\omega$ be an $n$-th 
root of unity in $\mathbb{F}$ such that $\omega$ is a generator of $H$. Then, we call the following polynomial base $\{L_i(X)\} _{0 \leq i\leq n-1}$ 
a Lagrange base, where $\forall i, 0 \leq i \leq n-1$, $L_i(X)$ is the unique polynomial in $\mathbb{F}_{<n}[X]$ such that 
$L_i(\omega^i) =1$ and $L_i(\omega^j) = 0, \forall j \neq i$.\\

\noindent Independent of the notion of Lagrange bases, but related to $n$ we define $\block$ also a power of 2 such that $\block < n$. 
We use $\block$ when defining one of our conditional NP relations in section \ref{sec:snarks}. In the following we assume 
$n = \mathsf{poly} (\lambda)$ and $\block = \Theta(\lambda)$ and $|\mathbb{F}|= 2^{\Theta(\lambda)}$.%, i.e., $|\mathbb{F}|$ is exponential in $\lambda$.}

 
%{\color{blue} TO DO: Make the statement about $|\mathbb{F}|$ stronger. In the following we assume 
%$n= \mathsf{poly} (\lambda)$ and $|\mathbb{F}|= \lambda^{\omega(1)}$, i.e., $|\mathbb{F}|$ is super-polynomial in $\lambda$. Also, is $\block$ a constant or is $\mathsf{poly} (\lambda)$? }



\section{Custom SNARKs for  Aggregation } \label{sec_apk_proofs}
\label{sec:snarks}

In this section we motivate the construction of two related SNARKS, each of them allowing a prover to convince an 
efficient verifier that an alleged aggregated public key has indeed been computed correctly as an aggregate 
of a vector of public keys for which two succinct commitments (to the $x$ and $y$ affine coordinates of points) 
are publicly known. The differences between the two constructions stem from how a \emph{bitvector} with one bit associated to each public key 
(necessary to signal the inclusion or omission of the respective public key w.r.t. the aggregate key) 
is used as part of the verifier's public input. We describe a 
\emph{basic accountable SNARK} (the bitvector is represented as a sequence of $0/1$ field elements) and a \emph{packed accountable SNARK} (the bitvector is 
partitioned into equal blocks of consecutive binary bits which are represented by one field element per block). 
%Each of our SNARKs implements a conditional NP relation bearing the 
%same name as the SNARK it implements. Note that the names ``basic accountable" (for short, ``basic'') and 
%``packed accountable" (for short, ``packed'') do not refer to the security of the respective SNARK but they summarise properties 
%of the underlying sets of constraints that define the SNARKs, and, hence their use case. 
We finally re-cast our basic and packed accountable SNARKs into SNARKs for specific types of conditional NP relations necessary for modelling and 
building accountable light client systems. %For reasons of space, the packed accountable scheme is described in full in Appendix \ref{sec_a}.

\noindent To compile our desired SNARKs we proceed as follows:
\begin{itemize}
\item In Sections \ref{sec_la} and \ref{sec_a} we define vector-based conditional NP relations $\Rla$ (i.e., basic accountable) and $\Ra$ (packed accountable) and we design two ranged polynomial protocols for these relations. The ranged polynomial protocol notion originates in~\cite{plonk}; 
we review it and define a refinement of it in Appendix~\ref{supplementary_poly_protocols_appendix};  
\item In Appendix~\ref{sec_two_step_compiler} we define a two-steps PLONK-inspired compiler which we use to compile the ranged polynomial protocols into 
SNARKs for two novel mixed vector and trusted polynomial commitments conditional NP relations which we denote by 
$\Rlacom$ and $\Racom$, respectively. 
\item In Section~\ref{sec:inst_committee_key} we give an instantiation for committee key scheme for aggregatable signatures which uses our custom SNARKs 
(obtained using our two-step compiler) and our instantiation for BLS aggregatable signatures from Appendix~\ref{sec:bls}. 
%\item We include in Section~\ref{suplementary_plonk_comparison} in the appendices a comparison between PLONK~\cite{plonk} and our custom SNARKs. 
\end{itemize}

\noindent We define our two specific conditional NP relations over $\mathbb{F}$, i.e., the base field of $\einn$. 
Our SNARKs provers' circuits are defined as well over $\mathbb{F}$ as the scalar field of $\eout$. The vector of public keys, which is part of the public input for both of our 
relations $\Rla$ and $\Ra$, and is denoted by $\mathbf{pk} = (\mathit{pk_0}, \ldots, \mathit{pk_{n-2}})$, is a vector of pairs with each component 
in $\mathbb{F}$. This vector has size $n-1$ ($n$ defined in 
Section~\ref{sec:lagrange}). For $\Rla$ we denote 
the $n$ components bitvector by $\mathbf{bit} = (\mathit{bit_0}, \ldots, \mathit{bit_{n-1}})$ 
(meaning that each component belongs to the set $\{0,1\} \subset \mathbb{F}$), 
while the $\Ra$ relation is defined using the \emph{compacted bitvector} 
$\mathbf{b'} = (\mathit{b'_{0}}, \ldots, \mathit{b'_{\frac{n}{\block}-1}})$ of $\frac{n}{\block}$ field elements, 
each of which is $\block$ binary bits long ($\block$ has been defined in Section~\ref{sec:lagrange}). 
Each of the bits in the bit representation of these field elements signals the 
inclusion (or exclusion) of the index-wise corresponding public keys into the aggregated public key $\mathit{apk}$. The last bit of field element $\mathit{b'_{\frac{n}{\block}-1}}$ as well as the $n$-th component $\mathit{bit_{n-1}}$ do not correspond to any public key, 
but, as will become clear in the following, they have been included for easier design of constraints. \\ 
\vspace{-0.009in}
\noindent We denote by $H$ the multiplicative subgroup of $\mathbb{F}$ generated 
by $\omega$ as defined in Section~\ref{sec:lagrange}. We denote by $\mathit{incl}(a_0, \ldots, a_{n-2})$ the inclusion 
predicate that checks if  $(a_0, \ldots, a_{n-2}) \in \ginn{1}^{n-1}$. Moreover let $h = (\mathit{h_x}, \mathit{h_y})$ 
be some fixed, publicly known element in $\einn \setminus \ginn{1}$. In case this is not possible, we chose $h$ according to Appendix~\ref{sec:other_choice_h}.  
Finally, we denote by $(a_x, a_y)$ the affine representation in 
$x$ and $y$ coordinates of $a \in \einn$ and by $\oplus$ the point addition in affine coordinates on the elliptic curve $\einn$. 
We denote $\mathbb{B} = \{0,1\} \subset \mathbb{F}$. \\
\vspace{-0.15in}

%\noindent Finally, as mentioned in section~\ref{sec:conditional_relations}, the interpretation of adding explicit domains to public 
%inputs in the definition of conditional NP relations is that the honest parties (in our case, both the polynomial protocol verifiers and the SNARKs verifiers 
%as defined in this section below) parse the public inputs according to the specified domains without any further checks. Any checks or 
%computations that the honest parties perform regarding the public inputs are explicitly described as part of the protocols followed by 
%the honest parties.
\vspace{-0.1in}
\subsection{Basic Accountable Protocol}
\label{sec_la}
%\vspace{0.1in}
%\noindent We start by describing our conditional basic accountable relation $\Rla$ and the 
%corresponding $H$-ranged polynomial protocol $\Pla$. %For brevity, we omit the security parameter 
%$\lambda$ whenever we refer to any conditional NP relations for which we build SNARKs. \\
%\vspace{-0.05in}
\noindent \textsf{Conditional Basic Accountable Relation $\Rla$}  
\vspace{-0.055in}
\begin{equation*}
\begin{split}
\Rla =  \{(\mathbf{pk} \in ({\mathbb{F}^2})^{n-1}, \mathbf{bit} \in \mathbb{B}^n,
\mathit{apk} \in \mathbb{F}^2; \_) : \mathit{apk} = \sum_{i=0}^{n-2} [\mathit{bit_i}] \cdot \mathit{pk_i} \ | \ \mathbf{pk} \in \ginn{1}^{n-1} \} 
\end{split}
\end{equation*}
\vspace{-0.008in}
\noindent where $\mathbf{pk} = (\mathit{pk_0}, \ldots, \mathit{pk_{n-2}})$ and $\mathbf{bit} = (\mathit{bit_0}, \ldots, \mathit{bit_{n-1}})$.\\ 

\noindent Next, we introduce the following Lagrange interpolation polynomials of degree at most $n-1 = |H|$ over cyclic group $H$ (Section~\ref{sec:lagrange}): 
$b(X)$ - interpolates the bits of bitvector $\mathit{bit}$; $pkx(X)$, $pky(X)$ - interpolate all public keys' 
$x$ and $y$ coordinates, respectively; $kaccx(X)$, $kaccy(X)$ - interpolate $x$ and $y$ coordinates, respectively, 
of the iterative partial aggregate sum of the actual signing validators' public keys. We also define five polynomial identities $id_1(X), \ldots, id_5(X)$ supporting the following intuition: $id_1(X)$, $id_2(X)$ 
ensure the $x$ and, respectively, the $y$ coordinates of the iterative partial aggregate sums of actual signing validators public keys (up to each index $i \leq n-2$) 
follow formulas $(\ast)$, $(\ast\ast)$ from Observation~\ref{obs} which gives all possible cases of 
complete curve point addition when the second curve point is multiplied by a bit; $id_3(X)$, $id_4(X)$ 
ensure first partial aggregate sum is $h$ and the total aggregate sum is $h + \mathit{apk}$; this is necessary in order to ensure the addition of the public keys (i.e., elliptic curve points) 
never falls into condition (3) defined in Observation~\ref{obs}, which recursively implies the partial aggregate sum at every step is a well defined curve point, hence, it is a suitable input for the next step consisting of an elliptic curve addition; 
$id_5(X)$ ensures $b(X)$ evaluates to bits over $H$. Together, $id_1(X)$ to $id_4(X)$ define 
the $H$-ranged polynomial protocol $\Pla$ for relation $\Rla$; $id_5(X)$ will be used to prove a more general result, applicable also for the 
$H$-ranged polynomial protocol $\Pa$ for relation $\Ra$ (Section~\ref{sec_a}). In more detail, we have:  \\

\noindent \textsf{Polynomials as Computed by Honest Parties} 
\vspace{-0.1cm}
\begin{align*}
&\mathsf{b(X)} = \sum_{i=0}^{n-1} \mathit{bit_i} \cdot \mathsf{L_i(X)}; 
 \mathsf{pkx(X)} =  \sum_{i=0}^{n-2} \mathit{pkx_i} \cdot \mathsf{L_i(X)}; \mathsf{pky(X)} =  \sum_{i=0}^{n-2} \mathit{pky_i} \cdot \mathsf{L_i(X)} \\
&\mathsf{kaccx(X)}  =  \sum_{i=0}^{n-1} \mathit{kaccx_i} \cdot \mathsf{L_i(X)}; \mathsf{kaccy(X)}  = \sum_{i=0}^{n-1} \mathit{kaccy_i} \cdot \mathsf{L_i(X)}, 
\end{align*}
\noindent where $(\mathit{pkx_0}, \ldots, \mathit{pkx_{n-2}})$ 
and $(\mathit{pky_0}, \ldots, \mathit{pky_{n-2}})$ are computed such that\\ $\forall i \in \{0, \ldots, n-2\}$, $\mathit{pk_i}$ 
is interpreted as a pair $(\mathit{pkx_i}, \mathit{pky_i})$ with its components in $\mathbb{F}$; we also have 
$$(\mathit{kaccx_{0}}, \mathit{kaccy_{0}}) = (\mathit{h_x}, \mathit{h_y})$$
$$(\mathit{kaccx_{i+1}}, \mathit{kaccy_{i+1}}) = (\mathit{kaccx_{i}}, \mathit{kaccy_{i}}) \oplus \mathit{bit_i}(\mathit{pkx_{i}}, \mathit{pky_{i}}), \forall i < n-1.$$ %Note that in the last relation $\mathit{bit_i}$ is not interpreted as a field element anymore but as a binary bit.

\vspace{0.1cm}

\noindent \textsf{Polynomial Identities} 
\begin{align*}
id_1(X) = & (X-\omega^{n-1}) \cdot \\
& \cdot [b(X) \cdot ((kaccx(X)-pkx(X))^2 \cdot (kaccx(X)+ pkx(X) + kaccx(\omega\cdot X)) - \\
&- (pky(X) - kaccy(X))^2) + (1-b(X)) \cdot (kaccy(\omega\cdot X) - kaccy(X))]. \\
id_2(X)  =  & (X-\omega^{n-1})\cdot \\ 
& \cdot [b(X) \cdot ((kaccx(X) - pkx(X)) \cdot (kaccy(\omega \cdot X) + kaccy(X)) -  \\
& - (pky(X) - kaccy(X)) \cdot (kaccx(\omega \cdot X) - kaccx(X))) + \\
& + (1-b(X)) \cdot (kaccx(\omega \cdot X) - kaccx(X))]. \\
id_3(X)  =  & (kaccx(X) - h_x)\cdot L_0(X) + (kaccx(X) - (h\oplus apk)_{x})  \cdot L_{n-1}(X).  \\
id_4(X) =  & (kaccy(X) - h_y)\cdot L_0(X) + (kaccy(X)  - (h\oplus apk)_{y}) \cdot L_{n-1}(X). \\
id_5(X) =  & b(X)(1-b(X)).
\end{align*}

\noindent %Polynomial identity 
$id_5(X)$ is not strictly needed for defining ranged polynomial protocols 
for $\Rla$, but included here to ease presentation and for consistency 
of results with those related to $\Ra$ described in Section~\ref{sec_a}. \\
\vspace{-0.10in}

\noindent \textsf{$H$-ranged Polynomial Protocol $\Pla$ for Conditional NP Relation $\Rla$} describes the interaction of the prover 
$\mathcal{P}_{poly}$, the verifier $\mathcal{V}_{poly}$ and the trusted third party $\mathcal{I}$ 
in accordance to Definition~\ref{def_ranged_poly_protocol} from Appendix~\ref{supplementary_poly_protocols_appendix}. \\
\vspace{-0.15in}
%\noindent \textsf{Protocol $\Pla$} \\

\noindent $\mathcal{P}_{poly}$ and $\mathcal{V}_{poly}$ know public input 
$\mathbf{bit} \in \mathbb{B}^n$, $\mathbf{pk} \in (\mathbb{F}^2)^{n-1}$ and $\mathit{apk} \in (\mathbb{F})^2$ 
which are interpreted as per their %respective 
domains.
\begin{enumerate}
\item $\mathcal{V}_{poly}$ computes $b(X)$, $pkx(X)$, $pky(X)$.
\item $\mathcal{P}_{poly}$ sends polynomials $kaccx(X)$ and $kaccy(X)$ to $\mathcal{I}$. 
\item $\mathcal{V}_{poly}$ asks $\mathcal{I}$ to check whether the following polynomial relations hold over range $H$ 
$$id_i(X) = 0, \forall i \in [4].$$
\item $\mathcal{V}_{poly}$  accepts if $\mathcal{I}$'s checks verify. 
\end{enumerate}

\noindent We show $\Pla$ is an $H$-ranged polynomial protocol for conditional NP relation $\Rla$. For this, we first prove:
\vspace{-0.05in}
\begin{test_claim} Assume that $\forall i < n-1$ such that $\mathit{bit}_i = 1$, we have that $$pk_i = (pkx_i, pky_i) \in \ginn{1}.$$ 
If polynomial identities $id_i(X) = 0, \forall i \in [5],$ hold over range 
$H$ and the polynomial $b(X)$ has been constructed via interpolation on $H$ such that $$b(\omega^i) = \mathit{bit}_i, \forall i <n$$ then the following four properties hold:
$$\mathit{bit}_i \in \mathbb{B} = \{0,1\} \subset \mathbb{F}, \forall i <n$$ 
$$(kaccx_{0}, kaccy_{0}) = (h_x, h_y)$$ 
$$(kaccx_{n-1}, kaccy_{n-1}) = (h_x, h_y)\oplus (apk_x, apk_y)$$ 
$$(kaccx_{i+1}, kaccy_{i+1}) =  (kaccx_{i}, kaccy_{i}) \oplus \mathit{bit}_i(pkx_{i}, pky_{i}), \forall i < n-1.$$
%where in the last relation $\mathit{bit_i}$ should not be interpreted as a field element but as a binary bit
\label{claim:keys_affine_comm}
\end{test_claim}
\vspace{-0.7cm}
\begin{proof} Everything but the last property in the claim is easy to derive from polynomial identities 
$id_3(X) =0, id_4(X )= 0, id_5(X) = 0$ holding over $H$. To prove the remaining property, we remind 
the incomplete addition formulae for curve points in affine coordinates, over elliptic curve in short Weierstrasse form and state:\\ 
\vspace{-0.3cm}

\begin{observation} \label{obs} Suppose that $\mathit{bit} \in \{0,1\}$, $(x_1,y_1)$ is a point on an elliptic curve in 
short Weierstrasse form, and, if $\mathit{bit} = 1$, so is $(x_2,y_2)$. We claim that the following equations: 
\begin{align*}
&\mathit{bit}((x_1 - x_2)^2 (x_1 + x_2 + x_3) - (y_2 - y_1)^2 ) + (1 - \mathit{bit})(y_3 - y_1) =0 \ (\ast)\\
&\mathit{bit}((x_1 - x_2)(y_3 + y_1) - (y_2 - y_1)(x_3 - x_1)) + (1 - \mathit{bit})(x_3 - x_1) =0 \ (\ast\ast)
\end{align*}

\noindent hold if and only if one of the following three conditions hold 

\begin{enumerate}
\item \label{cond1} $\mathit{bit}=1$ and $(x_1,y_1)\oplus(x_2,y_2)=(x_3,y_3)$ and $x_1 \neq x_2$
\item \label{cond2} $\mathit{bit}=0$ and $(x_3,y_3)=(x_1,y_1)$ 
\item  \label{cond3} $\mathit{bit}=1$ and $(x_1,y_1)=(x_2,y_2)$\footnote{Note that under condition~(\ref{cond3}), $(x_3,y_3)$ 
can be any point whatsoever, maybe not even on the curve. The same holds true for $(x_2, y_2)$ under the condition~(\ref{cond2}).}.
\end{enumerate}
\end{observation}

\noindent It is easy to see that each of the conditions~(\ref{cond1}),(\ref{cond2}),(\ref{cond3}) above implies equations $(\ast)$ and $(\ast \ast)$.
\noindent For the implication in the opposite direction, if we assume that $(\ast)$ and $(\ast \ast)$ hold, then \\
\vspace{-0.1in}

\noindent \textit{Case a:} For $\mathit{bit}=0$, the first term of each equation $(\ast)$ and $(\ast \ast)$ vanishes, 
leaving us with $y_3-y_1=0$ and $x_3-x_1=0$ which are equivalent to condition~(\ref{cond2}). \\
\vspace{-0.1in}

\noindent \textit{Case b:} For $\mathit{bit}=1$ and $x_1=x_2$, by simple substitution in $(\ast)$ and $(\ast \ast)$, 
we obtain $y_1 = y_2$, i.e., condition~(\ref{cond3}).  \\
\vspace{-0.1in}

\noindent \textit{Case c:} For $\mathit{bit}=1$ and $x_1 \neq x_2$, then we can substitute
$\beta=\frac{y_2-y_1}{x_2-x_1}$ into equations $(\ast)$ and $(\ast \ast)$, leaving us with
$$x_1+x_2+x_3=\beta^2 \textrm{ and } y_3+y_1=\beta(x_3-x_1).$$
which are the usual formulae for short Weierstrass form addition of affine coordinate points when $x_1 \neq x_2$ 
so this is equivalent to condition~(\ref{cond1}). \\
\vspace{-0.1in}

\noindent We apply the above Observation~\ref{obs} by noticing that if $id_1(X)$ and $id_2(X)$ hold over $H$, 
then $(\ast)$ and $(\ast \ast)$ hold with $(x_1, y_1)$ substituted by $(kaccx_i,kaccy_i)$, $(x_2, y_2)$ 
substituted by $(pkx_i, pky_i)$, $(x_3, y_3)$ substituted by $(kaccx_{i+1},kaccy_{i+1})$ and $\mathit{bit}$ 
substituted by $\mathit{bit}_i$ for $0 \leq i \leq n-2$ %, where $\mathit{bit_i}$ should not be interpreted as a field element but as binary  bit
. Moreover, since $(kaccx_{0}, kaccy_{0}) = (h_x, h_y) \in \einn \setminus \ginn{1}$ 
and if $(pkx_i, pky_i) \in \ginn{1}$ whenever $\mathit{bit}_i = 1$, then $\forall i < n-1$ 
equations $(\ast)$ and $(\ast \ast)$ obtained after the substitution defined above are equivalent to either 
condition~(\ref{cond1}) or condition~(\ref{cond2}), but never condition~(\ref{cond3}), so the result of the sum (i.e., $(kaccx_{i+1}, kaccy_{i+1})$, $0\leq i \leq n-2$) is, 
by induction, at each step a well-defined point on the curve.% and this concludes our proof.
\end{proof}
\vspace{-0.1in}

\begin{corollary} Assume $\forall i < n-1$ 
such that $\mathit{bit}_i = 1$, $pk_i = (pkx_i, pky_i) \in \ginn{1}$. 
If the polynomial identities $id_i(X) = 0, \forall i \in [4],$ hold over range $H$ and 
$\mathit{bit_i} \in \mathbb{B}$, $\forall i < n-1$ and $b(X) = \sum_{i=0}^{n-1} \mathit{bit_i} \cdot L_i(X)$
then:  \\
$$(kaccx_{0}, kaccy_{0}) = (h_x, h_y)$$ 
$$(kaccx_{n-1}, kaccy_{n-1}) = (h_x, h_y) \oplus (apk_x, apk_y)$$
$$(kaccx_{i+1}, kaccy_{i+1}) =  (kaccx_{i}, kaccy_{i}) \oplus \mathit{bit_i}(pkx_{i}, pky_{i}), \forall i < n-1.$$
%where in the last relation $\mathit{bit_i}$ should not be interpreted as a field element but as a binary bit.
\label{corollary:keys_affine_comm}
\end{corollary}
\vspace{-1cm}
\begin{proof}The proof follows trivially from the general result stated by Claim~\ref{claim:keys_affine_comm}. 
\end{proof}
\vspace{-0.1in}

\begin{lemma} 
\label{le:ba}
$\Pla$ as described above is an $H$-ranged polynomial 
protocol for conditional NP relation $\Rla$.
\end{lemma}
\vspace{-0.15in}

\begin{proof}
If $(\mathbf{bit},\mathbf{pk}, \mathit{apk}) \in \Rla$ holds, 
meaning that $\mathbf{bit} \in \mathbb{B}^n$ and $\mathbf{pk} \in \ginn{1}^{n-1}$ and $$\mathit{apk} = \sum_{i=0}^{n-2} [\mathit{bit_i}] \cdot \mathit{pk_i}$$ hold, 
then it is easy to see that the honest prover $\mathcal{P}_{poly}$ in $\Pla$ will convince the honest verifier $\mathcal{V}_{poly}$ in 
$\Pla$ to accept with probability $1$ so perfect completeness holds. 
For knowledge-soundness, if the verifier $\mathcal{V}_{poly}$ in $\Pla$ accepts, 
then the extractor $\mathcal{E}$ is trivial since $\Rla$ has no witness.  
We need to show that if $\mathbf{pk} \in \ginn{1}^{n-1}$ and the verifier in $\Pla$ accepts, 
then $$(\mathbf{bit},\mathbf{pk}, \mathit{apk}) \in \Rla$$ holds, which given our definition for conditional relation is 
equivalent to proving that $$\mathit{apk} = \sum_{i=0}^{n-2} [\mathit{bit_i}] \cdot \mathit{pk_i}$$ holds. This is due to 
Corollary~\ref{corollary:keys_affine_comm}. \end{proof}
\vspace{-0.15in}

\subsection{Packed Accountable Protocol}
\label{sec_a}
\section{Packed Accountable Protocol}
\label{sec_a}

Here we describe the packed accountable protocol, which differs from the basic accountable protocol gievn in Section \ref{sec_la}, in that the verifier performs far fewer field operations. It achieves this by partitioning the bitbector into field elements differently, instead of interpreting the bitvector is as a sequence of $0/1$ field elements, instead the bitvector is divided into sequences of $\block$ bits which are interpreted as field elements. Unpacking these to bits in the SNARK adds some complexity, but the verifier has to deal with $1/\block$ times fewer field elements and operations.


Let $\mathbb{F}_{|\block|} = \{0, \ldots, 2^{\block -1} \}$. %, with $\block$ defined in Section~\ref{sec:lagrange}.
Our conditional packed accountable relation $\Ra$ and the corresponding $H$-ranged polynomial protocol 
$\Pa$ are:\\
 
\noindent \textsf{Conditional Packed Accountable Relation $\Ra$} 
\begin{equation*}
\begin{split}
\Ra = & \{(\mathbf{pk} \in (\mathbb{F}^2)^{n-1},\mathbf{b'} \in \mathbb{F}_{|\block|}^{\frac{n}{\block}},
\mathit{apk} \in \mathbb{F}^2; \mathbf{bit}) : \\ 
 & \mathit{apk} = \sum_{i=0}^{n-2} [\mathit{bit_i}] \cdot \mathit{pk_i} \ | \ \mathbf{pk} \in \ginn{1}^{n-1} \ \wedge \\
 & \wedge \mathbf{bit} \in \mathbb{B}^n  \wedge b'_{j} = \sum_{i=0}^{\block -1}2^i \cdot \mathit{bit_{\block \cdot j + i}}, \forall j < \frac{n}{\block} \} 
\end{split}
\end{equation*}
where $\mathbf{b'} = (b'_{0}, \ldots, b'_{{\frac{n}{\block}} -1})$.

\noindent \textsf{New Polynomials as Computed by Honest Parties} 
\begin{align*}
aux(X) = & \sum_{i=0}^{n-1}aux_i \cdot L_i(X); c_{a}(X) \\
& = \sum_{i=0}^{n-1} c_{a,i} \cdot L_i(X); acc_{a}(X)  =  \sum_{i=0}^{n-1} acc_{a,i}  \cdot L_i(X)
\end{align*}

\noindent where $aux_{i} = 1 \in \mathbb{F}$ if $i$ is divisible with $\block$ and $aux_{i} = 0 \in \mathbb{F}$ otherwise, $\forall i < n$ 
and $c_{a,i} = 2^k \cdot r^j$, $k = i \mod \block$, $j = i \div \block$, $\forall i < n$ ($r \in \mathbb{F}$ is introduced in protocol $\Pa$) and $acc_{a,i}$ are components of $(0, \mathit{bit}_0 \cdot c_{a,0}, \mathit{bit}_0 \cdot c_{a,0}+ \mathit{bit}_1  \cdot c_{a,1}, \ldots, \sum_{i=0}^{n-2}\mathit{bit}_i \cdot c_{a,i})$, where $\mathit{bit_{0}}, \ldots, \mathit{bit_{n-1}}$ represent the first $n$ 
bits of the concatenation of the binary representation of 
$\mathit{b'_{0}}, \ldots, \mathit{b'_{\frac{n}{\block}-1}}$ each 
padded  with 0s  if necessary, to have an individual length of $\block$ bits. With this definition of $(\mathit{bit_{0}}, \ldots, \mathit{bit_{n-1}})$, $b(X)$ remains the same as in Section~\ref{sec_la}.\\

\noindent \textsf{New Polynomial Identities} 
\begin{align*}
id_6(X) & =  c_{a}(\omega \cdot X) - c_{a}(X) \cdot L_{n-1}(X). \\
& \cdot (2+ (\frac{r}{2^{\block -1}} -2)  \cdot aux(\omega \cdot X)) - (1 - r^{\frac{n}{\block}}) \\
id_7(X) & = acc_{a}(\omega \cdot X) - acc_{a}(X) - b(X)\cdot c_{a}(X)  \\
& +  \mathsf{sum} \cdot L_{n-1}(X),
\end{align*}

\noindent where $\mathsf{sum}$ is a field element known to both $\mathcal{P}_{poly}$ and $\mathcal{V}_{poly}$ and will be defined below. \\ 

\noindent \textsf{$H$-ranged Polynomial Protocol $\Pa$ for $\Ra$} \\

%\noindent In the following, we describe $H$-ranged polynomial protocol $\Pa$ for conditional relation 
%$\Ra$. \\

\noindent $\mathcal{P}_{poly}$ and $\mathcal{V}_{poly}$ know public inputs 
$\mathbf{b'} \in \mathbb{F}_{|\block|}^{\frac{n}{\block}}$ and 
$\mathbf{pk} \in (\mathbb{F}^2)^{n-1} $ and $\mathit{apk} \in \mathbf{F}^2$ which are interpreted as per their domains. 

\begin{enumerate}
\item $\mathcal{V}_{poly}$ computes $pkx(X)$, $pky(X)$ and $aux(X)$.
\item $\mathcal{P}_{poly}$ sends polynomials $b(X)$, $kaccx(X)$ and $kaccy(X)$ to $\mathcal{I}$. 
\item $\mathcal{V}_{poly}$ replies with a random value $r$ chosen from $\mathbb{F}$. 
\item $\mathcal{V}_{poly}$ computes $\mathsf{sum}$ as $\sum_{j=0}^{\frac{n}{\block}-1} \mathit{b'_{j}} \cdot r^j$.\footnote{Note that if 
$b'_{j} = \sum_{k=0}^{\block -1}2^k \cdot \mathit{bit_{\block \cdot j + k }}$, $\forall j < \frac{n}{\block}$ and $\mathit{bit_i} \in \mathbb{B}, \forall i <n$, 
then $\sum_{i=0}^{n-1} 2^{i \mod \block} \cdot r^{i \div \block} \cdot \mathit{bit}_{i} = \sum_{j=0}^{\frac{n}{\block}-1}(\sum_{i=0}^{\block -1}2^k \cdot \mathit{bit_{\block \cdot j + k }}) \cdot r^j= \sum_{j=0}^{\frac{n}{\block}-1} \mathit{b'_{j}} \cdot r^j$.}
\item $\mathcal{P}_{poly}$ sends polynomials $c_{a}(X)$ and $acc_{a}(X)$ to $\mathcal{I}$. 
\item $\mathcal{V}_{poly}$ asks $\mathcal{I}$ to check that $id_i(x) = 0, \forall x \in H, \forall i \in [7]$ and accepts if $\mathcal{I}$'s checks verify. 
\end{enumerate}

\noindent We show $\Pa$ is an $H$-ranged polynomial protocol 
for $\Ra$. First, we prove the following:

\begin{test_claim}
\label{claim:bitvector_comm}
If the polynomial identities $id_6(X) = 0, id_7(X) = 0$ hold over range $H$, then, 
\ewnp, 
we have $c_{a,i} =  2^{i \mod \block} \cdot r^{i \div \block}$, $\forall i < n$ and $\mathsf{sum} = \sum_{i=0}^{n-1}b_i \cdot c_{a,i}$, 
where $b_i = b(\omega^i), \forall i <n$. If, additionally, identity $id_5(X) = 0$ holds over $H$, 
$r$ has been randomly chosen in $\mathbb{F}$, $\mathsf{sum} = \sum_{j=0}^{\frac{n}{\block}-1} b'_{j}r^j$ 
(as computed by $\mathcal{V}_{poly}$) and $\mathit{bit_{i}} \in \mathbb{B}, \forall i < n$ and 
$b'_{j} = \sum_{k=0}^{\block -1}2^k \cdot \mathit{bit_{\block \cdot j + k}}, \forall 0 \leq j \leq \frac{n}{\block} -1$ 
(due to  $(b'_{0}, \ldots, b'^{\frac{n}{\block} -1})$ 
being interpreted by $\mathcal{V}_{poly}$ as in $\mathbb{F}_{|\block|}^{\frac{n}{\block}}$), then \ewnp, 
$b_i = \mathit{bit_{i}}, \forall i <n$.
\end{test_claim}
\vspace{-0.15in}

\begin{proof}
We show the first part of the claim by proving by contradiction that $c_{a,0} =1$ using the Schwartz-Zippel Lemma, the fact that $r$ has been 
randomly chosen, and, also the fact that $n$ is negligibly smaller than the size of $\mathbb{F}$. Finally, we expand $\mathit{id_7}(X) = 0$ 
over $H$, sum the LHS and the RHS, equate and obtain the desired property of $\mathsf{sum}$. We show the second part of 
the claim by expressing $\mathsf{sum}$ in two ways as $\sum_{j=0}^{\frac{n}{\block}-1} b'_{j}r^j $ and as $\sum_{i=0}^{n-1} b_i \cdot c_{a,i}$ and re-writing the 
latter as an inner product of a vector of field elements with the vector $(1, r, \ldots, r^{\frac{n}{\mathsf{block}}-1})$ and using the small exponents test~\cite{small_exponents}. 
Full proof can be found in Section~\ref{sec:missing_snark_proofs}.
\end{proof}
\vspace{-0.1in}

\begin{lemma} 
$\Pa$ is an $H$-ranged polynomial protocol $\Ra$.
\end{lemma}
\vspace{-0.15in}

\begin{proof} 
The proof follows an analogous logic as used for proving Lemma~\ref{le:ba}. We additionally use 
Claim~\ref{claim:bitvector_comm} and Corollary~\ref{corollary:keys_affine_comm}. Full proof can be found in Section~\ref{sec:missing_snark_proofs}.
\end{proof}
\vspace{-0.15in}


\subsection{Our Custom SNARKs}
\label{sec_custom_snarks}
To design our accountable light client systems as introduced in Section~\ref{sec_intro_committee} and fully detailed in Appendix~\ref{new_light_client}, 
we need to further refine and then compile relations $\Rla$ and $\Ra$ (introduced in Sections \ref{sec_la} and \ref{sec_a}, respectively) into appropriate 
SNARKs such that the long public input $\mathbf{pk} \in ({\mathbb{F}^2})^{n-1}$ is replaced by a 
pair of succinct commitments and, $\mathbf{pk}$ becomes a witness for the resulting refined relations. As far as we are aware, such a compiler does not exist. 
Thus, we design a two-step compiler to obtain SNARKs for relations with such properties. We are interested in relations $\Rlacom$ and $\Racom$, where 
$\Rlacom$ defined below and $\Racom$ defined in Appendix~\ref{compiler_step_2}.
 
\begin{align*}
{\Rlacom} = \{&(\mathbf{C} \in \mathcal{C}, \mathbf{bit} \in \mathbb{B}^n, \mathit{apk} \in \mathbb{F}^2; \mathbf{pk}) : \\
&\mathit{apk} = \sum_{i=0}^{n-2} [\mathit{bit_i}] \cdot \mathit{pk_i} \ | \ \mathbf{pk} \in \ginn{1}^{n-1} \ \wedge \  \mathbf{C} = \mathbf{Com}(\mathbf{pk}) \},
\end{align*}
\noindent where $\mathbf{C}$ is a commitment from a set $\mathcal{C}$ of commitments and $\mathbf{Com}(\mathbf{pk})$ is a commitment to a specific vector, 
namely $\mathbf{pk}$. 

%\vspace*{-0.5cm}
%\begin{align*}
 %{\Racom}  = \{&(\mathbf{C} \in \mathcal{C}, \mathbf{b'} \in \mathbb{F}_{|\block|}^{\frac{n}{\block}}, \mathit{apk} \in \mathbb{F}^2;\mathbf{pk}, \mathbf{bit}) : 
 %\mathit{apk} = \sum_{i=0}^{n-2} [\mathit{bit_i}] \cdot \mathit{pk_i} \ | \\
 %\vspace*{-0.75cm}
 %&\mathbf{pk} \in \ginn{1}^{n-1} \ \wedge \ \mathbf{bit} \in \mathbb{B}^n \ \wedge\ \ 
 %b'_{j} = \sum_{i=0}^{\block -1}2^i \cdot \mathit{bit_{\block \cdot j + i}}, \forall j < \frac{n}{\block}  \wedge \  \mathbf{C} = \mathbf{Com}(\mathbf{pk}) \} 
%\end{align*}
The two-step compiler is described in Appendix~\ref{sec_two_step_compiler}, with the first 
step being the standard PLONK compiler (Section 4.7 of 
\cite{plonk}) and the second step being amenable for compiling into SNARKs a specific 
generalisation of the two NP relations just detailed above. In fact, the second compilation step is simply a re-casting of the already compiled SNARKs in first step 
as SNARKs for new NP relations based on both commitments and vectors of field elements. 
This re-casting is possible due to a new security proof that holds under mild conditions. 
Overall, due to its generality, our two-step compiler may be of independent interest for other projects as well. Full details can be found in Appendix~\ref{sec_two_step_compiler}.
\vspace{-0.05in}

\subsection{Our Instantiation for CKS}
\label{sec:inst_committee_key}
\noindent Given relations $\Rlacom$ and $\Racom$ described in short in Section~\ref{sec_custom_snarks} (and in full in Appendix~\ref{sec_two_step_compiler}), 
%in full in Section~\ref{sec_two_step_compiler}
we present an instantiation for committee key scheme defined in Section~\ref{sec:committee_key}; 
this is used to build an accountable light client system. %(Section~\ref{new_light_client})

We instantiate $u$ and $v$ from Section~\ref{sec:committee_key} as $u = n-1$, ($n=|H|$ from 
Section~\ref{sec:lagrange}) and $v \in \mathbb{N}, n-1 \leq v$, $v = \mathsf{poly}(\lambda)$, where $v$ is the 
maximum number of validators.

\begin{construction}(Committee Key Scheme for Aggregatable Signatures)
\label{inst:cks} In our implementation we use the following instantiation of Definition~\ref{def: committee_key} for one of $\mathcal{R} \in \{\Rlacom, \Racom\}$:
\begin{itemize}
\item $\mathit{CKS_{\mathcal{R}}.Setup}(v)$ calls algorithms: 
\begin{enumerate}
\item $\mathit{pp} \leftarrow \mathit{AS.Setup}(\mathit{aux_{\mathit{AS}}}= v+1)$ with $\mathit{AS.Setup}$ part of Instantiation~\ref{insta:bls} and 
$\ginn{1}$ part of $\mathit{pp}$ (see notation in Appendix~\ref{sec:bls});
\item $\mathit{srs} \leftarrow \mathit{SNARK.Setup}(\mathit{aux_{\mathit{SNARK}}} = (v, 3v))$ with \\
$\mathit{srs}=([1]_{\indexoneout}, [\tau]_{\indexoneout}, \ldots, [\tau^{3v}]_{\indexoneout}, [1]_{\indextwoout}, [\tau]_{\indextwoout})$;
\item $(\mathit{rs}_{\mathit{pk}}, \mathit{rs}_{\mathit{vk}}) \leftarrow \mathit{SNARK.KeyGen}(\mathit{srs}, \mathcal{R})$ with \\
$(\mathit{rs}_{\mathit{pk}}, \mathit{rs}_{\mathit{vk}}) =  (([1]_{\indexoneout}, [\tau]_{\indexoneout}, \ldots, [\tau^{3v}]_{\indexoneout}), 
([1]_{\indexoneout}, [1]_{\indextwoout}, [\tau]_{\indextwoout}))$ where %$\mathcal{R} \in \{\Rlacom, \Racom \}$ is one of the accountable relations defined in the end of section~\ref{sec_two_step_compiler} and 
the notation $[\ldots]_{\indexoneout}$ and $[\ldots]_{\indextwoout}$ was defined in Section~\ref{sec:pairings}.
\end{enumerate}

\item $\mathit{ck} \leftarrow \mathit{CKS_{\mathcal{R}}.GenCommitteeKey}(\mathit{rs_{pk}}, (\mathit{pk_i})_{i=1}^{n-1})$,where 
$\mathit{CKS_{\mathcal{R}}.GenCommitteeKey}$ first checks whether $(\mathit{pk_i})_{i=1}^{n-1} \in \ginn{1}^{n-1}$; if this does not hold, it outputs $\bot$;
otherwise, $\mathit{CKS_{\mathcal{R}}.GenCommitteeKey}$ continues as: \\
\noindent Let $\mathbf{pkx} = (\mathit{pkx_{1}}, \ldots, \mathit{pkx_{n-1}})$, $\mathbf{pky} = (\mathit{pky_{1}}, \ldots, \mathit{pky_{n-1}})$, $\forall i \in [n-1]$, $\mathit{pk_i} = (\mathit{pkx_i}, \mathit{pky_i}) \in \mathbb{F}^{2}$. \\
\noindent Let $pkx(X) = \sum_{i=0}^{n-2} \mathit{pkx_{i+1}} \cdot L_i(X)$, $pky(X) = \sum_{i=0}^{n-2} \mathit{pky_{i+1}} \cdot L_i(X)$. \\
\noindent Let $[pkx]_{\indexoneout} = pkx(\tau) \cdot [1]_{\indexoneout}$, $[pky]_{\indexoneout} = pky(\tau) \cdot [1]_{\indexoneout}$. \\
Output $\mathit{ck} = ([pkx]_{\indexoneout}, [pky]_{\indexoneout})$.\\
\noindent Note that $\mathbb{F}$ and $\{L_i(X)\}_{i=1}^{n-2}$ are as defined in Section~\ref{sec:lagrange}. 

\item $(\pi, \mathit{ck}) \leftarrow \mathit{CKS_{\mathcal{R}}.Prove}
(\mathit{rs}_{\mathit{pk}}, (\mathit{pk_i})_{i=1}^{n-1}, (\mathit{bit_i})_{i=1}^{n-1})$ 
where $ \pi = (\pi_{SNARK}, \mathit{apk})$ and \\
$\mathit{ck} \leftarrow \mathit{CKS_{\mathcal{R}}.GenCommitteeKey}(\mathit{rs_{pk}}, (\mathit{pk_i})_{i=1}^{n-1}) $ and \\
 $\mathit{apk} = \sum_{i=1}^{n-1} \mathit{bit_i} \cdot \mathit{pk_i} \leftarrow \mathit{AS.AggregateKeys}(\mathit{pp}, (\mathit{pk_i})_{i:\mathit{bit_i = 1}})$ 
as defined in Instantiation~\ref{insta:bls} and \\ $\pi_{SNARK} \leftarrow \mathit{SNARK.Prove}(\mathit{rs_{pk}}, (x,w), \mathcal{R})$, 
for $\mathcal{R} \in \{\Rlacom, \Racom \}$ where 
\begin{equation*}
\begin{cases}
 (x = (\mathit{ck}, (\mathit{bit_i})_{i=1}^{n-1}||0, \mathit{apk}), w = ((\mathit{pk}_i)_{i=1}^{n-1}) & \text{ if } \mathcal{R} = \Rlacom,\\
 (x = (\mathit{ck}, \mathbf{b'}, \mathit{apk}), w = ((\mathit{pk}_i)_{i=1}^{n-1}, (\mathit{bit_i})_{i=1}^{n-1}||0) & \text{ if } \mathcal{R} = \Racom, \\
\end{cases}       
\end{equation*}
where $\mathbf{b'}$ is the vector of field elements formed from blocks of size $\mathsf{block}$ of bits from vector 
$(\mathit{bit_i})_{i=1}^{n-1}||0$ and $\mathsf{block}$ is the highest power of 2 smaller than the size of a field element in $\mathbb{F}$. 

\item $0/1 \leftarrow \mathit{CKS_{\mathcal{R}}.Verify}(\mathit{pp}, \mathit{rs}_{\mathit{vk}}, \mathit{ck}, m, \mathit{asig}, \pi, \mathbf{bitvector})$ 
parses $\pi$ to retrieve $\pi_{\mathit{SNARK}}$ and $\mathit{apk}$ and it calls $\mathit{AS.Verify(\mathit{pp}, \mathit{apk}, m, \mathit{asig})}$ 
as defined in Instantiation~\ref{insta:bls} and it also calls \\ $\mathit{SNARK.Verify}(\mathit{rs_{vk}}, x, \pi_{\mathit{SNARK}}, \mathcal{R})$ 
(where $\pi_{\mathit{SNARK}}$, $x$ and $\mathcal{R}$ are as defined in the item above with the only difference that $(\mathit{bit_i})_{i=1}^{n-1}$ 
represents the first $n-1$ bits of $\mathbf{bitvector}$, padded with $0$s, if not sufficiently many exist in $\mathbf{bitvector}$); 
overall, the output is $1$ if both algorithms output $1$ and the output is $0$ otherwise.
\end{itemize}
\end{construction}
\vspace{-0.07in}
\begin{theorem} 
% All of this was explained above so is redundant or is just redundant.
%Given the hybrid model SNARK scheme secure for relation $\mathcal{R} \in \{ \Rlacom, \Racom\}$ as 
%obtained using our two-step compiler in Section~\ref{sec_two_step_compiler} and the aggregatable signature scheme $\mathit{AS}$ 
 %                   as per Instantiation~\ref{insta:bls} (which fulfils Definition~\ref{def:aggregate_signatures}) with the additional 
 %                   specification that $\mathit{aux}_{\mathit{AS}} = v+1$ and choosing $v = n-1$, 
%if we assume that an efficient adversary (against the soundness of) $\mathit{CKS}_{\mathcal{R}}$ outputs public keys only from the source group $\ginn{1}$, then
The committee key scheme $\mathit{CKS}_{\mathcal{R}}$ in Instantiation~\ref{inst:cks} is secure with respect to Definition~\ref{def: committee_key}.
\end{theorem}
\vspace{-0.08in}
\begin{proof} We give a full proof in Appendix~\ref{supplementary_proof_sec_cks}. 
\end{proof}



\section{Conclusions}\label{conclusions}
In this work we have revisited the notion of O-SNARKs in relation with a wide class of oracles which we call AGM 
respecting. Our result shows that many modern SNARKs (i.e., PLONK~\cite{plonk}, Marlin~\cite{marlin}, Groth16~\cite{groth16}, etc.) 
are in fact O-SNARKs when combined with AGM respecting oracles. Additionally, we shed more light via a concrete 
example where and how O-SNARK security is useful for practical and more complex real-world applications. 



%\section{Acknowledgements} \label{sec_ack}
%We thank Handan K{\i}l{\i}n\c c Alper and Dario Fiore for useful comments and feedback and for reviewing drafts of this work. 




\bibliography{bibliography}
\bibliographystyle{ieeetr}
%%\bibliography{bibliography}
%%\bibliographystyle{splncs04}
\onecolumn
\appendix
\section*{Appendices}
\subsection{Aggregatable Signature Scheme Definition}
\label{sec:multisig}
%\label{suplementary_aggregatable}
An aggregatable signature scheme compresses signatures issued using possibly 
different signing keys into one signature. In this work we use an aggregatable 
signature scheme making explicit use of the proofs-of-possession (PoPs)~\cite{proofs_of_posession}.
For our concrete instantiation we use aggregatable BLS signatures with an 
efficient aggregation procedure, i.e., by adding together keys and by multiplying together 
signatures, and protect against rogue key attacks~\cite{proofs_of_posession} using PoPs. 
This is in contrast to other aggregation procedures that do not require PoPs for security 
but incur a higher computational cost (e.g., due to the use of multi-scalar multiplication~\cite{boneh_compact_multisig}). 
For our concrete use case of accountable light clients systems, our efficient signature aggregation method results 
in a simple and more efficient custom argument scheme (i.e., SNARK), which, in turn, compensates for the cost of having 
to work with PoPs. 
\begin{definition}
\label{def:aggregate_signatures}
(Aggregatable Signature Scheme) An aggregatable signature scheme consists of
the following tuple of algorithms ($\mathit{AS.Setup}$, $\mathit{AS.GenerateKeypair}$, $\mathit{AS.VerifyPoP}$, 
$\mathit{AS.Sign}$, \\ $\mathit{AS.AggregateKeys}$, $\mathit{AS.AggregateSignatures}$, $\mathit{AS.Verify}$) 
such that for implicit security parameter $\lambda$:
\begin{itemize}

\item $\mathit{pp} \leftarrow  \mathit{AS.Setup}(\mathit{aux_{\mathit{AS}}})$: a setup algorithm that, given an 
auxiliary parameter $\mathit{aux_{\mathit{AS}}}$, outputs public protocol parameters $\mathit{pp}$. 

\item $((\mathit{pk},\mathit{\pi_{PoP}}),\mathit{sk}) \leftarrow \mathit{AS.GenerateKeypair}(\mathit{pp})$:
a key pair generation algorithm that
outputs
a secret key $\mathit{sk}$,
and the corresponding public key $\mathit{pk}$
together with a proof of possession $\mathit{\pi_{PoP}}$ for the secret key.

\item $0/1 \leftarrow \mathit{AS.VerifyPoP}(\mathit{pp}, \mathit{pk},\mathit{\pi_{PoP}})$:
a public key verification algorithm that,
given a public key $\mathit{pk}$
and a proof of possession $\mathit{\pi_{PoP}}$,
outputs
$1$ if $\mathit{\pi_{PoP}}$ is valid for $\mathit{pk}$ and $0$ otherwise.

\item $\sigma \leftarrow \mathit{AS.Sign}(\mathit{pp}, \mathit{sk}, m)$:
a signing algorithm that,
given a secret key $\mathit{sk}$ and a message $m$ in $\{0, 1\}^*$, returns a signature $\sigma$.

\item $\mathit{apk} \leftarrow \mathit{AS.AggregateKeys}(\mathit{pp}, (\mathit{pk_i})_{i=1}^{u})$:
a public key aggregation algorithm that,
given a vector of public keys $(\mathit{pk_i})_{i=1}^u$,
returns
an aggregate public key $\mathit{apk}$.

\item $\mathit{asig} \leftarrow \mathit{AS.AggregateSignatures}(\mathit{pp}, (\sigma_i)_{i=1}^u)$:
a signature aggregation algorithm that,
given a vector of signatures $(\sigma_i)_{i=1}^u$,
returns
an aggregate signature $\mathit{asig}$.

\item $0/1 \leftarrow \mathit{AS.Verify}(\mathit{pp}, \mathit{apk}, m, \mathit{asig})$:
a signature verification algorithm that,
given an aggregate public key $\mathit{apk}$, a message $m \in \{0, 1\}^*$, and an aggregate signature $\sigma$,
returns
1 or 0 to indicate if the signature is valid.
\end{itemize}

\noindent We say ($\mathit{AS.Setup}$, $\mathit{AS.GenerateKeypair}$, $\mathit{AS.VerifyPoP}$, 
$\mathit{AS.Sign}$, $\mathit{AS.AggregateKeys}$, \\ $\mathit{AS.AggregateSignatures}$, 
$\mathit{AS.Verify}$) is an aggregatable signature scheme if it satisfies \emph{perfect completeness}  and 
\emph{perfect completeness for aggregation}  and \emph{unforgeability} as defined below. \\

\noindent \textbf{Perfect Completeness} An aggregatable signature scheme
($\mathit{AS.Setup}$, $\mathit{AS.GenerateKeypair}$, \\ $\mathit{AS.VerifyPoP}$, $\mathit{AS.Sign}$, $\mathit{AS.AggregateKeys}$,
$\mathit{AS.AggregateSignatures}$, $\mathit{AS.Verify}$) has perfect completeness if for any message $m \in \{0,1\}^*$ and any 
$u\in\mathbb{N}$ it holds that:
\begin{align*}
&\mathit{Pr} [\mathit{AS.Verify}(\mathit{pp}, \mathit{apk}, m, \mathit{asig})=1 \  \wedge \ \forall  i \in [u]\ \mathit{AS.VerifyPoP}(\mathit{pp}, \mathit{pk_i},\mathit{\pi_{\mathit{PoP},i}})=1\ |\\
& \mathit{pp} \leftarrow \mathit{AS.Setup}(\mathit{aux_{\mathit{AS}}}), \\
& ((pk_{i},\pi_{\mathit{PoP}, i}), sk_{i} ) \leftarrow \mathit{AS.GenerateKeypair}(\mathit{pp}),\ i=1,\ldots, u\\
&\mathit{apk} \leftarrow \mathit{AggregateKeys}(\mathit{pp}, (\mathit{pk}_{i})_{i=1}^{u}), \\
& \sigma_i \leftarrow \mathit{AS.Sign}(\mathit{pp}, \mathit{sk_i}, m),\ i=1,\ldots, u, \\
& \mathit{asig} \leftarrow \mathit{AS.AggregateSignatures(\mathit{pp}, (\sigma_{i})_{i=1}^{u})}] = 1.
\end{align*}
\noindent We note that an aggregatable signature scheme with perfect completeness implies the underlying signature scheme
has perfect completeness. \\

\noindent \textbf{Perfect Completeness for Aggregation} An aggregatable signature scheme 
($\mathit{AS.Setup}$, \\ $\mathit{AS.GenerateKeypair}$, $\mathit{AS.VerifyPoP}$, $\mathit{AS.Sign}$, 
$\mathit{AS.AggregateKeys}$, $\mathit{AS.AggregateSignatures}$, $\mathit{AS.Verify}$)
has perfect completeness for aggregation if, for every adversary $\mathcal{A}$
\begin{align*}
& \mathit{Pr}[\mathit{AS.Verify}(\mathit{pp}, \mathit{apk}, m, \mathit{asig}) = 1 \ | \ \mathit{pp} \leftarrow \mathit{AS.Setup}(\mathit{aux_{\mathit{AS}}}), \\
& ((\mathit{pk_i})_{i=1}^u, m, (\sigma_i)_{i=1}^{u}) \leftarrow \mathcal{A}(\mathit{\mathit{pp})} \ 
\textit{such that} \ \forall i \in [u], \mathit{AS.Verify}(\mathit{pp}, \mathit{pk_i}, m, \sigma_i) = 1, \\
& \mathit{apk} \leftarrow \mathit{AS.AggregateKeys}(\mathit{pp},  (\mathit{pk}_{i})_{i=1}^{u}), \\
&  \mathit{asig} \leftarrow \mathit{AS.AggregateSignatures}(\mathit{pp}, (\sigma_i)_{i=1}^u)] = 1.
\end{align*}

\noindent \textbf{Unforgeable Aggregatable Signature}
For an aggregatable signature scheme ($\mathit{AS.Setup}$, \\ $\mathit{AS.GenerateKeypair}$, $\mathit{AS.VerifyPoP}$, $\mathit{AS.Sign}$,
$\mathit{AS.AggregateKeys}$, $\mathit{AS.AggregateSignatures}$, $\mathit{AS.Verify}$)
the advantage of an adversary against unforgeability is defined by

$$\mathit{Adv}^{\mathit{forge}}_{\mathcal{A}}({\lambda}) = \mathit{Pr}[\mathit{Game}^{\mathit{forge}}_{\mathcal{A}}({\lambda}) =1]$$
\noindent where
\begin{align*}
&\mathit{Game}^{\mathit{forge}}_{\mathcal{A}}({\lambda}): \\
& \mathit{pp} \leftarrow \mathit{AS.Setup}(\mathit{aux_{\mathit{AS}}}) \\
& ((\mathit{pk}^*,\pi^*_{\mathit{PoP}}), \mathit{sk}^*) \leftarrow \mathit{AS.GenerateKeypair}(\mathit{pp})\\
& Q \leftarrow \emptyset \\
& ((\mathit{pk_i}, \pi_{\mathit{PoP},i})_{i=1}^{u}, m, \mathit{asig}) \leftarrow \mathcal{A}^{\mathit{OSign}}(\mathit{pp}, (\mathit{pk^*},\pi^*_{\mathit{PoP}})) \\
& \textit{If } \mathit{pk}^* \notin \{\mathit{pk_i}\}_{i=1}^{u} \vee m \in Q, \textit{ then return } 0 \\
& \textit{For } i \in [u] \\
& \ \ \ \ \ \textit{ If } \mathit{AS.VerifyPoP}(\mathit{pp}, \mathit{pk_i}, \pi_{\mathit{PoP},i})=0  \textit{ return } 0 \\
& \mathit{apk} \leftarrow \mathit{AS.AggregateKeys}(\mathit{pp}, (\mathit{pk_i})_{i=1}^{u}) \\
& \textit{Return } \mathit{AS.Verify}(\mathit{pp}, \mathit{apk}, m, \mathit{asig})
\end{align*}
\noindent and
\begin{align*}
& \mathit{OSign}(m_j): \\
& \sigma_j \leftarrow \mathit{AS.Sign}(\mathit{pp}, \mathit{sk}^*, m_j) \\
&  Q \leftarrow Q \cup \{m_j\} \\
& \textit{Return} \ \sigma_j
\end{align*}

\noindent and $\mathcal{A}^{\mathit{OSign}}$ denotes the adversary $\mathcal{A}$ with access to oracle $\mathit{OSign}$. \\

\noindent We say an aggregatable signature scheme is unforgeable if for all efficient adversaries
$\mathcal{A}$ it holds that $\mathit{Adv}^{\mathit{forge}}_{\mathcal{A}}({\lambda}) \leq \mathit{negl}(\lambda)$. 
\end{definition}

\subsubsection{An Aggregatable Signature Instantiation}
\label{sec:bls}
\noindent In the following, we instantiate the aggregatable signature definition given above with a scheme inspired by the BLS signature
scheme~\cite{BLS_signatures} and its follow-up variants~\cite{proofs_of_posession,boneh_compact_multisig}.

\begin{construction}(Aggregatable Signatures) 
\label{insta:bls}
In our implementation we call aggregatable signatures the following 
instantiation of aggregatable signatures definition. Note that in our implementation we instantiate $\einn$ with BLS12-377~\cite{zexe}.
\begin{itemize}
\item $(\ginn{1}, \sginn{1}, \ginn{2}, \sginn{2}, \gtinn, \epinn, \Hinn, \HPoP)$ from $\mathit{pp}$ where 
$\mathit{pp} \leftarrow  \mathit{AS.Setup}(\mathit{aux_{\mathit{AS}}})$, 
where $\ginn{1}$, $\sginn{1}$, $\ginn{2}$, $\sginn{2}$, $\gtinn$, $\epinn$ were defined in Section~\ref{sec:pairings} and 
$\Hinn: \{0,1\}^* \rightarrow \ginn{2}$ and $\HPoP: \{0,1\}^* \rightarrow \ginn{2}$ are two hash functions. 
The auxiliary parameter $\mathit{aux_{\mathit{AS}}}$ is such that there exists $N \in \mathbb{N}$, 
$N$ is the first component of the vector $\mathit{aux_{\mathit{AS}}}$ and there exists a subgroup of size at least $N$ in the multiplicative group of $\mathbb{F}$, where $\mathbb{F}$ 
is the base field of $\einn$, but also the size of the subgroup $\in O(N)$.

\item $(\mathit{pk},\mathit{sk}, \pi_{\PoP}) \leftarrow \mathit{AS.GenerateKeypair}(\mathit{pp})$, where $\mathit{sk} \xleftarrow{\$} \mathbb{Z}_{r}^{*}$  
and $\mathit{pk} = \mathit{sk} \cdot \sginn{1} \in \ginn{1}$ and $\pi_{\PoP} \leftarrow {\mathit{sk}} \cdot \HPoP(\mathit{pk})$ 
and $r$ was defined in Section~\ref{sec:pairings} as the characteristic of the scalar field of $\einn$.

\item $0/1 \leftarrow \mathit{AS.VerifyPoP}(\mathit{pp}, \mathit{pk}, \pi_{\PoP})$, where $\mathit{AS.VerifyPoP}$ outputs $1$ if 
$$\epinn( \sginn{1}, \pi_{\PoP}) = \epinn(\mathit{pk}, \HPoP(\mathit{pk}))$$ holds and $0$ otherwise. Note that implicitly, as part of running \\
$\mathit{AS.VerifyPoP}$, one checks that $\mathit{pk} \in \ginn{1}$ also holds.

\item $\sigma \leftarrow \mathit{AS.Sign}(\mathit{pp}, \mathit{sk}, m)$: 
where $\sigma = \mathit{sk} \cdot \Hinn(m) \in \ginn{2}$.

\item $\mathit{apk} \leftarrow \mathit{AS.AggregateKeys}(\mathit{pp}, (\mathit{pk_i})_{i=1}^{u})$, where  $\mathit{apk} = \sum_{i=1}^{u} \mathit{pk_i}$. 
Note that $\mathit{AS.AggregateKeys}$ checks whether $((\mathit{pk_i})_{i=1}^{u}) \in \ginn{1}^{u} (\ast)$ and, if that is not the case, it outputs $\bot$; 
if $(\ast)$ holds, the algorithm $\mathit{AS.AggregateKeys}$ continues with the computations described above. 


\item $\mathit{asig} \leftarrow \mathit{AS.AggregateSignatures}(\mathit{pp}, (\sigma_i)_{i=1}^u)$, where $\mathit{asig}$ = $\sum_{i=1}^{u} \sigma_i$.  

\item $0/1 \leftarrow  \mathit{AS.Verify}(\mathit{pp}, \mathit{apk}, m, \mathit{asig})$, where $\mathit{AS.Verify}$ outputs $1$ if $\mathit{apk} \neq \bot$ and
$\mathit{apk} \in \ginn{1}$ and $\epinn(\mathit{apk}, \Hinn(m)) = \epinn(\sginn{1}, \mathit{asig})$; otherwise, it outputs $0$.
\end{itemize}
\end{construction}
\section{Hybrid Model SNARKS}
\label{sec:snarks_defs}
\noindent When proving the security of our arguments, we use an extension of some of the more commonly employed SNARK definitions which we call a ``a hybrid model SNARK''. This resembles the existing notion of SNARKs with online-offline verifiers as described in~\cite{HP_paper}, where the 
verifier computation is split into two parts: during the offline phase some computation (possibly of commitments) happens; this computation takes some public inputs as parameters and, when not performed by the verifier, it may also be performed (in part) by the prover. The online phase is the main computation performed by the verifier. In the case of our hybrid model SNARKs, however, the input to the offline counterpart described above (which we call the $\mathit{PartInput}$ algorithm) may even be the witness or 
a part of the witness for the respective relation. For our custom SNARKs, $\mathit{PartInput}$ produces part of the public input used by the verifier; 
since for our use case, $\mathit{PartInput}$ does handle a portion of the witness, this operation cannot be performed by the verifier for that relation. 
Moreover, in our instantiation, $\mathit{PartInput}$ produces computationally binding commitment schemes that are opened by the prover. Both of these properties 
are not explicitly part of our general definition for hybrid model SNARKs, but they are crucial and explicitly assumed and used 
in proving the security for our compiler's second step (see Appendix~\ref{sec_two_step_compiler}). Intuitively, our commitments 
are the counterpart CP-SNARK subcomponent of a that computes a commitment (to part of the witness) linking different CP-SNARKs together. 
We do not need such a strong property of linking SNARKs; the commitments are used in our case for the efficiency they bring to the prover/ overall system.\\

\noindent The two SNARKs we design in this work have access to a \emph{structured reference string} (srs) of the form 
$$(\{[\tau^i]_1\}_{i=0}^{d}, \{[\tau^i]_2\}_{i=0}^{1})$$ where $\tau$ is a random (and allegedly secret) value in $\mathbb{F}$ and $d$ 
is bounded by a polynomial in $\lambda$. Such an srs is universal and updatable~\cite{updatable_universal_srs_2018}. 
We introduce a generalisation of the usual SNARK definition which we call a \emph{hybrid model SNARK} inspired by online-offline SNARKs~\cite{HP_paper}. We further refine it as described below:     

\begin{definition}(Hybrid Model SNARK)
\label{dfn_snark}
A hybrid model \emph{succinct non-interactive argument of knowledge for relation $\mathcal{R}$} is a tuple of PPT algorithms 
$(\mathit{SNARK.Setup}, \mathit{SNARK.KeyGen}, \mathit{SNARK.Prove},  \mathit{SNARK.Verify},  \\ \mathit{SNARK.PartInputs})$ 
such that for implicit security parameter $\lambda$: 

\begin{itemize}
\item $\mathit{srs} \leftarrow \mathit{SNARK.Setup} (\mathit{aux_{\mathit{SNARK}}})$: a setup algorithm that on input auxiliary parameter 
$\mathit{aux_{\mathit{SNARK}}}$ from some domain $\mathcal{D}$ outputs a universal structured reference string tuple $\mathit{srs}$, 

\item $(\mathit{srs_{pk}}, \mathit{srs_{vk}}) \leftarrow \mathit{SNARK.KeyGen}(\mathit{srs}, \mathcal{R})$: a key generation algorithm that on input a
universal structured reference string $\mathit{srs}$ and an NP relation $\mathcal{R}$ outputs a \emph{proving key} and 
a \emph{verification key} pair $(\mathit{srs_{pk}}, \mathit{srs_{vk}})$,

\item $\pi \leftarrow \mathit{SNARK.Prove}(\mathit{srs_{pk}}, (x,w), \mathcal{R})$: a proof generation algorithm that on input a proving key 
$\mathit{srs_{pk}}$ and a pair $(x,w) \in \mathcal{R}$ outputs \emph{proof} $\pi$, 

\item $0/1 \leftarrow \mathit{SNARK.Verify}(\mathit{srs_{vk}}, x, \pi, \mathcal{R})$: a proof verification algorithm that on input a verification key 
$\mathit{srs_{vk}}$, an instance $x$ and a proof $\pi$ outputs a bit that signals acceptance (if output is $1$) or rejection (if output is $0$),

\item $(x_1, \mathit{state}_2) \leftarrow \mathit{SNARK.PartInputs}(\mathit{srs}, \mathit{state}_1, \mathcal{R})$: a deterministic 
public inputs generation algorithm that takes as input a universal structured reference string $\mathit{srs}$, an NP relation $\mathcal{R}$ and 
 state $\mathit{state}_1$ and outputs updated state $\mathit{state}_2$ and partial public input $x_1$,

\end{itemize}
and satisfies completeness, knowledge soundness w.r.t. $\mathit{SNARK.PartInputs}$ and succinctness as defined below:

\noindent \textbf{Perfect Completeness} holds if an honest prover will always convince an honest verifier: for all  
$(x,w) \in \mathcal{R}$ and for all $\mathit{aux_{\mathit{SNARK}}} \in \mathcal{D}$
\begin{align*}
& \mathit{Pr}[\mathit{SNARK.Verify}(\mathit{srs_{vk}}, x, \pi, \mathcal{R}) = 1 \ | \  
\mathit{srs} \leftarrow \mathit{SNARK.Setup}(\mathit{aux_{\mathit{SNARK}}}), \\ 
& (\mathit{srs_{pk}}, \mathit{srs_{vk}})\leftarrow \mathit{SNARK.KeyGen}(\mathit{srs}, \mathcal{R}), \pi \leftarrow \mathit{SNARK.Prove}(\mathit{srs_{pk}}, (x,w), \mathcal{R}) \ ] = 1.
\end{align*}

\noindent \textbf{Notation} We denote by $\mathit{State_{\mathcal{R}}}$ the set of all states $\mathit{state}_1$ 
such that given some relation $\mathcal{R}$ and any possible $\mathit{srs}$, 
for any output $x_1$ of  $\mathit{SNARK.PartInputs}(\mathit{srs}, \mathit{state}_1, \mathcal{R})$ 
with $\mathit{state_1} \in \mathit{State_{\mathcal{R}}}$, there exists $x_2$ and $w$ with $(x=(x_1, x_2), w) \in \mathcal{R}$; 
we further assume $\mathit{State_{\mathcal{R}}} \neq \emptyset$.\\
\vspace{-0.08in}

\noindent \textbf{Knowledge-soundness with respect to $\mathit{SNARK.PartInputs}$}
holds if there exists a PPT extractor $\mathcal{E}$ such that for all PPT 
adversaries $\mathcal{A}$, for all $\mathit{aux_{\mathit{SNARK}}} \in \mathcal{D}$ and for all $\mathit{state_1} \in \mathit{State_{\mathcal{R}}}$
\begin{align*}
&\mathit{Pr}[(x = (x_1, x_2), w) \in \mathcal{R} \wedge 1 \leftarrow \mathit{SNARK.Verify}(\mathit{srs_{vk}}, x =  (x_1, x_2), \pi, \mathcal{R}) \ | \\
&\mathit{srs} \leftarrow \mathit{SNARK.Setup}(\mathit{aux_{\mathit{SNARK}}}), (\mathit{srs_{pk}}, \mathit{srs_{vk}})\leftarrow  \mathit{SNARK.KeyGen}(\mathit{srs}, \mathcal{R}), \\ 
& (x_1, \mathit{state}_2) \leftarrow \mathit{SNARK.PartInput}(\mathit{srs}, \mathit{state}_1, \mathcal{R}), (x_2, \pi) \leftarrow \mathcal{A}(\mathit{srs}, \mathit{state}_2, \mathcal{R}), \\
& w \leftarrow \mathcal{E}^{\mathcal{A}}(srs,\mathcal{R})]
\end{align*}
\noindent is overwhelming in $\lambda$, where by $\mathcal{E}^{\mathcal{A}}$ we denote the extractor $\mathcal{E}$ that has access to all of 
$\mathcal{A}$'s messages during the protocol with the honest verifier. \\ 
\noindent \textbf{Succinctness} holds if the size of the proof $\pi$ is $\mathsf{poly}(\lambda)$ and $\mathit{SNARK.Verify}$ runs in time 
$\mathsf{poly}(\lambda + |x|)$. % $+ \log{|w|}$ has been removed in both cases.
\end{definition}
\vspace{-0.04in}
\noindent Firstly, if $x_1$, $\mathit{state_1}$ and $\mathit{state_2}$ are the empty strings, we obtain the standard SNARK definition.
Secondly, $\mathcal{R}$ is not a component of the vector $\mathit{aux_{\mathit{SNARK}}}$ so even if $\mathit{SNARK.Setup}$ has 
$\mathit{aux_{\mathit{SNARK}}}$ as parameter, it is universal, 
i.e., it can be used to derive proving and verification keys for circuits of any size up to a polynomial in the security parameter $\lambda$,   
independently of any specific NP relation. Thirdly, for the SNARKs we design, the size of the key used by the honest verifier is much smaller than the size of the honest prover's key. To capture this special case we made the separation clear between the two keys; however, a potential adversarial prover has access to the complete $\mathit{srs}$ key. 
Moreover, our SNARKs are secure in the $\mathit{AGM}$ model~\cite{AGM_model}, i.e., security is w.r.t. $\mathit{AGM}$ 
adversaries only and by $\mathcal{E}^{\mathcal{A}}$ we denote the 
extractor $\mathcal{E}$ that has access to all of 
$\mathcal{A}$'s messages during the protocol with the honest 
verifier including the coefficients of the linear combinations of 
group elements used by the adversary at any protocol step for outputting new group elements at the next step. Finally, the auxiliary input (i.e., $\mathit{state_1}$) is required to 
be drawn from a ``benign distribution'' or else extraction may be 
impossible~\cite{extractability_limits_1,extractability_limits_2}. \\

\noindent We did not include the notion of zero-knowledge since it is not required.
\vspace{-0.1in}
\section{Ranged Polynomial Protocols for NP Relations}
\label{supplementary_poly_protocols_appendix}

\noindent In the following, we keep the convention that all algorithms receive an implicit security parameter $\lambda$. The definition below 
is a natural extension of the notions of polynomial protocols and polynomial protocols for relations from Section 4 of PLONK~\cite{plonk} to 
polynomial protocols over ranges for conditional NP relations with additional refinements required by our specific use case; these refinements are 
incorporated into steps (4), (5) and (6) as follows: 

\begin{definition}(Polynomial Protocols over Ranges for Conditional NP Relations)
\label{def_ranged_poly_protocol}
Assume three parties, a prover $\mathcal{P}_{poly}$, a verifier $\mathcal{V}_{poly}$ and a trusted party $\mathcal{I}$. 
Let $\mathcal{R}^c$ be a conditional NP relation (with $c$ being a predicate) and let $x$ be a public 
input both of which have been given to $\mathcal{P}_{poly}$ and $\mathcal{V}_{poly}$ by an $\mathit{InitGen}$ efficient algorithm. 
For positive integers $d$, $D$, $t$, $l$, $u$, $e$ and for set 
$S \subset \mathbb{F}$, an $S$-ranged $(d, D, t, l, u, e)$-polynomial protocol $\mathscr{P}_{\mathcal{R}^c}$ for relation $\mathcal{R}^c$ is a multi-round 
protocol between $\mathcal{P}_{poly}$, $\mathcal{V}_{poly}$ and $\mathcal{I}$ such that:

\begin{enumerate}
\item The protocol $\mathscr{P}_{\mathcal{R}^c}$ definition includes a set of pre-processed polynomials $g_1(X), \ldots, g_l(X) \in \mathbb{F}_{<d}[X]$. 

\item The messages of $\mathcal{P}_{poly}$ are sent to $\mathcal{I}$ and are of the form $f(X)$ for $f(X) \in \mathbb{F}_{<d}[X]$. 

If $\mathcal{P}_{poly}$ sends a message not of this form, the protocol is aborted.
\item The messages from $\mathcal{V}_{poly}$ to $\mathcal{P}_{poly}$ are random coins.

\item
$\mathcal{V}_{poly}$ may perform arithmetic computations using input $x$ and the random coins used in the 
communication with $\mathcal{P}_{poly}$. Let $(\mathit{res_1}, \ldots, \mathit{res_u})$ be the results of those computations 
which $\mathcal{V}_{poly}$ sends to $\mathcal{I}$. 

\item 
Using vectors which are part of input $x$ and/or other ad-hoc vectors which $\mathcal{V}_{poly}$ deems useful, $\mathcal{V}_{poly}$ 
may compute interpolation polynomials $s_1(X), \ldots, s_e(X)$ over domain $S$ such that $s_1(X), \ldots, s_e(X) \in \mathbb{F}_{<d}[X]$. 
$\mathcal{V}_{poly}$ sends $s_1(X), \ldots, s_e(X)$ to $\mathcal{I}$. 

\item 
At the end of the protocol, suppose $f_1(X), \ldots, f_t(X)$ are the polynomials that were sent from $\mathcal{P}_{poly}$ to 
$\mathcal{I}$. $\mathcal{V}_{poly}$ may ask $\mathcal{I}$ if certain polynomial identities hold between 
$$\{f_1(X), \ldots , f_t(X), g_1(X), \ldots, g_l(X), s_1(X), \ldots, s_e(X) \}$$ over set $S$ 
(i.e., if by evaluating all the polynomials that define the identity at each of the field elements from $S$ 
we obtain a true statement). Each such identity is of the form 

$$F(X) := G(X, h_1(v_1(X)), \dots , h_M(v_M(X))) \equiv 0,$$
for some $h_i(X) \in \{f_1(X), \ldots , f_t(X), g_1(X), \ldots , g_l(X), s_1(X), \ldots, s_e(X) \}$, \\ $G(X, X_1, \ldots, X_M) \in \mathbb{F}[X, X_1, \ldots , X_M]$, 
$v_1(X), \ldots , v_M(X) \in  \mathbb{F}_{<d}[X]$ such that $F(X) \in \mathbb{F}_{<D}[X]$ for every choice of 
$f_1(X), \ldots, f_t(X)$ made by $P_{poly}$ when following the protocol correctly. Note that some of the coefficients in the identities above may be 
from the set $\{\mathit{res_1}, \ldots, \mathit{res_u}\}$.
\item After receiving the answers from $I$ regarding the polynomial identities, 
$\mathcal{V}_{poly}$ outputs $\mathsf{acc}$ if all identities hold over set $S$, 
and outputs $\mathsf{rej}$ otherwise.
\end{enumerate}

\noindent Additionally, the following properties hold: \\

\noindent \textbf{Perfect Completeness:} If $\mathcal{P}_{poly}$ follows the protocol correctly and uses a witness $\omega$ with 
$(x, \omega) \in \mathcal{R}^c$, $\mathcal{V}_{poly}$ accepts with probability one.

\noindent \textbf{Knowledge Soundness:} There exists an efficient algorithm $E$, that given access to the messages of $\mathcal{P}_{poly}$ 
to $\mathcal{I}$ it outputs $\omega$ such that, for any strategy of $\mathcal{P}_{poly}$, the probability of $\mathcal{V}_{poly}$ 
outputting $\mathsf{acc}$ at the end of the protocol and, simultaneously, $(x, \omega) \in \mathcal{R}^c$ is overwhelming in $\lambda$.

\end{definition}

\noindent Our definition for polynomial protocols over ranges does not include a zero-knowledge property as it is not required in our current work. \\

\noindent Given the definition for polynomial protocols over ranges for conditional relations as detailed above, we are now ready to state the following result.
The proof follows with only minor changes from that of Lemmas 4.5. and 4.7. from~\cite{plonk}. 

\begin{lemma}(Compilation of Ranged Polynomial Protocols for Conditional NP Relations into Hybrid Model SNARKs using PLONK) 
\label{le:compilation_step_1}
Let $\mathscr{P}_{\mathcal{R}^c}$ be a public coin $S$-ranged $(d, D, t, l, u, e)$-polynomial protocol for relation $\mathcal{R}^c$ where only 
one identity is checked by $\mathcal{V}_{poly}$ and predicate $c$ from the definition of ${\mathcal{R}^c}$ needs to be fulfilled only by a part $x_1$ 
of the public input of the relation ${\mathcal{R}^c}$. Then one can construct a hybrid model SNARK protocol $\mathscr{P}^*_{\mathcal{R}^c}$ for relation 
$\mathscr{P}_{\mathcal{R}^c}$ with $\mathit{SNARK.PartInput}$ as defined below 
and with $\mathscr{P}^*_{\mathcal{R}^c}$ secure in the AGM under the $2d$-DLOG 
assumption\footnote{Definition 2.1. in PLONK~\cite{plonk} formally describes the $2d$-DLOG assumption.} such that:
\begin{enumerate}
\item The prover $\mathbf{P}$ in $\mathscr{P}^*_{\mathcal{R}^c}$ requires $\mathsf{e}(\mathscr{P}_{\mathcal{R}^c})$ $\gout{1}$-exponentiations where 
$\mathsf{e}(\mathscr{P}_{\mathcal{R}^c})$ is define analogously as in PLONK (see preamble of Section 4.2.), however it additionally takes into account 
polynomials $s_1(X), \ldots, s_e(X)$. 
\item The total prover communication consists of $t + t^*(\mathscr{P}_{\mathcal{R}^c}) + 1$ $\gout{1}$-elements and M $\mathbb{F}$-elements, where 
$t^*(\mathscr{P}_{\mathcal{R}^c})$ is defined identically as in PLONK (see preamble of Section 4.2.).
\item The verifier $\mathbf{V}$ in $\mathscr{P}^*_{\mathcal{R}^c}$ requires $t + t^*(\mathscr{P}_{\mathcal{R}^c})+1$ $\gout{1}$-exponentiations, 
two pairings and one evaluation of the polynomial $G$, and, additionally, the verifier in $\mathscr{P}^*_{\mathcal{R}^c}$ computes $e$ 
polynomial commitments to polynomials in the set $\{s_1(X), \ldots, s_e(X)\}$. 
\item For $x_1$ part of  $\mathit{state_1}$, the algorithm for computing partial inputs is defined as 
\begin{align*}
&\mathit{SNARK.PartInput}(\mathit{srs}, \mathit{state_1}, \mathcal{R}^c) \\
&\mathit{If \ } c(x_1) = 0 \\
&\ \ \ \ \mathit{Return} \\
&\mathit{Else } \\
&\ \ \ \ \mathit{Return} (\mathit{state_1}, x_1)
\end{align*}
\end{enumerate}
\end{lemma}
\section{Postponed Proofs for Packed Accountable Ranged Polynomial Protocol}
\label{sec:missing_snark_proofs}

\noindent We give below the missing proofs from Section~\ref{sec_a}:

\begin{test_claim}
\label{claim:bitvector_comm}
If the polynomial identities $id_6(X) = 0, id_7(X) = 0$ hold over range $H$, then, 
\ewnp, 
we have $c_{a,i} =  2^{i \mod \block} \cdot r^{i \div \block}$, $\forall i < n$ and $\mathsf{sum} = \sum_{i=0}^{n-1}b_i \cdot c_{a,i}$, 
where $b_i = b(\omega^i), \forall i <n$. If, additionally, identity $id_5(X) = 0$ holds over $H$, 
$r$ has been randomly chosen in $\mathbb{F}$, $\mathsf{sum} = \sum_{j=0}^{\frac{n}{\block}-1} b'_{j}r^j$ 
(as computed by $\mathcal{V}_{poly}$) and $\mathit{bit_{i}} \in \mathbb{B}, \forall i < n$ and 
$b'_{j} = \sum_{k=0}^{\block -1}2^k \cdot \mathit{bit_{\block \cdot j + k}}, \forall 0 \leq j \leq \frac{n}{\block} -1$ 
(due to the input $(b'_{0}, \ldots, b'^{\frac{n}{\block} -1})$ 
being interpreted by the verifier $\mathcal{V}_{poly}$ as in $\mathbb{F}_{|\block|}^{\frac{n}{\block}}$), then \ewnp, 
$b_i = \mathit{bit_{i}}, \forall i <n$.
\end{test_claim}

\begin{proof}
To prove the first part of the claim, assume by contradiction that $c_{a,0} = k  \neq 1$. 
Then, by induction, since $id_6(X) = 0$ on $H$, 
$$c_{a,i} = k \cdot 2^{i \mod \block} \cdot r^{i \div \block}, \forall 0<i<n.$$ 
Additionally, the property
\begin{align*} 
c_{a,0} = c_{a,n-1} \cdot (2+ (\frac{r}{2^{\block -1}} -2) \cdot 1) + (1 - r^{\frac{n}{\block}}) \tag{1}
\end{align*}  
\noindent holds (again, from $id_6(X) = 0$ on $H$). However, substituting $c_{a,0} = k$ 
and $c_{a,n-1} = k \cdot 2^{\block -1} \cdot r^{\frac{n}{\block} -1}$ in $(1)$, we obtain 
$k = k \cdot 2^{\block -1} \cdot r^{\frac{n}{\block} -1} \cdot \frac{r}{2^{\block -1}} +1 - r^{\frac{n}{\block}}$ which is equivalent to 
$k(1 - r^{\frac{n}{\block}}) = 1 - r^{\frac{n}{\block}}$, and, due to Schwartz-Zippel Lemma and the fact that degree $n$ is negligibly 
smaller compared to the size of $\mathbb{F}$, this implies \ewnp $k =1$ thus contradiction,
so the values $c_{a,i}$ have indeed the claimed form. \\
Next, by expanding $id_7(X) = 0$ over $H$, the following holds
\begin{align*}
acc_{a,1} &= acc_{a,0} + b_0 \cdot c_{a,0} \\
acc_{a,2} &= acc_{a,1} + b_1\cdot c_{a,1} \\
\ldots \\
acc_{a,n-1} &= acc_{a,n-2} + b_{n-2} \cdot c_{a,n-2} \\
acc_{a,0} &= acc_{a,n-1} + b_{n-1} \cdot c_{a,n-1} - \mathsf{sum}.
\end{align*}

\noindent By summing together the LHS and, respectively, the RHS of the equalities above and 
reducing the equal terms, we obtain $\mathsf{sum} = \sum_{i=0}^{n-1}b_i\cdot c_{a,i}$. \\

\noindent For the second part of the claim, since $id_5(X) = 0$ holds over $H$ then 
$b_i = b(\omega^i) \in \mathbb{B}, \forall i \leq n-1$. Finally, from 
verifier's computation and from the first part of the claim we have 
\begin{align*}
\sum_{j=0}^{\frac{n}{\block}-1} b'_{j}r^j = \mathsf{sum} =  \sum_{i=0}^{n-1} b_i \cdot c_{a,i} & = \sum_{i=0}^{n-1} b_i \cdot 2^{i \mod \block} \cdot r^{i \div \block} = \\
&= \sum_{j=0}^{\frac{n}{\block} -1} (\sum_{k=0}^{\block -1} 2^k \cdot b_{\block\cdot j +k}) \cdot r^j = \sum_{j=0}^{\frac{n}{\block} -1} b''_{j}r^j \tag{2},
\end{align*}
where $\forall j, b''_{j}$ are field elements equal to the binary representation that uses contiguous blocks 
of $\block$ components from the bitmask $(\mathit{b}_0, \ldots, \mathit{b}_{n-1})$.
Since both the LHS and the RHS of relation (2) represent two ways of computing $\mathsf{sum}$ as an inner product of a vector 
of field elements (on one hand, $(\mathit{b'_{0}}, \ldots, \mathit{b'_{\frac{n}{\block}-1}})$, on the other hand, 
$(\mathit{b''_{0}}, \ldots, \mathit{b''_{\frac{n}{\block}-1}})$ ) with the vector $(1, r, \ldots, r^{\frac{n}{\block}-1})$, 
where $r$ has been chosen at random, by the small exponents test \cite{small_exponents}, we obtain that \ewnp 
$\mathit{b''_{j}} = \mathit{b'_{j}}, \\ \forall \ 0 \leq j\leq \frac{n}{\block}-1$. Finally, if we equate the bit representation in $\mathbb{F}$ 
(i.e., using field elements from $\mathbb{B}$) of field elements $\mathit{b''_{j}}$ and $\mathit{b'_{j}}, \forall 0\leq j\leq \frac{n}{\block}-1$ and remember that, 
by verifier's check or by construction, respectively, each such field element has no more that \block binary bits, we can conclude that \ewnp 
$b_i = \mathit{bit_i}, \forall i <n$.
\end{proof}

\begin{lemma} 
$\Pa$ as described in Section~\ref{sec_a} is an $H$-ranged polynomial protocol for conditional NP relation $\Ra$.
\end{lemma}

\begin{proof} 
If $(\mathbf{b'}, \mathbf{pk}, \mathit{apk}, \mathbf{bit}) \in \Ra$, meaning that 
$\mathbf{pk} \in \ginn{1}^{n-1}$ and $\mathbf{bit} \in \mathbb{B}^n$ and $\mathit{apk} = \sum_{i=0}^{n-2} [\mathit{bit_i}] \cdot \mathit{pk_i}$ and 
 $b'_{j} = \sum_{i=0}^{\block -1}2^i \cdot \mathit{bit_{\block \cdot j + i}}, \forall j < \frac{n}{\block}$ 
hold then it is easy to see that the honest prover $\mathcal{P}_{poly}$ in $\Pa$ will convince the honest 
verifier $\mathcal{V}_{poly}$ in $\Pa$ to accept with probability $1$ so perfect completeness holds. Regarding knowledge-soundness, if the verifier $\mathcal{V}_{poly}$ in $\Pa$ accepts, 
then the extractor $\mathcal{E}$ sets $(\mathit{bit}_0, \ldots, \mathit{bit}_{n-1})$ as the vector of evaluations over $H$ of polynomial $b(X)$ sent by $\mathcal{P}_{poly}$ 
to $\mathcal{I}$. Next, we prove that if $(\mathit{pk_0}, \ldots, \mathit{pk_{n-2}}) \in \ginn{1}^{n-1}$ and the verifier in $\Pa$ accepts, 
then $$((\mathit{b'_{0}}, \ldots, \mathit{b'_{\frac{n}{\block}-1}}), (\mathit{pk_0}, \ldots, \mathit{pk_{n-2}}), \mathit{apk}, (\mathit{bit}_0, \ldots, \mathit{bit}_{n-1})) \in \Ra,$$ 
which is equivalent to proving that  $\mathit{apk} = \sum_{i=0}^{n-2} [\mathit{bit_i}] \cdot \mathit{pk_i}$ and 
 $\mathbf{bit} \in \mathbb{B}^n$ and  $$b'_{j} = \sum_{i=0}^{\block -1}2^i \cdot \mathit{bit_{\block \cdot j + i}}, \forall j < \frac{n}{\block}.$$

According to Claim~\ref{claim:bitvector_comm} and corollary~\ref{corollary:keys_affine_comm} this indeed holds \ewnp
\end{proof}

\section{Choosing $h$ when $\einn = \ginn{1}$}
\label{sec:other_choice_h}

\noindent For the polynomial protocols and custom SNARKs we have designed in Section~\ref{sec:snarks}, we have chosen 
$h \in \einn \setminus \ginn{1}$. However, we have not covered so far the case when $\einn = \ginn{1}$ and how to 
choose $h$ in such a situation. Our current section will give a guide for that. In fact, if $\einn = \ginn{1}$, we provide a 
method of choosing $h$ that will be suitable not only for our custom SNARKs from Section~\ref{sec:snarks}, but also for any other 
SNARK that proves the correctness of an aggregated public key (i.e., $\mathit{apk}$ for an aggregatable signature scheme), 
among other modelled constraints. \\

\noindent Let $\mathcal{H}$ be a hash function, $\mathcal{H}: \{0, 1\} \rightarrow \mathbb{F}$ such that $\mathcal{H}$ is used for 
the Fiat-Shamir transformation of a succinct argument of knowledge (including a sub-proof of correctness of $\mathit{apk}$) 
into its non-interactive version. Let $x$ be the public input corresponding to the above succinct argument of knowledge. 
Note that in case of the hybrid model SNARKs defined in this work (see Definition~\ref{dfn_snark}), the public input includes the partial input. 
For a concrete example of a partial input, see one of our custom SNARKs fully rolled out in Section~\ref{sec:rolled_out}. Then, the prover and 
the verifier compute $h$, for example as 
$$h = \mathcal{H}(h, \textit{"starting \ input \ point \ for \ public \ keys \ addition"}).$$
Intuitively, in the random oracle model (which is already an assumption we need for a secure Fiat-Shamir transformation that preserves 
knowledge-soundness), $h$ is thus an elliptic curve point on $\einn$, uniformly distributed on $\einn = \ginn{1}$. Moreover, by including 
the additional string in the computation of the hash function $\mathcal{H}$ we can re-use the same function for multiple specific tasks 
related to a SNARK (e.g., both for defining $h$ and for obtaining non-interactivity (N) in a SNARK). Hence, for a large enough elliptic 
curve group (i.e., an elliptic curve for which the number of elliptic curve points is $O(2^{\lambda})$ for security parameter $\lambda$), 
the probability of $h$ plus any elliptic curve point being equal to a fixed elliptic curve is negligible in $\lambda$. This, in turn, ensures that 
condition 3) from Observation~\ref{obs:incomplete_addition} is only met with negligible probability. Hence, knowledge soundness 
for our polynomial protocols in Sections~\ref{sec_la},\ref{sec_a} still holds with overwhelming probability, so all follow-up results in 
Section~\ref{sec:snarks} still hold. 

\section{Compiler for Hybrid Model SNARKs with Mixed Inputs}
\label{sec_two_step_compiler}
%\vspace{-0.05in} 
\subsection{Technical Challenges and Contributions Regarding our Custom SNARKs} 
\label{sec:technical_challenges}
%\vspace{-0.1cm}

In order to define and implement our committee key scheme accountable light client systems and in order to design the custom SNARKs that support our efficiency results, 
we had to tackle some technical challenges and make additional contributions as summarised below.

%\vspace{-0.2cm}
\paragraph{Extending PLONK Compiler to Mixed Commitment and Vectors NP Relations} Firstly, our custom SNARKs takes inspiration from PLONK \cite{plonk} in terms design of the proof system used, and of the circuits and gates. However, our SNARKs also have differences compared to PLONK. PLONK applies to NP relations  that use  vectors of field elements for 
public inputs and witnesses.  However we need SNARKs whose defining NP relations also have  polynomial commitments (in our case, the committee key $C$) as part of their public inputs. Hence, the original PLONK compiler does not suffice; we extend it with a second step in which we show that under certain conditions which our protocol fulfils, the SNARKs obtained using the original PLONK compiler are also SNARKs for a mixed type of NP relation containing both vectors and polynomial commitments. The full details and proofs can be found in Section~\ref{sec_two_step_compiler} 
and we believe this compiler extension to be of independent interest. 

%\vspace{-0.2cm}
\paragraph{Conditional NP Relations for Efficiency} Secondly, we also require NP relations that have a well-defined subpredicate which is verified outside the SNARKs. In a blockchain instantiation, 
any current validator set has to come to a consensus, among other things, on the next validator set, represented by a set of public keys. The validator set 
computes and signs a pair of polynomial commitments to the next set of validators' public keys. Before including a public key in the set, the validators perform several 
checks on the proposed public key, such as being in a particular subgroup of the elliptic curve. This check is not performed by the SNARKs' constraint system, but is 
required for the correctness of the statement the SNARKs prove. This design decision makes our SNARKs more efficient, but it also means we have to extend the 
usual definition of NP relations to conditional NP relations, where in fact, one of the subpredicates that define the conditional relation is checked outside the SNARKs 
or ensured due to a well-defined assumption. We introduce the general notion of conditional NP relation in Section~\ref{sec:conditional_relations} and describe our 
concrete conditional NP relations in Section~\ref{sec:snarks}.   

%\vspace{-0.2cm}
\paragraph{Hybrid Model SNARKs} In line with the two above technical challenges and the solutions we came up with, we extend the existing definitions 
related to SNARKs~\cite{groth16,plonk} by introducing an algorithm which we call $\mathit{PartInput}$. For our use case, this allows us to separate the public input for the NP relations that define our custom SNARKs in two: a part 
that is computed by the current set of validators on the blockchain in question and the rest of the public input plus the corresponding SNARK proof are 
computed by a (possibly malicious) prover interacting with the light client verifier. Our newly introduced notion of hybrid model SNARK (see Section~\ref{sec:snarks_defs}) 
generalises this public input separation concept and its definition is used to prove the security of our custom SNARKs in Section~\ref{sec_two_step_compiler}.

\subsection{SNARK compiler}

\noindent We present a two-steps PLONK-based compilation technique from 
ranged polynomial protocols for conditional NP relations (formal definition in Section~\ref{supplementary_poly_protocols_appendix}) to hybrid 
model SNARKs (Definition~\ref{dfn_snark}) such that the conditional NP relations that define the SNARKs we compile in the 
second step contain both polynomial commitments and vectors of field elements as public inputs. By using just the first step of our 
compiler which is equivalent to the original PLONK compiler~\cite{plonk}, one would 
not be able to obtain SNARKs with mixed public inputs consisting of both vectors of field elements and also poly commitments. 
In turn, this type of NP relations with mixed inputs is crucial for designing accountable light clients via the use of committee key schemes 
(see Section~\ref{sec:inst_committee_key}).\\
\vspace{-0.2in}
%\subsubsection{Our Compiler: Step 1} 
\subsection{Our Compiler: Step 1} 
\label{compiler_step_1}
\vspace{-0.05in}
%\noindent \textbf{(PLONK Compiler - from Polynomial Protocols to SNARKs)} \\

\noindent Our first step applies the PLONK compiler~\cite{plonk} (Lemma 4.7): we compile the information theoretical ranged polynomial protocols $\Pla$ and $\Pa$ 
for relations $\Rla$ and $\Ra$, respectively (see sections \ref{sec_la} and \ref{sec_a}) into 
(hybrid model) SNARKs $\Plastar$, and $\Pastar$, respectively. We can define this compilation step 
for any ranged polynomial protocols for relations (as per Definition~\ref{supplementary_poly_protocols_appendix} in section~\ref{def_ranged_poly_protocol}). 
\begin{comment}
In order to do that we need: 
\begin{itemize}
\item  The batched version of KZG polynomial commitments~\cite{KZG_10} described in section 3 of PLONK~\cite{plonk}.\footnote{In fact, 
one can replace the use of KZG polynomial commitments with any binding polynomial commitment that has knowledge-soundness, including non-homomorphic polynomial commitments, 
such as FRI-based polynomial commitments (e.g., RedShift~\cite{redshift}). If the optimisation gained from PLONK linearisation technique is a goal, 
then, with minimal changes one can use any homomorphic polynomial commitment, e.g., the discrete logarithm based polynomial commitment 
from Halo~\cite{halo}.}
\item A general compilation technique: such a technique has been already defined in Lemma 4.7 of PLONK; combined with Lemma 4.5 
from PLONK this technique can be applied with minor adaptations (this includes the corresponding technical measures) to the notion of ranged 
polynomial protocols.  
\item So far, both the ranged polynomial protocols for relations and the protocols resulted after the first compilation step have been explicitly defined as interactive 
protocols. In order to obtain the non-interactive version of the latter (essentially the N in SNARK) one has to apply the Fiat-Shamir 
transform~\cite{FS_transform}, \cite{FS_transform_with_proof}, \cite{SE_plonk}.
\end{itemize}

\end{comment}
%\begin{comment}
\noindent Let $\mathcal{R}$ be a (conditional) NP relation, let $\mathscr{P}_{\mathcal{R}}$ be a ranged polynomial protocol for 
relation $\mathcal{R}$ and let $\mathscr{P}^*_{\mathcal{R}}$ be the SNARK compiled from $\mathscr{P}_{\mathcal{R}}$ using the PLONK compiler.  
The compilation technique requires the SNARK prover of  $\mathscr{P}^*_{\mathcal{R}}$ to compute 
polynomial commitments to all polynomials that the prover $\mathcal{P}_{poly}$ in $\mathscr{P}_{\mathcal{R}}$ sent to the ideal party $\mathcal{I}$. Analogously, 
it requires the SNARK verifier of $\mathscr{P}^*_{\mathcal{R}}$ to compute polynomial commitments to all pre-processed polynomials\footnote{This is a one-time computation that is 
reused by the SNARK verifier for all SNARK proofs over the same circuit.} as well polynomial commitments to polynomials the verifier $\mathcal{V}_{poly}$ 
in $\mathscr{P}_{\mathcal{R}}$ sent to the ideal party $\mathcal{I}$. Then, the SNARK prover sends the SNARK verifier openings to 
all the polynomial commitments computed by him as well as the polynomial commitments computed by the SNARK verifier. The SNARK 
prover additionally sends the corresponding batched proofs for polynomial commitment openings. In turn, the SNARK verifier accepts or rejects based 
on the result of the verification of the batched polynomial commitment scheme. \\
%\end{comment}

%\noindent A more efficient compilation technique exists which reduces the number of polynomial commitments and alleged polynomial commitments openings 
%(i.e., both group elements and field elements) sent by the SNARK prover to the SNARK verifier; this, in turn, reduces the size of the SNARK proof. 
%This technique is called linearisation and is described, at a high level, after Lemma 4.7 in PLONK. The existing description however covers only the 
%SNARK prover side and it does not detail the SNARK verifier side so in the following we cover that. \\
\noindent PLONK proposes a more efficient compilation technique (i.e., linearisation, see explanation after Lemma 4.7 in PLONK) 
which reduces the SNARK proof size. 
%\noindent By functionality, the vectors that are handled by the the verifier $\mathcal{V}_{poly}$ are 
%of two types: pre-processed vectors and public input vectors. These two types of vectors are used by $\mathcal{V}_{poly}$ 
%to obtain, via interpolation over the range on which the respective range polynomial protocol is defined, pre-processed polynomials 
%(as used in the definition 2 in section 2 of supplementary material, e.g., polynomial $aux(X)$ used in section~\ref{sec_a}) and 
%public-inputs-derived polynomials (e.g., polynomials $pkx(X)$ and $pky(X)$ used in sections~\ref{sec_la} and ~\ref{sec_a} 
%and polynomial $b(X)$ used in section~\ref{sec_la}). The efficient linearisation technique allows the SNARK verifier to reduce the 
%number of polynomial commitments it has to compute compared to the general PLONK compiler in the following way. Instead of 
%having to compute polynomial commitments to all polynomials $\mathcal{V}_{poly}$ sends to $\mathcal{I}$ (including any corresponding 
%pre-processed polynomials), the SNARK verifier computes polynomial evaluations at one or multiple random points (as per the linearisation 
%step specific requirements) for all the polynomials that are either easy to evaluate (e.g., polynomial $aux(X)$ used in section~\ref{sec_a}) or 
%all the polynomials that are obtained from vectors that do not take up a large amount of memory (e.g., polynomial $b(X)$ used in section~\ref{sec_la}). 
%For the rest of the polynomials (e.g., $\mathit{pkx}(X)$ and $\mathit{pky}(X)$), the SNARK verifier computes polynomial commitments as before.\\
For our specific case, this allows the SNARK verifier to reduce the 
number of polynomial commitments it has to compute compared to the general PLONK compiler by computing 
polynomial evaluations at one or multiple random points (as per the linearisation step specific requirements) 
for all the polynomials that are either easy to evaluate (e.g., polynomial $aux(X)$ used in section~\ref{sec_a} and $\Ra$) or 
all the polynomials that are obtained from input vectors that do not take up a large amount of memory (e.g., polynomial $b(X)$ used in section~\ref{sec_la} and $\Rla$).
%\noindent We note we can apply all the techniques mentioned above, including the combined prover-and-verifier-side linearisation 
%to compile our ranged polynomial protocols $\Pla$ and $\Pa$ into the corresponding SNARKs $\Plastar$ and $\Pastar$, respectively.
Finally, we state in Section~\ref{def_ranged_poly_protocol}, Lemma~\ref{le:compilation_step_1} under which conditions and how efficiently 
one can compile ranged polynomial protocols for pure vector-based conditional NP relations 
into hybrid model SNARKs using only the original PLONK compiler and we give a more in-depth explanation of this step in section~\ref{first_step_compiler}.\\
\noindent \textbf{(PLONK Compiler - from Polynomial Protocols to SNARKs)} \\

\noindent We summarise and exemplify below the PLONK-based compilation technique~\cite{plonk} from 
ranged polynomial protocols for conditional NP relations (formal definition in Section~\ref{supplementary_poly_protocols_appendix}) to 
SNARKs for pure vector-based NP relations. This is also the first of our two-steps compiler. Concretely, our first step applies the PLONK compiler~\cite{plonk} (lemma 4.7): 
we compile the information theoretical ranged polynomial protocols $\Pla$ and $\Pa$ for relations $\Rla$ and $\Ra$, respectively (see Sections~\ref{sec_la},\ref{sec_a}) 
into SNARKs $\Plastar$, and $\Pastar$, respectively. We can define this compilation step for any ranged polynomial protocols for relations 
(as per Definition~\ref{supplementary_poly_protocols_appendix} in Section~\ref{def_ranged_poly_protocol}). In order to do that we need: 
\begin{itemize}
\item  The batched version of KZG polynomial commitments~\cite{KZG_10} described in Section 3 of PLONK~\cite{plonk}.\footnote{In fact, 
one can replace the use of KZG polynomial commitments with any binding polynomial commitment that has knowledge-soundness, including non-homomorphic polynomial commitments, 
such as FRI-based polynomial commitments (e.g., RedShift~\cite{redshift}). If the optimisation gained from PLONK linearisation technique is a goal, 
then, with minimal changes one can use any homomorphic polynomial commitment, e.g., the discrete logarithm based polynomial commitment 
from Halo~\cite{halo}.}
\item A general compilation technique: such a technique has been already defined in lemma 4.7 of PLONK; combined with lemma 4.5 
from PLONK this technique can be applied with minor adaptations (this includes the corresponding technical measures) to the notion of ranged 
polynomial protocols.  
\item So far, both the ranged polynomial protocols for relations and the protocols resulted after the first compilation step have been explicitly defined as interactive 
protocols. In order to obtain the non-interactive version of the latter (essentially the N in SNARK) one has to apply the Fiat-Shamir 
transform~\cite{FS_transform}, \cite{FS_transform_with_proof}, \cite{SE_plonk}.
\end{itemize}

\noindent Let $\mathcal{R}$ be a (conditional) NP relation, let $\mathscr{P}_{\mathcal{R}}$ be a ranged polynomial protocol for 
relation $\mathcal{R}$ and let $\mathscr{P}^*_{\mathcal{R}}$ be the SNARK compiled from $\mathscr{P}_{\mathcal{R}}$ using the PLONK compiler.  
The compilation technique requires the SNARK prover of  $\mathscr{P}^*_{\mathcal{R}}$ to compute 
polynomial commitments to all polynomials that the prover $\mathcal{P}_{poly}$ in $\mathscr{P}_{\mathcal{R}}$ sent to the ideal party $\mathcal{I}$. Analogously, 
it requires the SNARK verifier of $\mathscr{P}^*_{\mathcal{R}}$ to compute polynomial commitments to all pre-processed polynomials\footnote{This is a one-time computation that is 
reused by the SNARK verifier for all SNARK proofs over the same circuit.} as well polynomial commitments to polynomials the verifier $\mathcal{V}_{poly}$ 
in $\mathscr{P}_{\mathcal{R}}$ sent to the ideal party $\mathcal{I}$. Then, the SNARK prover sends the SNARK verifier openings to 
all the polynomial commitments computed by him as well as the polynomial commitments computed by the SNARK verifier. The SNARK 
prover additionally sends the corresponding batched proofs for polynomial commitment openings. In turn, the SNARK verifier accepts or rejects based 
on the result of the verification of the batched polynomial commitment scheme. \\

\noindent A more efficient compilation technique exists which reduces the number of polynomial commitments and alleged polynomial commitments openings 
(i.e., both group elements and field elements) sent by the SNARK prover to the SNARK verifier; this, in turn, reduces the size of the SNARK proof. 
This technique is called linearisation and is described, at a high level, after Lemma 4.7 in PLONK. The existing description however covers only the 
SNARK prover side and it does not detail the SNARK verifier side so in the following we cover that. \\

\noindent By functionality, the vectors that are handled by the the verifier $\mathcal{V}_{poly}$ are 
of two types: pre-processed vectors and public input vectors. These two types of vectors are used by $\mathcal{V}_{poly}$ 
to obtain, via interpolation over the range on which the respective range polynomial protocol is defined, pre-processed polynomials 
(as used in the Definition~\ref{supplementary_poly_protocols_appendix} in Section~\ref{def_ranged_poly_protocol}, e.g., polynomial $aux(X)$ used in Section~\ref{sec_la}) and 
public-inputs-derived polynomials (e.g., polynomials $pkx(X)$ and $pky(X)$ used in Sections~\ref{sec_la},\ref{sec_a})
and polynomial $b(X)$ used in Section~\ref{sec_la}). The efficient linearisation technique allows the SNARK verifier to reduce the 
number of polynomial commitments it has to compute compared to the general PLONK compiler in the following way. Instead of 
having to compute polynomial commitments to all polynomials $\mathcal{V}_{poly}$ sends to $\mathcal{I}$ (including any corresponding 
pre-processed polynomials), the SNARK verifier computes polynomial evaluations at one or multiple random points (as per the linearisation 
step specific requirements) for all the polynomials that are either easy to evaluate (e.g., polynomial $aux(X)$ used in Section~\ref{sec_a}) or 
all the polynomials that are obtained from vectors that do not take up a large amount of memory (e.g., polynomial $b(X)$ used in Section~\ref{sec_la}). 
For the rest of the polynomials (e.g., $\mathit{pkx}(X)$ and $\mathit{pky}(X)$), the SNARK verifier computes polynomial commitments as before.\\

\noindent We note we can apply all the techniques mentioned above, including the combined prover-and-verifier-side linearisation 
to compile our ranged polynomial protocols $\Pla$ and $\Pa$ into the corresponding SNARKs $\Plastar$ and $\Pastar$, respectively. 
To conclude this step, we formally state in Section~\ref{def_ranged_poly_protocol}, Lemma~\ref{le:compilation_step_1} under which condition and how efficiently 
one can compile ranged polynomial protocols for conditional NP relations (where the public inputs are interpreted as vector of field elements) 
into hybrid model SNARKs by using only the original PLONK compiler. \\

\vspace{-0.2in}
%\subsubsection{Our Compiler: Step 2}
\subsection{Our Compiler: Step 2}
\label{compiler_step_2}
\vspace{-0.05in}
\noindent \textbf{(Mixed Vector and Commitments based NP Relations and Associated SNARKs)} \\

\noindent The type of NP relations we have worked with so far as well as the more general PLONK NP relation 
(\cite{plonk}, Section 8.2) have vector of field elements as public inputs. Next we show that SNARKs 
compiled using Step 1 can become, under certain assumption, SNARKs for a new type of NP relation 
that specifically contains polynomial commitments as part of the input. Interpreting 
our already compiled SNARKs as SNARKs for this new type of NP relation is essential for designing 
accountable light client systems via committee key schemes (see Instantiation~\ref{inst:cks} 
in Section~\ref{sec:inst_committee_key}).  

\noindent Let conditional NP relation $\mathcal{R}_{\mathit{vec}}^c$  be:
\begin{align*}
\mathcal{R}_{\mathit{vec}}^c = \{&(\mathbf{input_1} \in \mathbf{\mathcal{D}_1}, \mathbf{input_2} \in\mathbf{\mathcal{D}_2}; \mathbf{witness_1}): \\  
&p_1(\mathbf{input_1}, \mathbf{input_2}, \mathbf{witness_1}) = 1 \ | \ c(\mathbf{input_1}) = 1 \ \wedge\ \\
&\wedge \ p_2(\mathbf{input_1}, \mathbf{input_2}, \mathbf{witness_1}) = 1 \},
\end{align*}
\noindent where $\mathbf{input_1}$, $\mathbf{input_2}$ are two sets of public input vectors 
belonging domains  $\mathcal{D}_1$, $\mathcal{D}_2$. $\mathbf{witness_1}$ is a set of witness vectors and $c$, $p_1$, $p_2$ are predicates. 
Let $\mathscr{P}_{\mathit{vec}}$ be a ranged polynomial protocol for relation $\mathcal{R}_{\mathit{vec}}^c$. Note that since $c$ applies 
only to a part of the public input for relation $\mathcal{R}_{\mathit{vec}}^c$ 
(i.e., $\mathbf{input_1}$), we can apply Lemma~\ref{le:compilation_step_1} of Section~\ref{supplementary_poly_protocols_appendix} and Step 1 
of our compiler to polynomial protocol $\mathscr{P}_{\mathit{vec}}$. \\

\noindent We make the following hybrid model assumptions:
\begin{itemize}
\item (HMA.1.) $\mathcal{V}_{poly}$ in $\mathscr{P}_{\mathit{vec}}$ computes 
$\mathit{Q_{1,\mathbf{input_1}}}(X), \ldots, \mathit{Q_{m, \mathbf{input_1}}}(X)$ which depend deterministically on $\mathbf{input_1}$ and sends them to $\mathcal{I}$. 
\item (HMA.2.) $\mathcal{V}_{poly}$ in $\mathscr{P}_{\mathit{vec}}$ does not use $\mathbf{input_1}$ in any further computation of 
any other polynomials or values its sends to $\mathcal{I}$.
\item (HMA.3.) By evaluating $\mathit{Q_{1,\mathbf{input_1}}}(X), \ldots, \mathit{Q_{m, \mathbf{input_1}}}(X)$ over the range on which 
$\mathscr{P}_{\mathit{vec}}$ is defined one obtains (using some efficiently computable and deterministic transformations) the set of vectors $\mathbf{input_1}$. 
\end{itemize} 
We denote by $\mathscr{P}^*_{\mathit{vec}}$ the hybrid model SNARK obtained after compiling $\mathscr{P}_{\mathit{vec}}$ using compilation Step 1. 
Due to (HMA.1.) and according to Step 1, the SNARK verifier in 
$\mathscr{P}^*_{\mathit{vec}}$ computes $$\mathit{Com_1} = \mathit{Com}(\mathit{Q_{1,\mathbf{input_1}}}), \ldots, \mathit{Com_m} = \mathit{Com}(\mathit{Q_{m,\mathbf{input_1}}})$$ 
which are KZG poly commitments to $\mathit{Q_{1,\mathbf{input_1}}}(X), \ldots, \mathit{Q_{m, \mathbf{input_1}}}(X)$. We denote vector
$(\mathit{Com_1}, \ldots, \mathit{Com_m})$ by $\mathbf{Com}(\mathbf{input_1})$ and we denote 
by $\mathcal{C}$ the set of all $\mathit{KZG}$ poly commitments or vectors of such poly commitments. We also define the relation: 
\begin{align*}
\mathcal{R}_{\mathit{vec}, \mathit{com}}^c = \{& \mathbf{C} \in \mathcal{C}, \mathbf{input_2} \in \mathbf{\mathcal{D}_2}; \mathbf{witness_1}, \mathbf{witness_2}):  \\
& p_1(\mathbf{witness_2}, \mathbf{input_2}, \mathbf{witness_1}) =1  \ |\ c(\mathbf{witness_2}) = 1  \ \wedge\  \\
& \wedge\ p_2(\mathbf{witness_2}, \mathbf{input_2}, \mathbf{witness_1}) = 1\ \wedge \\
& \wedge\ \mathbf{C} = \mathbf{Com}(\mathbf{witness_2})\}
\end{align*}
\noindent Finally, for $\mathbf{input_1}$ part of $\mathit{state_1}$, we define $\mathit{SNARK.PartInput}$:
\begin{align*} 
&\mathit{SNARK.PartInput}(\mathit{srs}, \mathit{state_1},\mathcal{R}_{\mathit{vec}, \mathit{com}}^c) \\  
& \mathit{If \ }  c(\mathbf{input_1}) = 0 \textit{ then} \ \mathit{Return} \\
& \mathit{Else} \\
& \ \ \ \ \textit{Compute via interpolation on } \mathscr{P}_{\mathit{vec}}  \textit{ range } 
\mathit{Q_{1,\mathbf{input_1}}}(X), \ldots, \mathit{Q_{m, \mathbf{input_1}}}(X).\\
& \ \ \ \ \mathbf{C} = (\mathit{Com}(Q_{1,\mathbf{input_1}}(X)), \ldots, \mathit{Com}(Q_{m,\mathbf{input_1}}(X))) \\
& \ \ \ \ \mathit{state_2} =  \mathit{state_1} \cup \{ \mathbf{C} \} \textit{ then} \ \mathit{Return} (\mathit{state_2,  \mathbf{C}})
\end{align*}

\noindent With the above notation, \textbf{our compiler's Step 2 is:} \\
\noindent The alleged hybrid model SNARK $\mathscr{P}_{\mathit{vec}}^{h}$ for relation $\mathcal{R}_{\mathit{vec}, \mathit{com}}^c$ is:
\begin{itemize}
\item $\mathit{SNARK.Setup}$ and $\mathit{SNARK.KeyGen}$ are as for relation $\mathcal{R}^{c}_{\mathit{vec}}$.
\item $\mathit{SNARK.PartInput}$ for relation $\mathcal{R}^{c}_{\mathit{vec}}$ 
(see Lemma~\ref{le:compilation_step_1} in Section~\ref{supplementary_poly_protocols_appendix}) 
is replaced with $\mathit{SNARK.PartInput}$ for relation $\mathcal{R}_{\mathit{vec}, \mathit{com}}^c$.
\item $\mathit{SNARK.Prover}$ for relation $\mathcal{R}_{\mathit{vec}, \mathit{com}}^c$ is identical with 
$\mathit{SNARK.Prover}$ for relation $\mathcal{R}^{c}_{\mathit{vec}}$ (as compiled using Step 1) with the appropriate 
re-interpretation of the public inputs and witness.
\item $\mathit{SNARK.Verifier}$ for relation $\mathcal{R}_{\mathit{vec}, \mathit{com}}^c$ is identical with 
$\mathit{SNARK.Verifier}$ for relation $\mathcal{R}^{c}_{\mathit{vec}}$ (as compiled using Step 1) with the appropriate 
re-interpretation of the public inputs and such that $\mathit{SNARK.Verifier}$ for $\mathcal{R}_{\mathit{vec}, \mathit{com}}^c$ does
not compute the polynomial commitments to the polynomials defined by assumption (HMA.1.).
\end{itemize}
%\vspace{-0.2in}
\noindent \begin{lemma} 
\label{sergey_type_relations} 
Let $\mathscr{P}_{\mathit{vec}}$ be a ranged polynomial protocol for relation $\mathcal{R}^c_{\mathit{vec}}$ defined above and let 
$\mathscr{P}_{\mathit{vec}}^{*}$ be the hybrid model SNARK for relation $\mathcal{R}^c_{\mathit{vec}}$ secure in the AGM obtained 
by compiling $\mathscr{P}_{\mathit{vec}}$ using our compiler's Step 1. If the hybrid model assumptions (HMA.1.) - (HMA.3.) hold w.r.t. 
protocol $\mathscr{P}_{\mathit{vec}}$ and $\mathit{State}_{\mathcal{R}_{\mathit{vec}, \mathit{com}}} \neq \emptyset $ then 
$\mathscr{P}_{\mathit{vec}}^{h}$ as compiled using our compiler's Step 2 is a hybrid model SNARK for relation 
$\mathcal{R}_{\mathit{vec}, \mathit{com}}^c$ secure also in the AGM.
\end{lemma}

\begin{proof} Let $\mathcal{E}_{\mathit{KZG}}$ and $\mathcal{E}$ be the extractors from the knowledge-soundness definitions for the 
$\mathit{KZG}$ batch polynomial commitment scheme (as in Definition 3.1, Section 3 in~\cite{plonk}) and the hybrid model 
SNARK $\mathscr{P}^*_{\mathcal{R}}$ for relation $\mathcal{R}^c_{\mathit{vec}}$ (as per Definition~\ref{dfn_snark}), respectively. 
Let $\mathcal{A}$ be an adversary against knowledge soundness in the hybrid model w.r.t. 
$\mathscr{P}_{\mathit{vec}}^{h}$ and relation $\mathcal{R}_{\mathit{vec}, \mathit{com}}^c$ and let $\mathit{aux}_{\mathit{SNARK}} \in \mathcal{D}$ 
and let $\mathit{state_1} \in \mathit{State}_{\mathcal{R}_{\mathit{vec}, \mathit{com}}}$; let 
$(\mathbf{C},\mathit{state_2}) = \mathit{SNARK.PartInput}(\mathit{srs}, \mathit{state_1}, \mathcal{R}_{\mathit{vec}, \mathit{com}}^c)$. 
By the definition of $\mathit{SNARK.PartInput}$ for $\mathscr{P}_{\mathit{vec}}^{h}$, there exists 
$\mathbf{input_1}$ such that $\mathbf{C} = \mathbf{Com}(\mathbf{input_1})$ and $c(\mathbf{input_1}) = 1$. 
We denote by $(\mathbf{input_2}, \pi)$ the output of $\mathcal{A}(\mathit{srs}, \mathit{state_2},  \mathcal{R}_{\mathit{vec}, \mathit{com}}^c)$ 
and let $\mathcal{A}_1$ be the part of $\mathcal{A}$ that sends openings and batched proofs for the polynomial commitments in 
$\mathbf{C}$. \\

\noindent On the one hand, if $\mathit{SNARK.Verifier}(\mathit{srs}_{\mathit{vk}}, (\mathbf{C},\mathbf{input_2}),\pi,\mathcal{R}_{\mathit{vec}, \mathit{com}}^c)$ 
in $\mathscr{P}_{\mathit{vec}}^{h}$ accepts, then the $\mathit{KZG}$ verifier corresponding to 
$\mathcal{A}_1$ also accepts. When such an event takes place, then, \ewnp $\mathcal{E}_{\mathit{KZG}}$ extracts polynomials 
$Q'_1(X), \ldots, Q'_m(X)$ that represent witnesses for the vector $\mathbf{C}$ of commitments and the alleged openings of $\mathcal{A}_1$. 
Because the $\mathit{KZG}$ polynomial commitment scheme is binding and by the definition of 
$\mathit{SNARK.PartInput}$ for $\mathscr{P}_{\mathit{vec}}^{h}$, we obtain that $Q'_1(X) = Q_1(X), \ldots, Q'_m(X) = Q_m(X).$ 
Since per (HMA.3.), the set $\{Q_1(X), \ldots, Q_m(X)\}$ evaluates to $\mathbf{input_1}$ over the range over which $\mathscr{P}_{\mathit{vec}}$ 
was defined, \ewnp the witness polynomials extracted by $\mathcal{E}_{\mathit{KZG}}$ evaluate to $\mathbf{input_1}$. \\

\noindent On the other hand, if $\mathit{SNARK.Verifier}(\mathit{srs}_{\mathit{vk}}, (\mathbf{C},\mathbf{input_2}),\pi,\mathcal{R}_{\mathit{vec}, \mathit{com}}^c)$ 
in $\mathscr{P}_{\mathit{vec}}^{h}$ accepts, then \\
$\mathit{SNARK.Verifier}(\mathit{srs}_{\mathit{vk}}, (\mathbf{input_1},\mathbf{input_2}),\pi,\mathcal{R}_{\mathit{vec}}^c)$ 
in $\mathscr{P}_{\mathit{vec}}^{*}$ also accepts. In turn, this acceptance together with the fact that $\mathscr{P}_{\mathit{vec}}^{*}$ 
has knowledge-soundness as per Definition~\ref{dfn_snark}, it implies $\mathcal{E}$ \ewnp extracts $\mathbf{witness_1}$ 
such that $(\mathbf{input_1}, \mathbf{input_2}, \mathbf{witness_1}) \in \mathcal{R}_{\mathit{vec}}^{c} \ (\#).$ 

\noindent By the definition of $\mathit{SNARK.PartInput}$ for $\mathscr{P}_{\mathit{vec}}^{h}$ and the way $\mathbf{input_1}$ was defined, 
it holds that $c(\mathbf{input_1}) = 1$. Due to $(\#)$ and by the definition of relation $\mathcal{R}_{\mathit{vec}}^{c}$, 
the predicates: $p_1$($\mathbf{input_1}$, $\mathbf{input_2}$, $\mathbf{witness_1}$) $= 1$ and 
$p_2(\mathbf{input_1}, \mathbf{input_2}, \mathbf{witness_1}) = 1$ hold. If we let 
$\mathbf{witness_2} = \mathbf{input_1}$, then 
$(\mathbf{C} = \mathbf{Com}(\mathbf{input_1}), \mathbf{input_2}, \mathbf{witness_1}, \mathbf{input_1}) \in \mathcal{R}_{\mathit{vec}, \mathit{com}}^c,$ so 
using $\mathcal{E}_{\mathit{KZG}}$ and $\mathcal{E}$ we can build an extractor for any knowledge-soundness adversary $\mathcal{A}$ for alleged 
hybrid model SNARK $\mathscr{P}_{\mathit{vec}}^{h}$ for relation $\mathcal{R}_{\mathit{vec}, \mathit{com}}^c$, which concludes the proof.
\end{proof}

\noindent It is straightforward to apply the technique described above to our SNARKs $\Plah$ and $\Pah$ 
compiled in Step 2 and obtain relations $\Rlacom$ and $\Racom$ as described below such that they fulfil Lemma~\ref{sergey_type_relations}.\footnote{Due to our specific application to proof-of-stake blockchain context in which we make use of our custom SNARKs, 
the assumption/requirement that  $\mathit{State}_{\mathcal{R}_{\mathit{vec}, \mathit{com}}} \neq \emptyset$ for 
$\mathcal{R}_{\mathit{vec}, \mathit{com}} \in \{\Rlacom, \Racom \}$ is fulfilled.}
%\vspace{-0.1in}
\begin{align*}
 {\Rlacom} = \{ & (\mathbf{C} \in \mathcal{C}, \mathbf{bit} \in \mathbb{B}^n, \mathit{apk} \in \mathbb{F}^2; \mathbf{pk}) : \mathit{apk} = \sum_{i=0}^{n-2} [\mathit{bit_i}] \cdot \mathit{pk_i} \ | \\ 
& \mathbf{pk} \in \ginn{1}^{n-1} \ \wedge \  \mathbf{C} = \mathbf{Com}(\mathbf{pk}) \} 
\end{align*}
%\vspace*{-0.75cm}
\begin{align*}
 {\Racom}  = \{ & (\mathbf{C} \in \mathcal{C}, \mathbf{b'} \in \mathbb{F}_{|\block|}^{\frac{n}{\block}}, \mathit{apk} \in \mathbb{F}^2;\mathbf{pk}, \mathbf{bit}) : \mathit{apk} = \sum_{i=0}^{n-2} [\mathit{bit_i}] \cdot \mathit{pk_i} \ | \\ 
& \mathbf{pk} \in \ginn{1}^{n-1} \ \wedge \ \mathbf{bit} \in \mathbb{B}^n  \wedge b'_{j} = \sum_{i=0}^{\block -1}2^i \cdot \mathit{bit_{\block \cdot j + i}}, \forall j < \frac{n}{\block}  \wedge \  \mathbf{C} = \mathbf{Com}(\mathbf{pk}) \} 
\end{align*}
For completeness, we also include the full rolled out SNARK $\Pah$ for relation $\Racom$ in Section~\ref{sec:rolled_out} and we provide a comparison between PLONK universal 
SNARK and our custom SNARKs in Section~\ref{suplementary_plonk_comparison}.  
%\vspace*{-0.75cm}
\section{Comparison between PLONK and our SNARKs}
\label{suplementary_plonk_comparison}

\noindent In the following, we briefly look at the differences between PLONK universal SNARK and the SNARKs designed in this work. We observe 
that while the NP relation that defines PLONK is more general, the relations that define our SNARKs are bespoke as we are only interested in efficiently proving 
public key aggregation. Because our relations are so bespoke, it turns out we do not require the full functionality that PLONK has to offer, and, in particular, our SNARKs  
do not require any permutation argument. \\

\noindent A second difference is that while PLONK's circuit is defined by a number of selector 
polynomials (which are a type of pre-processed polynomials) and a PLONK verifier needs to perform 
a one-time expensive computation of the polynomial commitments to those selector polynomials, our SNARK verifiers 
are able to avoid such a pre-processing phase. Indeed, in the case of $\mathscr{P}^h_{\mathsf{a}}$ (which is the only one of 
our three SNARKs that has a polynomial, namely $\mathit{aux}(X)$, that defines its circuit), our respective SNARK verifier does not need to compute a 
commitment to its only ``selector polynomial'' as, due to its structure, $\mathit{aux}(X)$ can be directly and efficiently evaluated by our SNARK verifier itself. \\ 
%Hence, in this case, we can use a simpler form of the corresponding linearisation polynomial which includes only the evaluation of 
%$aux(X)$\footnote{In fact, we need an evaluation of $aux(\omega \cdot X)$.} instead of $aux(X)$ itself.  \\

%\noindent A third difference is that PLONK relation (see section 8.2 in~\cite{plonk}) has a private witness while one of our three NP relations, 
%i.e., $\mathcal{R}^{\mathit{incl}}_{\mathsf{la}}$, has only public inputs. \\

%\noindent Another difference is that for all our three NP relations $\mathcal{R}^{\mathit{incl}}_{\mathsf{la}}$, $\mathcal{R}^{\mathit{incl}}_{\mathsf{a}}$ 
%and $\mathcal{R}^{\mathit{incl}}_{\mathsf{vt}}$ the public input contains a vector which has essentially the same length $O(n)$ 
%as the degree of the polynomials committed to by the SNARK prover. In particular this implies that after completing the first step of our compiler~\ref{sec_two_step_compiler} 
%(which is equivalent to running the original PLONK compiler with its linearisation optimisation) the SNARK verifier ends up parsing an $O(n)$ long vector 
%of public keys, computing the polynomials $pkx(X)$ and $pky(X)$ associated with the affine coordinates of the public keys and, finally, computing the 
%polynomial commitments to these two polynomials. Since public inputs, in general, and the vector of public keys, in particular, may differ between different 
%calls, the expensive computation described above cannot be reused by the SNARK verifier (as it is the case for the SNARK verifier in PLONK relative to the commitments to the 
%pre-processed polynomials) and, thus, would render our SNARK verifier less practical. However, the second and third steps of our compiler take care of this aspect by offloading 
%the expensive computation to a third trusted party. \\
\noindent A third difference is that using our two-steps compiler, our SNARKs verifiers are able to efficiently handle input vectors of length $O(n)$, 
where the degree of the polynomials committed to by our SNARK provers is also $O(n)$. Our SNARKs verifiers achieve efficiency by offloading the 
expensive polynomial commitment computation involving the public inputs to a trusted third party. \\

\noindent Moreover, while PLONK does not incorporate trusted inputs, one can easily apply the Step 2 of our compiler to PLONK. In particular, one could imagine a situation 
where a PLONK verifier is relying on a trusted party to compute some or all of the polynomial commitments to the circuit's selector polynomials. This is equivalent to our hybrid 
model SNARK definition applied to PLONK. The benefit is that by delegating such a computation, the PLONK verifier becomes more efficient. \\ %at the expense of relying on some trust assumptions. \\
%\noindent Finally, PLONK system does not incorporate by default trusted inputs in the sense we have defined in section~\ref{sec_two_step_compiler} but it can 
%be modified in a simple way to do that. For example, one could imagine a situation where a PLONK verifier is relying on a trusted party to compute some or all of 
%the polynomial commitments to the circuit's selector polynomials. This is the equivalent to our hybrid model applied to PLONK and the benefit is that by delegating this 
%computation the verifier becomes more efficient at the expense of relying on some trust assumptions.}
%{\color{blue}TO DO: Summarise how we can get rid of trust assumptions as well, in our case. For example: In a real world implementation, the trust assumption can, in turn, 
%be replaced by having, for example, the commitments signed by a set of parties whose public keys are known to the verifier and whom the verifier trusts that will not collude.}\\

%{\color{blue} \noindent TO DO Once we have defined the Polkadot instantiation for our light client in section~\ref{sec:instantiation} 
%and we have (an informal) sequential proof for soundness (for example), we can connect to that and mention that trusted inputs have even more of an efficiency 
%impact in case of our SNARKs since they replace computation that should be otherwise performed by our SNARK verifier at every step, while, in case of PLONK, 
%delegating the one-off computation of commitments to selector polynomials, even though beneficial as well, may not have had such an efficiency impact on the 
%verifier for the long term.}

\begin{comment}
The next paragraph should be commented out if we do not include in the submission the light client model instantiation section, i.e., section 5.3 of the eprint version.
\end{comment}

\noindent Finally, looking at our light client system instantiation in Section~\ref{sec:LCinstantiation} due to the inductive structure of the soundness proof (Theorem~\ref{th:soundness}), 
the efficiency of using a hybrid model SNARK has an even greater impact for the light client system verifier than that compared to verifying multiple instances of PLONK for the same circuit:
while for the latter the PLONK verifier has to compute commitments to selector polynomial only once anyway, in the case of the former, the commitments to public inputs may differ 
at very step hence a trusted third party relives a higher computation burden from the light client verifier overall. 



\section{Postponed Security Proof for Committee Key Scheme}
\label{supplementary_proof_sec_cks}

\begin{theorem} Given the hybrid model SNARK scheme secure for relation $\mathcal{R} \in \{ \Rlacom, \Racom\}$ as 
obtained using our two-step compiler in Section~\ref{sec_two_step_compiler} and the aggregatable signature scheme $\mathit{AS}$ 
                     as per Instantiation~\ref{insta:bls} (which fulfils Definition~\ref{def:aggregate_signatures}, with the additional 
                     specification that $\mathit{aux}_{\mathit{AS}} = v+1$ and choosing $v = n-1$, 
if we assume that an efficient adversary (against soundness of) $\mathit{CKS}_{\mathcal{R}}$ outputs public keys only from the source group $\ginn{1}$,  
then the committee key scheme $\mathit{CKS}_{\mathcal{R}}$ as per Instantiation~\ref{inst:cks} is secure with respect to Definition~\ref{def: committee_key}. 
\end{theorem}

\begin{proof} We prove below the statement only for $\Rlacom$. The statement can be proven analogously for $\Racom$. \\

In order to prove perfect completeness for $\mathit{CKS}_{\mathcal{R}}$ Instantiation~\ref{inst:cks} using a hybrid model SNARK secure for relation 
$\Rlacom$, we note that if $\mathit{AS.Verify}(\mathit{pp}, \mathit{apk}, m, \mathit{asig}) = 1$ holds, then due to the instantiation for \\
$\mathit{CKS_{\Rlacom}.Verify}$, we have that 
$$\mathit{CKS_{\Rlacom}.Verify}(\mathit{pp}, \mathit{rs}_{\mathit{vk}}, \mathit{ck}, m, \mathit{asig}, (\pi_{\mathit{SNARK}}, \mathit{apk}), (\mathit{bit_i})_{i=1}^{n-1}) =1$$ 
iff, in turn, 
$$\mathit{SNARK.Verify}(\mathit{rs_{vk}}, (\mathit{ck}, (\mathit{bit_i})_{i=1}^{n-1} || 0, \mathit{apk}), \pi_{\mathit{SNARK}}, \Rlacom) = 1 \ \ \ \ \ \ \ (1)$$ holds. 
Using the fact that the keys $\mathit{srs}$ and $(\mathit{rs}_{\mathit{pk}}, \mathit{rs}_{\mathit{vk}})$ for our hybrid model SNARK were generated correctly using 
$\mathit{SNARK.Setup}(v, 3v)$ and respectively $\mathit{SNARK.KeyGen}(\mathit{srs}, \Rlacom)$, 
 also since $(\mathit{pk_i})_{i=1}^{n-1} \in \ginn{1}^{n-1}$ as honestly generated by $\mathit{AS.GenerateKeypair}$, then 
$$(x = (\mathit{ck}, (\mathit{bit_i})_{i=1}^{n-1}||0, \mathit{apk}), w = (\mathit{pk}_i)_{i=1}^{n-1}) \in \Rlacom$$ 
(because $\mathit{apk} = \sum_{i=1}^{n-1} \mathit{bit_i} \cdot \mathit{pk_i}$ due to Instantiation~\ref{insta:bls} and 
$\mathit{ck}$ was honestly generated as  $\mathbf{Com}((\mathit{pk_i})_{i=1}^{n-1})$ as a pair of binding polynomial commitments to the $x$ and $y$ 
coordinates of the keys in $w$, respectively) and, finally, adding that the proof 
$\pi_{\mathit{SNARK}}$ was generated correctly as 
$$ \pi_{\mathit{SNARK}} \leftarrow \mathit{SNARK.Prove}(\mathit{rs_{pk}}, (x,w), \Rlacom),$$ 
then, by the perfect completeness property of the hybrid model SNARK for relation $\Rlacom$, we can conclude $(1)$.\\

\noindent The proof for the soundness property is described below. Let $\mathcal{A}$ be an efficient adversary that, 
whenever it outputs a vector of public keys $(\mathit{pk_i})_{i=1}^{n-1}$, the respective vector belongs to the set $\ginn{1}^{n-1}$. 
Assuming that the following holds 
$$\mathit{CKS_{\Rlacom}.Verify}(\mathit{pp}, \mathit{rs}_{\mathit{vk}}, \mathit{ck}, m, \mathit{asig}, \pi = (\pi_{\mathit{SNARK}}, \mathit{apk'}), (\mathit{bit_i})_{i=1}^{n-1}) =1,$$ 
then, according to instantiation for $\mathit{CKS_{\Rlacom}}$, it implies that both 
$$\mathit{AS.Verify(\mathit{pp}, \mathit{apk'}, m, \mathit{asig})} = 1 \ \ \ \ \ \ (2)$$ 
and 
$$\mathit{SNARK.Verify}(\mathit{rs_{vk}}, (\mathit{ck}, (\mathit{bit_i})_{i=1}^{n-1}||0, \mathit{apk'}), \pi_{\mathit{SNARK}}, \Rlacom) = 1  \ \ \ \ \ \ (3)$$
hold where $\mathit{apk'}$ was parsed from $\pi$. Since $\mathit{ck}$ was generated correctly as the pair of binding polynomial commitments 
$\mathbf{Com}((\mathit{pk_i})_{i=1}^{n-1)}$ using the vector $(\mathit{pk_i})_{i=1}^{n-1}$ output by the adversary $\mathcal{A}$ 
(which, as per adversary definition, belongs to $\ginn{1}^{n-1}$) and due to the knowledge 
soundness property of the SNARK scheme secure for relation $\Rlacom$, the knowledge soundness and the computational binding property 
of the polynomial commitment scheme (since for our $\mathit{CKS_{\mathcal{R}}}$ instantiation we use the KZG commitment scheme), it implies that, 
with overwhelming probability $$(x = (\mathit{ck}, (\mathit{bit_i})_{i=1}^{n-1}, \mathit{apk'}), w = (\mathit{pk}_i)_{i=1}^{n-1}) \in \Rlacom.$$ 
From this, in turn, by the definition of relation $\Rlacom$, we obtain that 
$\mathit{apk'} = \sum_{i=1}^{n-1} \mathit{bit_i} \cdot \mathit{pk_i}$. Moreover, by the instantiation of aggregatable signature scheme 
$\mathit{AS}$, we have that $\sum_{i=1}^{n-1} \mathit{bit_i} \cdot \mathit{pk_i} = \mathit{AS.AggregateKeys}(\mathit{pp}, (\mathit{pk_i})_{i:\mathit{bit_i = 1}})$ 
and, as per soundness challenge definition, it holds that \\
$\mathit{apk} \leftarrow \mathit{AS.AggregateKeys}(\mathit{pp}, (\mathit{pk_i})_{i:\mathit{bit_i = 1}})$. Hence $\mathit{apk'} = \mathit{apk}$.
Finally, due to $(2)$, we conclude that $$\mathit{AS.Verify(\mathit{pp}, \mathit{apk}, m, \mathit{asig})} = 1$$ holds with overwhelming probability (q.e.d.).

\end{proof}
\section{An Accountable Light Client System} \label{sec_light_client_model}
\section{An Accountable Light Client System}
\label{sec_light_client_model}
\label{new_light_client}

In this section, we give a model for the consensus systems that our light client system can be applied to and we define security properties for light client systems,
and, in particular accountable light client systems. Moreover, we present generic pseudocode for light client systems and prove that our implementation 
fulfils the security properties that we define for this notion.  

\subsection{Informal Model and Context}
\label{sec:LCinformal_model}

First, we informally describe our model, then we formalise it in \ref{sec:LCformal_model}.
There is a consensus system which we assume is a blockchain protocol. 
We consider consensus systems that make decisions based on signatures from a subset of validators, where the validator set may change periodically. 
Our model has the following entities: 

\paragraph{Full Nodes} - a full node maintains a view of the consensus decisions and stores the current state of the blockchain. 
A full node obtains both by running the consensus protocol correctly starting from the genesis state of the blockchain.

\paragraph{Validator} - a validator is a full node which the consensus protocol decides it belongs to a validator set. Once elected, 
validators take part in the consensus protocol and, in turn, their signatures determine what the consensus decides upon. 

\paragraph{Light Client Verifier} - a light client verifier is a node that does not keep the full state of the blockchain, but rather obtains (ideally short) proofs 
of parts of the blockchain state they are interested in; light client verifiers do this by being in communication with e.g., full nodes. In the optimistic scenario, 
where we have no adversary, the light client verifier can connect to a single full node and the full node should be able to convince the light client verifier of 
anything that the latter is interested in and the consensus system has agreed upon.

\paragraph{Adversary} The adversary controls a number of full nodes and validators. They are interested in convincing the light client verifier of things 
that may be in contradiction to what other (honest) nodes see as decided. The adversary, via the parties it controls, can try to  
double spend on the same blockchain or on another blockchain via a bridge. In the accountable case (which is the one we are interested in), 
the adversarial parties would like to ensure that if an attack is discovered, the honest validators and not the adversarial ones are to be blamed and punished.
In the pessimistic scenario, a light client verifier may only be connected to the adversary. In this scenario, we also assume that all full nodes, including honest 
validators are only connected to the adversary.

\paragraph{Validator Sets} As briefly mentioned above, the consensus protocol decides which entities are validators; the validators, in turn, agree on the consensus. 
The consensus protocol designates the next validator set which, in turn, is represented by the set of the corresponding entities' public keys. 

\subsubsection{Informal Security Properties}
\noindent We next informally describe the security properties that our light client system should satisfy. \\

\noindent {\bf Completeness}: If a full node sees that some fact was decided by the consensus, they can produce a proof that would convince a light client verifier of this fact.\\

\noindent {\bf Soundness}:  If, from some honest full nodes point of view, at least 1/3 of the validators in the validator set at any time are honest, then the light client verifier 
cannot be convinced of something incompatible with something the honest full node saw as decided. \\

\noindent For short, completeness and soundness mean, respectively, that in the optimistic scenario, 
a full node can always convince a light client verifier of some fact it sees as decided, and, in the pessimistic scenario, 
the adversary cannot convince the light client verifier of something that was not decided. \\

\noindent  Accountability means that if a light client verifier was convinced of an incorrect statement (in relation to what has been decided on the blockchain so far), 
then one can detect the misbehaving validators that contributed to that. We can separate this into two properties: \\%, that we can blame some validators and that those validators actually misbehaved:

\noindent {\bf Accountability Completeness}: If the light client verifier is convinced via a wrong proof of something which is incompatible 
with something a full node sees as decided, and then the light client verifier forwards the wrong proof to the full node, that full node can detect that some validators misbehaved. \\

\noindent {\bf Accountability Soundness}: If a full node is given a light client proof of something that is incompatible with something it sees as decided,
 then, when the full node detects that some validators misbehaved, indeed none of those validators are honest. \\
\vspace{-0.4cm}
\subsubsection{Consensus System Model}

\paragraph{Messages} For a full node to prove to a light client verifier that something has been decided, in the end it will prove that a {\it message} was signed by a quorum of validators from some {\it validator set}. 
Typically this message will not directly include the information the full node wants to convince the light client that it has been decided (during consensus), 
but the message will be a commitment to that information; hence, the full node can also include an opening of this commitment. \\
 
\noindent Our formal model will not mention blockchains, but it is useful to remember that in blockchain based consensus systems, often the message is a blockhash, which is a binding commitment to multiple types of data: 

\begin{enumerate}
\item the block header
\item all previous block headers, through parent hashes in block headers
\item the blocks themselves (whose hash is in the header)
%\item the state of a state machine, either because this is a function of the blocks or because a commitment, the state root, was explcitly included in a block header \cite{Ethereum_yellow_paper}.
\end{enumerate}

\noindent We define the {\it required data} of a message to be the data that the message is a binding commitment to 
and which all full nodes should know. We assume that if a full node sees a message as decided, it must have 
the corresponding required data. The required data of messages can overlap among each other and the full node 
would not need to store them separately, e.g. two block hashes for blocks in the same chain may have required data 
that overlap for a prefix of blocks in the chain, which may be many gigabytes of data.

\paragraph{Consensus Decisions, Validator Sets, Epochs and Consensus Views}

\noindent A message is decided if sufficient signatures corresponding to validators in the current validator set sign it. However the validator set may change. \\

\noindent We define an {\it epoch} as a period of time in which the validator set cannot change. During each epoch, the consensus determines the validator set 
for the next epoch. \\

\noindent We assume that the validator set size is bounded by some known constant $v$. Some threshold $t$ of validators are required to sign a message such that it is considered 
decided. $t$ may be a function of the size of the validator set of a given epoch, e.g. more than $2/3$ of the validators. We assume that the message itself indicates what epoch it belongs to, 
and only signatures from validators chosen for that epoch count for whether a message has been decided or not. \\

%When consensus works, full nodes would agree on things like what the validator set is for an epoch. However, sometimes we need to also consider the case when consensus fails and so these may be %subjective.

\noindent Each full node maintains a {\it consensus view}, i.e., its view of the protocol. The consensus view records the view of the validator set for each epoch, 
the messages that have been decided and the signatures on those messages. It also includes the required data for each decided message. \\

\noindent A well-defined function of the consensus view defines its validity. Full nodes should maintain only a valid consensus view, and must not include in their consensus view 
messages that would make the respective view invalid. 

\paragraph{Incompatible Messages}
\noindent There are some pairs of decisions that a consensus protocol cannot decide together without breaking validity. 
If the protocol ensures that honest validators do not sign messages corresponding 
to both decisions, then we can make signing such pairs of messages punishable. \\

\noindent Unfortunately the messages themselves need not be enough to judge their incompatibility. 
For example we would not want two block hashes to be decided if one is for a block of height 100 and the other is for a block of height 101, 
and the block of height 100 was not the parent of the block of height 101. However, if incompatibility is a function of the required data of one or both messages, 
then, because messages are binding commitments to their required data, this is still unambiguous for a pair of messages.

\subsubsection{Network Model}

\noindent When we need to assume a network model, the one we use is that all parties communicate only to the adversary, who may forward messages from one party to 
another when the adversary wants or not at all. Both our assumptions and our soundness and accountability soundness security definitions assume this networking model. \\

\noindent The proof of our security properties works in general for {\it asynchronously safe} protocols. These have a number of safety properties which hold with asynchronous networking. 
Asynchronous networking means that the adversary decides when a message is delivered but must deliver all messages eventually. For safety properties, those which have a 
statement that holds always or never, this is equivalent to our network model.

\begin{comment}
\subsection{Problem Description: an Informal Overview}

\noindent In general terms, we are interested in formalising a model in which a prover is able 
to convince a verifier that certain events happened in a consensus-based blockchain, while the verifier is minimally 
connected to the blockchain. We call such a verifier \textit{the light client verifier} or, simply, \textit{the light client}. \\

\noindent In more detail, the setting we consider is as follows. At any one time, there are at most some known maximum number of active 
validators. We assume that at least some threshold $t'$ of these are honest and, hence, they are able to achieve consensus on a blockchain 
using a Byzantine agreement protocol. However, the validator set is not fixed forever, but it changes regularly. 
We call \textit{epochs} the periods where each validator set does not change. At the end of an epoch, the current validators 
agree on the validator set for the next epoch. Depending on the exact details of the implementation, this operation may 
include checking the identity of each validator in the validator set for the next epoch and also recording those identities to the blockchain. \\

\noindent Given this setting, we are interested in a light client, an entity that is not part of the set of validators that allegedly work 
towards consensus on the blockchain, and a prover, a potentially malicious process that may not be part of any set of validators, 
but which listens to public messages sent by the validators. In order to model the bandwidth and/or CPU limitations of the 
device on which the light client runs, we assume the light client, once initialised (e.g., with some credentials/messages/events 
agreed upon by the first and publicly known validator set) is connected only to the prover and cannot listen to any 
consensus messages agreed upon by the validators. Hence, the light client completely relies on the prover to convince them that 
validators' consensus has been reached on a message or event that the light client is interested in. \\

\noindent Depending on the time frame (e.g., current epoch vs.\ multiple hop epoch), we can define two related problems. In the one epoch light client 
problem, the prover wants to convince the light client that the validator set in the current epoch agreed on something. In the multiple hop epoch light 
client problem, the prover needs to convince the light client that validators in a later epoch agreed on something. \\

\noindent Depending on the adversarial model, we have two cases as well. The first security model assumes that at least $t'$ validators 
in each validator set are honest, hence there cannot be a collusion between the possibly malicious prover and the honest validators reaching 
consensus on the blockchain. In turn, this implies that \textit{soundness} suffices as a security property, i.e., the prover should not be able to 
wrongly convince the verifier that consensus has been reached on any event or message outside of some small probability. The second security 
model strengthens the adversarial capability by not making any assumption regarding the fraction of honest validators in at least one of the validator 
sets. Hence a stronger security property is needed. We call it \textit{accountability}. This captures the intuition that if the light client is convinced of 
something that the blockchain did not achieve consensus on, and if the light client and prover's communication transcript is made public, then using 
it and other public information, it should be possible to identify an epoch and a number of dishonest validators equal to at least the total number of validators in 
that epoch minus $t'$.
\end{comment}

\subsection{A Formal Model for Consensus-based Accountable Light Client Design}
\label{sec:LCformal_model}

\noindent We need the following fundamental notions:

\begin{itemize}
\item some number $k$ of \textit{epochs} with ids $1,\dots, k$;
\item for each epoch id $i$, $1 \leq i \leq k$, the validators on the blockchain may agree on a subset of the \textit{set of possible consensus messages} $M_i$;
\item associated with each consensus message $m$ there may exist some \textit{required data} $d_{m} \in D$ for some set $D$; 
when such a $d_m$ exists, $m$ is a binding commitment to $d_m$; 
\item a secure aggregatable signature scheme $\mathit{AS}$ as defined in Section~\ref{sec:multisig}.
\end{itemize}

\noindent Building on the above notions, we also define \textit{a valid consensus view}. 

\begin{definition}(Consensus View) A consensus view $C$ for a set of epochs with ids $i$, $\forall i \in [k]$, 
for some $k$, contains for each epoch id $i$:
\begin{itemize}
\item a set $PK_i$ of public keys (we may also consider a list of public keys and weights, e.g. proportional to stake, but we focus here on  
the equal weight case for simplicity). 
\item a set $\{(m, \mathit{Signers}, \sigma) \ | \ m \in M_i, \mathit{Signers} \subseteq PK_i\}$ where 
$\sigma$ is a signature (or an aggregatable signature) on $m$ and the public key(s) of the signer(s) are $\mathit{Signers}$. 
\item some \textit{required data} $d_{m}$ associated with each message $m$, such that $m$ is a binding commitment to $d_{m}$. Note 
that some required data associated with different messages may overlap. 
\end{itemize}
In addition to the components mentioned above, a consensus view $C$ contains also a \emph{genesis state} $\mathsf{genstate}$; as a concrete example,
$\mathsf{genstate}$ may contain the set of public keys $\mathit{PK_1}$ for the first epoch and their proofs of possession. For each of the notions 
contained in some epoch of $C$ as well as for $\mathsf{genstate}$, we say they belong to $C$ and we simply denote that by ``$\in C$''. 
\end{definition}

\noindent In the following, we assume that all algorithms processing messages use a common efficient representation that implicitly 
includes for each of them an epoch id; this epoch id is retrieved using a function $\mathit{epoch}_{\mathit{id}}$. 

\begin{definition}(Deciding a Consensus Message) 
\label{def_decide}
Given a consensus view $C$, we say a message $m \in M_i$ is \emph{decided in $C$} 
if $C$ contains valid signatures from at least some threshold $t$ (e.g., more than 2/3) 
signers corresponding to public keys in $PK_i$ or, equivalently, a valid aggregatable 
signature of $t$ signers over $m$. Additionally, we denote by $(\mathit{m}, d_{\mathit{m}}) \in_{\mathit{decided}} C$ 
the fact that $m \in C$, $ \exists \ d_m \in C \cap D$, $d_m$ is the associated required data of $m$ and $m$ has been decided in $C$.  
\end{definition}

\begin{definition}(Valid Consensus View)
\label{def:valid_consensus}
\noindent 
We assume the following three \emph{functions used for validation} are efficiently computable and they are defined as: 

\begin{itemize}
\item $\mathit{VerifyData}: \cup_{i=1}^k M_i \times D \rightarrow \{1, 0\}$ such that it 
checks the validity of $m$ given the required data $\mathit{d_{m}}$;
\item $\mathit{HistoricVerifyData}: \{\mathsf{genstate}\} \times (\cup_{i=1}^k M_i \times D)^n \times (\cup_{i=1}^k \mathit{PK_i})^q \rightarrow \{1, 0\}$ 
such that it checks the validity of $\mathsf{genstate}$, some set of $n$ consensus messages and their required data and some set of $q$ public keys;
\item $\mathit{Incompatible}: \cup_{i=1}^k (M_i \times M_i) \times D \rightarrow \{0,1\}$ which 
given messages $m_1$, $m_2$ and potential required data $d_{m_1}$ for $m_1$ checks the incompatibility.
\end{itemize}

\noindent Let $m_1,\ldots, m_n$ be all the distinct consensus messages contained in $C$. Let $\mathit{pk_1},\ldots, \mathit{pk_q}$ be all the 
public keys, including repetitions, contained in $\mathit{PK_i}, \forall i \in[k]$.
We say \emph{the consensus view $C$ is valid} if: 
\begin{itemize}
\item $\exists \ d_{m_i} \in D \cap C$ such that $\mathit{VerifyData}(m_i, d_{m_i}) = 1$, $\forall 1\leq i \leq n$. 
\item $\mathit{HistoricVerifyData}(\mathsf{genstate}, m_1, d_{m_1}, \ldots, m_n, d_{m_n}, \mathit{pk_1},\ldots, \mathit{pk_q}) = 1$. 
\item There exists no pair $(i,j)$, $1 \leq i,j \leq k$, $i \neq j$ such that $\mathit{Incompatible}(m_i, m_j, d_{m_i}) = 1$ 
or \\ $\mathit{Incompatible}(m_j, m_i, d_{m_j}) = 1$.
\item We require that all consensus messages in $C$ are decided according to Definition~\ref{def_decide}.
\end{itemize}
\end{definition}

\noindent We conclude this subsection by defining what we mean by honest validator.
\begin{definition}(Honest Validator)
\label{def:honest_validator}
An honest full node of a blockchain is one that runs the protocol correctly starting from the genesis state of the blockchain. 
It maintains a valid consensus view of the system. An honest full node is a validator if they produced a public key that is in the set 
$\mathit{PK_i}$ in some epoch $i$ in some consensus view. An honest validator is an honest full node that is also a validator.
\end{definition}
 
\subsubsection{General Light Client Properties}

\noindent Next we define a light client system. 
\label{sec:soundness}
  
\begin{definition}(Light Client System)
\label{scheme_light_client} Let $\mathcal{R}$ be a (conditional) NP relation. A \emph{light client system} involves 
two parties - \emph{prover} and \emph{light client} (also called \emph{light client verifier}) - and it implements the following algorithms:
\begin{itemize}
\item $\mathit{pp_{\mathit{LC}}} \leftarrow \mathit{LC.Setup}(\mathcal{R})$: 
a setup algorithm that takes the security parameter $\lambda$ and a (conditional) NP relation $\mathcal{R}$ 
and outputs public parameters $\mathit{pp_{\mathit{LC}}}$.
\item $\pi \leftarrow \mathit{LC.GenerateProof}(\mathit{pp_{\mathit{LC}}}, C, m, \mathcal{R})$: a proof 
generation algorithm that takes a valid consensus view $C$, a message $m$ decided in consensus view $C$ 
and a (conditional) NP relation $\mathcal{R}$ and generates a proof $\pi$.
\item $\mathit{acc}/\mathit{rej}  \leftarrow  \mathit{LC.VerifyProof}(\mathit{pp_{\mathit{LC}}}, \LCseed, \pi, m, \mathcal{R})$: 
a proof verification algorithm that takes as input a genesis summary $\LCseed$ (whose properties are detailed in 
definition~\ref{def:genesis_summary}), a light client proof $\pi$ and a message $m$ and returns $\mathit{acc}$ if $\pi$ is a valid 
proof for $m$ and $\mathit{rej}$ otherwise.
\end{itemize}

\noindent We call the tuple ($\mathit{LC.Setup}$, $\mathit{LC.GenerateProof}$, 
$\mathit{LC.VerifyProof}$) a light client system if it fulfils 
perfect completeness and soundness as defined below. \\
\noindent \textbf{Perfect Completeness} A light client system is perfectly complete if a full node sees that any message $m$ has been decided, it can produce a proof that will convince a light client verifier of it. 
The full node should have a valid consensus view $C$ that decided $m$ which it can use as input in $\mathit{LC.GenerateProof}$ to obtain a proof $\pi$. The light client verifier will run 
$\mathit{LC.VerifyProof}$ with input $\pi$ and this should always accept.
Formally, we say ($\mathit{LC.Setup}$, $\mathit{LC.GenerateProof}$, $\mathit{LC.VerifyProof}$) has perfect completeness if 
for any valid consensus view $C$ and for any consensus message $m$ decided in $C$ we have that 
\begin{align*} 
\mathit{Pr} [&\mathit{LC.VerifyProof}(\mathit{pp_{\mathit{LC}}}, \LCseed, \pi, m, \mathcal{R}) = \mathit{acc} \ | \ \mathit{pp_{\mathit{LC}}} \leftarrow \mathit{LC.Setup}(\mathcal{R}),  \\
& \pi \leftarrow \mathit{LC.GenerateProof}(\mathit{pp_{\mathit{LC}}}, C, m, \mathcal{R})] = 1
\end{align*}
\noindent \textbf{Soundness} A light client protocol is sound if, under the assumption that $v-f$ validators in each epoch are honest, the light client cannot be convinced of a message $m$ unless $t-f$ honest validators have signed $m$. Here $f=v-t'$ is the a bound on the number of adversarial keys. Note that if $t-f$ honest validators sign $m$ and there are $f$ adversarial keys then additional signatures from these adversarial keys are enough to decide $m$. If the message $m$ belongs to epoch $k$, then we assume that there is a valid consensus view $C$ in which the validator sets for the first $k$ epochs have $t'$ honest validator's public keys. If this holds and less than $t-f$ honest validators signed $m$, then an adversary interacting with honest validators should not be able to generate a light client proof $\pi$ for $m$ that $LC.VerfifyProof$ accepts.

We say ($\mathit{LC.Setup}$, $\mathit{LC.GenerateProof}$, $\mathit{LC.VerifyProof}$) has soundness if, 
for every efficient malicious prover $\mathcal{A}$,  
\begin{align*} 
\mathit{Pr}[&\mathit{LC.VerifyProof}(\mathit{pp_{\mathit{LC}}}, \LCseed, \pi, m, \mathcal{R}) = \mathit{acc} \ | \ \mathit{pp_{\mathit{LC}}} \leftarrow \mathit{LC.Setup}( \mathcal{R}), \\
& \mathit{pp} \leftarrow \mathit{Parse}(\mathit{pp_{LC}}), (\pi, m, C) \leftarrow \mathcal{A}^{\mathit{HonestValidator}}(\mathit{pp}, \mathcal{R}), \\
& \mathit{CheckValidConsensus}(C) =1, \\ 
& \mathit{NumberHonestSigners}(m,\mathit{OGenerateKeypair}) < t+t'-v \\
& \mathit{HonestThreshold}(t', \mathit{OGenerateKeypair}, C) = 1] = \negl({\lambda}); 
\end{align*}
\noindent where 
\begin{itemize}
\item the predicate $\mathit{CheckValidConsensus}(C)$ checks if $C$ is valid 
w.r.t.\ Definition~\ref{def:valid_consensus} and outputs $1$ in that case (and $0$ otherwise); 
\item $\mathit{NumberHonestSigner}(m,\mathit{OGenerateKeypair})$ returns the number of public keys in $Q_{\mathit{pks}}$ from $\mathit{OGenerateKeypair}$ defined below.
\item $\mathcal{A}^{\mathit{HonestValidator}}$ represents the adversary $A$ in communication with the honest validators.
\item $\mathit{HonestThreshold}(t', \mathit{OGenerateKeypair}, C)$ checks 
that at least $t'$ of the public keys in each $\mathit{PK_i}$ of $C$ (for every epoch $i$ in $C$), are part of $Q_{\mathit{pks}}$ and outputs $1$ in 
that case (and $0$ otherwise). 
\end{itemize}
Finally, we assume that $\mathit{HonestValidator}$ (but not the adversary directly) makes oracles calls to $\mathit{OGenerateKeypair}(pp)$ (where $\mathit{pp}$ are the public parameters of aggregated signature scheme $\mathit{AS}$ are 
part of $\mathit{pp_{\mathit{LC}}}$) defined as
\begin{align*}
&\mathit{OGenerateKeypair}(pp): \\
& ((\mathit{pk}, \pi_{PoP}), \mathit{sk}) \leftarrow \mathit{AS.GenerateKeypair}(\mathit{pp}) \\
& Q_{\mathit{keys}} \leftarrow Q_{\mathit{keys}} \cup \{((\mathit{pk}, \pi_{PoP}), \mathit{sk})\}, Q_{\mathit{pks}} \leftarrow Q_{\mathit{pks}} \cup \{\mathit{pk},\}  \\
& \text{Output } ((\mathit{pk}, \pi_{PoP}), \mathit{sk}).
\end{align*}
\end{definition}

\noindent Finally, we define the genesis summary and its properties with respect to a light client system. 

\begin{definition}(Genesis Summary) 
\label{def:genesis_summary} Light client verifiers have access to a genesis summary $\mathit{LC.seed}$, 
which is a well defined deterministic function of the genesis state $\mathsf{genstate}$.
%$\mathit{LC.seed}$ contains a well defined commitment or set of commitments to the set $\mathit{PK_1}$. 
%We make the assumption that even if $\mathsf{genstate}$ has been generated by a potential adversary, 
%this adversary outputs the same $\mathsf{genstate}$ to all honest validators and, moreover, the genesis 
%summary $\mathit{LC.seed}$ equals the summary of this $\mathsf{genstate}$ given to honest validators. 
\end{definition}

\subsubsection{Accountable Light Client Properties}
\label{sec:accountability}

In the following, we extend our model above to include accountability. We provide the definition for an accountable light client system which subsumes the light client system definition given above.
An accountable light client has the property that if a full node with a consensus view $C$ that decides $m$ is given a light client proof $\pi$ for a message $m'$ that is incompatible with $m$, then 
it should be able to generate a proof that shows that some validators misbehaved. We need to add two more functions to our light client definition, the first one for detecting and generating proofs of 
misbehaviour, the second one for verifying the proofs of misbehaviour.

\begin{definition}(Accountable Light Client System)  
\label{def:lc_accountable}
Let $\mathcal{R}$ be a (conditional) NP relation. An accountable light client 
system implements algorithms ($\mathit{LC.Setup}$, $\mathit{LC.GenerateProof}$, $\mathit{LC.VerifyProof}$, \\
$\mathit{LC.DetectMisbehaviour}$, $\mathit{LC.VerifyMisbehaviour}$) where 
$\mathit{LC.Setup}$, $\mathit{LC.GenerateProof}$ and \\ $\mathit{LC.VerifyProof}$ are 
defined as in~\ref{scheme_light_client} and 
$$(i, S, \mathbf{bit}, \sigma, m'', m') \leftarrow \mathit{LC.DetectMisbehaviour}(\mathit{pp_{\mathit{LC}}}, \pi, m, C,\mathcal{R})$$ 
is an algorithm such that it takes a proof $\pi$ for message $m$, a consensus view $C$ and a (conditional) NP relation $\mathcal{R}$;
it outputs an epoch id $i$, a subset of misbehaving signers $S \subseteq \mathit{PK_i}$ in the same epoch as messages $m''$ and $m'$, 
with $m'$ decided in $C$ and $m''$ signed with signature $\sigma$ and using bitmask $\mathbf{bit}$ against the set $\mathit{PK_i}$ 
and
$$ \mathit{acc}/\mathit{rej} \leftarrow \mathit{LC.VerifyMisbehaviour}(\mathit{pp_{\mathit{LC}}}, i, S, \mathbf{bit}, \sigma, m'', m', C, \mathcal{R})$$ 
is an algorithm which takes the input of $\mathit{LC.DetectMisbehaviour}$ together with a consensus view $C$ and a (conditional) NP relation $\mathcal{R}$ and 
checks if indeed misbehaviour took place such that completeness, soundness, accountability and accountability soundness hold, where completeness and soundness 
are identical to Definition~\ref{scheme_light_client} and accountability completeness and accountability soundness are defined below.
\end{definition}

\noindent \textbf{Accountability Completeness} A light client protocol has accountability completeness if a full node sees a light client proof for a message $m$ and it sees that a message $m'$ 
has been decided that is incompatible with $m$, then it can identify and prove that a fraction of validators ($v+v'-t$ validators) have misbehaved.
The full node is given a proof $\pi$ of $m$. It has  a consensus view $C$ that decides $m'$, from the same epoch as $m$ with required data $d_{m'}$ that has $\mathit{Incompatible}(m', m, d_{m'})=1$.
Then it should be able to use $\mathit{LC.DetectMisbehaviour}$ to generate a proof that at least $v+v'-t$ validators misbehaved, that $\mathit{LC.VerifyMisbehaviour}$ will always accept.\\

\noindent Formally, we say ($\mathit{LC.Setup}$, $\mathit{LC.GenerateProof}$, 
$\mathit{LC.VerifyProof}$, \\ $\mathit{LC.DetectMisbehaviour}$, $\mathit{LC.VerifyMisbehaviour}$) 
achieves accountability completeness if for every efficient adversary $\mathcal{A}$ it holds that:
\begin{align*}
\mathit{Pr}[& \mathit{LC.VerifyMisbehaviour}(\mathit{pp_{\mathit{LC}}}, \mathit{LC.DetectMisbehaviour}(\mathit{pp_{\mathit{LC}}}, \pi, m, C, \mathcal{R}), C, \mathcal{R}) = \mathit{acc} \ | \ \\
& \mathit{pp_{\mathit{LC}}} \leftarrow \mathit{LC.Setup}( \mathcal{R}), (\pi, m, C) \leftarrow \mathcal{A}(\mathit{pp_{\mathit{LC}}}, \mathcal{R}), \\
& \mathit{LC.VerifyProof}(\mathit{pp_{\mathit{LC}}}, \LCseed, \pi, m, \mathcal{R}) = \mathit{acc}, \mathit{CheckValidConsensus}(C) =1, \\
& \exists \ (m', d_{m'}) \ \in_{\mathit{decided}} \ C, \mathit{Incompatible}(m', m, d_{m'})=1, \mathit{epoch}_{\mathit{id}}(m) = \mathit{epoch}_{\mathit{id}}(m')] = 1 - \negl(\lambda)
\end{align*}

\noindent \textbf{Accountability Soundness} A light client protocol has accountability soundness if an adversary interacting with 
a single honest validator cannot prove that the honest validator misbehaved. This holds even if all other validators are dishonest and the adversary controls the honest validator's view of the network.\\

\noindent Note that we assume that the adversary interacts with the honest validator, who generates their keys honestly in turn. The adversary can break accountability soundness if it can win the following game except with negligible probability. The adversary wins if they can produce an input $(i, S, \mathbf{bit}, \sigma, m'', m', C)$ to $\mathit{LC.VerifyMisbehaviour}$ such that $\mathit{LC.VerifyMisbehaviour}$ accepts, $C$ is a valid consensus view and $S$ contains a public key the honest validator generated.\\

\noindent Formally, we say ($\mathit{LC.Setup}$, $\mathit{LC.GenerateProof}$, 
$\mathit{LC.VerifyProof}$, \\ $\mathit{LC.DetectMisbehaviour}$, $\mathit{LC.VerifyMisbehaviour}$) 
achieves accountability soundness if for every efficient adversary $\mathcal{A}$ it holds that:
$$\mathit{Pr}[\mathit{Game}^{\mathit{accountability-soundness}}=1]=\negl(\lambda)$$
where

\begin{align*}
& \mathit{Game}^{\mathit{accountability-soundness}}(\lambda, \mathcal{R}): \\
&  Q_{\mathit{keys}} := \emptyset  \\ 
& \mathit{pp_{\mathit{LC}}} \leftarrow \mathit{LC.Setup}( \mathcal{R}) \\
& \mathit{pp} \leftarrow \mathit{Parse}(\mathit{pp_{\mathit{LC}}}) \\
&  (i, S, \mathbf{bit}, \sigma, m'', m', C) \leftarrow \mathcal{A}^{\mathit{HonestValidator}^{\mathit{OGenerateKeypair}} \mathit{}}(\mathit{pp},\mathit{pp_{\mathit{LC}}})   \\
& \text{If } \mathit{LC.VerifyMisbehaviour}(\mathit{pp_{\mathit{LC}}}, i, S, \mathbf{bit}, \sigma, m'', m', C, \mathcal{R}) = \mathit{rej} \text{ Return } 0 \\ 
&  \text{If } \mathit{CheckValidConsensus}(C) =0 \text{ Return } 0 \\
& \text{If } S \cap Q_{pks}=\emptyset \text{ Return } 0 \\
& \text{Return } 1
\end{align*}

%\subsection{A Light Client Instantiation for Polkadot}
%\label{sec:instantiation}

\subsection{Accountable Light Client System Instantiation}
\label{sec:LCinstantiation} 

\noindent We motivate our light client model from~\ref{sec:LCformal_model} by detailing below instantiations 
for a light client system that is accountable light client system. Both are 
compatible with proof-of-stake based blockchains and, in particular, Polkadot.

\subsubsection{Conventions and Assumptions}
\label{sec:conventions}

\noindent Before listing our light client systems' algorithms, we make several notational conventions:

\begin{itemize}
\item We use boldface font for denoting vectors. Furthermore, whenever necessary to avoid confusion, 
we denote by $\mathbf{Vec_i}(k)$ the $k$-th component of vector $\mathbf{Vec_i}$. 

\item In the following, unless otherwise stated, when we use $\mathcal{R}$, we mean one of the conditional relations from the set 
$\{ \Rlacom, \Racom\}$.% , \Rvtcom\}$.}

\item Given a valid consensus view $C$ over $i$ epochs, we assume there is a well-defined order on the set $\mathit{PK_j}$  
of public keys included in $C$, $\forall j \in [i]$; hence, in the following, we rename this set by $\mathbf{pk_j}$, $\forall j \in [i]$ 
and interpret it as a vector. Moreover, we instantiate honestly generated keys in $\mathbf{pk_j}$ with keys generated using 
$\mathit{AS.GenerateKeypair}$ as described in Instantiation~\ref{insta:bls}.
  
\item We remind the reader that by $\mathbf{Com}(\mathbf{pk})$ we denote the set of two computationally binding polynomial 
commitments to the polynomials obtained by interpolating the $x$ components of $\mathbf{pk}$ and, respectively, the $y$ 
components of $\mathbf{pk}$ over a range $H$ of size at least $v+1$, where $v$ is some maximum number of validators that the system allows. 
In our instantiations for (accountable) light client systems, we use the KZG polynomial commitments, but, as mentioned also in 
Section~\ref{sec_two_step_compiler}, the general results stated in this section hold for any binding polynomial commitments with a knowledge-soundness property.

\item We assume there is a fixed upper bound $v$ on the number of validators in each epoch and we use $v$ in the description of our algorithms. 
At the same time, for compatibility with the SNARKs that we build for relations $\Rlacom$ and $\Racom$ as defined 
in \ref{compiler_step_2}, when specifically using our Instantiation~\ref{inst:cks} of $\mathit{CKS_{\mathcal{R}}}$ or when proving our results in this section, we 
let $v$ equal $n-1$, where $n$ was defined in Section~\ref{sec:lagrange}. 

\item $\mathit{Parse}$ and $\mathit{Transform}$ denote functions performing the respective operations on the 
(accountable) light client algorithms' input in order to obtain the necessary components. $\mathit{Parse}$ and $\mathit{Transform}$ 
may additionally depend on the (conditional) relation $\mathcal{R}$ under consideration. If that is the case, we explicitly include 
$\mathcal{R}$. In particular, $\mathit{Parse}$ and $\mathit{Transform}$ functions which are part of $\mathit{LC.DetectMisbehaviour}$ 
work only for $\mathcal{R} \in \{ \Rlacom, \Racom \}$. 

\item The accountable light client systems use functions $f_x$ (deriving the public inputs), 
$\mathit{f_{\mathit{threshold}}}$ (deriving the Hamming weight), $\mathit{HammingWeight}$ (deriving the Hamming 
weight from consensus view elements) and $f_{\mathit{bit}}$ (deriving the bitmask corresponding to public keys that 
signed a given message). Before providing these functions' definitions, we make the convention that, 
whenever used as parameters/input to these functions, $\mathbf{bit}$, $\mathit{apk}$, $\mathbf{b'}$ and $s$ have 
the meaning and definition provided in Section~\ref{sec:snarks}. 

\begin{equation*}
 f_x (\mathbf{Com}(\mathbf{pk}), \mathbf{bit}, s,\mathit{apk}, \mathcal{R}) =
   \begin{cases}
    (\mathbf{Com}(\mathbf{pk}), \mathbf{bit}, \mathit{apk}) & \text{if } \mathcal{R} = \Rlacom \\
     (\mathbf{Com}(\mathbf{pk}), \mathbf{b'}, \mathit{apk}) & \text{if } \mathcal{R} = \Racom \\
   % (\mathbf{Com}(\mathbf{pk}), s, \mathit{apk}) & \text{if } \mathcal{R} = \Rvtcom \\
  \end{cases}       
\end{equation*}

%%\begin{equation*}
%% f_w (\mathbf{pk}, \mathbf{bit}, \mathcal{R}) =
%%   \begin{cases}
%%      \mathbf{pk} & \text{if } \mathcal{R} = \Rlacom \\
%%      (\mathbf{pk}, \mathbf{bit}) & \text{if } \mathcal{R} =  \Racom \\
%%     ( \mathbf{pk}, \mathbf{bit}) & \text{if } \mathcal{R} =  \Rvtcom \\
%%   \end{cases}       
%%\end{equation*}

\begin{equation*}
\mathit{HammingWeight^*}(\mathbf{vec}) = \mathit{HammingWeight}(\mathbf{vec}_{1}, \ldots, \mathbf{vec}_{|\mathit{vec}| - 1})
\end{equation*}

%\begin{equation*}
%\mathit{HammingWeight}(\mathbf{bit}, \mathcal{R}) =
%  \begin{cases}
%   \mathit{HammingWeight^*}(\mathbf{bit}) & \text{if } \mathcal{R} \in \{ \Rlacom, \Racom\} \\
%      \mathit{HammingWeight}(\mathbf{bit}) -1 & \text{if } \mathcal{R} =  \Rvtcom \\
%  \end{cases}     
%\end{equation*}

\begin{equation*}
 \mathit{f_{\mathit{threshold}}} (\mathbf{x}, \mathcal{R}) =
    \begin{cases}
      \mathit{HammingWeight^*}(\mathbf{bit}) & \text{if } \mathcal{R} = \Rlacom \\
     \sum_{j=1}^{\frac{v+1}{|\mathit{block}|}-1}\mathit{HammingWeight}(\mathbf{b'}_{j}) \ + \ \mathit{HammingWeight^*}(\mathbf{b'}_{\frac{v+1}{|\mathit{block}|}}) & \text{if } \mathcal{R} =  \Racom \\
      %s & \text{if } \mathcal{R} =  \Rvtcom \\
    \end{cases}      
\end{equation*}


\begin{equation*}
 \mathit{f_{\mathit{bit}}} (C, m, v) = ((\mathbf{bit_i}(k))_{k=1}^{v} ||\ 0, \mathbf{\sigma_i}),
\end{equation*}
\noindent where $i = \mathit{epoch}_{\mathit{id}}(m)$ and $\forall  \ k = 1, \ldots, v$, if  there exists 
$\sigma \in C \ \wedge \ \mathit{AS.Verify}(\mathit{pp}, \mathbf{pk_i}(k), \mathit{m}, \sigma) = 1$, 
we set $\mathbf{bit_i}(k) = 1$ and $\mathbf{\sigma_i}(k) = \sigma$, otherwise, we set $\mathbf{bit_i}(k) = 0$ and  $\mathbf{\sigma_i}(k) = \_$. 

\noindent Note that for each of our relations $\Rlacom$ and $\Racom$, $\mathit{apk}$ and $\mathbf{Com}(\mathbf{pk})$ 
are public inputs and $\mathbf{pk}$ is a witness. Moreover, for these relations $\Rlacom$ and $\Racom$, we build an accountable light client system. 
%\noindent Note that for each of our relations $\Rlacom$, $\Racom$, $\Rvtcom$, $\mathit{apk}$ and $\mathbf{Com}(\mathbf{pk})$ 
%are public inputs and $\mathbf{pk}$ is a witness. Moreover, for relations $\Rlacom$ and $\Racom$, we build an accountable light client system 
%while for relation $\Rvtcom$ we build a light client system that is not accountable.
\item We make the following instantiations: $\mathsf{genstate}$ is the set of public keys in $\mathbf{pk_1}$ and their alleged proofs of possession; 
$\mathit{LC.seed} = \mathbf{Com}(\mathbf{pk_1})$.
\end{itemize}

%\noindent We have a parametrisation convention:

%\begin{itemize}
%\item We assume there is a fixed upper bound $v$ on the number of validators in each epoch. In particular, for 
%compatibility with the SNARKs that we build for relations $\Rlacom$, $\Racom$ and $\Rvtcom$ as defined 
%in \ref{compiler_step_2}, we let $v$ equal $n-1$, where $n$ was defined in Section~\ref{sec:lagrange}.  
%\end{itemize}

%\noindent Finally, we have security definition convention:
%\begin{itemize}
%\item Our instantiation algorithms $\mathit{LC.DetectMisbehaviour}$, $\mathit{LC.VerifyMisbehaviour}$ and $\mathit{GenerateProof}$ 
%as described below may also return no values, usually in case $\mathit{CheckValidConsensus}(C)=0$. However, this does not affect the 
%security definitions in Section~\ref{sec:LCformal_model} since in completeness and both the accountability related definitions we assume 
%$\mathit{CheckValidConsensus}(C) = 1$, while our soundness definition does not involve the above three algorithms. Moreover, we use 
%the implicit convention that all properties fail if one of the algorithms that are part of its definition return no value.
%\end{itemize}

\subsubsection{The Algorithms}
\label{sec:instantiation}

\noindent The setup algorithm used by the accountable light client system is: 
\begin{itemize}
\item $\mathit{LC.Setup(\mathcal{R})}$
\begin{align*}
%& \mathit{pp} \leftarrow \mathit{AS.Setup}(\lambda) \\
%& \mathit{srs} \leftarrow \mathit{SNARK.Setup}(\lambda) \\
%& (\mathit{rs_{\mathit{pk}}}, \mathit{rs_{\mathit{vk}}}) \leftarrow \mathit{SNARK.KeyGen}(\mathit{srs}, \mathcal{R}) \\
%& \mathit{Return} \ (\mathit{pp}, \mathit{rs_{\mathit{pk}}}, \mathit{rs_{\mathit{vk}}})
(\mathit{pp}, \mathit{rs_{\mathit{pk}}}, \mathit{rs_{\mathit{vk}}}) \leftarrow \mathit{CKS_{\mathcal{R}}.Setup}(v) \\ 
\mathit{Return} \ (\mathit{pp}, \mathit{rs_{\mathit{pk}}}, \mathit{rs_{\mathit{vk}}})
\end{align*}
\end{itemize}

\noindent The four algorithms that are part of the accountable light client system are: 
\begin{itemize}
\item $\mathit{LC.GenerateProof}(\mathit{pp}, \mathit{rs_{\mathit{pk}}}, C, m, \mathcal{R})$
\begin{align*}
&\ \ \ \ \mathbf{\Pi} = (); \ \mathbf{\Sigma} = () \\
&\ \ \ \ i = \mathit{epoch_{id}}(m) \\
&\ \ \ \ \mathit{For} \ j = 1, \ldots, i \\
&\ \ \ \ \ \ \ \ \mathit{If} \  j < i  \\
&\ \ \ \ \ \ \ \ \ \ \ \ \mathit{m_j}= (j,\mathbf{Com}(\mathbf{pk_{j+1}})) \\
&\ \ \ \ \ \ \ \ \mathit{Else} \\
&\ \ \ \ \ \ \ \ \ \ \ \ \mathit{m_j} = m  \\
&\ \ \ \ \ \ \ \ (\mathbf{bit_j}, \mathbf{\sigma_j}) = \mathit{f_{bit}}(C, m_j, v) \\
%&\ \ \ \ \ \ \ \ \mathit{For} \ k = 1, \ldots, v \\
%&\ \ \ \ \ \ \ \ \ \ \ \ \textit{If exists }  \ \sigma  \in C \ \wedge \ \mathit{AS.Verify}(\mathit{pp}, \mathbf{pk_j}(k), \mathit{m_j}, \sigma) = 1 \\
%& \ \ \ \ \ \ \ \ \ \ \ \ \ \ \ \ \mathbf{bit_j}(k) = 1 \ \wedge \ \mathbf{\sigma_j}(k) = \sigma \\
%&\ \ \ \ \ \ \ \ \ \ \ \ \textit{Else} \\ 
%& \ \ \ \ \ \ \ \ \ \ \ \ \ \ \ \ \mathbf{bit_j}(k) = 0 \ \wedge \ \mathbf{\sigma_j}(k) = \_ \\
%&\ \ \ \ \ \ \ \ s = \mathit{HammingWeight}(\mathbf{bit_j}, \mathcal{R}) \\
%&\ \ \ \ \ \ \ \ \mathit{apk_{j}} = \mathit{AS.AggregateKeys}(\mathit{pp}, (\mathbf{bit_j}(k) \cdot \mathbf{pk_j}(k))_{k=1}^v) \\
&\ \ \ \ \ \ \ \ \mathbf{\Sigma}(j) \leftarrow \mathit{AS.AggregateSignatures}(\mathit{pp}, (\mathbf{\sigma_j}(k))_{k=1}^v) \\ 
%&\ \ \ \ \ \ \ \ \mathbf{x_j} = f_x(\mathbf{Com}(\mathbf{pk_j}), \mathbf{bit_j}, s,  \mathit{apk_{j}}, \mathcal{R}) ; \ \mathbf{w_j} = f_w( \mathbf{pk_j}, \mathbf{bit_j}, \mathcal{R}) \\
%&\ \ \ \ \ \ \ \ \mathbf{\pi_{j}} = \mathit{SNARK.Prove}(\mathit{srs}_{pk} , (\mathbf{x_j}, \mathbf{w_j}), \mathcal{R})\\
&\ \ \ \ \ \ \ \ (\mathit{\pi_{\mathit{SNARK},j}}, \mathit{apk_j}, \mathbf{Com}(\mathbf{pk_{j}})) \leftarrow \mathit{CKS_{\mathcal{R}}.Prove}(\mathit{rs}_{pk}, (\mathbf{pk_{j}}(k))_{k=1}^{v}, (\mathbf{bit_j}(k))_{k=1}^{v})\\
&\ \ \ \ \ \ \ \ \mathbf{x_j} = f_x(\mathbf{Com}(\mathbf{pk_j}), \mathbf{bit_j}, s,  \mathit{apk_{j}}, \mathcal{R}) \\
%&\ \ \ \ \ \ \ \ \mathbf{\Pi}(j) = (\mathbf{x_j},\mathbf{\pi_{j}}); \ \mathbf{\Sigma}(j) = \mathit{\Sigma}_{j} \\
&\ \ \ \ \ \ \ \ \mathbf{\Pi}(j) = (\mathbf{x_j}, \mathit{\pi_{\mathit{SNARK}, j}}) \\
&\ \ \ \ \mathit{Return} \ (\mathbf{\Pi}, \mathbf{\Sigma})  
\end{align*}
\end{itemize}
\vspace{-0.5cm}
%\noindent The two algorithms that are only part of the accountable light client system are: 
\begin{itemize}
\item $\mathit{LC.VerifyProof}(\mathit{pp}, \mathit{rs_{\mathit{vk}}}, \LCseed, \pi, m, \mathcal{R})$
\vspace{-0.9cm}
\begin{align*}
& i = \mathit{epoch_{id}}(m) \\
& (\mathbf{\Pi}, \mathbf{\Sigma}) = \mathit{Parse}(\pi);  \\
& \mathit{For} \ j = 1, \ldots, i \\
& \ \ \ \ (\mathbf{x_j}, \mathbf{\pi_{\mathit{SNARK},j}}) = \mathbf{\Pi}(j); \ (\mathit{com_j}, \mathbf{bit_j}, \mathit{apk_j}) = \mathit{Parse}(\mathbf{x_j}, \mathcal{R}) \\
& \mathit{If} \ \LCseed \neq \mathit{com_1} \\
& \ \ \ \ \mathit{Return} \ \mathit{rej} \\
& \mathit{For} \ j = 1, \ldots, i \\
& \ \ \ \ \mathit{If} \ j < i\\
& \ \ \ \ \ \ \ \ \ \mathit{m_j} = (j, \mathit{com_{j+1}}) \\
& \ \ \ \ \mathit{Else}\\
& \ \ \ \ \ \ \ \ \ \mathit{m_j} = m \\
& \ \ \ \ \mathit{threshold_j} = \mathit{f_{\mathit{threshold}}}(\mathbf{x_j}, \mathcal{R})  \\
%& \ \ \ \ \mathit{If} \ (\mathit{SNARK.Verify}(\mathit{srs_{vk}}, \mathbf{x_j}, \mathbf{\pi_j}, \mathcal{R}) = 0)  \ \vee  \ (\mathit{AS.Verify}(\mathit{pp}, \mathit{apk_j}, \mathit{m_j}, \mathbf{\Sigma}(j)) = 0) \ \vee (s < t) \\
& \ \ \ \ \mathit{If} \ (\mathit{CKS_{\mathcal{R}}.Verify}(\mathit{pp},  \mathit{rs_{\mathit{vk}}}, \mathit{com_j}, m_j,  \mathbf{\Sigma}(j), (\mathbf{\pi_{\mathit{SNARK},j}},\mathit{apk_j}), 
\mathbf{bit_j}) = 0)  \vee (\mathit{threshold_j} < t) \\
& \ \ \ \ \ \ \ \ \ \mathit{Return} \ \mathit{rej} \\
& \mathit{Return} \ \mathit{acc} \\
\end{align*}
\end{itemize}
\vspace{-0.5cm}
\begin{itemize}
\item $\mathit{LC.VerifyMisbehaviour}(\mathit{pp}, i, S, \mathbf{bit}, \sigma, m'', m', C)$
\vspace{-0.25cm}
\begin{align*}
%& \mathit{If} \ (\mathit{CheckValidConsensus}(C) =1) \\
& \ \ \ \ \mathit{apk} = \mathit{AS.AggregateKeys}(\mathit{pp}, (\mathbf{bit}(k) \cdot \mathbf{pk_i}(k))_{k=1}^v) \\
&\ \ \ \ (\mathbf{bit'}, \_ ) = \mathit{f_{bit}}(C, m', v) \\ 
%& \ \ \ \ \mathit{For} \ k = 1, \ldots, v \\
%& \ \ \ \ \ \ \ \ \textit{If exists }  \ (\mathit{sig}  \in C) \ \wedge \ (\mathit{AS.Verify}(\mathit{pp}, \mathbf{pk_{\mathbf{i}}}(k), m', \mathit{sig}) = 1) \\
%& \ \ \ \ \ \ \ \ \ \ \ \mathbf{bit'}(k) = 1 \\
%& \ \ \ \ \ \ \ \ \textit{Else} \\ 
%& \ \ \ \ \ \ \ \ \ \ \ \mathbf{bit'}(k) = 0 \\
& \ \ \ \ \mathit{Compute} \ S_{m''} = \{ \mathbf{pk}_{\mathbf{i}}(k) \ | \ \mathbf{bit}(k) = 1, k \in [v] \} \\
& \ \ \ \ \mathit{Compute} \ S_{m'} =  \{ \mathbf{pk}_{\mathbf{i}}(k) \ | \ \mathbf{bit'}(k) = 1, k \in [v] \} \\
& \ \ \ \ \mathit{If} \ (\mathit{AS.Verify}(\mathit{pp}, \mathit{apk}, m'', \sigma) = 1) \ \wedge (S_{m''} \cap S_{m'} = S) \ \wedge \ (|S_{m'}| \geq t) \ \wedge \
 (|S_{m''}| \geq t) \ \wedge \\
%& \hspace{1cm} \wedge \ {\color{red}(m' \in C )} \ \wedge \ {\color{red} (\exists \ d_{m'} \in C\cap D, \mathit{VerifyData}(m', d_{m'})=1)} \ \wedge\ \\
& \hspace{1cm} \wedge ((\mathit{m'}, d_{\mathit{m'}}) \in_{\mathit{decided}} C) \ \wedge \ \\
& \hspace{1cm} \wedge \ (i = \mathit{epoch_{id}}(m'') = \mathit{epoch_{id}}(m')) \ \wedge \ (\mathit{Incompatible(m'', m', d_{m'})} = 1)  \\
& \ \ \ \ \ \ \ \ \mathit{Return} \ \mathit{acc} \\
& \ \ \ \ \mathit{Else} \\
& \ \ \ \ \ \ \ \ \mathit{Return} \ \mathit{rej} 
%& \mathit{Else} \\
%& \ \ \ \ \mathit{Return} 
\end{align*}

\item $\mathit{LC.DetectMisbehaviour}(\mathit{pp}, \mathit{rs_{\mathit{vk}}}, \pi, m, C, \mathcal{R})$
\begin{align*}
& (\mathbf{\Pi}, \mathbf{\Sigma}) = \mathit{Parse}(\pi)  \\
& i = \mathit{epoch_{id}}(m) \\
& \mathit {index} = i \\
& m'' = m \\
& \mathit{For} \ j = 1, \ldots, i \\
& \ \ \ \ (\mathbf{x_j}, \mathit{\pi_{\mathit{SNARK},j}}) = \mathbf{\Pi}(j); \ (\mathit{apk_j}, \mathit{com_j}) = \mathit{Parse}(\mathbf{x_j}) \\
& \mathit{If} \ (\mathit{LC.VerifyProof}(\mathit{pp}, \mathit{rs_{\mathit{vk}}}, \LCseed, \pi, m, \mathcal{R}) = 1) \ \wedge \  ( \exists \ \mathit{min} \ 2 \leq j \leq i, \ \mathit{com_j} \neq \mathbf{Com}(\mathbf{pk_j}))\\ % \ \wedge \\
%&  \hspace{5.2cm} \ \wedge \ (\mathit{CheckValidConsensus}(C) = 1) \\
& \ \ \ \ \ \ \ m'' = (j-1,\mathit{com_j}); \ m' = (j-1, \mathbf{Com}(\mathbf{pk_j})); \mathit{index} = j-1\\
& \mathit{ElseIf} \ (\mathit{LC.VerifyProof}(\mathit{pp}, \mathit{rs_{\mathit{vk}}}, \LCseed, \pi, m, \mathcal{R}) = 1) \  \wedge \ (\forall \ 2 \leq j \leq i, \ \mathit{com_j} = \mathbf{Com}(\mathbf{pk_j})) \ \wedge \\ 
& \hspace{1cm} \wedge (\exists \ (\mathit{aux}, d_{\mathit{aux}}) \in_{\mathit{decided}} C) \ \wedge \ \mathit{Incompatible}(\mathit{aux}, m'',  d_{\mathit{aux}})=1) \\
%& \hspace{1cm} \wedge \ (\exists \ {\color{red} \mathit{aux}}, d_{{\color{red} \mathit{aux}}} \in C, {\color{blue} \mathit{aux} \textit{ decided  in }  C}, \mathit{Incompatible}({\color{red} \mathit{aux}}, {\color{red} m''},  d_{ {\color{red} \mathit{aux}}})=1) \\
%& \hspace{1cm} \wedge \ (\mathit{CheckValidConsensus}(C) = 1) \\
& \ \ \ \ \ \ \ m' =   \mathit{aux}\\
&\mathit{Else} \ \mathit{Return} \\
& \mathbf{bit} = \mathit{Transform}(\mathit{Parse}(\mathbf{x_{\mathit{index}}}, \mathcal{R}), \mathcal{R}) \\
& \mathit{Compute} \ S_{m''} = \{ \mathbf{pk}_{\mathbf{index}}(k) \ | \ \mathbf{bit}(k) = 1, k \in [v] \} \\
%& \mathit{For} \ k = 1, \ldots, v \\
%& \ \ \ \ \textit{If exists }  \ (\sigma  \in C) \ \wedge \ (\mathit{AS.Verify}(\mathit{pp}, \mathbf{pk_{\mathbf{index}}}(k), m', \sigma) = 1) \\
%& \ \ \ \ \ \ \ \mathbf{bit'}(k) = 1 \\
%& \ \ \ \ \textit{Else} \\ 
%& \ \ \ \ \ \ \ \mathbf{bit'}(k) = 0 \\
& (\mathbf{bit'}, \_ ) = \mathit{f_{bit}}(C, m', v) \\ 
& \mathit{Compute} \ S_{m'} = \{ \mathbf{pk}_{\mathbf{index}}(k) \ | \ \mathbf{bit'}(k) = 1, k \in [v] \} \\
& \mathit{Return} \ (\mathit{index}, S_{m''} \cap S_{m'}, \mathbf{bit}, {\mathbf{\Sigma}}(\mathit{index}), {m''}, m' ) 
\end{align*}
\end{itemize}

\subsubsection{Assumptions and Security Proofs}
\label{sec:assumptions}
\label{sec:security_proofs}

\noindent We complete our instantiation by proving the security properties of our light client and accountable light client systems according to 
definitions introduced in Sections~\ref{sec:soundness} and \ref{sec:accountability}. However, beforehand, we present the assumptions we use, of 
which there are six classes, i.e., there are assumptions about honest validators' behaviour (B), about consensus (C), about parameters (P), about instantiation of primitives (S), 
about genesis state (G) and assumptions about light client integration (I). \\

\noindent The assumptions about honest validators' behaviour are:
\begin{itemize}
\item (B.1.) An honest validator never signs a message $m$ unless it knows some required data $d_m$  
such that $\mathit{VerifyData}(m, d_m) = 1$ holds.
\item (B.2.) An honest validator never signs a message $m$ such that $\mathit{VerifyData}(m, d_m) = 1$ holds 
if they have previously signed $m'$ such that $\mathit{VerifyData}(m', d_{m'}) = 1$ holds and \\ $\mathit{Incompatible}(m, m',d_m) = 1$ 
or $\mathit{Incompatible}(m', m, d_{m'}) =1$ hold.
\item (B.3.) An honest validator does not sign any message in $M_i$ unless they have a valid consensus view $C$ (with $M_i \subset C$) 
for which their public key is in $\mathbf{pk_i}$ with $\mathbf{pk_i} \in C$.
\end{itemize}

%{\color{blue} TO DO: CHECK the assumptions below for correctness}
\noindent The assumptions about consensus are:
\begin{itemize}
\item (C.1.) The adversary interacting with honest validators should not except with negligible probability be able to produce both:
(i) a valid consensus view $C$ in which at least $t'$ validators in every epoch are honest that decides some message $m$ 
with $d_m$ such that $\mathit{VerifyData}(m, d_m) =1$ and 
(ii) a valid consensus view $C'$ with the same genesis state as $C$ (in particular, with the same $\mathbf{pk_1} \subset \mathsf{genstate}$) 
which decides some message $m'$ in the same epoch as $m$, with $\mathit{Incompatible}(m, d_m, m') = 1$. 
\item (C.2.) The adversary interacting with honest validators should not except with negligible probability be able to produce both:
(i) a valid consensus view $C$ in which at least $t'$ validators in every epoch are honest and 
(ii)  a valid consensus view $C'$ with the same genesis state as $C$ (in particular, with the same $\mathbf{pk_1} \subset \mathsf{genstate}$) 
in which there is some epoch $i$ that $C$ and $C'$ both reach with $\mathbf{pk_i} \neq \mathbf{pk'_i}$.
\end{itemize}

\noindent The assumptions about parameters are:
\begin{itemize}
\item (P.1.) $2t -v > 0$
\item (P.2.) $t + t' > v$
\end{itemize}

\noindent The assumption about instantiation of primitives is:
\begin{itemize}
\item (S.1.) We instantiate the aggregatable signature scheme $\mathit{AS}$ 
such that the oracle $\mathit{OSign}$ in Definition~\ref{def:aggregate_signatures} (in particular in the 
unforgeability property definition), is replaced with $\mathit{OSpecialSign}$. It is easy to see that if $\mathit{AS}$ 
is an aggregatable signature scheme secure according to Definition~\ref{def:aggregate_signatures},  
then $\mathit{AS}$ is also an aggregatable signature with oracle $\mathit{OSign}$ replaced by 
$\mathit{OSpecialSign}$ in Definition~\ref{def:aggregate_signatures}.
\end{itemize}

\noindent The assumptions about genesis state are:
\begin{itemize}
\item (G.1.) In a valid consensus view, $\mathit{HistoricVerifyData}$ checks, among others, that 
a) every $\mathit{pk} \in \mathbf{pk_1}$ is also part of  $\mathsf{genstate}$, b) that every $\mathit{pk} \in \mathbf{pk_1}$ is in $\ginn{1}$ 
and c) that the proofs of possession for each of the public keys in $ \mathbf{pk_1}$ pass the verification in $\mathit{AS.VerifyPoP}$.
\item  (G.2.) We assume that all honest full nodes and validators have access to the same genesis state 
$\mathsf{genstate}$ even when the genesis state is generated by a potential adversary.
\end{itemize}

Before the last class of assumptions, we add two notational conventions in the form of two functions:
\begin{itemize}
\item $\mathit{NextEpochKeys}(m, d_m)$ returns $\bot$ or a list of public keys; if $\mathit{epoch_{id}}(m) = i$, 
these keys are supposed to be the public keys of epoch $i+1$. 
%, where $d_m$ with $\mathit{ValidateData}(m, d_m) =1$. 
%If $\mathit{epoch_{id}}(m) = i$ and $m \in C$, $\mathit{NextEpochKeys}(m, d_m) \neq \bot$, for some valid consensus view $C$, 
%then $\mathit{NextEpochKeys}(m, d_m)= \mathbf{pk_{i+1}}$ and $\mathbf{pk_{i+1}} \subset d_m$.} 
\item  $\mathit{IsCommitment}(m)$ returns $0$ or $1$; $\mathit{IsCommitment}(m) = 1$ iff  there exists some $i$ such that 
$m = (i, \mathbf{Com}(\mathbf{pk_{i+1}}))$.
%\item {\color{red} $\mathit{IsCommitment}(m)$ returns $0$ or $1$. If $\mathit{epoch_{id}}(m) = i$ and $m \in C$, 
%for some valid consensus view $C$, then $\mathit{IsCommitment}(m) = 1$ iff $m = \mathbf{Com}(\mathbf{pk_{i+1}})$.}
%\item {\color{red} We say predicate $P(m, d_m, \mathbf{pk}) = 1$ iff 
%the blockhash $m$ is final ({\color{blue} What does that mean?}), $d_m$ is the blockchain leading up to and including 
%the block for which $m$ is a blockhash and $\mathbf{pk}$ is the set of public keys which determines the next validator 
%set and is also a deterministic function of $d_m$.}
\end{itemize}

\noindent Finally, we make the following light client integration assumptions, i.e., these are assumptions that 
apply to our specific light client instantiation: 
\begin{itemize}
\item (I.1.) If $m$ and $m'$ are such that 
$\mathit{epoch_{id}}(m) =\mathit{epoch_{id}}(m')$ and 
$\mathit{NextEpochKeys}(m, d_m) \neq \bot$ and \\
$\mathit{IsCommitment}(m') = 1$ and 
$m' \neq (\mathit{epoch_{id}}(m), \mathbf{Com}(\mathit{NextEpochKeys}(m, d_m)))$ 
then $$\mathit{Incompatible}(m, m', d_m)=1.$$
\item (I.2.) If $\mathit{epoch_{id}}(m) =i$ and $\mathit{NextEpochKeys}(m, d_m)=\mathbf{pk_{i+1}}$, 
then $\mathit{ValidateData}(m, d_m)$ must call \\ $\mathit{AS.VerifyPoP}(\mathit{pp}, \mathit{pk}, \mathit{\pi_{POP}})$ 
for each $pk \in \mathbf{pk_{i+1}}$ and some data $\mathit{\pi_{POP}} \in d_m$ and also check that
 $\mathit{pk} \in \mathbb{G}_{1,\mathit{inn}}$; if any of these checks fails, then $\mathit{ValidateData}(m, d_m)$ fails.
\item (I.3.) An honest validator with a valid consensus view $C$, does not sign a message $m'$ 
with \\ $\mathit{IsCommitment}(m')=1$ unless there exists a message $m$ decided in $C$ 
and its required data $d_m$ (i.e., $\mathit{ValidateData}(m, d_m) =1$) such that 
$$m'=(\mathit{epoch_{id}}(m), \mathbf{Com}(\mathit{NextEpochKeys}(m, d_m))).$$ 
\item (I.4.) If $\mathit{HistoricVerifyData}$ outputs $1$ and there exist a message $m \in C$ 
that has been decided in epoch $i$, then for all $1 \leq j < i$, $(j, \mathbf{Com}(\mathbf{pk_{j+1}}))$ was decided in epoch $j$. 
\item (I.5.) If $\mathit{HistoricVerifyData}$ outputs $1$ and a message $m'$ has been decided in $C$ 
such that \\ $\mathit{IsCommitment}(m') = 1$, then there exist $m, d_m \in C$ with $\mathit{ValidateData}(m, d_m) = 1$, $m$ decided in $C$
and $\mathit{epoch_{id}}(m) = \mathit{epoch_{id}}(m')$ such that 
$$\mathbf{pk_{\mathit{epoch_{id}}(m)+1}} = \mathit{NextEpochKeys}(m, d_m).$$
\end{itemize}
%{\color{blue} I propose changing the order of assumptions to (I.2), (I.1), (I.4), (I.5), (I.3).} \\

\noindent We are now ready to state and prove the security properties of our (accountable) light client systems. 

\begin{theorem}
\label{th:soundness} 
If $\mathit{AS}$ is the secure aggregatable signature scheme defined in Instantiation~\ref{insta:bls} and if 
$\mathit{CKS_{\mathcal{R}}}$ is the secure committee key scheme defined in Instantiation~\ref{inst:cks}, then, together with 
the assumptions stated at the beginning of Section~\ref{sec:assumptions} and for $\mathcal{R} \in \{ \Rlacom, \Racom \}$, the tuple 
($\mathit{LC.Setup}$, $\mathit{LC.GenerateProof}$, $\mathit{LC.VerifyProof}$) as instantiated in Section~\ref{sec:instantiation} is a light client system.
\end{theorem}


\begin{proof} 
We include the full proof in Section~\ref{supplementary_proof_sec_soundness}.   
\end{proof}

\begin{theorem} 
\label{th:accountability_results}
If $\mathit{AS}$ is the secure aggregatable signature scheme defined in Instantiation~\ref{insta:bls} and if 
$\mathit{CKS_{\mathcal{R}}}$ is the secure committee key scheme defined in Instantiation~\ref{inst:cks}, then, together 
with the assumptions stated at the beginning of Section~\ref{sec:assumptions} and for $\mathcal{R} \in \{ \Rlacom, \Racom \}$, the tuple 
($\mathit{LC.Setup}$, $\mathit{LC.GenerateProof}$, $\mathit{LC.VerifyProof}$, $\mathit{LC.DetectMisbehaviour}$, 
$\mathit{LC.VerifyMisbehaviour}$) as instantiated in Section~\ref{sec:instantiation} is an accountable light client system.
\end{theorem}
\begin{proof}
\noindent Due to theorem~\ref{th:soundness}, the tuple 
($\mathit{LC.Setup}$, $\mathit{LC.GenerateProof}$, $\mathit{LC.VerifyProof}$, \\ $\mathit{LC.DetectMisbehaviour}$, 
$\mathit{LC.VerifyMisbehaviour}$) as instantiated in Section~\ref{sec:instantiation} is already a light client system. 
It is only left to show that both accountability completeness and accountability soundness also hold and we include the full details of this proof in 
Section~\ref{supplementary_proof_sec_accountability}.   
\end{proof}

\begin{corollary} In an accountable light client system, the number of misbehaving validators output by \\ 
$\mathit{LC.DetectMisbehaviour}$ is $|S|$ and $|S| > 0$.
\end{corollary}
\begin{proof} Due to theorem~\ref{th:accountability_results} and accountability completeness, 
given a valid consensus view $C$, a verifying light client proof $\pi$ for a message $m''$ 
and given the existence in $C$ of a message $m'$ incompatible with $m''$, the number of 
validators that $\mathit{LC.DetectMisbehaviour}$ is able to catch is at least $|S|$. Moreover, due 
to accountability soundness, any public key output by $\mathit{LC.DetectMisbehaviour}$, \ewnp, belongs to a misbehaving validator. 
Finally, using again accountability completeness and, in particular, since $\mathit{LC.VerifyMisbehaviour}$ accepts 
with overwhelming probability the output of an honest party running $\mathit{LC.DetectMisbehaviour}$ and due to assumption (P.1.) it holds that:
$$ |S| = |S_{m'} \cap S_{m''}| = |S_{m'}| + |S_{m''}| - |S_{m'} \cup S_{m''}| \geq t + t - v > 0.$$ 
\end{proof}

\section{Postponed Security Proof for Light Client Systems}
\label{supplementary_proof_sec_soundness}

\noindent In this section we prove the following theorem:

\begin{theorem} 
\label{thm_lc_soundness}
If $\mathit{AS}$ is the secure aggregatable signature scheme defined in instantiation~\ref{insta:bls} and if 
$\mathit{CKS_{\mathcal{R}}}$ is the secure committee key scheme defined in instantiation~\ref{inst:cks}, 
then, together with the assumptions stated at the beginning of Section~\ref{sec:LCinstantiation} and for 
$\mathcal{R} \in \{ \Rlacom, \Racom \}$, the tuple ($\mathit{LC.Setup}$, $\mathit{LC.GenerateProof}$, $\mathit{LC.VerifyProof}$) 
as instantiated in Section~\ref{sec:LCinstantiation} is a light client  as formalised by Definition~\ref{scheme_light_client}.
\end{theorem}

\begin{proof}
\noindent \textit{Perfect Completeness:} 
Let $m$ be a message decided in some epoch $i$ of a valid consensus view $C$. 
Since $C$ is a valid consensus view, this implies $\mathit{HistoricVerifyData}$ outputs $1$. Adding that $m$ has been decided in epoch $i$ 
and using assumption (I.4.), we have that for each previous epoch $j \in [i-1]$, $(j, \mathbf{Com}(\mathbf{pk_{j+1}}))$ 
was decided in epoch $j$; we denote this as property $(\ast)$. Since $$\mathit{IsCommitment}(j, \mathbf{Com}(\mathbf{pk_{j+1}})) = 1, \forall j \in [i-1]$$  
holds and using assumptions (I.5.), (I.2.) and (G.1.), we conclude the proofs of possession for each of the public keys in 
$\mathbf{pk_{j}}$, $j  \in [i]$ pass the verification $\mathit{AS.VerifyPoP}$ (property $(**)$) and, as a consequence, each of the public keys in $\mathbf{pk_{j}}$, $j  \in [i]$ belong to $\ginn{1}$ (property $(***)$). 
The main fact we have to show (with the notation used in the description of 
$\mathit{LC.VerifyProof}$), is that the following two predicates hold:
$$\mathit{AS.Verify}(\mathit{pp}, \mathit{apk_j}, m_j, \mathbf{\Sigma}(j)) = 1, \forall j \in [i]  \ \ \ \ \ (1)$$ 
and 
$$\mathit{threshold_j} \geq t, \forall j \in [i] \ \ \ \ \ (2).$$
\noindent Indeed, $(1)$ holds due to perfect completeness for aggregation for secure signature scheme instantiation 
$\mathit{AS}$ which applies because: (a) for every epoch $j \in [i]$, as computed by $\mathit{LC.GenerateProof}$, 
each of the individual signatures aggregated into $\mathbf{\Sigma}(j)$ passes $\mathit{AS.Verify}$, (b)
the aggregation $\mathbf{\Sigma}(j)$ is computed correctly as per $\mathit{LC.GenerateProof}$, (c) the proofs of possession have been checked 
for each of the public keys in $\mathbf{pk_{j}}$, $\forall j  \in [i]$ (see property $(**)$), and, finally, (d) the aggregation of public 
keys denoted by $\mathit{apk_j}$, $\forall \ j \in [i]$, has been computed correctly as $(\mathbf{bit_j}(k) \cdot \mathbf{pk_j}(k))_{k=1}^{v}$ 
due to property $(***)$ and the perfect completeness of the SNARK scheme for relation $\mathcal{R}$ invoked by the instantiation of 
$\mathit{CKS}_{\mathcal{R}}.\mathit{Prove}$. \\
Moreover, due to definition of $f_{\mathit{threshold}}$ and the fact that 
$m_j = (j, \mathbf{Com}(\mathbf{pk_{j+1}})), \forall j \in [i-1]$ and, respectively,  $m_i = m$ have been 
decided in their respective epochs as per $(*)$, we have that $(2)$ holds. \\

\noindent Finally, using $(1)$, letting $ \mathit{ck_j} = \mathit{com_j} = \mathbf{Com}(\mathbf{pk_{j+1}})$, $ \forall \ j \in [i]$ and since 
$\forall \ j \in [i]$, $\pi_{\mathit{SNARK,j}}$, $\mathit{apk_j}$ and $\mathit{ck_j}$ are honestly computed as described by  
$\mathit{LC.GenerateProof}$ and invoking the perfect completeness property of the $\mathit{CKS_{\mathcal{R}}}$ committee key scheme, 
we obtain that $$\mathit{CKS_{\mathcal{R}}.Verify}(\mathit{pp}, \mathit{rs}_{\mathit{vk}}, \mathbf{Com}(\mathbf{pk}_{\mathbf{j+1}}), 
m_j, \mathbf{\Sigma}(j), (\pi_{\mathit{SNARK,j}}, \mathit{apk_j}), \mathbf{bit_j}) = 1, \forall j \in [i]  \ \ \ \ \ (3).$$
In turn, the fact that $(2)$ and $(3)$ hold with probability $1$ immediately implies 
$$\mathit{LC.VerifyProof}(\mathit{pp}_{\mathit{LC}},\mathit{LC.seed}, \mathit{LC .GenerateProof}(\mathit{pp}_{\mathit{LC}}, C, m, \mathcal{R}),m, \mathcal{R}) = \mathit{acc}$$ with probability 1 (q.e.d.). \\

\noindent \textit{Soundness:} 
\noindent In order to prove soundness, we first state and prove the following: 
\begin{proposition} 
\label{le:lc_soundness}
Given an efficient adversary $\mathcal{A}$ as defined in the soundness game 
(Definition~\ref{scheme_light_client}) and let $(\pi, m, C)$ be its corresponding output. 

Let $i = \mathit{epoch}_{\mathit{id}}(m)$. Assuming that 
$$\mathit{LC.VerifyProof}(\mathit{pp}_{\mathit{LC}}, \mathit{LC.seed}, m, \mathcal{R}) = \mathit{acc}$$ and 
$\mathit{CheckValidConsensus}(C) = 1$ and 
$\mathit{HonestThreshold}(t', \mathit{OGenerateKeypair}, C) = 1$ (i.e., the light client proof $\pi$ is accepted, 
$C$ is a valid consensus view as per Definition~\ref{def:valid_consensus} and for each epoch $k$ in $C$, $PK_k$ 
contains at least $t'$ honest validators), then:
\begin{itemize}
\item Statement $A(j)$: For $ j < i$, assuming further that $\mathit{com_j} = \mathbf{Com}(\mathbf{pk_j})$, %and that the validity of the proofs of 
%possession for each key in $\mathbf{pk_j}$ has been checked (either as a property of $\mathsf{genstate}$ for $j=1$ 
%or by the honest validators in $\mathbf{pk_{j-1}}$) and $\mathbf{pk_j} \in \ginn{1}^{n-1} $, 
then there exists some honest validator whose key is in $\mathbf{pk_j}$ such that it signed 
$m_j = (j, \mathbf{Com}(\mathbf{pk_{j+1}}))$, except with negligible probability.
\item Statement $B(j)$: For $j < i$, if an honest validator whose key is in $\mathbf{pk_j}$ 
signed $m_j$ with \\ $\mathit{epoch_{\mathit{id}}}(m_j) = j$ and 
$\mathit{IsCommitment}(m_j) =1$ then $m_j= (j, \mathbf{Com}(\mathbf{pk_{j+1}}))$.
\end{itemize}
\end{proposition}

\begin{proof}(Proposition) We prove the proposition above by induction. Moreover, we prove the proposition 
only for $ \mathcal{R} = \Rlacom$. The proposition can be proven analogously for $\mathcal{R} = \Racom$. 
Proving the base case, namely that $A(1)$ holds under the assumption G.1. and proving that $A(j)$ holds 
if $B(j-1)$ holds follows a very similar proof structure hence we give complete details only for the latter and add 
only the differences for the former. We complete the induction step by proving that if $A(j)$ holds then $B(j)$ holds. \\

\noindent First proof of the induction step: Assume that statement $B(j-1)$ holds. We have to prove that $A(j)$ holds. 
Due to the assumption that the light client proof $\pi$ is accepted and due to 
the definition of step $j$ in algorithm $\mathit{LC.VerifyProof}$, we have that properties 
$(1)$ and $(2)$ as described below hold, except with negligible probability, where

$$(\mathit{CKS}_{\mathcal{R}}.\mathit{Verify}(\mathit{pp},  \mathit{rs_{\mathit{vk}}}, \mathit{com_j}, m_j, 
 \mathbf{\Sigma}(j), (\mathbf{\pi_{\mathit{SNARK},j}}, \mathit{apk_j}), 
\mathbf{bit_j}) = 1) \ \ \ \ \  (1)$$ 
and
$$(\mathit{threshold_j} \geq t) \ \ \ \ \  (2)$$

\noindent Due to instantiation~\ref{inst:cks}, $(1)$, in turn, is equivalent to properties $(3)$ and $(4)$ holding, 
except with negligible probability, where:

$$ \mathit{AS.Verify}(\mathit{pp}, \mathit{apk_j}, m_j, \mathbf{\Sigma}(j)) = 1 \ \ \ \ \  (3)$$ 
and
$$\mathit{SNARK.Verify}(\mathit{rs}_{\mathit{vk}}, (\mathit{com_j}, \mathbf{bit_{j}} || 0, \mathit{apk}), \pi_{\mathit{SNARK}}, \mathcal{R}) = 1 \ \ \ \ \  (4)$$ 

\noindent By the knowledge soundness property of the hybrid model SNARK for relation $\mathcal{R}$ and 
algorithm $\mathit{SNARK.PartInputs}$ defined in Section~\ref{sec_two_step_compiler}  
(where $c(\mathbf{pk_j}) = \mathit{incl}(\mathbf{pk_j}) =1$ 
iff $\mathbf{pk_j} \in \ginn{1}^{n-1}$ holds) and since $(4)$ holds and since $\mathbf{pk_j} \in \ginn{1}^{n-1}$ holds 
as a consequence of the fact that the proofs of possession for each of the public
keys in $\mathbf{pk_j}$ pass the verification in $\mathit{AS.VerifyPoP}$ (which, in turn, holds since 
$B(j-1)$ holds plus due to integration assumptions I.1.- I.3. and the definition of $\mathit{IsCommitment}$), 
it means that, extractor $\mathcal{E}$ (as described in Definition~\ref{dfn_snark} can extract 
$w$ such that $$(\mathbf{x_j} = (\mathit{com_j},\mathbf{bit_j}||0,\mathit{apk_j}), w = \mathbf{pk'}) \in \mathcal{R},$$ 
except with negligible probability. In particular, this means $\mathit{apk_j} = \sum_{k=1}^{n-1} \mathbf{bit_j}(k) \cdot \mathbf{pk'}(k)$ and 
$\mathbf{Com}(\mathbf{pk'}) = \mathit{com_j}$. By the computational binding of the KZG commitment used in defining $\mathit{com_j}$ 
and by the fact that  $\mathit{com_j} = \mathbf{Com}(\mathbf{pk_j})$ by assumption (i), we obtain that $\mathbf{pk'} = \mathbf{pk_j}$, 
hence
$$\mathit{apk_j} = \sum_{k=1}^{n-1} \mathbf{bit_j}(k) \cdot \mathbf{pk_j}(k) \ \ \ \ \ (5)$$ 
which, in turn, by the definition of aggregatable signature scheme $\mathit{AS}$ in instantiation~\ref{sec:bls} is equivalent to: 

$$\mathit{apk_j} = \mathit{AS.AggKeys}(\mathit{pp}, (\mathbf{pk_j}(k))_{k=1}^{n-1}) \ \ \ \ \ (6)$$ 

\noindent Next, we look at $(2)$ which is equivalent to $\mathit{HammingWeight^*}(\mathbf{bit_j}) \geq t \ (7)$; (7)  
together with the fact that there are at least $t'$ honest validators in 
$\mathbf{pk}$ (since $\mathit{HonestThreshold}(t', \mathit{OGenerateKeypair}, C) = 1$) and the assumption P.2. that 
$t+t' \geq v=n-1$, we obtain that there exists at least an honest validator in $\mathbf{pk_j}$ 
whose public key is aggregated into $\mathit{apk_j}$. We denote this as property $(8)$. \\

\noindent Finally, it is clear that due to $(3), (6), (8)$ and since the proofs of possession for each of the public 
keys in $\mathbf{pk_j}$ pass the verification in AS.VerifyPoP (in turn, since $B(j-1)$ holds and due to integration assumptions 
I.1.- I.3. and the definition of $\mathit{IsCommitment}$), the statement $A(j)$ 
becomes equivalent to showing that the advantage
$\mathit{Adv}^{\mathit{multiforge}}_{\mathcal{A}_{\mathit{sound}}}({\lambda})$ in the following game is negligible $(9)$,
where, in general, 
$$\mathit{Adv}^{\mathit{multiforge}}_{\mathcal{A}}({\lambda}) = \mathit{Pr}[\mathit{Game}^{\mathit{multiforge}}_{\mathcal{A}}({\lambda}) =1]$$
\noindent and 
\begin{align*}
&\mathit{Game}^{\mathit{multiforge}}_{\mathcal{A}}({\lambda}): \\
& \mathit{pp} \leftarrow \mathit{AS.Setup}(\mathit{aux_{\mathit{AS}}}) \\
& ((\mathit{pk}_{k}^*,\pi^*_{k, \mathit{PoP}}), \mathit{sk}_{k}^*)_{k=1}^{t'} \leftarrow \mathit{AS.GenKeypair}(\mathit{pp})\\
& Q \leftarrow \emptyset \ \\
& ((\mathit{pk_k}, \pi_{k,\mathit{PoP}})_{k=1}^{u}, m, \mathit{asig}) \leftarrow \mathcal{A}^{\mathit{OMSign}}(\mathit{pp}, (\mathit{pk_k^*},\pi^*_{k,\mathit{PoP}})_{k=1}^{t'}) \\
& \textit{If } \exists \ k \in [t'], \mathit{pk}_{k}^*  \notin \{ \mathit{pk_i} \}_{i=1}^{u}  \vee (m, \mathit{pk}^*_{k}) \in Q, \textit{ then return } 0 \\
& \textit{For } i \in [u] \\
& \ \ \ \ \ \textit{ If } \mathit{AS.VerifyPoP}(\mathit{pp}, \mathit{pk_i}, \pi_{\mathit{PoP},i})=0  \textit{ return } 0 \\
& \mathit{apk} \leftarrow \mathit{AS.AggKeys}(\mathit{pp}, (\mathit{pk_i})_{i=1}^{u}) \\
& \textit{Return } \mathit{AS.Verify}(\mathit{pp}, \mathit{apk}, m, \mathit{asig})
\end{align*}
\noindent and
\begin{align*}
& \mathit{OMSign}(m_k, \mathit{pk}^*): \\
& \textit{If } \mathit{pk}^* \in Q_{\mathit{keys}|\mathit{pk}} \\
& \ \ \ \ \sigma_j \leftarrow \mathit{AS.Sign}(\mathit{pp}, \mathit{sk}^*, m_k) \\
& \ \ \ \  Q \leftarrow Q \cup \{(m_k,  \mathit{pk}^*) \} \\
& \ \ \ \ \textit{return } \ \sigma_k \\
& Else \\
& \ \ \ \ \textit{return}
\end{align*}

\noindent and $\mathcal{A}_{\mathit{sound}}$ is defined such that $\mathit{asig} = \mathbf{\Sigma(j)}$, $m = m_j$, 
$\mathit{apk} = \mathit{apk_j}$ and the public keys output by $\mathcal{A}_{\mathit{sound}}$ are the non-zero public 
keys from the vector $(\mathbf{bit_j}(k) \cdot \mathbf{pk_j}(k))_{k=1}^{n-1}$.

\noindent We prove statement (9) by contradiction: if we assume the advantage 
$\mathit{Adv}^{\mathit{multiforge}}_{\mathcal{A}_{\mathit{sound}}}({\lambda})$ is non-negligible, 
then, using a standard hybrid argument and reducing the game 
$\mathit{Game}^{\mathit{multiforge}}_{\mathcal{A}}({\lambda})$ to the game \\
$\mathit{Game}^{\mathit{forge}}_{\mathcal{A}}({\lambda})$ as per Definition~\ref{def:aggregate_signatures}, 
the advantage $\mathit{Adv}^{\mathit{forge}}_{\mathcal{A}_{\mathit{sound}}}({\lambda})$ is also non-negligible; 
however, this, in turn, contradicts the fact that the instantiation $\mathit{AS}$ is an unforgeable 
aggregatable signature scheme, hence our proof for $A(j)$ is complete.  \\

\noindent Observation: In the case of the proof for $A(1)$, the only difference is that the proofs of possession for each of the public 
keys in $\mathbf{pk_1}$ pass the verification in $\mathit{AS.VerifyPoP}$ by assumption G.1. By the definition of aggregatable signature 
scheme $\mathit{AS}$, as the consequence, $\mathbf{pk_1} \in \ginn{1}^{n-1}$.\\

\noindent Second proof of the induction step: Assume that statement $A(j)$ holds. 
 Assume by contradiction that $B(j)$ does not hold, i.e., an honest validator $\mathit{HVal}$ whose key is in $\mathbf{pk_j}$ signed $m_j$ such that 
$\mathit{IsCommitment}(m_j) = 1$ and $m_j \neq (j, \mathbf{Com}(\mathbf{pk_{j+1}}))$ (we call this property (10)).
%where $\mathbf{pk_{j+1}}$ is the validator set for epoch $j+1$ in $C$. \\

\noindent Due to assumption I.3, $\mathit{HVal}$ does not sign $m_j$ unless $\mathit{HVal}$ has a valid consensus view $C'$ deciding 
a message $m'$ with required data $d_{m'}$ and  $m_j = (j, \mathbf{Com}(\mathit{NextEpochKeys}(m', d_{m'}))$ 
(we call this property (11)). By (10) and (11) and the fact that the commitment scheme used to compute $\mathbf{Com}(\cdot)$ is binding,  we obtain:
$$\mathit{NextEpochKeys}(m', r_{m'}) \neq \mathbf{pk_{j+1}} \ \ \ \ \ (12).$$

%{\color{blue}\noindent Due to G.2. and by consensus assumption C.2., except with negligible probability, $C$ and $C'$ 
%have the same validator sets, in particular $\mathbf{pk_j}$ for epoch $j$. Do we need this? How is it used? Clarify and add to text.}

\noindent By assumptions I.3. and I.4, there exists in epoch $j$ of valid consensus view $C$ some decided message $m'_j$ with 
$\mathit{epoch}_{\mathit{id}}(m'_j) = j$ and $m'_j =\mathbf{Com}(\mathbf{pk_{j+1}})$.
Then, by assumption I.1, $m'_j$ and $m'$ are incompatible. This, in turn, contradicts assumption C.1. combined with assumption G.2. 
since $C$ and $C'$ decided in epoch $j$ messages $m'_j$ and $m'$, respectively.
Hence our initial assumption is false and $B(j)$ is proven to hold. And our proposition proof is complete.
\end{proof}

\noindent 
We are now able to prove the soundness property. Given an efficient adversary $\mathcal{A}$ 
as defined in the soundness game (Definition~\ref{scheme_light_client}) and let $(\pi, m, C)$ 
be its corresponding output. Let $i = \mathit{epoch}_{\mathit{id}}(m)$. Assuming that 
$$\mathit{LC.VerifyProof}(\mathit{pp}_{\mathit{LC}}, \mathit{LC.seed}, m, \mathcal{R}) = \mathit{acc}$$ and 
$\mathit{CheckValidConsensus}(C) = 1$ and $\mathit{HonestThreshold}(t', \mathit{OGenerateKeypair}, C) = 1$, 
then, using the proposition above, we obtain that statement $B(i-1)$ holds. Then, letting $m_i = m$ and with an analogous 
reasoning used for proving the induction step, namely that $A(j)$ holds when $B(j)$ holds (please see proof above) we are able to conclude 
that $m$ was signed by an honest validator only with negligible probability (q.e.d).
\end{proof}

\section{Postponed Security Proof for Accountable Light Client Systems}
\label{supplementary_proof_sec_accountability}

\noindent In this section we prove the following theorem:

\begin{theorem} 
\label{th:accountability_results}
If $\mathit{AS}$ is the secure aggregatable signature scheme defined in instantiation~\ref{insta:bls} and if 
$\mathit{CKS_{\mathcal{R}}}$ is the secure committee key scheme defined in instantiation~\ref{inst:cks}, then, together 
with the assumptions stated at the beginning of Section~\ref{sec:assumptions} and for $\mathcal{R} \in \{ \Rlacom, \Racom \}$, the tuple 
($\mathit{LC.Setup}$, $\mathit{LC.GenerateProof}$, $\mathit{LC.VerifyProof}$, $\mathit{LC.DetectMisbehaviour}$, 
$\mathit{LC.VerifyMisbehaviour}$) as instantiated in Section~\ref{sec:instantiation} is an accountable light client system.
\end{theorem}
\begin{proof}
\noindent Due to theorem~\ref{th:soundness}, the tuple 
($\mathit{LC.Setup}$, $\mathit{LC.GenerateProof}$, $\mathit{LC.VerifyProof}$, \\ $\mathit{LC.DetectMisbehaviour}$, 
$\mathit{LC.VerifyMisbehaviour}$) as instantiated in Section~\ref{sec:instantiation} is already a light client system. 
It is only left to show that both accountability completeness and accountability soundness also hold. Indeed: \\

\noindent \textit{Accountability Completeness:} %This follows immediately from the description of $\mathit{LC.DetectMisbehaviour}$ 
%and $\mathit{LC.VerifyMisbehaviour}$, from the fact that $$\mathit{LC.VerifyProof}(\pi, m, \mathcal{R}) = \mathit{acc}$$ 
%and $$\mathit{CheckValidConsensus}(C)=1$$ and from the existence of a message $\mathit{aux}$ in $C$ that has 
%been decided and is incompatible with the message $m$ supported by $\pi$. \\
Let $\mathcal{A}$ be an efficient adversary that on input $\mathit{pp_{\mathit{LC}}}$ and $\mathcal{R}$ outputs 
 $\pi$, $m$ and $C$. It is easy to see that if the descriptions of $\mathit{LC.DetectMisbehaviour}$ and $\mathit{LC.VerifyMisbehaviour}$ are followed honestly,  
then the predicate $S_{m''} \cap S_{m'} = S$ checked in the end of $\mathit{LC.VerifyMisbehaviour}$ is fulfilled. Moreover, due to the satisfied predicate
$$\mathit{LC.VerifyProof}(\mathit{pp}, \mathit{rs_{\mathit{vk}}}, \LCseed, \pi, m, \mathcal{R}) = 1 \ \ \ \ \ \ (1)$$ %{\color{blue} due to the description of 
%$\mathit{LC.DetectMisbehaviour}$ plus the fact that $C$ is a valid consensus view as per the definition of accountability security challenge, }
it holds that all $m_j, j \in [i]$ (as defined in $\mathit{LC.VerifyProof}$) are decided in $C$. Due to the way $m'$ and $m''$ are computed 
by $\mathit{LC.DetectMisbehaviour}$ from the messages $(m_j)_{j=1}^i$, this implies $|S_{m'}| \geq t$, $|S_{m''}| \geq t$ and $(\mathit{m'}, d_{\mathit{m'}}) \in_{\mathit{decided}} C$ and 
 $$\mathit{Incompatible(m'', m', d_{m'})} = 1.$$ We are only left to show that 
 $$\mathit{AS.Verify}(\mathit{pp}, \mathit{apk}, m'', \sigma) =1 \ \ \ \ (*)$$ 
\noindent holds with overwhelming probability. Indeed, since $(1)$ holds then, for every epoch $j \in [i]$ it holds that 
$$\mathit{CKS_{\mathcal{R}}.Verify}(\mathit{pp},  \mathit{rs_{\mathit{vk}}}, \mathit{com_j}, m_j,  \mathbf{\Sigma}(j), (\mathbf{\pi_{j}},\mathit{apk_j}), 
\mathbf{bit_j}) = 1 \ \ \ \ (2).$$
\noindent In particular, $(2)$ holds for $j = \mathit{index}$. Due to soundness property of the committee key scheme $\mathit{CKS_{\mathcal{R}}}$, 
since $\mathit{com_{\mathit{index}}} = \mathbf{Com}(\mathbf{pk_{\mathbf{index}}})$ by the definition of $\mathit{index}$ and 
$\mathit{LC.DetectMisbehaviour}$, since $$\mathit{apk_{\mathit{index}}} = \mathit{AS.AggKeys}(\mathit{pp}, 
(\mathbf{bit_{\mathbf{index}}}(k) \cdot \mathbf{pk_{\mathbf{index}}}(k))_{k=1}^v)$$ as computed by $\mathit{LC.DetectMisbehaviour}$, 
since also $m'' = m_{\mathit{index}}$ (with $m_j, \forall j \in [i]$ defined in $\mathit{LC.VerifyProof}$ and $\mathit{index}$ defined 
in $\mathit{LC.DetectMisbehaviour}$) and, finally, since $\mathbf{\Sigma}(\mathit{index}) = \sigma$, as defined in $\mathit{LC.DetectMisbehaviour}$, 
it follows that $(*)$ holds with overwhelming probability (q.e.d.). \\
 
\noindent \textit{Accountability Soundness:} Let $\mathcal{A}$ be an efficient adversary who interacts with an honest validator. If \\ $\mathit{LC.VerifyMisbehaviour}(\mathit{pp}, i, S, \mathbf{bit}, \sigma, m'', m', C)$ 
outputs $\mathit{acc}$ $(\ast)$, its checks together with completeness for aggregation imply 
$$\mathit{AS.Verify}(\mathit{pp}, \sigma', m', \mathit{apk_S}) = 1 \ \ (\ast\ast),$$
where 
\begin{equation*}
  \mathbf{\sigma_i}(j) =
    \begin{cases}
      \mathit{sig} & \text{if }  \exists \ \mathit{sig} \in C, \mathit{AS.Verify}(\mathit{pp}, \mathit{sig}, m', \mathbf{pk_i}(j)) \\
      \_ & \text{otherwise} \\
    \end{cases}       
\end{equation*}
\begin{equation*}
  \mathbf{b_{S}}(j) =
    \begin{cases}
      1 & \text{if }  \mathbf{pk_i}(j) \in S \\
      0 & \text{otherwise} \\
    \end{cases}       
\end{equation*}
$$ \sigma' \leftarrow  \mathit{AS.AggSigs}(\mathit{pp}, (\mathbf{b_{S}}(j) \cdot \mathbf{\sigma_i}(j))_{j=1}^v),$$
$$\mathit{apk_S} \leftarrow \mathit{AS.AggKeys}(\mathit{pp},(\mathbf{b_{S}}(j) \cdot \mathbf{pk_i}(j))_{j=1}^v),$$

\noindent Additionally, since $(\ast)$ holds and for $\mathit{apk}$ as defined in $\mathit{LC.VerifyMisbehaviour}$, we obtain
$$\mathit{AS.Verify}(\mathit{pp}, \sigma, m'', \mathit{apk}) = 1 \ \ (\ast \ast').$$
%Note that $m'$, $m''$, $\sigma$ are part of the input of $\mathit{LC.VerifyMisbehaviour}$ and 
%\noindent Note that $(\ast)$ holds due to perfect completeness of aggregation 
%and $(\ast')$ holds as part of the checks that $\mathit{LC.VerifyMisbehaviour}$ outputs $1$. \\
\noindent Since $\mathit{CheckValidConsensus}(C) = 1$ holds and $m'$ has been decided in epoch $i$ of $C$ 
and $d_{m'}$ is the required data associated with $m'$, due to assumptions (I.5.), (I.4.) and (I.2.) we have 
that $d_{m'}$ contains correct proofs of possession for all keys in $\mathbf{pk_i}$ \ $(\ast\ast\ast)$. \\ 
\noindent We assume by contradiction that $S \cap Q_{\mathit{pks}}$ is non-empty with more than negligible probability. 
Since the following check (which is part of $\mathit{LC.VerifyMisbehaviour}$) passes:  
$$S_{m''} \cap S_{m'} =S,$$ any $\mathit{pk} \in S$ is aggregated into $\mathit{apk}$ and also into $\mathit{apk_S}$; 
this includes $\mathit{pk^*}$. Since the aggregate signature instantiation $\mathit{AS}$ is unforgeable 
(see Definition~\ref{def:aggregate_signatures} plus the assumption (S.1.)), due to $(\ast\ast)$, $(\ast\ast')$, $(\ast\ast\ast)$ and 
$(\ast\ast\ast\ast)$ we have that, with more than negligible probability, both $m'$ and $m''$ have 
been signed by the honest validator. However, this contradicts that 
$\mathit{Incompatible}(m', m'', d_{m'}) = 1$ which is ensured by assumption (B.2.). Hence, our assumption is false and $S \cap \mathit{Q_{pks}} = \emptyset$, so the 
probability defined in the accountability soundness property is indeed negligible. 

%If $\mathit{LC.VerifyMisbehaviour}$ outputs $1$, its checks imply that the 
%aggregate signature $\sigma$ (used as input by $\mathit{LC.VerifyMisbehaviour}$) on $m''$ passes the verification of 
%$\mathit{AS.Verify}$; analogously, since $m'$ with required data $d_m'$ was decided in $C$, there is an aggregate 
%signature of at least $t$ individual signatures on $m'$ that pass verification of $\mathit{AS.Verify}$. Moreover, each 
%of the public keys in $S$ sign both $m'$ and $m''$ (relation we denote by $(\ast)$) and 
%$\mathit{Incompatible}(m', m'', d_m') = 1$ and $\mathit{epoch_{id}}(m') = \mathit{epoch_{id}}(m'') = i$ hold (two 
%relations we together denote by $(\ast\ast)$). 
%We assume by contradiction that with more than a negligible probability $S \cap \mathit{Q_{keys}} \neq \emptyset$.  
\end{proof}

%\section{PLONK Compiler for SNARKs}
\label{first_step_compiler}

\noindent We summarise and exemplify below the PLONK-based compilation technique~\cite{plonk} from 
ranged polynomial protocols for conditional NP relations (formal definition in Section~\ref{supplementary_poly_protocols_appendix}) to 
SNARKs for pure vector-based NP relations. This is a more detailed version of the first step of our two-step PLONK based compiler as 
defined in Section~\ref{sec_two_step_compiler}. Concretely, our first step applies the PLONK compiler~\cite{plonk} (Lemma 4.7): 
we compile the information theoretical ranged polynomial protocols $\Pla$ and $\Pa$ for relations $\Rla$ and $\Ra$, respectively (see Sections~\ref{sec_la,sec_a}) 
into SNARKs $\Plastar$, and $\Pastar$, respectively. We can define this compilation step for any ranged polynomial protocols for relations 
(as per Definition~\ref{supplementary_poly_protocols_appendix} in Section~\ref{def_ranged_poly_protocol}). In order to do that we need: 
\begin{itemize}
\item  The batched version of KZG polynomial commitments~\cite{KZG_10} described in Section 3 of PLONK~\cite{plonk}.\footnote{In fact, 
one can replace the use of KZG polynomial commitments with any binding polynomial commitment that has knowledge-soundness, including non-homomorphic polynomial commitments, 
such as FRI-based polynomial commitments (e.g., RedShift~\cite{redshift}). If the optimisation gained from PLONK linearisation technique is a goal, 
then, with minimal changes one can use any homomorphic polynomial commitment, e.g., the discrete logarithm based polynomial commitment 
from Halo~\cite{halo}.}
\item A general compilation technique: such a technique has been already defined in Lemma 4.7 of PLONK; combined with Lemma 4.5 
from PLONK this technique can be applied with minor adaptations (this includes the corresponding technical measures) to the notion of ranged 
polynomial protocols.  
\item So far, both the ranged polynomial protocols for relations and the protocols resulted after the first compilation step have been explicitly defined as interactive 
protocols. In order to obtain the non-interactive version of the latter (essentially the N in SNARK) one has to apply the Fiat-Shamir 
transform~\cite{FS_transform}, \cite{FS_transform_with_proof}, \cite{SE_plonk}.
\end{itemize}

\noindent Let $\mathcal{R}$ be a (conditional) NP relation, let $\mathscr{P}_{\mathcal{R}}$ be a ranged polynomial protocol for 
relation $\mathcal{R}$ and let $\mathscr{P}^*_{\mathcal{R}}$ be the SNARK compiled from $\mathscr{P}_{\mathcal{R}}$ using the PLONK compiler.  
The compilation technique requires the SNARK prover of  $\mathscr{P}^*_{\mathcal{R}}$ to compute 
polynomial commitments to all polynomials that the prover $\mathcal{P}_{poly}$ in $\mathscr{P}_{\mathcal{R}}$ sent to the ideal party $\mathcal{I}$. Analogously, 
it requires the SNARK verifier of $\mathscr{P}^*_{\mathcal{R}}$ to compute polynomial commitments to all pre-processed polynomials\footnote{This is a one-time computation that is 
reused by the SNARK verifier for all SNARK proofs over the same circuit.} as well polynomial commitments to polynomials the verifier $\mathcal{V}_{poly}$ 
in $\mathscr{P}_{\mathcal{R}}$ sent to the ideal party $\mathcal{I}$. Then, the SNARK prover sends the SNARK verifier openings to 
all the polynomial commitments computed by him as well as the polynomial commitments computed by the SNARK verifier. The SNARK 
prover additionally sends the corresponding batched proofs for polynomial commitment openings. In turn, the SNARK verifier accepts or rejects based 
on the result of the verification of the batched polynomial commitment scheme. \\

\noindent A more efficient compilation technique exists which reduces the number of polynomial commitments and alleged polynomial commitments openings 
(i.e., both group elements and field elements) sent by the SNARK prover to the SNARK verifier; this, in turn, reduces the size of the SNARK proof. 
This technique is called linearisation and is described, at a high level, after Lemma 4.7 in PLONK. The existing description however covers only the 
SNARK prover side and it does not detail the SNARK verifier side so in the following we cover that. \\

\noindent By functionality, the vectors that are handled by the the verifier $\mathcal{V}_{poly}$ are 
of two types: pre-processed vectors and public input vectors. These two types of vectors are used by $\mathcal{V}_{poly}$ 
to obtain, via interpolation over the range on which the respective range polynomial protocol is defined, pre-processed polynomials 
(as used in the Definition~\ref{supplementary_poly_protocols_appendix} in Section~\ref{def_ranged_poly_protocol}, e.g., polynomial $aux(X)$ used in Section~\ref{sec_la}) and 
public-inputs-derived polynomials (e.g., polynomials $pkx(X)$ and $pky(X)$ used in Sections~\ref{sec_la,sec_a})
and polynomial $b(X)$ used in Section~\ref{sec_la}). The efficient linearisation technique allows the SNARK verifier to reduce the 
number of polynomial commitments it has to compute compared to the general PLONK compiler in the following way. Instead of 
having to compute polynomial commitments to all polynomials $\mathcal{V}_{poly}$ sends to $\mathcal{I}$ (including any corresponding 
pre-processed polynomials), the SNARK verifier computes polynomial evaluations at one or multiple random points (as per the linearisation 
step specific requirements) for all the polynomials that are either easy to evaluate (e.g., polynomial $aux(X)$ used in Section~\ref{sec_a}) or 
all the polynomials that are obtained from vectors that do not take up a large amount of memory (e.g., polynomial $b(X)$ used in Section~\ref{sec_la}). 
For the rest of the polynomials (e.g., $\mathit{pkx}(X)$ and $\mathit{pky}(X)$), the SNARK verifier computes polynomial commitments as before.\\

\noindent We note we can apply all the techniques mentioned above, including the combined prover-and-verifier-side linearisation 
to compile our ranged polynomial protocols $\Pla$ and $\Pa$ into the corresponding SNARKs $\Plastar$ and $\Pastar$, respectively. 
To conclude this step, we formally state in Section~\ref{def_ranged_poly_protocol}, Lemma~\ref{le:compilation_step_1} under which conditions and how efficiently 
one can compile ranged polynomial protocols for conditional NP relations (where the public inputs are interpreted as vector of field elements) 
into hybrid model SNARKs by using only the original PLONK compiler. \\

\section{Appendix B - Rolled out Protocol $\Pah$ for Conditional NP Relation $\Racom$}
\label{sec:rolled_out}

\noindent  We give below the full rolled-out hybrid model protocol $\Pah$ for conditional NP relation $\Racom$. This is obtained by applying 
our two-steps compiler from section~\ref{sec_two_step_compiler} to polynomial protocol $\Pa$. In order to obtain the non-interactive version 
(i.e., the N from SNARK) we have additionally applied the Fiat-Shamir transform. In the following, by \textbf{transcript} at a certain point in time we denote the concatenation 
of the global constant, verification key, trusted public input, other public input and the proof elements created by the prover up to that point in time.
\noindent $\mathcal{H}$ is a hash function, $\mathcal{H}: \{0, 1\} \rightarrow \mathbb{F}$ and it emulates the random oracle.  
In the following, $\oplus$ is the addition operation on $\einn$ in affine coordinates. Note that in our implementation we instantiate 
$\einn$ with BLS12-377~\cite{zexe} and $\eout$ with BW6-761~\cite{BW6}, while we choose $\block$ to be $256$ as this is the largest power of 2 smaller 
than the size of a field element in $\mathbb{F}$ (i.e., the base field for BLS12-377 which is the same as the scalar field of BW6-761). 
Finally, $n$ has been defined as per section~\ref{sec:lagrange}, i.e., $n$ is a large enough power of 2; 
moreover, we let $v= n-1$ and we let $N = n$. This, in turn, ensures that $N$ has been chosen according to the properties stated in instantiation~\ref{insta:bls}, in particular 
when defining $\mathit{AS.Setup}$. \\

\noindent \textbf{Public Parameters:} \\
$(\ginn{1}, \sginn{1}, \ginn{2}, \sginn{2}, \gtinn, \epinn, \Hinn, \HPoP)$ from $\mathit{pp}$ with $\mathit{pp} \leftarrow \mathit{AS.Setup}(\mathit{aux_{\mathit{AS}}}= n)$ \\

\noindent \textbf{Global constant:} $h \in \einn \setminus \ginn{1}$ \\ 

\noindent \textbf{Trusted Setup:} $\mathit{srs} \leftarrow \mathit{SNARK.Setup}(\mathit{aux_{\mathit{SNARK}}} = (n, 3n-3))$,\\ 
where $\mathit{srs} =([1]_{\indexoneout}, [\tau]_{\indexoneout}, [\tau^2]_{\indexoneout}, \ldots, [\tau^{3n-3}]_{\indexoneout}, [1]_{\indextwoout}, [\tau]_{\indextwoout})$ \\

\noindent \textbf{Proving and Verifying Key Generation:} $(\mathit{srs}_{\mathit{pk}}, \mathit{srs}_{\mathit{vk}}) \leftarrow \mathit{SNARK.KeyGen}(\mathit{srs}, \Racom)$, \\
where $(\mathit{srs}_{\mathit{pk}}, \mathit{srs}_{\mathit{vk}}) = 
(([1]_{\indexoneout}, [\tau]_{\indexoneout}, [\tau^2]_{\indexoneout}, \ldots, [\tau^{3n-3}]_{\indexoneout}), ([1]_{\indexoneout}, [1]_{\indextwoout}, [\tau]_{\indextwoout}))$ \\

\noindent For $(\mathit{pk_0}, \ldots, \mathit{pk_{n-2}})$ part of $\mathit{state_1}$, we define 
\textbf{Partial Input:} $(x_1, \mathit{state_2}) \leftarrow \mathit{SNARK.PartInput}(\mathit{srs}, \mathit{state_1}, \Racom)$, \\
where if $(\mathit{pk_0}, \ldots, \mathit{pk_{n-2}})) \notin \ginn{1}^{n-1}$, $\mathit{SNARK.PartInput}(\mathit{srs}, \mathit{state_1}, \Racom)$ outputs the empty string, \\ otherwise 
 $\mathit{SNARK.PartInput}$ outputs $x_1 = ([pkx]_{\indexoneout}$, $[pky]_{\indexoneout})$ and $\mathit{state_2} = \mathit{state_1} \cup \{ x_1\}$, where $\forall i \in \{0, \ldots, n-2\}$, $\mathit{pk_i}$ as an element of the curve $\einn$ 
has the affine representation $(\mathit{pkx_i}, \mathit{pky_i})$. The polynomials $pkx(X)$ and $pky(X)$ are computed as $pkx(X) = \sum_{i=0}^{n-2} \mathit{pkx_i} \cdot L_i(X)$ 
and $pky(X) = \sum_{i=0}^{n-2} \mathit{pky_i} \cdot L_i(X)$ and finally, the polynomial commitments are computed as 
$[pkx]_{\indexoneout} = pkx(\tau) \cdot [1]_{\indexoneout}$ and $[pky]_{\indexoneout} = pky(\tau) \cdot [1]_{\indexoneout}$.\\

\noindent \textbf{Public input:} $x_1 = ([pkx]_{\indexoneout}, [pky]_{\indexoneout})$, $x_2 = ((\mathit{b'_{0}}, \ldots, \mathit{b'_{\frac{n}{\block}-1}}), \mathit{apk})$\\

\noindent \textbf{Witness:}
$w = ((\mathit{pk_0}, \ldots, \mathit{pk_{n-2}})$, $(\mathit{bit}_0, \ldots, \mathit{bit_{n-1}}))$ \\

\noindent \textbf{Prover's Algorithm:} $ \pi \leftarrow \mathit{SNARK.Prove}(\mathit{srs_{pk}}, ((x_1, x_2), w), \Racom)$, where\\

\noindent \textbf{Step 1:} \\
\noindent Compute the affine representation $h = (h_x, h_y)$ and  $apk \oplus h = ((apk \oplus h)_{x}, (apk \oplus h)_{y})$. \\

\noindent Compute $\mathbf{pkx} = (\mathit{pkx_0}, \ldots, \mathit{pkx_{n-2}})$ and $\mathbf{pky} = (\mathit{pky_0}, \ldots, \mathit{pky_{n-2}})$ s.\ t.\  
$\forall i \in \{0, \ldots, n-2\}$, $\mathit{pk_i}$ as an element of the curve $\einn$ has the affine representation $(\mathit{pkx_i}, \mathit{pky_i})$. \\

\noindent Let $(kaccx_{0}, kaccy_{0}) = (h_x, h_y)$ and 
compute $(kaccx_{i+1}, kaccy_{i+1}) =  (kaccx_{i}, kaccy_{i}) \oplus \mathit{bit_i}(pkx_{i}, pky_{i})$, $\forall i < n-1$. \\ 

\noindent Compute polynomials 

$$b(X) = \sum_{i=0}^{n-1} \mathit{bit_i} \cdot L_i(X),$$

$$kaccx(X) = \sum_{i=0}^{n-1} kaccx_i \cdot L_i(X),$$

$$kaccy(X) = \sum_{i=0}^{n-1} kaccy_i \cdot L_i(X),$$

$$pkx(X) = \sum_{i=0}^{n-2} pkx_i \cdot L_i(X),$$

$$pky(X) = \sum_{i=0}^{n-2} pky_i \cdot L_i(X).$$

\noindent Compute $[b]_{\indexoneout} = b(\tau)\cdot[1]_{\indexoneout}$, $[kaccx]_{\indexoneout} = kaccx(\tau) \cdot [1]_{\indexoneout}$, $[kaccy]_{\indexoneout} = kaccy(\tau)\cdot [1]_{\indexoneout}$. \\

\noindent The first output of the prover is ($[b]_{\indexoneout}$, $[kaccx]_{\indexoneout}$, $[kaccy]_{\indexoneout}$). \\

%\noindent Note that if we assume the commitments $[pkx]_{\indexoneout}$ and $[pky]_{\indexoneout}$ are generated by a trusted third party 
%(or, in our case, by a set of trusted validators), then it should hold that $[pkx]_{\indexoneout} = pkx(\tau)\cdot [1]_{\indexoneout}$ and 
%$[pky]_{\indexoneout} = pky(\tau)\cdot [1]_{\indexoneout}.$

\noindent \textbf{Step 2:} \\
\noindent Compute the sum challenge $r = \mathcal{H}(\mathbf{transcript})$. \\

\noindent Compute $sum = \sum_{j=0}^{\frac{n}{\block}-1} b'_{j} r^j$.\\

\noindent Compute: $\frac{r}{2^{\block -1}}, r^{\frac{n}{\block}}$. \\

\noindent Compute polynomials 

$$c(X) = \sum_{i=0}^{n-1} c_i \cdot L_{i}(X),$$ 
where $c_i =  2^{i\mod \block}\cdot r^{i\div \block}, 0 \leq i \leq n-1 $. 

$$acc(X) = \sum_{i=0}^{n-1} acc_i \cdot L_{i}(X),$$ 
where $acc_0 = 0$ and $acc_i = \sum_{j=0}^{i-1} \mathit{bit_j} \cdot c_j, 0 < i \leq n-1$. 
$$aux(X) = \sum_{i=0}^{n-1} aux_i \cdot L_{i}(X),$$
where $aux_{i} = 1$ if $i$ is divisible with $\block$ and $aux_{i} = 0$ otherwise, $\forall i < n$ \\

\noindent Compute $[c]_{\indexoneout} = c(\tau) \cdot [1]_{\indexoneout}$, $[acc]_{\indexoneout} = acc(\tau) \cdot [1]_{\indexoneout}$. \\

\noindent The second output of the prover is $([c]_{\indexoneout}, [acc]_{\indexoneout})$. \\

\noindent \textbf{Step 3:} \\
\noindent Compute the quotient challenge $\alpha = \mathcal{H}(\mathbf{transcript})$. \\

\noindent Compute the polynomial $t(X)$ of degree at most $3\cdot n - 3$  where 
%\begin{align*}
%t(X)(X^n-1)  = \ & a_1(X)(X-\omega^{n-1})+\alpha a_2(X)(X-\omega^{n-1}) +\alpha^2 a_3(X)  \\
%& +\alpha^3 a_4(X) +\alpha^4 a_5(X)+\alpha^5 a_6(X)+\alpha^6 a_7(X) \; .
%\end{align*}
%\noindent or, if we expand all the constraints defined by the polynomials $a_1(X), \ldots, a_7(X)$, we have:
\begin{align*}
&t(X)(X^n-1)  = \\  
&(X-\omega^{n-1}) \cdot [b(X) \cdot ((kaccx(X)-pkx(X))^2 \cdot (kaccx(X)+ pkx(X) + kaccx(\omega\cdot X)) - (pky(X)-kaccy(X))^2) + \\ 
& + (1-b(X))\cdot (kaccy(\omega\cdot X) - kaccy(X))] + \\
& +\alpha (X-\omega^{n-1})\cdot [b(X) \cdot ((kaccx(X) - pkx(X)) \cdot (kaccy(\omega \cdot X) + kaccy(X)) - (pky(X) - kaccy(X)) \cdot \\
&\cdot (kaccx(\omega \cdot X) - kaccx(X))) + (1-b(X)) \cdot (kaccx(\omega \cdot X) - kaccx(X))] + \\
& +\alpha^2 \cdot [b(X) \cdot (1-b(X))] + \\
& +\alpha^3 \cdot [c(\omega \cdot X) - c(X)\cdot (2+ (\frac{r}{2^{\block -1}} -2) \cdot aux(\omega \cdot X))- (1 - r^{\frac{n}{\block}}) \cdot L_{n-1}(X)] + \\ 
& +\alpha^4 \cdot [(kaccx(X) - h_x)\cdot L_0(X) + (kaccx(X) - (h + apk)_{x}) \cdot L_{n-1}(X)] + \\ 
& +\alpha^5 \cdot [(kaccy(X) - h_y)\cdot L_0(X) + (kaccy(X) - (h + apk)_{y}) \cdot L_{n-1}(X)] + \\
& +\alpha^6 \cdot [ acc(\omega \cdot X) - acc(X) - b(X)\cdot c(X) + \mathit{sum} \cdot L_{n-1}(X)] \; .
\end{align*}

\noindent Compute $[t]_{\indexoneout} = t(\tau) \cdot [1]_{\indexoneout}$.  \\

\noindent The third output of the prover is $[t]_{\indexoneout}$. \\

\noindent \textbf{Step 4:} \\
\noindent Compute evaluation challenge $\zeta = \mathcal{H}(\mathbf{transcript})$. \\

\noindent Compute evaluations: 
$\overline{pkx} = pkx(\zeta)$, $\overline{pky}=pky(\zeta)$, $ \overline{b}=b(\zeta)$, $\overline{kaccx}=kaccx(\zeta)$, $\overline{kaccy}=kaccy(\zeta)$, 
$\overline{c}=c(\zeta)$, $\overline{acc}=acc(\zeta)$, $\overline{t}=t(\zeta)$. \\

\noindent Compute linearisation polynomial: 
\begin{align*}
r(X) = (\zeta - \omega^{n-1}) \cdot &[\bar{b} \cdot (\overline{kaccx} - \overline{pkx})^2 \cdot kaccx( X) + (1 - \bar{b})\cdot kaccy(X)]+ \\
&+ \alpha \cdot (\zeta - \omega^{n-1}) \cdot [\bar{b} \cdot ((\overline{kaccx} - \overline{pkx}) \cdot kaccy(X) - (\overline{pky} - \overline{kaccy}) \cdot kaccx(X)) + (1 - \bar{b}) \cdot kaccx(X)]+ \\
&+\alpha^3 \cdot c(X)+ \\
&+\alpha^6 \cdot acc(X).
\end{align*}

\noindent Compute evaluation of linearisation polynomial $\overline{r_{\omega}} = r(\omega \cdot \zeta)$. \\

\noindent The fourth output of the prover is $(\overline{pkx}, \overline{pky}, \overline{b}, \overline{kaccx}, \overline{kaccy}, \overline{c}, \overline{acc},\overline{r_{\omega}})$. \\

\noindent \textbf{Step 5:} \\
\noindent Compute opening challenge $\nu = \mathcal{H}(\mathbf{transcript})$.  \\

\noindent Compute first opening proof polynomial

\begin{align*} 
W_{\zeta}(X) = \frac{1}{X-\zeta}&(t(X) - \bar{t}+ \\ 
&+ \nu(pkx(X) - \overline{pkx}) +\\
&+  \nu^2(\cdot pky(X) - \overline{pky}) + \\ 
&+ \nu^3 (b(X) - \bar{b}) + \\
&+ \nu^4( kaccx(X) - \overline{kaccx}) + \\  
&+ \nu^5(kaccy(X) - \overline{kaccy}) +  \\ 
&+ \nu^6 (c(X) -\bar{c}) + \\ 
&+ \nu^7 (acc(X) - \overline{acc}))
\end{align*}

\noindent and second opening proof polynomial

\begin{align*}
W_{\zeta \cdot \omega}(X) = \frac{1}{X-\zeta \cdot \omega}&(r(X) - \overline{r_{\omega}}).
\end{align*}

\noindent Compute $[W_{\zeta}]_{\indexoneout} = W_{\zeta}(\tau) \cdot [1]_{\indexoneout}$ and $[W_{\zeta \cdot \omega}]_{\indexoneout} = W_{\zeta \cdot \omega}(\tau) \cdot [1]_{\indexoneout}.$ \\

\noindent The fifth output of the prover is $([W_{\zeta}]_{\indexoneout}, [W_{\zeta \cdot \omega}]_{\indexoneout})$. \\

\noindent Compute the multipoint evaluation challenge $u = \mathcal{H}(transcript)$. \\

\noindent Return $\pi$ = ($[b]_{\indexoneout}$, $[kaccx]_{\indexoneout}$, $[kaccy]_{\indexoneout}$, $[c]_{\indexoneout}$, $[acc]_{\indexoneout}$, $[t]_{\indexoneout}$, $[W_{\zeta}]_{\indexoneout}$, 
$[W_{\zeta \cdot \omega}]_{\indexoneout}$,  $\overline{pkx}$, $\overline{pky}$, $\overline{b}$, $\overline{kaccx}$, $\overline{kaccy}$, $\overline{c}$, 
$\overline{acc}$, $\overline{r}_{\omega}$) \\  

\noindent \textbf{Verifier's Algorithm:} $0/1 \leftarrow \mathit{SNARK.Verify}(\mathit{srs_{vk}}, (x_1, x_2), \pi, \Racom)$, where\\ 

\noindent \textbf{Step 1:} \\
\noindent Compute the affine representation $h = (h_x, h_y)$ and  $apk \oplus h = ((apk \oplus h)_{x}, (apk \oplus h)_{y})$.\\

\noindent \textbf{Step 2:} \\
\noindent Validate proof elements ($[b]_{\indexoneout}$, $[kaccx]_{\indexoneout}$, $[kaccy]_{\indexoneout}$, $[c]_{\indexoneout}$, $[acc]_{\indexoneout}$, $[t]_{\indexoneout}$, 
$[W_{\zeta}]_{\indexoneout}$, $[W_{\zeta \cdot \omega}]_{\indexoneout}$) $ \in \gout{1}^{8}$. \\ 

\noindent \textbf{Step 3:} \\
\noindent Validate proof elements ($\overline{pkx}$, $\overline{pky}$, $\overline{b}$, $\overline{kaccx}$, 
$\overline{kaccy}$, $\overline{c}$, $\overline{acc}$, $\overline{r}_{\omega}$) $\in \mathbb{F}^{8}$. \\

\noindent \textbf{Step 4:} \\
\noindent Compute challenges $(r, \alpha, \zeta, \nu, u)$ as in the prover $P_{\mathit{pa, com}}^{\mathit{SNARK}}$ description from the common input, trusted public input, public input and respective necessary parts of the $\mathbf{transcript}$ using elements of $\pi_{\mathit{pa}}$. \\

\noindent \textbf{Step 5:} \\
\noindent Compute: $sum = \sum_{j=0}^{\frac{n}{\block}-1} b'_{j} r^j$. \\

\noindent Compute: $\frac{r}{2^{\block -1}}, r^{\frac{n}{\block}}$. \\

\noindent \textbf{Step 6:} \\
\noindent Compute polynomial evaluations $\zeta^{n} -1$ and $\overline{aux}_{\omega} = aux(\omega \cdot \zeta)$\footnote{We have $aux(\omega \cdot \zeta)= 1$ if $(\omega\cdot \zeta)^{\frac{n}{\block}} =1$ and $aux(\omega \cdot \zeta) = \frac{1}{\block} \cdot \frac{\zeta^n -1}{(\omega\cdot\zeta)^{\frac{n}{\block}} -1}$ otherwise.} and Lagrange basis polynomials 
$L_0(\zeta)= \frac{\zeta^n - 1}{n \cdot (\zeta-1)}$ and $L_{n-1}(\zeta)= \frac{(\zeta^n - 1) \cdot \omega^{n-1}}{n \cdot (\zeta - \omega^{n-1})}$. \\

\noindent \textbf{Step 7\footnote{This step can be optimised in obvious ways in order to reduce the number of field operations necessary to compute $\bar{t}$. We choose to include the non-compact formula in this write-up such that the reader is able to follow the linearisation process to a larger extent than via a more compact formula.}:} \\
\noindent Compute quotient polynomial evaluation $$\bar{t} = 
\frac{\overline{r_{\omega}} + [\bar{b}((\overline{kaccx} - \overline{pkx})^2 \cdot (\overline{kaccx} + \overline{pkx})- (\overline{pky} - \overline{kaccy})^2) - (1-\bar{b})\cdot \overline{kaccy}]\cdot (\zeta - \omega^{n-1})}{\zeta^{n} - 1} +$$

$$+ \frac{\alpha \cdot [\bar{b} \cdot ((\overline{kaccx} - \overline{pkx}) \cdot \overline{kaccy} + (\overline{pky} - \overline{kaccy}) \cdot \overline{kaccx}) - (1 - \bar{b}) \cdot \overline{kaccx}] \cdot (\zeta - \omega^{n-1})}{\zeta^{n} - 1}+$$

$$+\frac{\alpha^2 \cdot \bar{b} \cdot (1 - \bar{b})}{\zeta^{n} - 1} +$$

$$-\frac{\alpha^3 \cdot[(1 - r^{\frac{n}{\block}}) \cdot L_{n-1}(\zeta)]}{\zeta^{n}-1}  \ - \alpha^3 \cdot  \bar{c} \cdot (2+ (\frac{r}{2^{\block -1 }} - 2)) \cdot \overline{aux}_{\omega} + $$

$$+\frac{\alpha^4 \cdot [(\overline{kaccx} - h_x) \cdot L_0(\zeta) + (\overline{kaccx} - (h + apk)_x) \cdot L_{n-1}(\zeta)]}{\zeta^{n} - 1} +$$

$$+\frac{\alpha^5 \cdot[(\overline{kaccy} - h_y) \cdot L_0(\zeta) + (\overline{kaccy} - (h + apk)_y) \cdot L_{n-1}(\zeta)]}{\zeta^{n} - 1} +$$

$$+\frac{\alpha^6 \cdot [- \overline{acc} - \bar{b} \cdot \bar{c} + \mathit{sum} \cdot L_{n-1}(\zeta)]}{\zeta^{n} - 1}.$$

\noindent \textbf{Step 8:} \\
\noindent Compute full batched polynomial commitment $[F]_{\indexoneout}$. 
\begin{align*}
[F]_{\indexoneout} =&  [t]_{\indexoneout} + \nu \cdot [pkx]_{\indexoneout} + \nu^2 \cdot [pky]_{\indexoneout} + \nu^3 \cdot [b]_{\indexoneout} \ + \\
& + (u \cdot (\zeta - \omega^{n-1})\cdot (\bar{b} \cdot ((\overline{kaccx} - \overline{pkx})^2 + \alpha \cdot (\overline{pky} - \overline{kaccy}))+\alpha \cdot (1 - \bar{b})) + \nu^4) \cdot [kaccx]_{\indexoneout} \ + \\  
& +(u \cdot (\zeta - \omega^{n-1})(\alpha \cdot \bar{b}(\overline{kaccx} - \overline{pkx})+ (1- \bar{b})) + \nu^5) \cdot [kaccy]_{\indexoneout} \ + \\  
& +(u\cdot \alpha^3+ \nu^6) \cdot [c]_{\indexoneout} \ + \\  
& +(u\cdot \alpha^6+ \nu^7)\cdot [acc]_{\indexoneout}. \ \\ 
\end{align*}

\noindent \textbf{Step 9:} \\
\noindent Compute group-encoded batch evaluation $[E]_{\indexoneout}$ 
$$[E]_{\indexoneout} = (\bar{t} + \nu \cdot \overline{pkx} + \nu^2 \cdot \overline{pky} + \nu^3 \cdot \bar{b} +
\nu^4 \cdot \overline{kaccx} + \nu^5 \cdot \overline{kaccy} + \nu^6 \cdot \bar{c} + \nu^7 \cdot \overline{acc} + u \cdot  \overline{r_{w}}) \cdot [1]_{\indexoneout}$$

\noindent \textbf{Step 10:} \\
\noindent Batch validate all evaluations by checking that the following holds \\
$$\epout([W_{\zeta}]_{\indexoneout} + u \cdot [W_{\zeta \cdot \omega}]_{\indexoneout},[\tau]_{\indextwoout}) = \epout(\zeta \cdot [W_{\zeta}]_{\indexoneout} + u \cdot \zeta \cdot \omega \cdot [W_{\zeta \cdot \omega}]_{\indexoneout} + [F]_{\indexoneout} - [E]_{\indexoneout}, [1]_{\indextwoout}).$$
\section{Is our protocol applicable to Ethereum?} \label{sec:ethereum}

We believe that our protocol could not feasibly be directly applied to Ethereum as it is without a hard fork, but it would be easy to apply it with changes that might feasibly be implemented, even with changes not specifically designed with our protocol in mind.
 Ethereum already uses BLS signatures on the BLS12-381 curve in consensus. To work with BLS12-381, our protocol would use KZG commitments on the BW6-767 curve\cite{bw6767}.
 Because the base field of BL12-381 is not highly 2-adic, a prover would need a more complicated FFT algorithm but this is feasible \cite{bw6767}. We also need an appropriately sized subgroup of the multiplicative field to use for our Lagrange basis commitments. An easy calculation gives the small prime factors $2,3^2,11,23,47,10177$ and 859267 for the order of the multiplicative group in BLS12-381 and any product of these larger than the number of validators gives a usable subgroup.
 
 Next we consider who constructs the KZG commitment to the validators public keys. For the shortest light client proofs validators would construct and sign this commitment, which would require a change to the consensus logic. As an alternative, a smart contract could compute the commitments on chain.
 This requires the EVM to have access to the active validator's public keys and would also require a precompile for BW6-767 operations to be feasible.
 We note that there have been many suggestions for adding elliptic curve operations for different curves to Ethereum (e.g. EIPs 2537,2538,3026[\cite{EIPs}) but few have been implemented so far.
 We would expect this to be the bottleneck for implementing our protocol on Ethereum.
 
 Finally we compare running our scheme on the full validator set to Ethereum's current light client design\cite{ethlight}.
 That uses s subset of 1024 public keys that changes every epoch (i.e. 64 blocks or 12.8 minutes). It is not accountable because it would take less than 1024 validators misbehaving to deceive a light client. We remark that 1024 384-bit public keys is comparable in size to 1 bit for all of Ethereum's over 500000 validators, and as a result our light client scheme can be used for an accountable light client with a similar overhead to Ethereum's existing unaccountable scheme.

 With epochless Casper FFG \cite{Gasper}, 64 aggregated signatures are required to represent a single Casper FFG vote, 2 of which are required for a proof. The form of accountable safety satisfied by Casper FFG \cite{CasperFFG} suffices for us: if two forks are finalised, then then 2/3 of the validator set voted on two messages such that signing both is punishable.  One complication of a Casper FFG light client is that valid votes include two blockhashes, one of which is required to be the ancestor of the other. There needs to be an efficient "proof of ancestry" such as introducing a more efficient commitment to previous block hashes, e.g. Merkle Mountain Range of blockhashes as suggested in \cite{flyclient}.


\end{document}
