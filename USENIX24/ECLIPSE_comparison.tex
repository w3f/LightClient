ECLIPSE~\cite{eclipse} presents a compiler that starts off with popular SNARKs (e.g., Sonic, PLONK, 
Marlin) and via a new general compilation method generates CP-variants for these SNARKs. 
Our proposed compiler uses as a first step the standard PLONK compiler (with only very mild 
adaptations to incorporate our extended definition of ranged polynomial protocols). 
As a second step we simply re-cast the SNARK resulted in the first step as a SNARK for a new relation. 
The security of the recasting holds under mild conditions that deterministically relate some polynomials 
processed by the verifier in the ranged polynomial protocol (before applying PLONK compiler) to some 
public inputs. To our knowledge, our re-casting conditions are less stringent than the conditions needed 
by the ECLIPSE compiler to work.

We cannot use the ECLIPSE compilation technique either in full or in part to compile our custom SNARKs 
in this work since the types of NP relations derived after compilation are simply incompatible. While in the 
case of ECLIPSE, the witnesses for the NP relations before compilation remain witnesses also for the relations 
after compilation, in our case, some part of the public input before compilation becomes witness after the 
re-casting of the SNARK for a new NP relation. So overall we solve a related but different problem to the one 
solved by ECLIPSE. Finally, our compilation method requires only the PLONK compiler without additional computational 
steps so it is more efficient than the one in~\ref{eclipse}. 

  
%Short conclusion: our compiler work in LC paper is not implied by ECLIPSE compiler; in fact, what I have called "second %compilation step" in our case is a misnomer. All we do in the second compilation step is to re-cast the SNARK already obtained %as a result of the standard PLONK compilation as a SNARK for a new relation. This holds under some mild conditions that %deterministically relate some polynomials processed by the verifier to some public inputs (in the polynomial protocol).  NOTE: %our conditions for re-casting are, I believe, less stringent than the conditions needed by the ECLIPSE compiler to work.


%Moreover, the reason why the ECLIPSE compiler does not give as outcome the SNARKs we use in our LC paper is because %ECLIPSE's witnesses are the same in both relations before and after compilation. Their second relation just adds explicit %commitments (to some part of the witness) as public inputs. In our case, some part of the public input before compilation %becomes witness after the re-casting of the SNARK for a new relation. We also add, of course, commitments to this witness as %part of the public input in the second relation (after re-casting). So overall we solve related but different problem to the one in %ECLIPSE
