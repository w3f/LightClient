\label{sec:snarks}

In the following, we construct two related SNARKS, each of them allowing a prover to convince an 
efficient verifier that an alleged aggregated public key has indeed been computed correctly as an aggregate 
of a vector of public keys for which two succinct commitments (one to the vector of x affine coordinates and the other 
to the vector of y affine coordinates, respectively) are publicly known. The differences between the two  
constructions stem from how a \emph{bitmask} (also called a \emph{bitvector}) with one bit associated to each public key 
(necessary to signal the inclusion or omission of the respective public key w.r.t. the aggregate key) 
is used as part of the verifier's public input. We describe a 
\emph{basic accountable SNARK} (the bitmask is represented as a sequence of $0/1$ field elements) and a \emph{packed accountable SNARK} (the bitmask is 
partitioned into equal blocks of consecutive binary bits, and, in turn, each block is represented as a field element). 
%Each of our SNARKs implements a conditional NP relation bearing the 
%same name as the SNARK it implements. Note that the names ``basic accountable" (for short, ``basic'') and 
%``packed accountable" (for short, ``packed'') do not refer to the security of the respective SNARK but they summarise properties 
%of the underlying sets of constraints that define the SNARKs, and, hence their use case. 
We finally transform basic and packed accountable SNARKs into SNARKs for building accountable light client systems. \\

\noindent In order to compile our desired SNARKs we proceed as follows:
\begin{itemize}
\item In Sections \ref{sec_la} and \ref{sec_a} we define vector-based conditional NP relations $\Rla$ (i.e., basic accountable) and $\Ra$ (packed accountable).  
\item Correspondingly, we design two ranged polynomial protocols for the above relations. The ranged polynomial protocol notion originates in~\cite{plonk}; 
for convenience, we remind it to the reader (including a refinement) in Section~\ref{supplementary_poly_protocols_appendix};  
\item In Section~\ref{sec_two_step_compiler} we define a two-steps PLONK-inspired compiler which we use to compile the two ranged polynomial protocols into 
SNARKs for two novel mixed vector and trusted polynomial commitments conditional NP relations which we denote by 
$\Rlacom$ and $\Racom$, respectively. 
\item In Section~\ref{sec:inst_committee_key} we give an instantiation for committee key scheme for aggregatable signatures which uses, in turn, our SNARKS 
compiled in Section~\ref{sec_two_step_compiler} and our instantiation for BLS aggregatable signatures from Section~\ref{sec:bls}. 
\item We include in Section~\ref{suplementary_plonk_comparison} a comparison between PLONK~\cite{plonk} and our custom SNARKs. 
\end{itemize}

\noindent In more detail, we define our conditional NP relations over $\mathbb{F}$, i.e., the base field of $\einn$. 
Our SNARKs provers' circuits are defined as well over $\mathbb{F}$ as the scalar field of $\eout$. The vector of public keys, which is part of the public input for both of our 
relations $\Rla$ and $\Ra$, and is denoted by $\mathbf{pk} = (\mathit{pk_0}, \ldots, \mathit{pk_{n-2}})$, is a vector of pairs with each component 
in $\mathbb{F}$. This vector has size $n-1$ ($n$ defined in 
Section~\ref{sec:lagrange}). For $\Rla$ we denote 
the $n$ components bitmask by $\mathbf{bit} = (\mathit{bit_0}, \ldots, \mathit{bit_{n-1}})$ 
(meaning that each component belongs to the set $\{0,1\} \subset \mathbb{F}$), 
while the $\Rla$ relation is defined using the \emph{compacted bitmask} 
$\mathbf{b'} = (\mathit{b'_{0}}, \ldots, \mathit{b'_{\frac{n}{\block}-1}})$ of $\frac{n}{\block}$ field elements, 
each of which is $\block$ binary bits long ($\block$ has been defined in Section~\ref{sec:lagrange}). 
Each of the bits in the bit representation of these field elements signals the 
inclusion (or exclusion) of the index-wise corresponding public keys into the aggregated public key $\mathit{apk}$. In fact, 
the last bit of field element $\mathit{b'_{\frac{n}{\block}-1}}$ as well as the $n$-th component $\mathit{bit_{n-1}}$ do not correspond to any public key, 
but, as will become clear in the following, they have been included for easier design of constraints. \\ 

\noindent We denote by $H$ the multiplicative subgroup of $\mathbb{F}$ generated 
by $\omega$ as defined in Section~\ref{sec:lagrange}. We denote by  $\mathit{incl}(a_0, \ldots, a_{n-2})$ the inclusion 
predicate that checks if $(a_0, \ldots, a_{n-2}) \in \ginn{1}^{n-1}$. Moreover let $h = (\mathit{h_x}, \mathit{h_y})$ 
be some fixed, publicly known element in $\einn \setminus \ginn{1}$. We denote by $(a_x, a_y)$ the affine representation in 
$x$ and $y$ coordinates of $a \in \einn$ and by $\oplus$ the point addition in affine coordinates on the elliptic curve $\einn$. 
We denote by $\mathbb{B} = \{0,1\} \subset \mathbb{F}$. \\

%\noindent Finally, as mentioned in section~\ref{sec:conditional_relations}, the interpretation of adding explicit domains to public 
%inputs in the definition of conditional NP relations is that the honest parties (in our case, both the polynomial protocol verifiers and the SNARKs verifiers 
%as defined in this section below) parse the public inputs according to the specified domains without any further checks. Any checks or 
%computations that the honest parties perform regarding the public inputs are explicitly described as part of the protocols followed by 
%the honest parties.
%\vspace{-0.17in}
\subsection{Basic Accountable Ranged Polynomial Protocol}
\label{sec_la}
%\vspace{0in}
%\noindent We start by describing our conditional basic accountable relation $\Rla$ and the 
%corresponding $H$-ranged polynomial protocol $\Pla$. %For brevity, we omit the security parameter 
%$\lambda$ whenever we refer to any conditional NP relations for which we build SNARKs. \\

\noindent \textsf{Conditional Basic Accountable Relation $\Rla$}  
%\vspace{-0.05in}
\begin{equation*}
\begin{split}
\Rla = & \{(\mathbf{pk} \in ({\mathbb{F}^2})^{n-1}, \mathbf{bit} \in \mathbb{B}^n,
\mathit{apk} \in \mathbb{F}^2; \_) : \mathit{apk} = \sum_{i=0}^{n-2} [\mathit{bit_i}] \cdot \mathit{pk_i} \ | \ \mathbf{pk} \in \ginn{1}^{n-1} \} 
\end{split}
\end{equation*}
\noindent where $\mathbf{pk} = (\mathit{pk_0}, \ldots, \mathit{pk_{n-2}})$ and $\mathbf{bit} = (\mathit{bit_0}, \ldots, \mathit{bit_{n-1}})$. 
\noindent In this section we use the following polynomials and polynomial identities: \\

\noindent \textsf{Polynomials as Computed by Honest Parties} 
\begin{align*}
&\mathsf{b(X)} = \sum_{i=0}^{n-1} \mathit{bit_i} \cdot \mathsf{L_i(X)}; \mathsf{pkx(X)} =  \sum_{i=0}^{n-2} \mathit{pkx_i} \cdot \mathsf{L_i(X)}; \mathsf{pky(X)} =  \sum_{i=0}^{n-2} \mathit{pky_i} \cdot \mathsf{L_i(X)} \\
&\mathsf{kaccx(X)}  =  \sum_{i=0}^{n-1} \mathit{kaccx_i} \cdot \mathsf{L_i(X)}; \mathsf{kaccy(X)}  = \sum_{i=0}^{n-1} \mathit{kaccy_i} \cdot \mathsf{L_i(X)}, 
\end{align*}
\noindent where $(\mathit{pkx_0}, \ldots, \mathit{pkx_{n-2}})$ 
and $(\mathit{pky_0}, \ldots, \mathit{pky_{n-2}})$ are computed such that $\forall i \in \{0, \ldots, n-2\}$, $\mathit{pk_i}$ 
is interpreted as a pair $(\mathit{pkx_i}, \mathit{pky_i})$ with its components in $\mathbb{F}$; we also have 
$(\mathit{kaccx_{0}}, \mathit{kaccy_{0}}) = (\mathit{h_x}, \mathit{h_y})$ and 
$(\mathit{kaccx_{i+1}}, \mathit{kaccy_{i+1}}) =  (\mathit{kaccx_{i}}, \mathit{kaccy_{i}}) \oplus \mathit{bit_i}(\mathit{pkx_{i}}, \mathit{pky_{i}})$, 
$\forall i < n-1$. Note that in the last relation $\mathit{bit_i}$ is not interpreted as a field element anymore but as a binary bit.\\

\noindent \textsf{Polynomial Identities} 
\begin{align*}
& id_1(X) = (X-\omega^{n-1}) \cdot [b(X) \cdot ((kaccx(X)-pkx(X))^2 \cdot (kaccx(X)+ pkx(X) + \\ 
& + kaccx(\omega\cdot X)) - (pky(X) - kaccy(X))^2) +  (1-b(X))\cdot (kaccy(\omega\cdot X) - kaccy(X))] \\
& id_2(X)  =  (X-\omega^{n-1})\cdot [b(X) \cdot ((kaccx(X) - pkx(X)) \cdot (kaccy(\omega \cdot X) + kaccy(X)) - \\
& - (pky(X) - kaccy(X)) \cdot (kaccx(\omega \cdot X) - kaccx(X))) +  (1-b(X)) \cdot \\ 
& \cdot(kaccx(\omega \cdot X) - kaccx(X))] \\
& id_3(X)  =  (kaccx(X) - h_x)\cdot L_0(X) + (kaccx(X) - (h\oplus apk)_{x}) \cdot L_{n-1}(X)  \\
& id_4(X) =  (kaccy(X) - h_y)\cdot L_0(X) + (kaccy(X) - (h\oplus apk)_{y}) \cdot L_{n-1}(X) \\
& id_5(X) =  b(X)(1-b(X)).
\end{align*}

\noindent Polynomial identity $id_5(X)$ is not needed for defining ranged polynomial protocols for $\Rla$, however it is included 
here to ease presentation and for proofs consistency for the following section. \\

\noindent \textsf{$H$-ranged Polynomial Protocol $\Pla$ for Relation $\Rla$} \\

\noindent $H$-ranged polynomial protocol $\Pla$ for conditional relation $\Rla$ describes the interaction of the prover 
$\mathcal{P}_{poly}$, the verifier $\mathcal{V}_{poly}$ and the trusted third party $\mathcal{I}$ in accordance to 
Definition~\ref{def_ranged_poly_protocol} from Section~\ref{supplementary_poly_protocols_appendix}. \\

%\noindent \textsf{Protocol $\Pla$} \\

\noindent $\mathcal{P}_{poly}$ and $\mathcal{V}_{poly}$ know public input 
$\mathbf{bit} \in \mathbb{B}^n$, $\mathbf{pk} \in (\mathbb{F}^2)^{n-1}$ and $\mathit{apk} \in (\mathbb{F})^2$ 
which are interpreted as per their respective domains.
\begin{enumerate}
\item $\mathcal{V}_{poly}$ computes $b(X)$, $pkx(X)$, $pky(X)$.
\item $\mathcal{P}_{poly}$ sends polynomials $kaccx(X)$ and $kaccy(X)$ to $\mathcal{I}$. 
\item $\mathcal{V}_{poly}$ asks $\mathcal{I}$ to check whether the following polynomial relations hold over range $H$ 
$$id_i(X) = 0, \forall i \in [4].$$
\item $\mathcal{V}_{poly}$  accepts if $\mathcal{I}$'s checks verify. 
\end{enumerate}

\noindent We show that protocol $\Pla$ is an $H$-ranged polynomial protocol for 
conditional NP relation $\Rla$. For this, we first prove that:
\begin{test_claim} Assume that $\forall i < n-1$ such that $\mathit{bit}_i = 1$, $pk_i = (pkx_i, pky_i) \in \ginn{1}$. 
If polynomial identities $id_i(X) = 0, \forall i \in [5],$ hold over range 
$H$ and and the polynomial $b(X)$ has been constructed via interpolation on $H$ such that $b(\omega^i) = \mathit{bit}_i, \forall i <n$ then $\mathit{bit}_i \in \mathbb{B} = \{0,1\} \subset \mathbb{F}, \forall i <n$ \\
$(kaccx_{0}, kaccy_{0}) = (h_x, h_y)$, $(kaccx_{n-1}, kaccy_{n-1}) = (h_x, h_y) \oplus (apk_x, apk_y)$, \\
$(kaccx_{i+1}, kaccy_{i+1}) =  (kaccx_{i}, kaccy_{i}) \oplus \mathit{bit}_i(pkx_{i}, pky_{i})$, $\forall i < n-1$, 
where in the last relation $\mathit{bit_i}$ should not be interpreted as a field element but as a binary bit.
\label{claim:keys_affine_comm}
\end{test_claim}

\begin{proof} Everything but the last property in the claim is easy to derive from polynomial identities 
$id_3(X) =0, id_4(X )= 0, id_5(X) = 0$ holding over $H$. In order to prove the remaining property, we remind 
the incomplete addition formulae for curve points in affine coordinates, over elliptic curve in short Weierstrasse form and state:\\ 

\noindent \textit{Observation:} Suppose that $\mathit{bit} \in \{0,1\}$, $(x_1,y_1)$ is a point on an elliptic curve in 
short Weierstrasse form, and, if $\mathit{bit} = 1$, so is $(x_2,y_2)$. We claim that the following equations: 
\begin{align*}
&\mathit{bit}((x_1 - x_2)^2 (x_1 + x_2 + x_3) - (y_2 - y_1)^2 ) + (1 - \mathit{bit})(y_3 - y_1) =0 \ (\ast)\\
&\mathit{bit}((x_1 - x_2)(y_3 + y_1) - (y_2 - y_1)(x_3 - x_1)) + (1 - \mathit{bit})(x_3 - x_1) =0 \ (\ast\ast)
\end{align*}

\noindent hold if and only if one of the following three conditions hold 

\begin{enumerate}
\item \label{cond1} $\mathit{bit}=1$ and $(x_1,y_1)\oplus(x_2,y_2)=(x_3,y_3)$ and $x_1 \neq x_2$
\item \label{cond2} $\mathit{bit}=0$ and $(x_3,y_3)=(x_1,y_1)$ 
\item  \label{cond3} $\mathit{bit}=1$ and $(x_1,y_1)=(x_2,y_2)$\footnote{Note that under condition~\ref{cond3}, $(x_3,y_3)$ 
can be any point whatsoever, maybe not even on the curve. The same holds true for $(x_2, y_2)$ under the condition~\ref{cond2}.}.
\end{enumerate}

\noindent It is easy to see that each of the conditions~\ref{cond1},\ref{cond2},\ref{cond3} above implies equations $(\ast)$ and $(\ast \ast)$.
\noindent For the implication in the opposite direction, if we assume that $(\ast)$ and $(\ast \ast)$ hold, then \\

\noindent \textit{Case a:} For $\mathit{bit}=0$, the first term of each equation $(\ast)$ and $(\ast \ast)$ vanishes, 
leaving us with $y_3-y_1=0$ and $x_3-x_1=0$ which are equivalent to condition~\ref{cond2}. \\

\noindent \textit{Case b:} For $\mathit{bit}=1$ and $x_1=x_2$, by simple substitution in $(\ast)$ and $(\ast \ast)$, 
we obtain $y_1 = y_2$, i.e., condition~\ref{cond3}.  \\

\noindent \textit{Case c:} For $\mathit{bit}=1$ and $x_1 \neq x_2$, then we can substitute
$\beta=\frac{y_2-y_1}{x_2-x_1}$ into equations $(\ast)$ and $(\ast \ast)$, leaving us with
$$x_1+x_2+x_3=\beta^2 \textrm{ and } y_3+y_1=\beta(x_3-x_1).$$
which are the usual formulae for short Weierstrass form addition of affine coordinate points when $x_1 \neq x_2$ 
so this is equivalent to condition~\ref{cond1}. \\

\noindent We apply the above \textit{Observation} by noticing that if $id_1(X)$ and $id_2(X)$ hold over $H$, 
then $(\ast)$ and $(\ast \ast)$ hold with $(x_1, y_1)$ substituted by $(kaccx_i,kaccy_i)$, $(x_2, y_2)$ 
substituted by $(pkx_i, pky_i)$, $(x_3, y_3)$ substituted by $(kaccx_{i+1},kaccy_{i+1})$ and $\mathit{bit}$ 
substituted by $\mathit{bit}_i$ for $0 \leq i \leq n-2$, where $\mathit{bit_i}$ should not be interpreted as a field element but as binary 
bit. Moreover, since $(kaccx_{0}, kaccy_{0}) = (h_x, h_y) \in \einn \setminus \ginn{1}$ 
and if $(pkx_i, pky_i) \in \ginn{1}$ whenever $\mathit{bit}_i = 1$, then $\forall i < n-1$ 
equations $(\ast)$ and $(\ast \ast)$ obtained after the substitution defined above are equivalent to either 
condition~\ref{cond1} or condition~\ref{cond2}, but never condition~\ref{cond3}, so the result of the sum (i.e., $(kaccx_{i+1}, kaccy_{i+1})$, $0\leq i \leq n-2$) is, 
by induction, at each step a well-defined point on the curve and this concludes our proof.
\end{proof}

\begin{corollary} Assume $\forall i < n-1$ 
such that $\mathit{bit}_i = 1$, $pk_i = (pkx_i, pky_i) \in \ginn{1}$. 
If the polynomial identities $id_i(X) = 0, \forall i \in [4],$ hold over range $H$ and 
$\mathit{bit_i} \in \mathbb{B}$, $\forall i < n-1$ and $b(X) = \sum_{i=0}^{n-1} \mathit{bit_i} \cdot L_i(X)$
then:  \\
$(kaccx_{0}, kaccy_{0}) = (h_x, h_y)$, \\
$(kaccx_{n-1}, kaccy_{n-1}) = (h_x, h_y) \oplus (apk_x, apk_y)$, \\
$(kaccx_{i+1}, kaccy_{i+1}) =  (kaccx_{i}, kaccy_{i}) \oplus \mathit{bit_i}(pkx_{i}, pky_{i})$, $\forall i < n-1$, where in the last relation 
$\mathit{bit_i}$ should not be interpreted as a field element but as a binary bit.
\label{corollary:keys_affine_comm}
\end{corollary}

\begin{proof}The proof follows trivially from the general result stated by Claim~\ref{claim:keys_affine_comm}. 
\end{proof}

\begin{lemma} 
\label{le:ba}
$\Pla$ as described above is an $H$-ranged polynomial 
protocol for conditional NP relation $\Rla$.
\end{lemma}

\begin{proof}
If $(\mathbf{bit},\mathbf{pk}, \mathit{apk}) \in \Rla$ holds, 
meaning that $\mathbf{bit} \in \mathbb{B}^n$ and $\mathbf{pk} \in \ginn{1}^{n-1}$ and $\mathit{apk} = \sum_{i=0}^{n-2} [\mathit{bit_i}] \cdot \mathit{pk_i}$ hold, 
then it is easy to see that the honest prover $\mathcal{P}_{poly}$ in $\Pla$ will convince the honest verifier $\mathcal{V}_{poly}$ in 
$\Pla$ to accept with probability $1$ so perfect completeness holds. \\
Regarding knowledge-soundness, if the verifier $\mathcal{V}_{poly}$ in $\Pla$ accepts, 
then the extractor $\mathcal{E}$ does not have to do anything as the relation $\Rla$ does not have a witness.  
However, we have to prove that if $\mathbf{pk} \in \ginn{1}^{n-1}$ and the verifier in $\Pla$ accepts, 
then $(\mathbf{bit},\mathbf{pk}, \mathit{apk}) \in \Rla$ holds, which given our definition for conditional relation is 
equivalent to proving that $\mathit{apk} = \sum_{i=0}^{n-2} [\mathit{bit_i}] \cdot \mathit{pk_i}$ holds. This is indeed the case due to 
corollary~\ref{corollary:keys_affine_comm}.

\end{proof}
%\vspace{-0.15in}
\subsection{Packed Accountable Ranged Polynomial Protocol}
\label{sec_a}
Let $\mathbb{F}_{|\block|}$ be the subset of field elements in $\mathbb{F}$ that can be represented using at 
most $\block$ bits, i.e., the set $\{0, \ldots, 2^{\block -1} \}$, where $\block$ has been defined in Section ~\ref{sec:lagrange}. 
Our conditional packed accountable relation $\Ra$ and the corresponding $H$-ranged polynomial protocol 
$\Pa$ are defined as follows:\\
 
\noindent \textsf{Conditional Packed Accountable Relation $\Ra$} 
\begin{equation*}
\begin{split}
\Ra = & \{(\mathbf{pk} \in (\mathbb{F}^2)^{n-1},\mathbf{b'} \in \mathbb{F}_{|\block|}^{\frac{n}{\block}},
\mathit{apk} \in \mathbb{F}^2; \mathbf{bit}) : \\ 
 & \mathit{apk} = \sum_{i=0}^{n-2} [\mathit{bit_i}] \cdot \mathit{pk_i} \ | \ \mathbf{pk} \in \ginn{1}^{n-1} \ \wedge \\
 & \wedge \mathbf{bit} \in \mathbb{B}^n  \wedge b'_{j} = \sum_{i=0}^{\block -1}2^i \cdot \mathit{bit_{\block \cdot j + i}}, \forall j < \frac{n}{\block} \} 
\end{split}
\end{equation*}
where $\mathbf{b'} = (b'_{0}, \ldots, b'_{{\frac{n}{\block}} -1})$. We define new polynomials and polynomial identities: \\

\noindent \textsf{New Polynomials as Computed by Honest Parties} 
$$ aux(X) = \sum_{i=0}^{n-1}aux_i \cdot L_i(X); c_{a}(X) = \sum_{i=0}^{n-1} c_{a,i} \cdot L_i(X); acc_{a}(X)  =  \sum_{i=0}^{n-1} acc_{a,i}  \cdot L_i(X)$$
\noindent where $aux_{i} = 1 \in \mathbb{F}$ if $i$ is divisible with $\block$ and $aux_{i} = 0 \in \mathbb{F}$ otherwise, $\forall i < n$ 
and $c_{a,i} = 2^k \cdot r^j$, $k = i \mod \block$, $j = i \div \block$, $\forall i < n$ ($r \in \mathbb{F}$ is introduced in protocol $\Pa$) and $acc_{a,i}$ are components of the vector $(0, \mathit{bit}_0 \cdot c_{a,0}, \mathit{bit}_0 \cdot c_{a,0}+ \mathit{bit}_1  \cdot c_{a,1}, \ldots, \sum_{i=0}^{n-2}\mathit{bit}_i \cdot c_{a,i})$, where $\mathit{bit_{0}}, \ldots, \mathit{bit_{n-1}}$ represent the first $n$ 
bits (however, we interpret them as elements in $\mathbb{B}$) of the concatenation of the binary representation of 
$\mathit{b'_{0}}, \ldots, \mathit{b'_{\frac{n}{\block}-1}}$.\footnote{As 
part of a correct public input for relation $\Ra$, each field element in the set 
$\{\mathit{b'_{0}}, \ldots, \mathit{b'_{\frac{n}{\block}-1}} \}$ is at most $\block$ binary bits long. If any such field element has fewer than $\block$ bits, 
then the honest prover will pad it with $0$s starting from the most significant bit up to a total individual length of $\block$ bits.} 
With this definition of $(\mathit{bit_{0}}, \ldots, \mathit{bit_{n-1}})$, 
the definition of $b(X)$ remains the same as in Section \ref{sec_la}.\\

\noindent \textsf{New Polynomial Identities} 
\begin{align*}
id_6(X) & =  c_{a}(\omega \cdot X) - c_{a}(X)\cdot (2+ (\frac{r}{2^{\block -1}} -2) \cdot aux(\omega \cdot X)) - (1 - r^{\frac{n}{\block}}) \cdot L_{n-1}(X).\\
id_7(X) & =  acc_{a}(\omega \cdot X) - acc_{a}(X) - b(X)\cdot c_{a}(X) + \mathsf{sum} \cdot L_{n-1}(X),
\end{align*}

\noindent where $\mathsf{sum}$ is a field element known to both $\mathcal{P}_{poly}$ and $\mathcal{V}_{poly}$ and will be defined below. \\ 

\noindent \textsf{$H$-ranged Polynomial Protocol $\Pa$ for Relation $\Ra$} \\

%\noindent In the following, we describe $H$-ranged polynomial protocol $\Pa$ for conditional relation 
%$\Ra$. \\

%\noindent \textsf{Protocol $\Pa$} \\

\noindent $\mathcal{P}_{poly}$ and $\mathcal{V}_{poly}$ know public inputs 
$\mathbf{b'} \in \mathbb{F}_{|\block|}^{\frac{n}{\block}}$ and 
$\mathbf{pk} \in (\mathbb{F}^2)^{n-1} $ and $\mathit{apk} \in \mathbf{F}^2$ which are interpreted as per their respective domains. \\

\begin{enumerate}
\item $\mathcal{V}_{poly}$ computes $pkx(X)$, $pky(X)$ and $aux(X)$.
\item $\mathcal{P}_{poly}$ sends polynomials $b(X)$, $kaccx(X)$ and $kaccy(X)$ to $\mathcal{I}$. 
\item $\mathcal{V}_{poly}$ replies with a random value $r$ chosen from $\mathbb{F}$. 
\item $\mathcal{V}_{poly}$ computes $\mathsf{sum}$ as $\sum_{j=0}^{\frac{n}{\block}-1} \mathit{b'_{j}} \cdot r^j$.\footnote{Note that if 
$b'_{j} = \sum_{k=0}^{\block -1}2^k \cdot \mathit{bit_{\block \cdot j + k }}$, $\forall j < \frac{n}{\block}$ and $\mathit{bit_i} \in \mathbb{B}, \forall i <n$, 
then $\sum_{i=0}^{n-1} 2^{i \mod \block} \cdot r^{i \div \block} \cdot \mathit{bit}_{i} = \sum_{j=0}^{\frac{n}{\block}-1}(\sum_{i=0}^{\block -1}2^k \cdot \mathit{bit_{\block \cdot j + k }}) \cdot r^j= \sum_{j=0}^{\frac{n}{\block}-1} \mathit{b'_{j}} \cdot r^j$.}
\item $\mathcal{P}_{poly}$ sends polynomials $c_{a}(X)$ and $acc_{a}(X)$ to $\mathcal{I}$. 
\item $\mathcal{V}_{poly}$ asks $\mathcal{I}$ to check whether the following polynomial relations hold over range $H$: 
$$id_i(X) = 0, \forall i \in [7].$$
\item $\mathcal{V}_{poly}$ accepts if $\mathcal{I}$'s checks verify. 
\end{enumerate}

\noindent We show $\Pa$ is an $H$-ranged polynomial protocol 
for relation $\Ra$. First, we prove the following:

\begin{test_claim}
\label{claim:bitvector_comm}
If the polynomial identities $id_6(X) = 0, id_7(X) = 0$ hold over range $H$, then, 
\ewnp, 
we have $c_{a,i} =  2^{i \mod \block} \cdot r^{i \div \block}$, $\forall i < n$ and $\mathsf{sum} = \sum_{i=0}^{n-1}b_i \cdot c_{a,i}$, 
where $b_i = b(\omega^i), \forall i <n$. If, additionally, identity $id_5(X) = 0$ holds over $H$, 
$r$ has been randomly chosen in $\mathbb{F}$, $\mathsf{sum} = \sum_{j=0}^{\frac{n}{\block}-1} b'_{j}r^j$ 
(as computed by $\mathcal{V}_{poly}$) and $\mathit{bit_{i}} \in \mathbb{B}, \forall i < n$ and 
$b'_{j} = \sum_{k=0}^{\block -1}2^k \cdot \mathit{bit_{\block \cdot j + k}}, \forall 0 \leq j \leq \frac{n}{\block} -1$ 
(due to the input $(b'_{0}, \ldots, b'^{\frac{n}{\block} -1})$ 
being interpreted by the verifier $\mathcal{V}_{poly}$ as in $\mathbb{F}_{|\block|}^{\frac{n}{\block}}$), then \ewnp, 
$b_i = \mathit{bit_{i}}, \forall i <n$.
\end{test_claim}

\begin{proof}
We show the first part of the claim by proving by contradiction that $c_{a,0} =1$ using the Schwartz-Zippel Lemma, the fact that $r$ has been 
randomly chosen, and, also the fact that $n$ is negligibly smaller compared to the size of $\mathbb{F}$. Finally, we appropriately expand $\mathit{id_7}(X) = 0$ 
over $H$, sum the LHS on one hand and the RHS on the other hand, equate and obtain the desired property of $\mathsf{sum}$. We show the second part of 
the claim by expressing $\mathsf{sum}$ in two ways as $\sum_{j=0}^{\frac{n}{\block}-1} b'_{j}r^j $ and as $\sum_{i=0}^{n-1} b_i \cdot c_{a,i}$ and re-writing the 
latter as an inner product of a vector of field elements with the vector $(1, r, \ldots, r^{\frac{n}{\mathsf{block}}-1})$ and using the small exponents test~\cite{small_exponents}. 
Full proof can be found in Section~\ref{sec:missing_snark_proofs}.
\end{proof}

\begin{lemma} 
$\Pa$ is an $H$-ranged polynomial protocol for conditional NP relation $\Ra$.
\end{lemma}

\begin{proof} 
The proof follows an analogous logic as used for proving Lemma~\ref{le:ba}. We additionally use 
claim~\ref{claim:bitvector_comm} and corollary~\ref{corollary:keys_affine_comm}. Full proof can be found in Section~\ref{sec:missing_snark_proofs}.
\end{proof}
%\vspace{-0.1in}

\subsection{Compiler for Hybrid Model SNARKs with Mixed Inputs}
\label{sec_two_step_compiler}
%\vspace{-0.05in}

\noindent We present a two-steps PLONK-based compilation technique from 
ranged polynomial protocols for conditional NP relations (formal definition in Section~\ref{supplementary_poly_protocols_appendix}) to hybrid 
model SNARKs (Definition~\ref{dfn_snark}) such that the conditional NP relations that define the SNARKs we compile in the 
second step contain both polynomial commitments and vectors of field elements as public inputs. By using just the first step of our 
compiler which is equivalent to the original PLONK compiler~\cite{plonk}, one would 
not be able to obtain SNARKs with mixed public inputs consisting of both vectors of field elements and also poly commitments. 
In turn, this type of NP relations with mixed inputs is crucial for designing accountable light clients via the use of committee key schemes 
(see Section~\ref{sec:inst_committee_key}).\\
\vspace{-0.2in}
\subsubsection{Our Compiler: Step 1} 
\label{compiler_step_1}
\vspace{-0.05in}
%\noindent \textbf{(PLONK Compiler - from Polynomial Protocols to SNARKs)} \\

\noindent Our first step applies the PLONK compiler~\cite{plonk} (Lemma 4.7): we compile the information theoretical ranged polynomial protocols $\Pla$ and $\Pa$ 
for relations $\Rla$ and $\Ra$, respectively (see sections \ref{sec_la} and \ref{sec_a}) into 
(hybrid model) SNARKs $\Plastar$, and $\Pastar$, respectively. We can define this compilation step 
for any ranged polynomial protocols for relations (as per Definition~\ref{supplementary_poly_protocols_appendix} in section~\ref{def_ranged_poly_protocol}). 
\begin{comment}
In order to do that we need: 
\begin{itemize}
\item  The batched version of KZG polynomial commitments~\cite{KZG_10} described in section 3 of PLONK~\cite{plonk}.\footnote{In fact, 
one can replace the use of KZG polynomial commitments with any binding polynomial commitment that has knowledge-soundness, including non-homomorphic polynomial commitments, 
such as FRI-based polynomial commitments (e.g., RedShift~\cite{redshift}). If the optimisation gained from PLONK linearisation technique is a goal, 
then, with minimal changes one can use any homomorphic polynomial commitment, e.g., the discrete logarithm based polynomial commitment 
from Halo~\cite{halo}.}
\item A general compilation technique: such a technique has been already defined in Lemma 4.7 of PLONK; combined with Lemma 4.5 
from PLONK this technique can be applied with minor adaptations (this includes the corresponding technical measures) to the notion of ranged 
polynomial protocols.  
\item So far, both the ranged polynomial protocols for relations and the protocols resulted after the first compilation step have been explicitly defined as interactive 
protocols. In order to obtain the non-interactive version of the latter (essentially the N in SNARK) one has to apply the Fiat-Shamir 
transform~\cite{FS_transform}, \cite{FS_transform_with_proof}, \cite{SE_plonk}.
\end{itemize}

\end{comment}
%\begin{comment}
\noindent Let $\mathcal{R}$ be a (conditional) NP relation, let $\mathscr{P}_{\mathcal{R}}$ be a ranged polynomial protocol for 
relation $\mathcal{R}$ and let $\mathscr{P}^*_{\mathcal{R}}$ be the SNARK compiled from $\mathscr{P}_{\mathcal{R}}$ using the PLONK compiler.  
The compilation technique requires the SNARK prover of  $\mathscr{P}^*_{\mathcal{R}}$ to compute 
polynomial commitments to all polynomials that the prover $\mathcal{P}_{poly}$ in $\mathscr{P}_{\mathcal{R}}$ sent to the ideal party $\mathcal{I}$. Analogously, 
it requires the SNARK verifier of $\mathscr{P}^*_{\mathcal{R}}$ to compute polynomial commitments to all pre-processed polynomials\footnote{This is a one-time computation that is 
reused by the SNARK verifier for all SNARK proofs over the same circuit.} as well polynomial commitments to polynomials the verifier $\mathcal{V}_{poly}$ 
in $\mathscr{P}_{\mathcal{R}}$ sent to the ideal party $\mathcal{I}$. Then, the SNARK prover sends the SNARK verifier openings to 
all the polynomial commitments computed by him as well as the polynomial commitments computed by the SNARK verifier. The SNARK 
prover additionally sends the corresponding batched proofs for polynomial commitment openings. In turn, the SNARK verifier accepts or rejects based 
on the result of the verification of the batched polynomial commitment scheme. \\
%\end{comment}

%\noindent A more efficient compilation technique exists which reduces the number of polynomial commitments and alleged polynomial commitments openings 
%(i.e., both group elements and field elements) sent by the SNARK prover to the SNARK verifier; this, in turn, reduces the size of the SNARK proof. 
%This technique is called linearisation and is described, at a high level, after Lemma 4.7 in PLONK. The existing description however covers only the 
%SNARK prover side and it does not detail the SNARK verifier side so in the following we cover that. \\
\noindent PLONK proposes a more efficient compilation technique (i.e., linearisation, see explanation after Lemma 4.7 in PLONK) 
which reduces the SNARK proof size. 
%\noindent By functionality, the vectors that are handled by the the verifier $\mathcal{V}_{poly}$ are 
%of two types: pre-processed vectors and public input vectors. These two types of vectors are used by $\mathcal{V}_{poly}$ 
%to obtain, via interpolation over the range on which the respective range polynomial protocol is defined, pre-processed polynomials 
%(as used in the definition 2 in section 2 of supplementary material, e.g., polynomial $aux(X)$ used in section~\ref{sec_a}) and 
%public-inputs-derived polynomials (e.g., polynomials $pkx(X)$ and $pky(X)$ used in sections~\ref{sec_la} and ~\ref{sec_a} 
%and polynomial $b(X)$ used in section~\ref{sec_la}). The efficient linearisation technique allows the SNARK verifier to reduce the 
%number of polynomial commitments it has to compute compared to the general PLONK compiler in the following way. Instead of 
%having to compute polynomial commitments to all polynomials $\mathcal{V}_{poly}$ sends to $\mathcal{I}$ (including any corresponding 
%pre-processed polynomials), the SNARK verifier computes polynomial evaluations at one or multiple random points (as per the linearisation 
%step specific requirements) for all the polynomials that are either easy to evaluate (e.g., polynomial $aux(X)$ used in section~\ref{sec_a}) or 
%all the polynomials that are obtained from vectors that do not take up a large amount of memory (e.g., polynomial $b(X)$ used in section~\ref{sec_la}). 
%For the rest of the polynomials (e.g., $\mathit{pkx}(X)$ and $\mathit{pky}(X)$), the SNARK verifier computes polynomial commitments as before.\\
For our specific case, this allows the SNARK verifier to reduce the 
number of polynomial commitments it has to compute compared to the general PLONK compiler by computing 
polynomial evaluations at one or multiple random points (as per the linearisation step specific requirements) 
for all the polynomials that are either easy to evaluate (e.g., polynomial $aux(X)$ used in section~\ref{sec_a} and $\Ra$) or 
all the polynomials that are obtained from input vectors that do not take up a large amount of memory (e.g., polynomial $b(X)$ used in section~\ref{sec_la} and $\Rla$).
%\noindent We note we can apply all the techniques mentioned above, including the combined prover-and-verifier-side linearisation 
%to compile our ranged polynomial protocols $\Pla$ and $\Pa$ into the corresponding SNARKs $\Plastar$ and $\Pastar$, respectively.
Finally, we state in Section~\ref{def_ranged_poly_protocol}, Lemma~\ref{le:compilation_step_1} under which conditions and how efficiently 
one can compile ranged polynomial protocols for pure vector-based conditional NP relations 
into hybrid model SNARKs using only the original PLONK compiler and we give a more in-depth explanation of this step in section~\ref{first_step_compiler}.\\
\noindent \textbf{(PLONK Compiler - from Polynomial Protocols to SNARKs)} \\

\noindent We summarise and exemplify below the PLONK-based compilation technique~\cite{plonk} from 
ranged polynomial protocols for conditional NP relations (formal definition in Section~\ref{supplementary_poly_protocols_appendix}) to 
SNARKs for pure vector-based NP relations. This is also the first of our two-steps compiler. Concretely, our first step applies the PLONK compiler~\cite{plonk} (lemma 4.7): 
we compile the information theoretical ranged polynomial protocols $\Pla$ and $\Pa$ for relations $\Rla$ and $\Ra$, respectively (see Sections~\ref{sec_la},\ref{sec_a}) 
into SNARKs $\Plastar$, and $\Pastar$, respectively. We can define this compilation step for any ranged polynomial protocols for relations 
(as per Definition~\ref{supplementary_poly_protocols_appendix} in Section~\ref{def_ranged_poly_protocol}). In order to do that we need: 
\begin{itemize}
\item  The batched version of KZG polynomial commitments~\cite{KZG_10} described in Section 3 of PLONK~\cite{plonk}.\footnote{In fact, 
one can replace the use of KZG polynomial commitments with any binding polynomial commitment that has knowledge-soundness, including non-homomorphic polynomial commitments, 
such as FRI-based polynomial commitments (e.g., RedShift~\cite{redshift}). If the optimisation gained from PLONK linearisation technique is a goal, 
then, with minimal changes one can use any homomorphic polynomial commitment, e.g., the discrete logarithm based polynomial commitment 
from Halo~\cite{halo}.}
\item A general compilation technique: such a technique has been already defined in lemma 4.7 of PLONK; combined with lemma 4.5 
from PLONK this technique can be applied with minor adaptations (this includes the corresponding technical measures) to the notion of ranged 
polynomial protocols.  
\item So far, both the ranged polynomial protocols for relations and the protocols resulted after the first compilation step have been explicitly defined as interactive 
protocols. In order to obtain the non-interactive version of the latter (essentially the N in SNARK) one has to apply the Fiat-Shamir 
transform~\cite{FS_transform}, \cite{FS_transform_with_proof}, \cite{SE_plonk}.
\end{itemize}

\noindent Let $\mathcal{R}$ be a (conditional) NP relation, let $\mathscr{P}_{\mathcal{R}}$ be a ranged polynomial protocol for 
relation $\mathcal{R}$ and let $\mathscr{P}^*_{\mathcal{R}}$ be the SNARK compiled from $\mathscr{P}_{\mathcal{R}}$ using the PLONK compiler.  
The compilation technique requires the SNARK prover of  $\mathscr{P}^*_{\mathcal{R}}$ to compute 
polynomial commitments to all polynomials that the prover $\mathcal{P}_{poly}$ in $\mathscr{P}_{\mathcal{R}}$ sent to the ideal party $\mathcal{I}$. Analogously, 
it requires the SNARK verifier of $\mathscr{P}^*_{\mathcal{R}}$ to compute polynomial commitments to all pre-processed polynomials\footnote{This is a one-time computation that is 
reused by the SNARK verifier for all SNARK proofs over the same circuit.} as well polynomial commitments to polynomials the verifier $\mathcal{V}_{poly}$ 
in $\mathscr{P}_{\mathcal{R}}$ sent to the ideal party $\mathcal{I}$. Then, the SNARK prover sends the SNARK verifier openings to 
all the polynomial commitments computed by him as well as the polynomial commitments computed by the SNARK verifier. The SNARK 
prover additionally sends the corresponding batched proofs for polynomial commitment openings. In turn, the SNARK verifier accepts or rejects based 
on the result of the verification of the batched polynomial commitment scheme. \\

\noindent A more efficient compilation technique exists which reduces the number of polynomial commitments and alleged polynomial commitments openings 
(i.e., both group elements and field elements) sent by the SNARK prover to the SNARK verifier; this, in turn, reduces the size of the SNARK proof. 
This technique is called linearisation and is described, at a high level, after Lemma 4.7 in PLONK. The existing description however covers only the 
SNARK prover side and it does not detail the SNARK verifier side so in the following we cover that. \\

\noindent By functionality, the vectors that are handled by the the verifier $\mathcal{V}_{poly}$ are 
of two types: pre-processed vectors and public input vectors. These two types of vectors are used by $\mathcal{V}_{poly}$ 
to obtain, via interpolation over the range on which the respective range polynomial protocol is defined, pre-processed polynomials 
(as used in the Definition~\ref{supplementary_poly_protocols_appendix} in Section~\ref{def_ranged_poly_protocol}, e.g., polynomial $aux(X)$ used in Section~\ref{sec_la}) and 
public-inputs-derived polynomials (e.g., polynomials $pkx(X)$ and $pky(X)$ used in Sections~\ref{sec_la},\ref{sec_a})
and polynomial $b(X)$ used in Section~\ref{sec_la}). The efficient linearisation technique allows the SNARK verifier to reduce the 
number of polynomial commitments it has to compute compared to the general PLONK compiler in the following way. Instead of 
having to compute polynomial commitments to all polynomials $\mathcal{V}_{poly}$ sends to $\mathcal{I}$ (including any corresponding 
pre-processed polynomials), the SNARK verifier computes polynomial evaluations at one or multiple random points (as per the linearisation 
step specific requirements) for all the polynomials that are either easy to evaluate (e.g., polynomial $aux(X)$ used in Section~\ref{sec_a}) or 
all the polynomials that are obtained from vectors that do not take up a large amount of memory (e.g., polynomial $b(X)$ used in Section~\ref{sec_la}). 
For the rest of the polynomials (e.g., $\mathit{pkx}(X)$ and $\mathit{pky}(X)$), the SNARK verifier computes polynomial commitments as before.\\

\noindent We note we can apply all the techniques mentioned above, including the combined prover-and-verifier-side linearisation 
to compile our ranged polynomial protocols $\Pla$ and $\Pa$ into the corresponding SNARKs $\Plastar$ and $\Pastar$, respectively. 
To conclude this step, we formally state in Section~\ref{def_ranged_poly_protocol}, Lemma~\ref{le:compilation_step_1} under which condition and how efficiently 
one can compile ranged polynomial protocols for conditional NP relations (where the public inputs are interpreted as vector of field elements) 
into hybrid model SNARKs by using only the original PLONK compiler. \\

\vspace{-0.2in}
\subsubsection{Our Compiler: Step 2}
\label{compiler_step_2}
\vspace{-0.05in}
\noindent \textbf{(Mixed Vector and Commitments based NP Relations and Associated SNARKs)} \\

\noindent The type of NP relations we have worked with so far as well as the more general PLONK NP relation 
(\cite{plonk}, Section 8.2) have vector of field elements as public inputs. Next we show that SNARKs 
compiled using Step 1 can become, under certain assumption, SNARKs for a new type of NP relation 
that specifically contains polynomial commitments as part of the input. Interpreting 
our already compiled SNARKs as SNARKs for this new type of NP relation is essential for designing 
accountable light client systems via committee key schemes (see Instantiation~\ref{inst:cks} 
in Section~\ref{sec:inst_committee_key}).  

\noindent Let conditional NP relation $\mathcal{R}_{\mathit{vec}}^c$  be:
\begin{align*}
\mathcal{R}_{\mathit{vec}}^c = \{&(\mathbf{input_1} \in \mathbf{\mathcal{D}_1}, \mathbf{input_2} \in\mathbf{\mathcal{D}_2}; \mathbf{witness_1}): \\  
&p_1(\mathbf{input_1}, \mathbf{input_2}, \mathbf{witness_1}) = 1 \ | \ c(\mathbf{input_1}) = 1 \ \wedge\ \\
&\wedge \ p_2(\mathbf{input_1}, \mathbf{input_2}, \mathbf{witness_1}) = 1 \},
\end{align*}
\noindent where $\mathbf{input_1}$, $\mathbf{input_2}$ are two sets of public input vectors 
belonging domains  $\mathcal{D}_1$, $\mathcal{D}_2$. $\mathbf{witness_1}$ is a set of witness vectors and $c$, $p_1$, $p_2$ are predicates. 
Let $\mathscr{P}_{\mathit{vec}}$ be a ranged polynomial protocol for relation $\mathcal{R}_{\mathit{vec}}^c$. Note that since $c$ applies 
only to a part of the public input for relation $\mathcal{R}_{\mathit{vec}}^c$ 
(i.e., $\mathbf{input_1}$), we can apply Lemma~\ref{le:compilation_step_1} of Section~\ref{supplementary_poly_protocols_appendix} and Step 1 
of our compiler to polynomial protocol $\mathscr{P}_{\mathit{vec}}$. \\

\noindent We make the following hybrid model assumptions:
\begin{itemize}
\item (HMA.1.) $\mathcal{V}_{poly}$ in $\mathscr{P}_{\mathit{vec}}$ computes 
$\mathit{Q_{1,\mathbf{input_1}}}(X), \ldots, \mathit{Q_{m, \mathbf{input_1}}}(X)$ which depend deterministically on $\mathbf{input_1}$ and sends them to $\mathcal{I}$. 
\item (HMA.2.) $\mathcal{V}_{poly}$ in $\mathscr{P}_{\mathit{vec}}$ does not use $\mathbf{input_1}$ in any further computation of 
any other polynomials or values its sends to $\mathcal{I}$.
\item (HMA.3.) By evaluating $\mathit{Q_{1,\mathbf{input_1}}}(X), \ldots, \mathit{Q_{m, \mathbf{input_1}}}(X)$ over the range on which 
$\mathscr{P}_{\mathit{vec}}$ is defined one obtains (using some efficiently computable and deterministic transformations) the set of vectors $\mathbf{input_1}$. 
\end{itemize} 
We denote by $\mathscr{P}^*_{\mathit{vec}}$ the hybrid model SNARK obtained after compiling $\mathscr{P}_{\mathit{vec}}$ using compilation Step 1. 
Due to (HMA.1.) and according to Step 1, the SNARK verifier in 
$\mathscr{P}^*_{\mathit{vec}}$ computes $$\mathit{Com_1} = \mathit{Com}(\mathit{Q_{1,\mathbf{input_1}}}), \ldots, \mathit{Com_m} = \mathit{Com}(\mathit{Q_{m,\mathbf{input_1}}})$$ 
which are KZG poly commitments to $\mathit{Q_{1,\mathbf{input_1}}}(X), \ldots, \mathit{Q_{m, \mathbf{input_1}}}(X)$. We denote vector
$(\mathit{Com_1}, \ldots, \mathit{Com_m})$ by $\mathbf{Com}(\mathbf{input_1})$ and we denote 
by $\mathcal{C}$ the set of all $\mathit{KZG}$ poly commitments or vectors of such poly commitments. We also define the relation: 
\begin{align*}
\mathcal{R}_{\mathit{vec}, \mathit{com}}^c = \{& \mathbf{C} \in \mathcal{C}, \mathbf{input_2} \in \mathbf{\mathcal{D}_2}; \mathbf{witness_1}, \mathbf{witness_2}):  \\
& p_1(\mathbf{witness_2}, \mathbf{input_2}, \mathbf{witness_1}) =1  \ |\ c(\mathbf{witness_2}) = 1  \ \wedge\  \\
& \wedge\ p_2(\mathbf{witness_2}, \mathbf{input_2}, \mathbf{witness_1}) = 1\ \wedge \\
& \wedge\ \mathbf{C} = \mathbf{Com}(\mathbf{witness_2})\}
\end{align*}
\noindent Finally, for $\mathbf{input_1}$ part of $\mathit{state_1}$, we define $\mathit{SNARK.PartInput}$:
\begin{align*} 
&\mathit{SNARK.PartInput}(\mathit{srs}, \mathit{state_1},\mathcal{R}_{\mathit{vec}, \mathit{com}}^c) \\  
& \mathit{If \ }  c(\mathbf{input_1}) = 0 \textit{ then} \ \mathit{Return} \\
& \mathit{Else} \\
& \ \ \ \ \textit{Compute via interpolation on } \mathscr{P}_{\mathit{vec}}  \textit{ range } 
\mathit{Q_{1,\mathbf{input_1}}}(X), \ldots, \mathit{Q_{m, \mathbf{input_1}}}(X).\\
& \ \ \ \ \mathbf{C} = (\mathit{Com}(Q_{1,\mathbf{input_1}}(X)), \ldots, \mathit{Com}(Q_{m,\mathbf{input_1}}(X))) \\
& \ \ \ \ \mathit{state_2} =  \mathit{state_1} \cup \{ \mathbf{C} \} \textit{ then} \ \mathit{Return} (\mathit{state_2,  \mathbf{C}})
\end{align*}

\noindent With the above notation, \textbf{our compiler's Step 2 is:} \\
\noindent The alleged hybrid model SNARK $\mathscr{P}_{\mathit{vec}}^{h}$ for relation $\mathcal{R}_{\mathit{vec}, \mathit{com}}^c$ is:
\begin{itemize}
\item $\mathit{SNARK.Setup}$ and $\mathit{SNARK.KeyGen}$ are as for relation $\mathcal{R}^{c}_{\mathit{vec}}$.
\item $\mathit{SNARK.PartInput}$ for relation $\mathcal{R}^{c}_{\mathit{vec}}$ 
(see Lemma~\ref{le:compilation_step_1} in Section~\ref{supplementary_poly_protocols_appendix}) 
is replaced with $\mathit{SNARK.PartInput}$ for relation $\mathcal{R}_{\mathit{vec}, \mathit{com}}^c$.
\item $\mathit{SNARK.Prover}$ for relation $\mathcal{R}_{\mathit{vec}, \mathit{com}}^c$ is identical with 
$\mathit{SNARK.Prover}$ for relation $\mathcal{R}^{c}_{\mathit{vec}}$ (as compiled using Step 1) with the appropriate 
re-interpretation of the public inputs and witness.
\item $\mathit{SNARK.Verifier}$ for relation $\mathcal{R}_{\mathit{vec}, \mathit{com}}^c$ is identical with 
$\mathit{SNARK.Verifier}$ for relation $\mathcal{R}^{c}_{\mathit{vec}}$ (as compiled using Step 1) with the appropriate 
re-interpretation of the public inputs and such that $\mathit{SNARK.Verifier}$ for $\mathcal{R}_{\mathit{vec}, \mathit{com}}^c$ does
not compute the polynomial commitments to the polynomials defined by assumption (HMA.1.).
\end{itemize}
%\vspace{-0.2in}
\noindent \begin{lemma} 
\label{sergey_type_relations} 
Let $\mathscr{P}_{\mathit{vec}}$ be a ranged polynomial protocol for relation $\mathcal{R}^c_{\mathit{vec}}$ defined above and let 
$\mathscr{P}_{\mathit{vec}}^{*}$ be the hybrid model SNARK for relation $\mathcal{R}^c_{\mathit{vec}}$ secure in the AGM obtained 
by compiling $\mathscr{P}_{\mathit{vec}}$ using our compiler's Step 1. If the hybrid model assumptions (HMA.1.) - (HMA.3.) hold w.r.t. 
protocol $\mathscr{P}_{\mathit{vec}}$ and $\mathit{State}_{\mathcal{R}_{\mathit{vec}, \mathit{com}}} \neq \emptyset $ then 
$\mathscr{P}_{\mathit{vec}}^{h}$ as compiled using our compiler's Step 2 is a hybrid model SNARK for relation 
$\mathcal{R}_{\mathit{vec}, \mathit{com}}^c$ secure also in the AGM.
\end{lemma}

\begin{proof} Let $\mathcal{E}_{\mathit{KZG}}$ and $\mathcal{E}$ be the extractors from the knowledge-soundness definitions for the 
$\mathit{KZG}$ batch polynomial commitment scheme (as in Definition 3.1, Section 3 in~\cite{plonk}) and the hybrid model 
SNARK $\mathscr{P}^*_{\mathcal{R}}$ for relation $\mathcal{R}^c_{\mathit{vec}}$ (as per Definition~\ref{dfn_snark}), respectively. 
Let $\mathcal{A}$ be an adversary against knowledge soundness in the hybrid model w.r.t. 
$\mathscr{P}_{\mathit{vec}}^{h}$ and relation $\mathcal{R}_{\mathit{vec}, \mathit{com}}^c$ and let $\mathit{aux}_{\mathit{SNARK}} \in \mathcal{D}$ 
and let $\mathit{state_1} \in \mathit{State}_{\mathcal{R}_{\mathit{vec}, \mathit{com}}}$; let 
$(\mathbf{C},\mathit{state_2}) = \mathit{SNARK.PartInput}(\mathit{srs}, \mathit{state_1}, \mathcal{R}_{\mathit{vec}, \mathit{com}}^c)$. 
By the definition of $\mathit{SNARK.PartInput}$ for $\mathscr{P}_{\mathit{vec}}^{h}$, there exists 
$\mathbf{input_1}$ such that $\mathbf{C} = \mathbf{Com}(\mathbf{input_1})$ and $c(\mathbf{input_1}) = 1$. 
We denote by $(\mathbf{input_2}, \pi)$ the output of $\mathcal{A}(\mathit{srs}, \mathit{state_2},  \mathcal{R}_{\mathit{vec}, \mathit{com}}^c)$ 
and let $\mathcal{A}_1$ be the part of $\mathcal{A}$ that sends openings and batched proofs for the polynomial commitments in 
$\mathbf{C}$. \\

\noindent On the one hand, if $\mathit{SNARK.Verifier}(\mathit{srs}_{\mathit{vk}}, (\mathbf{C},\mathbf{input_2}),\pi,\mathcal{R}_{\mathit{vec}, \mathit{com}}^c)$ 
in $\mathscr{P}_{\mathit{vec}}^{h}$ accepts, then the $\mathit{KZG}$ verifier corresponding to 
$\mathcal{A}_1$ also accepts. When such an event takes place, then, \ewnp $\mathcal{E}_{\mathit{KZG}}$ extracts polynomials 
$Q'_1(X), \ldots, Q'_m(X)$ that represent witnesses for the vector $\mathbf{C}$ of commitments and the alleged openings of $\mathcal{A}_1$. 
Because the $\mathit{KZG}$ polynomial commitment scheme is binding and by the definition of 
$\mathit{SNARK.PartInput}$ for $\mathscr{P}_{\mathit{vec}}^{h}$, we obtain that $Q'_1(X) = Q_1(X), \ldots, Q'_m(X) = Q_m(X).$ 
Since per (HMA.3.), the set $\{Q_1(X), \ldots, Q_m(X)\}$ evaluates to $\mathbf{input_1}$ over the range over which $\mathscr{P}_{\mathit{vec}}$ 
was defined, \ewnp the witness polynomials extracted by $\mathcal{E}_{\mathit{KZG}}$ evaluate to $\mathbf{input_1}$. \\

\noindent On the other hand, if $\mathit{SNARK.Verifier}(\mathit{srs}_{\mathit{vk}}, (\mathbf{C},\mathbf{input_2}),\pi,\mathcal{R}_{\mathit{vec}, \mathit{com}}^c)$ 
in $\mathscr{P}_{\mathit{vec}}^{h}$ accepts, then \\
$\mathit{SNARK.Verifier}(\mathit{srs}_{\mathit{vk}}, (\mathbf{input_1},\mathbf{input_2}),\pi,\mathcal{R}_{\mathit{vec}}^c)$ 
in $\mathscr{P}_{\mathit{vec}}^{*}$ also accepts. In turn, this acceptance together with the fact that $\mathscr{P}_{\mathit{vec}}^{*}$ 
has knowledge-soundness as per Definition~\ref{dfn_snark}, it implies $\mathcal{E}$ \ewnp extracts $\mathbf{witness_1}$ 
such that $(\mathbf{input_1}, \mathbf{input_2}, \mathbf{witness_1}) \in \mathcal{R}_{\mathit{vec}}^{c} \ (\#).$ 

\noindent By the definition of $\mathit{SNARK.PartInput}$ for $\mathscr{P}_{\mathit{vec}}^{h}$ and the way $\mathbf{input_1}$ was defined, 
it holds that $c(\mathbf{input_1}) = 1$. Due to $(\#)$ and by the definition of relation $\mathcal{R}_{\mathit{vec}}^{c}$, 
the predicates: $p_1$($\mathbf{input_1}$, $\mathbf{input_2}$, $\mathbf{witness_1}$) $= 1$ and 
$p_2(\mathbf{input_1}, \mathbf{input_2}, \mathbf{witness_1}) = 1$ hold. If we let 
$\mathbf{witness_2} = \mathbf{input_1}$, then 
$(\mathbf{C} = \mathbf{Com}(\mathbf{input_1}), \mathbf{input_2}, \mathbf{witness_1}, \mathbf{input_1}) \in \mathcal{R}_{\mathit{vec}, \mathit{com}}^c,$ so 
using $\mathcal{E}_{\mathit{KZG}}$ and $\mathcal{E}$ we can build an extractor for any knowledge-soundness adversary $\mathcal{A}$ for alleged 
hybrid model SNARK $\mathscr{P}_{\mathit{vec}}^{h}$ for relation $\mathcal{R}_{\mathit{vec}, \mathit{com}}^c$, which concludes the proof.
\end{proof}

\noindent It is straightforward to apply the technique described above to our SNARKs $\Plah$ and $\Pah$ 
compiled in Step 2 and obtain relations $\Rlacom$ and $\Racom$ as described below such that they fulfil Lemma~\ref{sergey_type_relations}.\footnote{Due to our specific application to proof-of-stake blockchain context in which we make use of our custom SNARKs, 
the assumption/requirement that  $\mathit{State}_{\mathcal{R}_{\mathit{vec}, \mathit{com}}} \neq \emptyset$ for 
$\mathcal{R}_{\mathit{vec}, \mathit{com}} \in \{\Rlacom, \Racom \}$ is fulfilled.}
%\vspace{-0.1in}
\begin{align*}
& {\Rlacom} = \{(\mathbf{C} \in \mathcal{C}, \mathbf{bit} \in \mathbb{B}^n, \mathit{apk} \in \mathbb{F}^2; \mathbf{pk}) : 
\mathit{apk} = \sum_{i=0}^{n-2} [\mathit{bit_i}] \cdot \mathit{pk_i} \ | \\ & | \ \mathbf{pk} \in \ginn{1}^{n-1} \ \wedge \  \mathbf{C} = \mathbf{Com}(\mathbf{pk}) \} 
\end{align*}
%\vspace*{-0.75cm}
\begin{align*}
& {\Racom}  = \{(\mathbf{C} \in \mathcal{C}, \mathbf{b'} \in \mathbb{F}_{|\block|}^{\frac{n}{\block}}, \mathit{apk} \in \mathbb{F}^2;\mathbf{pk}, \mathbf{bit}) : 
\mathit{apk} = \sum_{i=0}^{n-2} [\mathit{bit_i}] \cdot \mathit{pk_i} \ | \\ & | \ \mathbf{pk} \in \ginn{1}^{n-1} \ \wedge \ \mathbf{bit} \in \mathbb{B}^n  \wedge b'_{j} = \sum_{i=0}^{\block -1}2^i \cdot \mathit{bit_{\block \cdot j + i}}, \forall j < \frac{n}{\block}  \wedge \  \mathbf{C} = \mathbf{Com}(\mathbf{pk}) \} 
\end{align*}
For completeness, we also include the full rolled out SNARK $\Pah$ for relation $\Racom$ in Section~\ref{sec:rolled_out} and we provide a comparison between PLONK universal 
SNARK and our custom SNARKs in Section~\ref{suplementary_plonk_comparison}.  
%\vspace*{-0.75cm}
\subsection{Our Instantiation for Committee Key Scheme}
\label{sec:inst_committee_key}
\noindent Given relations $\Rlacom$ and $\Racom$ described in Section~\ref{sec_two_step_compiler}, 
we are ready to present an instantiation for committee key scheme for aggregatable signatures 
(see definition in Section~\ref{sec:committee_key}); this, in turn, can be used to build an accountable light client. 
We instantiate $u$ and $v$ introduced in Section~\ref{sec:committee_key} as follows: let $u = n-1$, where $n$ was defined in 
Section~\ref{sec:lagrange} and we let $v \in \mathbb{N}, n-1 \leq v$, $v = \mathsf{poly}(\lambda)$, where by $v$ we denote the 
maximum number of validators that the system allows.

\begin{construction}(Committee Key Scheme for Aggregatable Signatures)
\label{inst:cks} In our implementation we call committee key scheme for 
aggregatable signatures the following instantiation of Definition~\ref{def: committee_key}, where $\mathcal{R} \in \{\Rlacom, \Racom\}$ 
as defined in the end of Section~\ref{sec_two_step_compiler}:
\begin{itemize}
\item $\mathit{CKS_{\mathcal{R}}.Setup}(v)$ calls the following algorithms: 
\begin{enumerate}
\item $\mathit{pp} \leftarrow \mathit{AS.Setup}(\mathit{aux_{\mathit{AS}}}= v+1)$ with  $\mathit{AS.Setup}$ part of Instantiation~\ref{insta:bls} where \\
$\ginn{1}$ is part of $\mathit{pp}$ (see notation in Section~\ref{sec:bls});
\item $\mathit{srs} \leftarrow \mathit{SNARK.Setup}(\mathit{aux_{\mathit{SNARK}}} = (v, 3v))$ with \\
$\mathit{srs}=([1]_{\indexoneout}, [\tau]_{\indexoneout}, \ldots, [\tau^{3v}]_{\indexoneout}, [1]_{\indextwoout}, [\tau]_{\indextwoout})$ ;
\item $(\mathit{rs}_{\mathit{pk}}, \mathit{rs}_{\mathit{vk}}) \leftarrow \mathit{SNARK.KeyGen}(\mathit{srs}, \mathcal{R})$ with \\ 
$(\mathit{rs}_{\mathit{pk}}, \mathit{rs}_{\mathit{vk}}) =  (([1]_{\indexoneout}, [\tau]_{\indexoneout}, \ldots, [\tau^{3v}]_{\indexoneout}),([1]_{\indexoneout}, [1]_{\indextwoout}, [\tau]_{\indextwoout}))$ \\
where %$\mathcal{R} \in \{\Rlacom, \Racom \}$ is one of the accountable relations defined in the end of section~\ref{sec_two_step_compiler} and 
the notation $[\ldots]_{\indexoneout}$ and $[\ldots]_{\indextwoout}$ was defined in Section~\ref{sec:pairings}.
\end{enumerate}

\item $\mathit{ck} \leftarrow \mathit{CKS_{\mathcal{R}}.GenerateCommitteeKey}(\mathit{rs_{pk}}, (\mathit{pk_i})_{i=1}^{n-1})$, where 
$\mathit{CKS_{\mathcal{R}}.GenerateCommitteeKey}$ first checks whether $(\mathit{pk_i})_{i=1}^{n-1} \in \ginn{1}^{n-1}$; 
if this holds, it outputs $\bot$; otherwise, the algorithm \\ $\mathit{CKS_{\mathcal{R}}.GenerateCommitteeKey}$ continues with the 
computations described below: \\
\noindent Let $\mathbf{pkx} = (\mathit{pkx_{1}}, \ldots, \mathit{pkx_{n-1}})$, $\mathbf{pky} = (\mathit{pky_{1}}, \ldots, \mathit{pky_{n-1}})$, $\forall i \in [n-1]$, $\mathit{pk_i} = (\mathit{pkx_i}, \mathit{pky_i}) \in \mathbb{F}^{2}$. \\
\noindent Let $pkx(X) = \sum_{i=0}^{n-2} \mathit{pkx_{i+1}} \cdot L_i(X)$, $pky(X) = \sum_{i=0}^{n-2} \mathit{pky_{i+1}} \cdot L_i(X)$.\\ 
\noindent Let $[pkx]_{\indexoneout} = pkx(\tau) \cdot [1]_{\indexoneout}$, $[pky]_{\indexoneout} = pky(\tau) \cdot [1]_{\indexoneout}$.\\ 
\noindent Let $\mathit{ck} = ([pkx]_{\indexoneout}, [pky]_{\indexoneout})$.\\
\noindent Note that $\mathbb{F}$ and $\{L_i(X)\}_{i=1}^{n-2}$ are as defined in Section~\ref{sec:lagrange}. 

\item $\pi = (\pi_{SNARK}, \mathit{apk}) \leftarrow \mathit{CKS_{\mathcal{R}}.Prove}
(\mathit{rs}_{\mathit{pk}}, \mathit{ck}, (\mathit{pk_i})_{i=1}^{n-1}, (\mathit{bit_i})_{i=1}^{n-1})$ 
where $\mathit{CKS_{\mathcal{R}}.Prove}$ calls \\ 
$\mathit{apk} = \sum_{i=1}^{n-1} \mathit{bit_i} \cdot \mathit{pk_i} \leftarrow \mathit{AS.AggregateKeys}(\mathit{pp}, (\mathit{pk_i})_{i:\mathit{bit_i = 1}})$ 
as defined in Instantiation~\ref{insta:bls} and $\pi_{SNARK} \leftarrow \mathit{SNARK.Prove}(\mathit{rs_{pk}}, (x,w), \mathcal{R})$, 
for $\mathcal{R} \in \{\Rlacom, \Racom \}$ where 
\begin{equation*}
\begin{cases}
 (x = (\mathit{ck}, (\mathit{bit_i})_{i=1}^{n-1}||0, \mathit{apk}), w = ((\mathit{pk}_i)_{i=1}^{n-1}) & \text{ if } \mathcal{R} = \Rlacom,\\
 (x = (\mathit{ck}, \mathbf{b'}, \mathit{apk}), w = ((\mathit{pk}_i)_{i=1}^{n-1}, (\mathit{bit_i})_{i=1}^{n-1}||0) & \text{ if } \mathcal{R} = \Racom, \\
\end{cases}       
\end{equation*}
where $\mathbf{b'}$ is the vector of field elements formed from blocks of size $\mathsf{block}$ of bits from vector 
$(\mathit{bit_i})_{i=1}^{n-1}||0$ and $\mathsf{block}$ is the highest power of 2 smaller than the size of a field element in $\mathbb{F}$. 

\item $0/1 \leftarrow \mathit{CKS_{\mathcal{R}}.Verify}(\mathit{pp}, \mathit{rs}_{\mathit{vk}}, \mathit{ck}, m, \mathit{asig}, \pi, \mathbf{bitvector})$ 
parses $\pi$ to retrieve $\pi_{\mathit{SNARK}}$ and $\mathit{apk}$ and it calls $\mathit{AS.Verify(\mathit{pp}, \mathit{apk}, m, \mathit{asig})}$ 
as defined in Instantiation~\ref{insta:bls} and it also calls \\ $\mathit{SNARK.Verify}(\mathit{rs_{vk}}, x, \pi_{\mathit{SNARK}}, \mathcal{R})$ 
(where $\pi_{\mathit{SNARK}}$, $x$ and $\mathcal{R}$ are as defined in the paragraph above with the only difference that $(\mathit{bit_i})_{i=1}^{n-1}$ 
represents the first $n-1$ bits of $\mathbf{bitvector}$, padded with $0$s, if not sufficiently many exist in $\mathbf{bitvector}$); 
it outputs $1$ if both algorithms output $1$ and it outputs $0$ otherwise.
\end{itemize}
\end{construction}

\begin{theorem}Given the hybrid model SNARK scheme secure for relation $\mathcal{R} \in \{ \Rlacom, \Racom\}$ as 
obtained using our two-step compiler in Section~\ref{sec_two_step_compiler} and the aggregatable signature scheme $\mathit{AS}$ 
                     as per Instantiation~\ref{insta:bls} (which fulfils Definition~\ref{def:aggregate_signatures}) with the additional 
                     specification that $\mathit{aux}_{\mathit{AS}} = v+1$ and choosing $v = n-1$, 
if we assume that an efficient adversary (against the soundness of) $\mathit{CKS}_{\mathcal{R}}$ outputs public keys only from the source group $\ginn{1}$,  
then the committee key scheme $\mathit{CKS}_{\mathcal{R}}$ as per Instantiation~\ref{inst:cks} is secure with respect to Definition~\ref{def: committee_key}.
\end{theorem}
\begin{proof} We give a full proof in Section~\ref{supplementary_proof_sec_cks}. 
\end{proof}