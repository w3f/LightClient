\section{Choosing $h$ when $\einn = \ginn{1}$}
\label{sec:other_choice_h}

\noindent For the polynomial protocols and custom SNARKs we have designed in Section~\ref{sec:snarks}, we have chosen 
$h \in $\einn \setminus \ginn{1}$. However, we have not covered so far the case when $\einn = \ginn{1}$ and how to 
choose $h$ in such a situation. Our current section will give a guide for that. In fact, if $\einn = \ginn{1}$, we provide a 
method of choosing $h$ that will be suitable not only for our custom SNARKs, but also for any other SNARK that proves the 
correctness of an aggregated public key (i.e., $\mathit{apk}$) for an aggregatable signature scheme, among other modelled constraints. 
Let $\mathcal{H}$ be a hash function, $\mathcal{H}: \{0, 1\} \rightarrow \mathbb{F}$ such that $\mathcal{H}$ is used for the 
Fiat-Shamir transformation of a (custom) succinct argument of knowledge (including the proof of correctness of $\mathit{apk}$) 
into its non-interactive version. Let $x$ be the public input corresponding to the above (custom) succinct argument of knowledge. 
Note that in case of hybrid model SNARKs as we define in this work, the public input includes the partial input. For a concrete example, see 
our full rolled out custom SNARK in Section~\ref{sec:rolled_out}. Then, the prover and the verifier compute $h$, for example as 
$h = \mathcal{H}(h,"input dependent elliptic curve addition starting point")$. Intuitively, in the random oracle model 
(which is already an assumption we have to use for the Fiat-Shamir transformation), $h$ is thus an elliptic curve point on $\einn$, uniformly 
distributed on $\einn = \ginn{1}$. Hence, for a large enough elliptic curve group (i.e., an elliptic curve group for which the number of points is exponential 
in the security parameter), the probability of $h$ plus some elliptic curve points
