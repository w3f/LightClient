\label{sec:snarks}

We construct two related SNARKS, each of them allowing a prover to convince an 
efficient verifier that an alleged aggregated public key has indeed been computed correctly as an aggregate 
of a vector of public keys for which two succinct commitments (to the $x$ and $y$ affine coordinates of points) are publicly known. The differences between the two  
constructions stem from how a \emph{bitvector} with one bit associated to each public key 
(necessary to signal the inclusion or omission of the respective public key w.r.t. the aggregate key) 
is used as part of the verifier's public input. We describe a 
\emph{basic accountable SNARK} (the bitvector is represented as a sequence of $0/1$ field elements) and a \emph{packed accountable SNARK} (the bitvector is 
partitioned into equal blocks of consecutive binary bits which are represented by a field element per block). 
%Each of our SNARKs implements a conditional NP relation bearing the 
%same name as the SNARK it implements. Note that the names ``basic accountable" (for short, ``basic'') and 
%``packed accountable" (for short, ``packed'') do not refer to the security of the respective SNARK but they summarise properties 
%of the underlying sets of constraints that define the SNARKs, and, hence their use case. 
We finally transform basic and packed accountable SNARKs into SNARKs for building accountable light client systems. For reasons of space, the packed accountable scheme is described in Appendix \ref{sec_a}.

\noindent To compile our desired SNARKs we proceed as follows:
\begin{itemize}
\item In Sections \ref{sec_la} and \ref{sec_a} we define vector-based conditional NP relations $\Rla$ (i.e., basic accountable) and $\Ra$ (packed accountable) and we design two ranged polynomial protocols for these relations. The ranged polynomial protocol notion originates in~\cite{plonk}; 
we review it (including a refinement) in Section~\ref{supplementary_poly_protocols_appendix};  
\item In Section~\ref{sec_two_step_compiler} we define a two-steps PLONK-inspired compiler which we use to compile the ranged polynomial protocols into 
SNARKs for two novel mixed vector and trusted polynomial commitments conditional NP relations which we denote by 
$\Rlacom$ and $\Racom$, respectively. 
\item In Section~\ref{sec:inst_committee_key} we give an instantiation for committee key scheme for aggregatable signatures which uses our SNARKS 
 and our instantiation for BLS aggregatable signatures from Section~\ref{sec:bls}. 
%\item We include in Section~\ref{suplementary_plonk_comparison} in the appendices a comparison between PLONK~\cite{plonk} and our custom SNARKs. 
\end{itemize}

\noindent We define our conditional NP relations over $\mathbb{F}$, i.e., the base field of $\einn$. 
Our SNARKs provers' circuits are defined as well over $\mathbb{F}$ as the scalar field of $\eout$. The vector of public keys, which is part of the public input for both of our 
relations $\Rla$ and $\Ra$, and is denoted by $\mathbf{pk} = (\mathit{pk_0}, \ldots, \mathit{pk_{n-2}})$, is a vector of pairs with each component 
in $\mathbb{F}$. This vector has size $n-1$ ($n$ defined in 
Section~\ref{sec:lagrange}). For $\Rla$ we denote 
the $n$ components bitvector by $\mathbf{bit} = (\mathit{bit_0}, \ldots, \mathit{bit_{n-1}})$ 
(meaning that each component belongs to the set $\{0,1\} \subset \mathbb{F}$), 
while the $\Ra$ relation is defined using the \emph{compacted bitvector} 
$\mathbf{b'} = (\mathit{b'_{0}}, \ldots, \mathit{b'_{\frac{n}{\block}-1}})$ of $\frac{n}{\block}$ field elements, 
each of which is $\block$ binary bits long ($\block$ has been defined in Section~\ref{sec:lagrange}). 
Each of the bits in the bit representation of these field elements signals the 
inclusion (or exclusion) of the index-wise corresponding public keys into the aggregated public key $\mathit{apk}$. The last bit of field element $\mathit{b'_{\frac{n}{\block}-1}}$ as well as the $n$-th component $\mathit{bit_{n-1}}$ do not correspond to any public key, 
but, as will become clear in the following, they have been included for easier design of constraints. \\ 
\vspace{-0.009in}
\noindent We denote by $H$ the multiplicative subgroup of $\mathbb{F}$ generated 
by $\omega$ as defined in Section~\ref{sec:lagrange}. We denote by \\ $\mathit{incl}(a_0, \ldots, a_{n-2})$ the inclusion 
predicate that checks if \\ $(a_0, \ldots, a_{n-2}) \in \ginn{1}^{n-1}$. Moreover let $h = (\mathit{h_x}, \mathit{h_y})$ 
be some fixed, publicly known element in $\einn \setminus \ginn{1}$. We denote by $(a_x, a_y)$ the affine representation in 
$x$ and $y$ coordinates of $a \in \einn$ and by $\oplus$ the point addition in affine coordinates on the elliptic curve $\einn$. 
We denote by $\mathbb{B} = \{0,1\} \subset \mathbb{F}$. \\
\vspace{-0.15in}

%\noindent Finally, as mentioned in section~\ref{sec:conditional_relations}, the interpretation of adding explicit domains to public 
%inputs in the definition of conditional NP relations is that the honest parties (in our case, both the polynomial protocol verifiers and the SNARKs verifiers 
%as defined in this section below) parse the public inputs according to the specified domains without any further checks. Any checks or 
%computations that the honest parties perform regarding the public inputs are explicitly described as part of the protocols followed by 
%the honest parties.
\vspace{-0.1in}
\subsection{Basic Accountable Protocol}
\label{sec_la}
%\vspace{0.1in}
%\noindent We start by describing our conditional basic accountable relation $\Rla$ and the 
%corresponding $H$-ranged polynomial protocol $\Pla$. %For brevity, we omit the security parameter 
%$\lambda$ whenever we refer to any conditional NP relations for which we build SNARKs. \\
%\vspace{-0.05in}
\noindent \textsf{Conditional Basic Accountable Relation $\Rla$}  
\vspace{-0.055in}
\begin{equation*}
\begin{split}
\Rla = & \{(\mathbf{pk} \in ({\mathbb{F}^2})^{n-1}, \mathbf{bit} \in \mathbb{B}^n,
\mathit{apk} \in \mathbb{F}^2; \_) : \mathit{apk} = \\ &\sum_{i=0}^{n-2} [\mathit{bit_i}] \cdot \mathit{pk_i} \ | \ \mathbf{pk} \in \ginn{1}^{n-1} \} 
\end{split}
\end{equation*}
\vspace{-0.008in}
\noindent where $\mathbf{pk} = (\mathit{pk_0}, \ldots, \mathit{pk_{n-2}})$ and $\mathbf{bit} = (\mathit{bit_0}, \ldots, \mathit{bit_{n-1}})$.\\ 

\noindent Next, we introduce the following Lagrange interpolation polynomials of degree at most $n-1 = |H|$ over cyclic group $H$ (Section~\ref{sec:lagrange}): 
$b(X)$ - interpolates the bits of bitvector $\mathit{bit}$; $pkx(X)$, $pky(X)$ - interpolate all public keys' 
$x$ and $y$ coordinates, respectively; $kaccx(X)$, $kaccy(X)$ - interpolate $x$ and $y$ coordinates, respectively, 
of the iterative partial aggregate sum of the actual signing validators' public keys. We also define five polynomial identities $id_1(X), \ldots, id_5(X)$ supporting the following intuition: $id_1(X)$, $id_2(X)$ 
ensure the $x$ and, respectively, the $y$ coordinates of the iterative partial aggregate sums of actual signing validators public keys (up to each index $i \leq n-2$) 
follow formulas $(\ast)$, $(\ast\ast)$ from Observation 1 which gives all possible cases of 
complete curve point addition when the second curve point is multiplied by a bit; $id_3(X)$, $id_4(X)$ 
ensure first partial aggregate sum is $h$ and the total aggregate sum is $h + \mathit{apk}$; this is necessary in order to ensure the addition of the public keys (i.e., elliptic curve points) 
never falls into condition (3) defined in Observation 1, which recursively implies the partial aggregate sum at every step is a well defined curve point, hence, it is a suitable input for the next step; 
$id_5(X)$ ensures $b(X)$ evaluates to bits over $H$. Together, $id_1(X)$ to $id_4(X)$ define 
the $H$-ranged polynomial protocol $\Pla$ for relation $\Rla$; $id_5(X)$ will be used to prove a more general result, applicable also for the 
$H$-ranged polynomial protocol $\Pa$ for relation $\Ra$ (Section~\ref{sec_a}). In more detail, we have:  \\

\noindent \textsf{Polynomials as Computed by Honest Parties} 
\vspace{-0.1cm}
\begin{align*}
&\mathsf{b(X)} = \sum_{i=0}^{n-1} \mathit{bit_i} \cdot \mathsf{L_i(X)}; \\ 
& \mathsf{pkx(X)} =  \sum_{i=0}^{n-2} \mathit{pkx_i} \cdot \mathsf{L_i(X)}; \mathsf{pky(X)} =  \sum_{i=0}^{n-2} \mathit{pky_i} \cdot \mathsf{L_i(X)} \\
&\mathsf{kaccx(X)}  =  \sum_{i=0}^{n-1} \mathit{kaccx_i} \cdot \mathsf{L_i(X)}; \mathsf{kaccy(X)}  = \sum_{i=0}^{n-1} \mathit{kaccy_i} \cdot \mathsf{L_i(X)}, 
\end{align*}
\noindent where $(\mathit{pkx_0}, \ldots, \mathit{pkx_{n-2}})$ 
and $(\mathit{pky_0}, \ldots, \mathit{pky_{n-2}})$ are computed such that $\forall i \in \{0, \ldots, n-2\}$, $\mathit{pk_i}$ 
is interpreted as a pair $(\mathit{pkx_i}, \mathit{pky_i})$ with its components in $\mathbb{F}$; we also have 
$(\mathit{kaccx_{0}}, \mathit{kaccy_{0}}) = (\mathit{h_x}, \mathit{h_y})$ and 
$(\mathit{kaccx_{i+1}}, \mathit{kaccy_{i+1}}) = \\ (\mathit{kaccx_{i}}, \mathit{kaccy_{i}}) \oplus \mathit{bit_i}(\mathit{pkx_{i}}, \mathit{pky_{i}})$, 
$\forall i < n-1$. %Note that in the last relation $\mathit{bit_i}$ is not interpreted as a field element anymore but as a binary bit.
\\
\vspace{-0.15in}

\noindent \textsf{Polynomial Identities} 
\begin{align*}
& id_1(X) = (X-\omega^{n-1}) \cdot [b(X) \\ &\cdot ((kaccx(X)-pkx(X))^2 \cdot (kaccx(X)+ pkx(X) + \\ 
& + kaccx(\omega\cdot X)) - (pky(X) - kaccy(X))^2) +  (1-b(X))\\& \cdot (kaccy(\omega\cdot X) - kaccy(X))] \\
& id_2(X)  =  (X-\omega^{n-1})\cdot [b(X) \cdot ((kaccx(X) - pkx(X)) \\& \cdot (kaccy(\omega \cdot X) + kaccy(X)) - \\
& - (pky(X) - kaccy(X)) \cdot (kaccx(\omega \cdot X) - kaccx(X))) + \\& (1-b(X)) \cdot   \cdot(kaccx(\omega \cdot X) - kaccx(X))] \\
& id_3(X)  =  (kaccx(X) - h_x)\cdot L_0(X) + (kaccx(X) - (h\oplus apk)_{x}) \\ & \cdot L_{n-1}(X)  
 id_4(X) =  (kaccy(X) - h_y)\cdot L_0(X) + (kaccy(X) \\ & - (h\oplus apk)_{y}) \cdot L_{n-1}(X) \\
& id_5(X) =  b(X)(1-b(X)).
\end{align*}

\noindent %Polynomial identity 
$id_5(X)$ is not strictly needed for % defining ranged polynomial protocols for 
$\Rla$, but %included 
%here to ease presentation and  for %proofs consistency for the next 
for the section~\ref{sec_a}. \\
\vspace{-0.10in}

\noindent \textsf{$H$-ranged Polynomial Protocol $\Pla$ for Conditional NP Relation $\Rla$} describes the interaction of the prover 
$\mathcal{P}_{poly}$, the verifier $\mathcal{V}_{poly}$ and the trusted third party $\mathcal{I}$ 
in accordance to 
Definition~\ref{def_ranged_poly_protocol} from Section~\ref{supplementary_poly_protocols_appendix}. \\
\vspace{-0.15in}
%\noindent \textsf{Protocol $\Pla$} \\

\noindent $\mathcal{P}_{poly}$ and $\mathcal{V}_{poly}$ know public input 
$\mathbf{bit} \in \mathbb{B}^n$, $\mathbf{pk} \in (\mathbb{F}^2)^{n-1}$ and $\mathit{apk} \in (\mathbb{F})^2$ 
which are interpreted as per their %respective 
domains.
\begin{enumerate}
\item $\mathcal{V}_{poly}$ computes $b(X)$, $pkx(X)$, $pky(X)$.
\item $\mathcal{P}_{poly}$ sends polynomials $kaccx(X)$ and $kaccy(X)$ to $\mathcal{I}$. 
\item $\mathcal{V}_{poly}$ asks $\mathcal{I}$ to check whether the following polynomial relations hold over range $H$ 
$$id_i(X) = 0, \forall i \in [4].$$
\item $\mathcal{V}_{poly}$  accepts if $\mathcal{I}$'s checks verify. 
\end{enumerate}

\noindent We show $\Pla$ is an $H$-ranged polynomial protocol for 
conditional NP relation $\Rla$. For this, we first prove:
\vspace{-0.05in}
\begin{test_claim} Assume that $\forall i < n-1$ such that $\mathit{bit}_i = 1$, $pk_i = (pkx_i, pky_i) \in \ginn{1}$. 
If polynomial identities $id_i(X) = 0, \forall i \in [5],$ hold over range 
$H$ and the polynomial $b(X)$ has been constructed via interpolation on $H$ such that $b(\omega^i) = \mathit{bit}_i, \forall i <n$ then $\mathit{bit}_i \in \mathbb{B} = \{0,1\} \subset \mathbb{F}, \forall i <n$ \\
$(kaccx_{0}, kaccy_{0}) = (h_x, h_y)$, $(kaccx_{n-1}, kaccy_{n-1}) = (h_x, h_y)\\ \oplus (apk_x, apk_y)$, 
$(kaccx_{i+1}, kaccy_{i+1}) =  (kaccx_{i}, kaccy_{i}) \oplus \mathit{bit}_i(pkx_{i}, pky_{i})$, $\forall i < n-1$.
%where in the last relation $\mathit{bit_i}$ should not be interpreted as a field element but as a binary bit
\label{claim:keys_affine_comm}
\end{test_claim}
\vspace{-0.08in}

\begin{proof} Everything but the last property in the claim is easy to derive from polynomial identities 
$id_3(X) =0, id_4(X )= 0, id_5(X) = 0$ holding over $H$. To prove the remaining property, we remind 
the incomplete addition formulae for curve points in affine coordinates, over elliptic curve in short Weierstrasse form and state:\\ 
\vspace{-0.1in}

\noindent \textit{Observation 1:} \label{obs} Suppose that $\mathit{bit} \in \{0,1\}$, $(x_1,y_1)$ is a point on an elliptic curve in 
short Weierstrasse form, and, if $\mathit{bit} = 1$, so is $(x_2,y_2)$. We claim that the following equations: 
\begin{align*}
&\mathit{bit}((x_1 - x_2)^2 (x_1 + x_2 + x_3) - (y_2 - y_1)^2 ) + (1 - \mathit{bit})(y_3 - y_1) \\ & =0 \ (\ast)\\
&\mathit{bit}((x_1 - x_2)(y_3 + y_1) - (y_2 - y_1)(x_3 - x_1)) + (1 - \mathit{bit})(x_3 - x_1) \\ & =0 \ (\ast\ast)
\end{align*}

\noindent hold if and only if one of the following three conditions hold 

\begin{enumerate}
\item \label{cond1} $\mathit{bit}=1$ and $(x_1,y_1)\oplus(x_2,y_2)=(x_3,y_3)$ and $x_1 \neq x_2$
\item \label{cond2} $\mathit{bit}=0$ and $(x_3,y_3)=(x_1,y_1)$ 
\item  \label{cond3} $\mathit{bit}=1$ and $(x_1,y_1)=(x_2,y_2)$\footnote{Note that under condition~\ref{cond3}, $(x_3,y_3)$ 
can be any point whatsoever, maybe not even on the curve. The same holds true for $(x_2, y_2)$ under the condition~\ref{cond2}.}.
\end{enumerate}

\noindent It is easy to see that each of the conditions~\ref{cond1},\ref{cond2},\ref{cond3} above implies equations $(\ast)$ and $(\ast \ast)$.
\noindent For the implication in the opposite direction, if we assume that $(\ast)$ and $(\ast \ast)$ hold, then \\
\vspace{-0.1in}

\noindent \textit{Case a:} For $\mathit{bit}=0$, the first term of each equation $(\ast)$ and $(\ast \ast)$ vanishes, 
leaving us with $y_3-y_1=0$ and $x_3-x_1=0$ which are equivalent to condition~\ref{cond2}. \\
\vspace{-0.1in}

\noindent \textit{Case b:} For $\mathit{bit}=1$ and $x_1=x_2$, by simple substitution in $(\ast)$ and $(\ast \ast)$, 
we obtain $y_1 = y_2$, i.e., condition~\ref{cond3}.  \\
\vspace{-0.1in}

\noindent \textit{Case c:} For $\mathit{bit}=1$ and $x_1 \neq x_2$, then we can substitute
$\beta=\frac{y_2-y_1}{x_2-x_1}$ into equations $(\ast)$ and $(\ast \ast)$, leaving us with
$$x_1+x_2+x_3=\beta^2 \textrm{ and } y_3+y_1=\beta(x_3-x_1).$$
which are the usual formulae for short Weierstrass form addition of affine coordinate points when $x_1 \neq x_2$ 
so this is equivalent to Condition~\ref{cond1}. \\
\vspace{-0.1in}

\noindent We apply the above \textit{Observation} by noticing that if $id_1(X)$ and $id_2(X)$ hold over $H$, 
then $(\ast)$ and $(\ast \ast)$ hold with $(x_1, y_1)$ substituted by $(kaccx_i,kaccy_i)$, $(x_2, y_2)$ 
substituted by $(pkx_i, pky_i)$, $(x_3, y_3)$ substituted by $(kaccx_{i+1},kaccy_{i+1})$ and $\mathit{bit}$ 
substituted by $\mathit{bit}_i$ for $0 \leq i \leq n-2$ %, where $\mathit{bit_i}$ should not be interpreted as a field element but as binary  bit
. Moreover, since $(kaccx_{0}, kaccy_{0}) = (h_x, h_y) \in \einn \setminus \ginn{1}$ 
and if $(pkx_i, pky_i) \in \ginn{1}$ whenever $\mathit{bit}_i = 1$, then $\forall i < n-1$ 
equations $(\ast)$ and $(\ast \ast)$ obtained after the substitution defined above are equivalent to either 
condition~\ref{cond1} or condition~\ref{cond2}, but never condition~\ref{cond3}, so the result of the sum (i.e., $(kaccx_{i+1}, kaccy_{i+1})$, $0\leq i \leq n-2$) is, 
by induction, at each step a well-defined point on the curve.% and this concludes our proof.
\end{proof}
\vspace{-0.1in}

\begin{corollary} Assume $\forall i < n-1$ 
such that $\mathit{bit}_i = 1$, $pk_i = (pkx_i, pky_i) \in \ginn{1}$. 
If the polynomial identities $id_i(X) = 0, \forall i \in [4],$ hold over range $H$ and 
$\mathit{bit_i} \in \mathbb{B}$, $\forall i < n-1$ and $b(X) = \sum_{i=0}^{n-1} \mathit{bit_i} \cdot L_i(X)$
then:  \\
$(kaccx_{0}, kaccy_{0}) = (h_x, h_y)$, 
$(kaccx_{n-1}, kaccy_{n-1}) = (h_x, h_y) \\ \oplus (apk_x, apk_y)$, 
$(kaccx_{i+1}, kaccy_{i+1}) =  (kaccx_{i}, kaccy_{i})\\ \oplus \mathit{bit_i}(pkx_{i}, pky_{i})$, $\forall i < n-1$.
%where in the last relation $\mathit{bit_i}$ should not be interpreted as a field element but as a binary bit.
\label{corollary:keys_affine_comm}
\end{corollary}
\vspace{-0.2in}

\begin{proof}The proof follows trivially from the general result stated by Claim~\ref{claim:keys_affine_comm}. 
\end{proof}
\vspace{-0.1in}

\begin{lemma} 
\label{le:ba}
$\Pla$ as described above is an $H$-ranged polynomial 
protocol for conditional NP relation $\Rla$.
\end{lemma}
\vspace{-0.15in}

\begin{proof}
If $(\mathbf{bit},\mathbf{pk}, \mathit{apk}) \in \Rla$ holds, 
meaning that $\mathbf{bit} \in \mathbb{B}^n$ and $\mathbf{pk} \in \ginn{1}^{n-1}$ and $\mathit{apk} = \sum_{i=0}^{n-2} [\mathit{bit_i}] \cdot \mathit{pk_i}$ hold, 
then it is easy to see that the honest prover $\mathcal{P}_{poly}$ in $\Pla$ will convince the honest verifier $\mathcal{V}_{poly}$ in 
$\Pla$ to accept with probability $1$ so perfect completeness holds. 
For knowledge-soundness, if the verifier $\mathcal{V}_{poly}$ in $\Pla$ accepts, 
then the extractor $\mathcal{E}$ is trivial since $\Rla$ has no witness.  
We need to show that if $\mathbf{pk} \in \ginn{1}^{n-1}$ and the verifier in $\Pla$ accepts, 
then $(\mathbf{bit},\mathbf{pk}, \mathit{apk}) \in \Rla$ holds, which given our definition for conditional relation is 
equivalent to proving that $\mathit{apk} = \sum_{i=0}^{n-2} [\mathit{bit_i}] \cdot \mathit{pk_i}$ holds. This is due to 
Corollary~\ref{corollary:keys_affine_comm}. \end{proof}
\vspace{-0.15in}

\subsection{Our Custom SNARKs}
\label{sec_custom_snarks}
To design our accountable light client systems as introduced in Section~\ref{sec_intro_committee} and fully detailed in Section~\ref{new_light_client}, 
we need to further refine and then compile relations $\Rla$ and $\Ra$ into appropriate SNARKs such that 
the long public input $\mathbf{pk} \in ({\mathbb{F}^2})^{n-1}$ is replaced by a 
pair of succinct commitments and, $\mathbf{pk}$ becomes a witness for the resulting refined relations. As far as we are aware, such a compiler does not exist. 
Thus, we design a two-step compiler to obtain SNARKs for relations with such properties. We are interested in relations below. 
 \begin{align*}
& {\Rlacom} = \{(\mathbf{C} \in \mathcal{C}, \mathbf{bit} \in \mathbb{B}^n, \mathit{apk} \in \mathbb{F}^2; \mathbf{pk}) : \\
& \mathit{apk} = \sum_{i=0}^{n-2} [\mathit{bit_i}] \cdot \mathit{pk_i} \ | \ \mathbf{pk} \in \ginn{1}^{n-1} \ \wedge \  \mathbf{C} = \mathbf{Com}(\mathbf{pk}) \} 
\end{align*}
%\vspace*{-0.75cm}
\begin{align*}
& {\Racom}  = \{(\mathbf{C} \in \mathcal{C}, \mathbf{b'} \in \mathbb{F}_{|\block|}^{\frac{n}{\block}}, \mathit{apk} \in \mathbb{F}^2;\mathbf{pk}, \mathbf{bit}) : \\
& \mathit{apk} = \sum_{i=0}^{n-2} [\mathit{bit_i}] \cdot \mathit{pk_i} \ | \ \mathbf{pk} \in \ginn{1}^{n-1} \ \wedge \ \mathbf{bit} \in \mathbb{B}^n \ \wedge\\ 
& \wedge b'_{j} = \sum_{i=0}^{\block -1}2^i \cdot \mathit{bit_{\block \cdot j + i}}, \forall j < \frac{n}{\block}  \wedge \  \mathbf{C} = \mathbf{Com}(\mathbf{pk}) \} 
\end{align*}
The two-step compiler is described  
in Section~\ref{sec_two_step_compiler}, with the first 
step being the standard PLONK compiler %(Section 4.7 of 
\cite{plonk} and the second step being amenable for compiling into SNARKs a %specific 
generalisation of the two NP relations just detailed above.
\vspace{-0.05in}

\subsection{Our Instantiation for CKS}
\label{sec:inst_committee_key}
\noindent Given relations $\Rlacom$ and $\Racom$ described in short in Section~\ref{sec_custom_snarks} and %in full in Section~\ref{sec_two_step_compiler}
, we present an instantiation for committee key scheme defined in Section~\ref{sec:committee_key}; this is used to build an accountable light client system %(Section~\ref{new_light_client})
. 
We instantiate $u$ and $v$ from Section~\ref{sec:committee_key} as $u = n-1$, ($n=|H|$ from 
Section~\ref{sec:lagrange}) and $v \in \mathbb{N}, n-1 \leq v$, $v = \mathsf{poly}(\lambda)$, where $v$ is the 
maximum number of validators.

\begin{construction}(Committee Key Scheme for Aggregatable Signatures)
\label{inst:cks} In our implementation we use the following instantiation of Definition~\ref{def: committee_key} for one of $\mathcal{R} \in \{\Rlacom, \Racom\}$:
\begin{itemize}
\item $\mathit{CKS_{\mathcal{R}}.Setup}(v)$ calls algorithms: 
\begin{enumerate}
\item $\mathit{pp} \leftarrow \mathit{AS.Setup}(\mathit{aux_{\mathit{AS}}}= v+1)$ with  $\mathit{AS.Setup}$ part of Instantiation~\ref{insta:bls} and 
$\ginn{1}$ part of $\mathit{pp}$ (see notation in Section~\ref{sec:bls});
\item $\mathit{srs} \leftarrow \mathit{SNARK.Setup}(\mathit{aux_{\mathit{SNARK}}} = (v, 3v))$ with \\
$\mathit{srs}=([1]_{\indexoneout}, [\tau]_{\indexoneout}, \ldots, [\tau^{3v}]_{\indexoneout}, [1]_{\indextwoout}, [\tau]_{\indextwoout})$;
\item $(\mathit{rs}_{\mathit{pk}}, \mathit{rs}_{\mathit{vk}}) \leftarrow \mathit{SNARK.KeyGen}(\mathit{srs}, \mathcal{R})$ with \\ 
$(\mathit{rs}_{\mathit{pk}}, \mathit{rs}_{\mathit{vk}}) =  (([1]_{\indexoneout}, [\tau]_{\indexoneout}, \ldots, [\tau^{3v}]_{\indexoneout}), \\ 
([1]_{\indexoneout}, [1]_{\indextwoout}, [\tau]_{\indextwoout}))$ where %$\mathcal{R} \in \{\Rlacom, \Racom \}$ is one of the accountable relations defined in the end of section~\ref{sec_two_step_compiler} and 
the notation $[\ldots]_{\indexoneout}$ and $[\ldots]_{\indextwoout}$ was defined in Section~\ref{sec:pairings}.
\end{enumerate}

\item $\mathit{ck} \leftarrow \mathit{CKS_{\mathcal{R}}.GenCommitteeKey}(\mathit{rs_{pk}}, (\mathit{pk_i})_{i=1}^{n-1})$,\\ where 
$\mathit{CKS_{\mathcal{R}}.GenCommitteeKey}$ first checks whether \\ $(\mathit{pk_i})_{i=1}^{n-1} \in \ginn{1}^{n-1}$; \\ if this does not hold, it outputs $\bot$; \\
otherwise, $\mathit{CKS_{\mathcal{R}}.GenCommitteeKey}$ continues as: \\
\noindent Let $\mathbf{pkx} = (\mathit{pkx_{1}}, \ldots, \mathit{pkx_{n-1}})$, $\mathbf{pky} = (\mathit{pky_{1}}, \ldots, \mathit{pky_{n-1}})$, $\forall i \in [n-1]$, $\mathit{pk_i} = (\mathit{pkx_i}, \mathit{pky_i}) \in \mathbb{F}^{2}$. \\
\noindent Let $pkx(X) = \sum_{i=0}^{n-2} \mathit{pkx_{i+1}} \cdot L_i(X)$, $pky(X) = \sum_{i=0}^{n-2} \mathit{pky_{i+1}} \cdot L_i(X)$. Let $[pkx]_{\indexoneout} = pkx(\tau) \cdot [1]_{\indexoneout}$, $[pky]_{\indexoneout} = pky(\tau) \cdot [1]_{\indexoneout}$. Output $\mathit{ck} = ([pkx]_{\indexoneout}, [pky]_{\indexoneout})$.\\
\noindent Note that $\mathbb{F}$ and $\{L_i(X)\}_{i=1}^{n-2}$ are as defined in Section~\ref{sec:lagrange}. 

\item $\pi \leftarrow \mathit{CKS_{\mathcal{R}}.Prove}
(\mathit{rs}_{\mathit{pk}}, \mathit{ck}, (\mathit{pk_i})_{i=1}^{n-1}, (\mathit{bit_i})_{i=1}^{n-1})$ 
where $ \pi = (\pi_{SNARK}, \mathit{apk})$ and $\mathit{CKS_{\mathcal{R}}.Prove}$ calls \\ 
$\mathit{apk} = \sum_{i=1}^{n-1} \mathit{bit_i} \cdot \mathit{pk_i} \leftarrow \mathit{AS.AggregateKeys}(\mathit{pp}, (\mathit{pk_i})_{i:\mathit{bit_i = 1}})$ 
as defined in Instantiation~\ref{insta:bls} and \\ $\pi_{SNARK} \leftarrow \mathit{SNARK.Prove}(\mathit{rs_{pk}}, (x,w), \mathcal{R})$, \\ 
for $\mathcal{R} \in \{\Rlacom, \Racom \}$ where 
\begin{equation*}
\begin{cases}
 (x = (\mathit{ck}, (\mathit{bit_i})_{i=1}^{n-1}||0, \mathit{apk}), w = ((\mathit{pk}_i)_{i=1}^{n-1}) & \text{ if } \mathcal{R} = \Rlacom,\\
 (x = (\mathit{ck}, \mathbf{b'}, \mathit{apk}), w = ((\mathit{pk}_i)_{i=1}^{n-1}, (\mathit{bit_i})_{i=1}^{n-1}||0) & \text{ if } \mathcal{R} = \Racom, \\
\end{cases}       
\end{equation*}
where $\mathbf{b'}$ is the vector of field elements formed from blocks of size $\mathsf{block}$ of bits from vector 
$(\mathit{bit_i})_{i=1}^{n-1}||0$ and $\mathsf{block}$ is the highest power of 2 smaller than the size of a field element in $\mathbb{F}$. 

\item $0/1 \leftarrow \mathit{CKS_{\mathcal{R}}.Verify}(\mathit{pp}, \mathit{rs}_{\mathit{vk}}, \mathit{ck}, m, \mathit{asig}, \pi, \mathbf{bitvector})$ 
parses $\pi$ to retrieve $\pi_{\mathit{SNARK}}$ and $\mathit{apk}$ and it calls \\$\mathit{AS.Verify(\mathit{pp}, \mathit{apk}, m, \mathit{asig})}$ 
as defined in Instantiation~\ref{insta:bls} and it also calls $\mathit{SNARK.Verify}(\mathit{rs_{vk}}, x, \pi_{\mathit{SNARK}}, \mathcal{R})$ 
(where $\pi_{\mathit{SNARK}}$, $x$ and $\mathcal{R}$ are as defined in the paragraph above with the only difference that $(\mathit{bit_i})_{i=1}^{n-1}$ 
represents the first $n-1$ bits of $\mathbf{bitvector}$, padded with $0$s, if not sufficiently many exist in $\mathbf{bitvector}$); 
it outputs $1$ if both algorithms output $1$ and it outputs $0$ otherwise.
\end{itemize}
\end{construction}
\vspace{-0.07in}
\begin{theorem} 
% All of this was explained above so is redundant or is just redundant.
%Given the hybrid model SNARK scheme secure for relation $\mathcal{R} \in \{ \Rlacom, \Racom\}$ as 
%obtained using our two-step compiler in Section~\ref{sec_two_step_compiler} and the aggregatable signature scheme $\mathit{AS}$ 
 %                   as per Instantiation~\ref{insta:bls} (which fulfils Definition~\ref{def:aggregate_signatures}) with the additional 
 %                   specification that $\mathit{aux}_{\mathit{AS}} = v+1$ and choosing $v = n-1$, 
%if we assume that an efficient adversary (against the soundness of) $\mathit{CKS}_{\mathcal{R}}$ outputs public keys only from the source group $\ginn{1}$, then
The committee key scheme $\mathit{CKS}_{\mathcal{R}}$ in Instantiation~\ref{inst:cks} is secure with respect to Definition~\ref{def: committee_key}.
\end{theorem}
\vspace{-0.08in}
\begin{proof} We give a full proof in Appendix~\ref{supplementary_proof_sec_cks}. 
\end{proof}
