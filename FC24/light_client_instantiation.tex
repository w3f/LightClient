\subsection{Accountable Light Client System Instantiation}
\label{sec:LCinstantiation} 

\noindent We motivate our light client model from~\ref{sec:LCformal_model} by detailing below instantiations 
for a light client system that is accountable light client system. Both are 
compatible with proof-of-stake based blockchains and, in particular, Polkadot.

\subsubsection{Conventions and Assumptions}
\label{sec:conventions}

\noindent Before listing our light client systems' algorithms, we make several notational conventions:

\begin{itemize}
\item We use boldface font for denoting vectors. Furthermore, whenever necessary to avoid confusion, 
we denote by $\mathbf{Vec_i}(k)$ the $k$-th component of vector $\mathbf{Vec_i}$. 

\item In the following, unless otherwise stated, when we use $\mathcal{R}$, we mean one of the conditional relations from the set 
$\{ \Rlacom, \Racom\}$.% , \Rvtcom\}$.}

\item Given a valid consensus view $C$ over $i$ epochs, we assume there is a well-defined order on the set $\mathit{PK_j}$  
of public keys included in $C$, $\forall j \in [i]$; hence, in the following, we rename this set by $\mathbf{pk_j}$, $\forall j \in [i]$ 
and interpret it as a vector. Moreover, we instantiate honestly generated keys in $\mathbf{pk_j}$ with keys generated using 
$\mathit{AS.GenerateKeypair}$ as described in instantiation~\ref{insta:bls}.
  
\item We remind the reader that by $\mathbf{Com}(\mathbf{pk})$ we denote the set of two computationally binding polynomial 
commitments to the polynomials obtained by interpolating the $x$ components of $\mathbf{pk}$ and, respectively, the $y$ 
components of $\mathbf{pk}$ over a range $H$ of size at least $v+1$, where $v$ is some maximum number of validators that the system allows. 
In our instantiations for (accountable) light client systems, we use the KZG polynomial commitments, but, as mentioned also in 
Section~\ref{sec_two_step_compiler}, the general results stated in this section hold for any binding polynomial commitments with a knowledge-soundness property.

\item We assume there is a fixed upper bound $v$ on the number of validators in each epoch and we use $v$ in the description of our algorithms. 
At the same time, for compatibility with the SNARKs that we build for relations $\Rlacom$ and $\Racom$ as defined 
in \ref{compiler_step_2}, when specifically using our instantiation~\ref{inst:cks} of $\mathit{CKS_{\mathcal{R}}}$ or when proving our results in this section, we 
let $v$ equal $n-1$, where $n$ was defined in Section~\ref{sec:lagrange}. 

\item $\mathit{Parse}$ and $\mathit{Transform}$ denote functions performing the respective operations on the 
(accountable) light client algorithms' input in order to obtain the necessary components. $\mathit{Parse}$ and $\mathit{Transform}$ 
may additionally depend on the (conditional) relation $\mathcal{R}$ under consideration. If that is the case, we explicitly include 
$\mathcal{R}$. In particular, $\mathit{Parse}$ and $\mathit{Transform}$ functions which are part of $\mathit{LC.DetectMisbehaviour}$ 
work only for $\mathcal{R} \in \{ \Rlacom, \Racom \}$. 

\item The accountable light client systems use functions $f_x$ (deriving the public inputs), 
$\mathit{f_{\mathit{threshold}}}$ (deriving the Hamming weight), $\mathit{HammingWeight}$ (deriving the Hamming 
weight from consensus view elements) and $f_{\mathit{bit}}$ (deriving the bitmask corresponding to public keys that 
signed a given message). Before providing these functions' definitions, we make the convention that, 
whenever used as parameters/input to these functions, $\mathbf{bit}$, $\mathit{apk}$, $\mathbf{b'}$ and $s$ have 
the meaning and definition provided in Section~\ref{sec:snarks}. 

\begin{equation*}
 f_x (\mathbf{Com}(\mathbf{pk}), \mathbf{bit}, s,\mathit{apk}, \mathcal{R}) =
   \begin{cases}
    (\mathbf{Com}(\mathbf{pk}), \mathbf{bit}, \mathit{apk}) & \text{if } \mathcal{R} = \Rlacom \\
     (\mathbf{Com}(\mathbf{pk}), \mathbf{b'}, \mathit{apk}) & \text{if } \mathcal{R} = \Racom \\
   % (\mathbf{Com}(\mathbf{pk}), s, \mathit{apk}) & \text{if } \mathcal{R} = \Rvtcom \\
  \end{cases}       
\end{equation*}

%%\begin{equation*}
%% f_w (\mathbf{pk}, \mathbf{bit}, \mathcal{R}) =
%%   \begin{cases}
%%      \mathbf{pk} & \text{if } \mathcal{R} = \Rlacom \\
%%      (\mathbf{pk}, \mathbf{bit}) & \text{if } \mathcal{R} =  \Racom \\
%%     ( \mathbf{pk}, \mathbf{bit}) & \text{if } \mathcal{R} =  \Rvtcom \\
%%   \end{cases}       
%%\end{equation*}

\begin{equation*}
\mathit{HammingWeight^*}(\mathbf{vec}) = \mathit{HammingWeight}(\mathbf{vec}_{1}, \ldots, \mathbf{vec}_{|\mathit{vec}| - 1})
\end{equation*}

%\begin{equation*}
%\mathit{HammingWeight}(\mathbf{bit}, \mathcal{R}) =
%  \begin{cases}
%   \mathit{HammingWeight^*}(\mathbf{bit}) & \text{if } \mathcal{R} \in \{ \Rlacom, \Racom\} \\
%      \mathit{HammingWeight}(\mathbf{bit}) -1 & \text{if } \mathcal{R} =  \Rvtcom \\
%  \end{cases}     
%\end{equation*}

\begin{equation*}
 \mathit{f_{\mathit{threshold}}} (\mathbf{x}, \mathcal{R}) =
    \begin{cases}
      \mathit{HammingWeight^*}(\mathbf{bit}) & \text{if } \mathcal{R} = \Rlacom \\
     \sum_{j=1}^{\frac{v+1}{|\mathit{block}|}-1}\mathit{HammingWeight}(\mathbf{b'}_{j}) \ + \ \mathit{HammingWeight^*}(\mathbf{b'}_{\frac{v+1}{|\mathit{block}|}}) & \text{if } \mathcal{R} =  \Racom \\
      %s & \text{if } \mathcal{R} =  \Rvtcom \\
    \end{cases}      
\end{equation*}


\begin{equation*}
 \mathit{f_{\mathit{bit}}} (C, m, v) = ((\mathbf{bit_i}(k))_{k=1}^{v} ||\ 0, \mathbf{\sigma_i}),
\end{equation*}
\noindent where $i = \mathit{epoch}_{\mathit{id}}(m)$ and $\forall  \ k = 1, \ldots, v$, if  there exists 
$\sigma \in C \ \wedge \ \mathit{AS.Verify}(\mathit{pp}, \mathbf{pk_i}(k), \mathit{m}, \sigma) = 1$, 
we set $\mathbf{bit_i}(k) = 1$ and $\mathbf{\sigma_i}(k) = \sigma$, otherwise, we set $\mathbf{bit_i}(k) = 0$ and  $\mathbf{\sigma_i}(k) = \_$. 

\noindent Note that for each of our relations $\Rlacom$ and $\Racom$, $\mathit{apk}$ and $\mathbf{Com}(\mathbf{pk})$ 
are public inputs and $\mathbf{pk}$ is a witness. Moreover, for these relations $\Rlacom$ and $\Racom$, we build an accountable light client system. 
%\noindent Note that for each of our relations $\Rlacom$, $\Racom$, $\Rvtcom$, $\mathit{apk}$ and $\mathbf{Com}(\mathbf{pk})$ 
%are public inputs and $\mathbf{pk}$ is a witness. Moreover, for relations $\Rlacom$ and $\Racom$, we build an accountable light client system 
%while for relation $\Rvtcom$ we build a light client system that is not accountable.
\item We make the following instantiations: $\mathsf{genstate}$ is the set of public keys in $\mathbf{pk_1}$ and their alleged proofs of possession; 
$\mathit{LC.seed} = \mathbf{Com}(\mathbf{pk_1})$.
\end{itemize}

%\noindent We have a parametrisation convention:

%\begin{itemize}
%\item We assume there is a fixed upper bound $v$ on the number of validators in each epoch. In particular, for 
%compatibility with the SNARKs that we build for relations $\Rlacom$, $\Racom$ and $\Rvtcom$ as defined 
%in \ref{compiler_step_2}, we let $v$ equal $n-1$, where $n$ was defined in Section~\ref{sec:lagrange}.  
%\end{itemize}

%\noindent Finally, we have security definition convention:
%\begin{itemize}
%\item Our instantiation algorithms $\mathit{LC.DetectMisbehaviour}$, $\mathit{LC.VerifyMisbehaviour}$ and $\mathit{GenerateProof}$ 
%as described below may also return no values, usually in case $\mathit{CheckValidConsensus}(C)=0$. However, this does not affect the 
%security definitions in Section~\ref{sec:LCformal_model} since in completeness and both the accountability related definitions we assume 
%$\mathit{CheckValidConsensus}(C) = 1$, while our soundness definition does not involve the above three algorithms. Moreover, we use 
%the implicit convention that all properties fail if one of the algorithms that are part of its definition return no value.
%\end{itemize}

\subsubsection{The Algorithms}
\label{sec:instantiation}

\noindent The setup algorithm used by the accountable light client system is: 
\begin{itemize}
\item $\mathit{LC.Setup(\mathcal{R})}$
\begin{align*}
%& \mathit{pp} \leftarrow \mathit{AS.Setup}(\lambda) \\
%& \mathit{srs} \leftarrow \mathit{SNARK.Setup}(\lambda) \\
%& (\mathit{rs_{\mathit{pk}}}, \mathit{rs_{\mathit{vk}}}) \leftarrow \mathit{SNARK.KeyGen}(\mathit{srs}, \mathcal{R}) \\
%& \mathit{Return} \ (\mathit{pp}, \mathit{rs_{\mathit{pk}}}, \mathit{rs_{\mathit{vk}}})
(\mathit{pp}, \mathit{rs_{\mathit{pk}}}, \mathit{rs_{\mathit{vk}}}) \leftarrow \mathit{CKS_{\mathcal{R}}.Setup}(v) \\ 
\mathit{Return} \ (\mathit{pp}, \mathit{rs_{\mathit{pk}}}, \mathit{rs_{\mathit{vk}}})
\end{align*}
\end{itemize}

\noindent The four algorithms that are part of the accountable light client system are: 
\begin{itemize}
\item $\mathit{LC.GenerateProof}(\mathit{pp}, \mathit{rs_{\mathit{pk}}}, C, m, \mathcal{R})$
\begin{align*}
&\ \ \ \ \mathbf{\Pi} = (); \ \mathbf{\Sigma} = () \\
&\ \ \ \ i = \mathit{epoch_{id}}(m) \\
&\ \ \ \ \mathit{For} \ j = 1, \ldots, i \\
&\ \ \ \ \ \ \ \ \mathit{If} \  j < i  \\
&\ \ \ \ \ \ \ \ \ \ \ \ \mathit{m_j}= (j,\mathbf{Com}(\mathbf{pk_{j+1}})) \\
&\ \ \ \ \ \ \ \ \mathit{Else} \\
&\ \ \ \ \ \ \ \ \ \ \ \ \mathit{m_j} = m  \\
&\ \ \ \ \ \ \ \ (\mathbf{bit_j}, \mathbf{\sigma_j}) = \mathit{f_{bit}}(C, m_j, v) \\
%&\ \ \ \ \ \ \ \ \mathit{For} \ k = 1, \ldots, v \\
%&\ \ \ \ \ \ \ \ \ \ \ \ \textit{If exists }  \ \sigma  \in C \ \wedge \ \mathit{AS.Verify}(\mathit{pp}, \mathbf{pk_j}(k), \mathit{m_j}, \sigma) = 1 \\
%& \ \ \ \ \ \ \ \ \ \ \ \ \ \ \ \ \mathbf{bit_j}(k) = 1 \ \wedge \ \mathbf{\sigma_j}(k) = \sigma \\
%&\ \ \ \ \ \ \ \ \ \ \ \ \textit{Else} \\ 
%& \ \ \ \ \ \ \ \ \ \ \ \ \ \ \ \ \mathbf{bit_j}(k) = 0 \ \wedge \ \mathbf{\sigma_j}(k) = \_ \\
%&\ \ \ \ \ \ \ \ s = \mathit{HammingWeight}(\mathbf{bit_j}, \mathcal{R}) \\
%&\ \ \ \ \ \ \ \ \mathit{apk_{j}} = \mathit{AS.AggregateKeys}(\mathit{pp}, (\mathbf{bit_j}(k) \cdot \mathbf{pk_j}(k))_{k=1}^v) \\
&\ \ \ \ \ \ \ \ \mathbf{\Sigma}(j) \leftarrow \mathit{AS.AggregateSignatures}(\mathit{pp}, (\mathbf{\sigma_j}(k))_{k=1}^v) \\ 
%&\ \ \ \ \ \ \ \ \mathbf{x_j} = f_x(\mathbf{Com}(\mathbf{pk_j}), \mathbf{bit_j}, s,  \mathit{apk_{j}}, \mathcal{R}) ; \ \mathbf{w_j} = f_w( \mathbf{pk_j}, \mathbf{bit_j}, \mathcal{R}) \\
%&\ \ \ \ \ \ \ \ \mathbf{\pi_{j}} = \mathit{SNARK.Prove}(\mathit{srs}_{pk} , (\mathbf{x_j}, \mathbf{w_j}), \mathcal{R})\\
&\ \ \ \ \ \ \ \ (\mathit{\pi_{\mathit{SNARK},j}}, \mathit{apk_j}, \mathbf{Com}(\mathbf{pk_{j}})) \leftarrow \mathit{CKS_{\mathcal{R}}.Prove}(\mathit{rs}_{pk}, (\mathbf{pk_{j}}(k))_{k=1}^{v}, (\mathbf{bit_j}(k))_{k=1}^{v})\\
&\ \ \ \ \ \ \ \ \mathbf{x_j} = f_x(\mathbf{Com}(\mathbf{pk_j}), \mathbf{bit_j}, s,  \mathit{apk_{j}}, \mathcal{R}) \\
%&\ \ \ \ \ \ \ \ \mathbf{\Pi}(j) = (\mathbf{x_j},\mathbf{\pi_{j}}); \ \mathbf{\Sigma}(j) = \mathit{\Sigma}_{j} \\
&\ \ \ \ \ \ \ \ \mathbf{\Pi}(j) = (\mathbf{x_j}, \mathit{\pi_{\mathit{SNARK}, j}}) \\
&\ \ \ \ \mathit{Return} \ (\mathbf{\Pi}, \mathbf{\Sigma})  
\end{align*}
\end{itemize}
%\vspace{-0.5cm}
%\noindent The two algorithms that are only part of the accountable light client system are: 
\begin{itemize}
\item $\mathit{LC.VerifyProof}(\mathit{pp}, \mathit{rs_{\mathit{vk}}}, \LCseed, \pi, m, \mathcal{R})$
\begin{align*}
& i = \mathit{epoch_{id}}(m) \\
& (\mathbf{\Pi}, \mathbf{\Sigma}) = \mathit{Parse}(\pi);  \\
& \mathit{For} \ j = 1, \ldots, i \\
& \ \ \ \ (\mathbf{x_j}, \mathbf{\pi_{\mathit{SNARK},j}}) = \mathbf{\Pi}(j); \ (\mathit{com_j}, \mathbf{bit_j}, \mathit{apk_j}) = \mathit{Parse}(\mathbf{x_j}, \mathcal{R}) \\
& \mathit{If} \ \LCseed \neq \mathit{com_1} \\
& \ \ \ \ \mathit{Return} \ \mathit{rej} \\
& \mathit{For} \ j = 1, \ldots, i \\
& \ \ \ \ \mathit{If} \ j < i\\
& \ \ \ \ \ \ \ \ \ \mathit{m_j} = (j, \mathit{com_{j+1}}) \\
& \ \ \ \ \mathit{Else}\\
& \ \ \ \ \ \ \ \ \ \mathit{m_j} = m \\
& \ \ \ \ \mathit{threshold_j} = \mathit{f_{\mathit{threshold}}}(\mathbf{x_j}, \mathcal{R})  \\
%& \ \ \ \ \mathit{If} \ (\mathit{SNARK.Verify}(\mathit{srs_{vk}}, \mathbf{x_j}, \mathbf{\pi_j}, \mathcal{R}) = 0)  \ \vee  \ (\mathit{AS.Verify}(\mathit{pp}, \mathit{apk_j}, \mathit{m_j}, \mathbf{\Sigma}(j)) = 0) \ \vee (s < t) \\
& \ \ \ \ \mathit{If} \ (\mathit{CKS_{\mathcal{R}}.Verify}(\mathit{pp},  \mathit{rs_{\mathit{vk}}}, \mathit{com_j}, m_j,  \mathbf{\Sigma}(j), (\mathbf{\pi_{\mathit{SNARK},j}},\mathit{apk_j}), 
\mathbf{bit_j}) = 0)  \vee (\mathit{threshold_j} < t) \\
& \ \ \ \ \ \ \ \ \ \mathit{Return} \ \mathit{rej} \\
& \mathit{Return} \ \mathit{acc} \\
\end{align*}
\end{itemize}
%\vspace{-0.5cm}
\begin{itemize}
\item $\mathit{LC.VerifyMisbehaviour}(\mathit{pp}, i, S, \mathbf{bit}, \sigma, m'', m', C)$
\begin{align*}
%& \mathit{If} \ (\mathit{CheckValidConsensus}(C) =1) \\
& \ \ \ \ \mathit{apk} = \mathit{AS.AggregateKeys}(\mathit{pp}, (\mathbf{bit}(k) \cdot \mathbf{pk_i}(k))_{k=1}^v) \\
&\ \ \ \ (\mathbf{bit'}, \_ ) = \mathit{f_{bit}}(C, m', v) \\ 
%& \ \ \ \ \mathit{For} \ k = 1, \ldots, v \\
%& \ \ \ \ \ \ \ \ \textit{If exists }  \ (\mathit{sig}  \in C) \ \wedge \ (\mathit{AS.Verify}(\mathit{pp}, \mathbf{pk_{\mathbf{i}}}(k), m', \mathit{sig}) = 1) \\
%& \ \ \ \ \ \ \ \ \ \ \ \mathbf{bit'}(k) = 1 \\
%& \ \ \ \ \ \ \ \ \textit{Else} \\ 
%& \ \ \ \ \ \ \ \ \ \ \ \mathbf{bit'}(k) = 0 \\
& \ \ \ \ \mathit{Compute} \ S_{m''} = \{ \mathbf{pk}_{\mathbf{i}}(k) \ | \ \mathbf{bit}(k) = 1, k \in [v] \} \\
& \ \ \ \ \mathit{Compute} \ S_{m'} =  \{ \mathbf{pk}_{\mathbf{i}}(k) \ | \ \mathbf{bit'}(k) = 1, k \in [v] \} \\
& \ \ \ \ \mathit{If} \ (\mathit{AS.Verify}(\mathit{pp}, \mathit{apk}, m'', \sigma) = 1) \ \wedge (S_{m''} \cap S_{m'} = S) \ \wedge \ (|S_{m'}| \geq t) \ \wedge \
 (|S_{m''}| \geq t) \ \wedge \\
%& \hspace{1cm} \wedge \ {\color{red}(m' \in C )} \ \wedge \ {\color{red} (\exists \ d_{m'} \in C\cap D, \mathit{VerifyData}(m', d_{m'})=1)} \ \wedge\ \\
& \hspace{1cm} \wedge ((\mathit{m'}, d_{\mathit{m'}}) \in_{\mathit{decided}} C) \ \wedge \ \\
& \hspace{1cm} \wedge \ (i = \mathit{epoch_{id}}(m'') = \mathit{epoch_{id}}(m')) \ \wedge \ (\mathit{Incompatible(m'', m', d_{m'})} = 1)  \\
& \ \ \ \ \ \ \ \ \mathit{Return} \ \mathit{acc} \\
& \ \ \ \ \mathit{Else} \\
& \ \ \ \ \ \ \ \ \mathit{Return} \ \mathit{rej} 
%& \mathit{Else} \\
%& \ \ \ \ \mathit{Return} 
\end{align*}

\item $\mathit{LC.DetectMisbehaviour}(\mathit{pp}, \mathit{rs_{\mathit{vk}}}, \pi, m, C, \mathcal{R})$
\begin{align*}
& (\mathbf{\Pi}, \mathbf{\Sigma}) = \mathit{Parse}(\pi)  \\
& i = \mathit{epoch_{id}}(m) \\
& \mathit {index} = i \\
& m'' = m \\
& \mathit{For} \ j = 1, \ldots, i \\
& \ \ \ \ (\mathbf{x_j}, \mathit{\pi_{\mathit{SNARK},j}}) = \mathbf{\Pi}(j); \ (\mathit{apk_j}, \mathit{com_j}) = \mathit{Parse}(\mathbf{x_j}) \\
& \mathit{If} \ (\mathit{LC.VerifyProof}(\mathit{pp}, \mathit{rs_{\mathit{vk}}}, \LCseed, \pi, m, \mathcal{R}) = 1) \ \wedge \  ( \exists \ \mathit{min} \ 2 \leq j \leq i, \ \mathit{com_j} \neq \mathbf{Com}(\mathbf{pk_j}))\\ % \ \wedge \\
%&  \hspace{5.2cm} \ \wedge \ (\mathit{CheckValidConsensus}(C) = 1) \\
& \ \ \ \ \ \ \ m'' = (j-1,\mathit{com_j}); \ m' = (j-1, \mathbf{Com}(\mathbf{pk_j})); \mathit{index} = j-1\\
& \mathit{ElseIf} \ (\mathit{LC.VerifyProof}(\mathit{pp}, \mathit{rs_{\mathit{vk}}}, \LCseed, \pi, m, \mathcal{R}) = 1) \  \wedge \ (\forall \ 2 \leq j \leq i, \ \mathit{com_j} = \mathbf{Com}(\mathbf{pk_j})) \ \wedge \\ 
& \hspace{1cm} \wedge (\exists \ (\mathit{aux}, d_{\mathit{aux}}) \in_{\mathit{decided}} C) \ \wedge \ \mathit{Incompatible}(\mathit{aux}, m'',  d_{\mathit{aux}})=1) \\
%& \hspace{1cm} \wedge \ (\exists \ {\color{red} \mathit{aux}}, d_{{\color{red} \mathit{aux}}} \in C, {\color{blue} \mathit{aux} \textit{ decided  in }  C}, \mathit{Incompatible}({\color{red} \mathit{aux}}, {\color{red} m''},  d_{ {\color{red} \mathit{aux}}})=1) \\
%& \hspace{1cm} \wedge \ (\mathit{CheckValidConsensus}(C) = 1) \\
& \ \ \ \ \ \ \ m' =   \mathit{aux}\\
&\mathit{Else} \ \mathit{Return} \\
& \mathbf{bit} = \mathit{Transform}(\mathit{Parse}(\mathbf{x_{\mathit{index}}}, \mathcal{R}), \mathcal{R}) \\
& \mathit{Compute} \ S_{m''} = \{ \mathbf{pk}_{\mathbf{index}}(k) \ | \ \mathbf{bit}(k) = 1, k \in [v] \} \\
%& \mathit{For} \ k = 1, \ldots, v \\
%& \ \ \ \ \textit{If exists }  \ (\sigma  \in C) \ \wedge \ (\mathit{AS.Verify}(\mathit{pp}, \mathbf{pk_{\mathbf{index}}}(k), m', \sigma) = 1) \\
%& \ \ \ \ \ \ \ \mathbf{bit'}(k) = 1 \\
%& \ \ \ \ \textit{Else} \\ 
%& \ \ \ \ \ \ \ \mathbf{bit'}(k) = 0 \\
& (\mathbf{bit'}, \_ ) = \mathit{f_{bit}}(C, m', v) \\ 
& \mathit{Compute} \ S_{m'} = \{ \mathbf{pk}_{\mathbf{index}}(k) \ | \ \mathbf{bit'}(k) = 1, k \in [v] \} \\
& \mathit{Return} \ (\mathit{index}, S_{m''} \cap S_{m'}, \mathbf{bit}, {\mathbf{\Sigma}}(\mathit{index}), {m''}, m' ) 
\end{align*}
\end{itemize}

\subsubsection{Assumptions and Security Proofs}
\label{sec:assumptions}
\label{sec:security_proofs}

\noindent We complete our instantiation by proving the security properties of our light client and accountable light client systems according to 
definitions introduced in Sections~\ref{sec:soundness} and \ref{sec:accountability}. However, beforehand, we present the assumptions we use, of 
which there are six classes, i.e., there are assumptions about honest validators' behaviour (B), about consensus (C), about parameters (P), about instantiation of primitives (S), 
about genesis state (G) and assumptions about light client integration (I). \\

\noindent The assumptions about honest validators' behaviour are:
\begin{itemize}
\item (B.1.) An honest validator never signs a message $m$ unless it knows some required data $d_m$  
such that $\mathit{VerifyData}(m, d_m) = 1$ holds.
\item (B.2.) An honest validator never signs a message $m$ such that $\mathit{VerifyData}(m, d_m) = 1$ holds 
if they have previously signed $m'$ such that $\mathit{VerifyData}(m', d_{m'}) = 1$ holds and \\ $\mathit{Incompatible}(m, m',d_m) = 1$ 
or $\mathit{Incompatible}(m', m, d_{m'}) =1$ hold.
\item (B.3.) An honest validator does not sign any message in $M_i$ unless they have a valid consensus view $C$ (with $M_i \subset C$) 
for which their public key is in $\mathbf{pk_i}$ with $\mathbf{pk_i} \in C$.
\end{itemize}

%{\color{blue} TO DO: CHECK the assumptions below for correctness}
\noindent The assumptions about consensus are:
\begin{itemize}
\item (C.1.) The adversary interacting with honest validators should not except with negligible probability be able to produce both:
(i) a valid consensus view $C$ in which at least $t'$ validators in every epoch are honest that decides some message $m$ 
with $d_m$ such that $\mathit{VerifyData}(m, d_m) =1$ and 
(ii) a valid consensus view $C'$ with the same genesis state as $C$ (in particular, with the same $\mathbf{pk_1} \subset \mathsf{genstate}$) 
which decides some message $m'$ in the same epoch as $m$, with $\mathit{Incompatible}(m, d_m, m') = 1$. 
\item (C.2.) The adversary interacting with honest validators should not except with negligible probability be able to produce both:
(i) a valid consensus view $C$ in which at least $t'$ validators in every epoch are honest and 
(ii)  a valid consensus view $C'$ with the same genesis state as $C$ (in particular, with the same $\mathbf{pk_1} \subset \mathsf{genstate}$) 
in which there is some epoch $i$ that $C$ and $C'$ both reach with $\mathbf{pk_i} \neq \mathbf{pk'_i}$.
\end{itemize}

\noindent The assumptions about parameters are:
\begin{itemize}
\item (P.1.) $2t -v > 0$
\item (P.2.) $t + t' > v$
\end{itemize}

\noindent The assumption about instantiation of primitives is:
\begin{itemize}
\item (S.1.) We instantiate the aggregatable signature scheme $\mathit{AS}$ 
such that the oracle $\mathit{OSign}$ in Definition~\ref{def:aggregate_signatures} (in particular in the 
unforgeability property definition), is replaced with $\mathit{OSpecialSign}$. It is easy to see that if $\mathit{AS}$ 
is an aggregatable signature scheme secure according to Definition~\ref{def:aggregate_signatures},  
then $\mathit{AS}$ is also an aggregatable signature with oracle $\mathit{OSign}$ replaced by 
$\mathit{OSpecialSign}$ in Definition~\ref{def:aggregate_signatures}.
\end{itemize}

\noindent The assumptions about genesis state are:
\begin{itemize}
\item (G.1.) In a valid consensus view, $\mathit{HistoricVerifyData}$ checks, among others, that 
a) every $\mathit{pk} \in \mathbf{pk_1}$ is also part of  $\mathsf{genstate}$, b) that every $\mathit{pk} \in \mathbf{pk_1}$ is in $\ginn{1}$ 
and c) that the proofs of possession for each of the public keys in $ \mathbf{pk_1}$ pass the verification in $\mathit{AS.VerifyPoP}$.
\item  (G.2.) We assume that all honest full nodes and validators have access to the same genesis state 
$\mathsf{genstate}$ even when the genesis state is generated by a potential adversary.
\end{itemize}

Before the last class of assumptions, we add two notational conventions in the form of two functions:
\begin{itemize}
\item $\mathit{NextEpochKeys}(m, d_m)$ returns $\bot$ or a list of public keys; if $\mathit{epoch_{id}}(m) = i$, 
these keys are supposed to be the public keys of epoch $i+1$. 
%, where $d_m$ with $\mathit{ValidateData}(m, d_m) =1$. 
%If $\mathit{epoch_{id}}(m) = i$ and $m \in C$, $\mathit{NextEpochKeys}(m, d_m) \neq \bot$, for some valid consensus view $C$, 
%then $\mathit{NextEpochKeys}(m, d_m)= \mathbf{pk_{i+1}}$ and $\mathbf{pk_{i+1}} \subset d_m$.} 
\item  $\mathit{IsCommitment}(m)$ returns $0$ or $1$; $\mathit{IsCommitment}(m) = 1$ iff  there exists some $i$ such that 
$m = (i, \mathbf{Com}(\mathbf{pk_{i+1}}))$.
%\item {\color{red} $\mathit{IsCommitment}(m)$ returns $0$ or $1$. If $\mathit{epoch_{id}}(m) = i$ and $m \in C$, 
%for some valid consensus view $C$, then $\mathit{IsCommitment}(m) = 1$ iff $m = \mathbf{Com}(\mathbf{pk_{i+1}})$.}
%\item {\color{red} We say predicate $P(m, d_m, \mathbf{pk}) = 1$ iff 
%the blockhash $m$ is final ({\color{blue} What does that mean?}), $d_m$ is the blockchain leading up to and including 
%the block for which $m$ is a blockhash and $\mathbf{pk}$ is the set of public keys which determines the next validator 
%set and is also a deterministic function of $d_m$.}
\end{itemize}

\noindent Finally, we make the following light client integration assumptions, i.e., these are assumptions that 
apply to our specific light client instantiation: 
\begin{itemize}
\item (I.1.) If $m$ and $m'$ are such that 
$\mathit{epoch_{id}}(m) =\mathit{epoch_{id}}(m')$ and 
$\mathit{NextEpochKeys}(m, d_m) \neq \bot$ and 
$\mathit{IsCommitment}(m') = 1$ and 
$m' \neq (\mathit{epoch_{id}}(m), \mathbf{Com}(\mathit{NextEpochKeys}(m, d_m)))$ 
then $$\mathit{Incompatible}(m, m', d_m)=1.$$
\item (I.2.) If $\mathit{epoch_{id}}(m) =i$ and $\mathit{NextEpochKeys}(m, d_m)=\mathbf{pk_{i+1}}$, 
then $\mathit{ValidateData}(m, d_m)$ must call $\mathit{AS.VerifyPoP}(\mathit{pp}, \mathit{pk}, \mathit{\pi_{POP}})$ 
for each $pk \in \mathbf{pk_{i+1}}$ and some data $\mathit{\pi_{POP}} \in d_m$ and also check that
 $\mathit{pk} \in \mathbb{G}_{1,\mathit{inn}}$; if any of these checks fails, then $\mathit{ValidateData}(m, d_m)$ fails.
\item (I.3.) An honest validator with a valid consensus view $C$, does not sign a message $m'$ 
with $\mathit{IsCommitment}(m')=1$ unless there exists a message $m$ decided in $C$ 
and its required data $d_m$ (i.e., $\mathit{ValidateData}(m, d_m) =1$) such that 
$$m'=(\mathit{epoch_{id}}(m), \mathbf{Com}(\mathit{NextEpochKeys}(m, d_m))).$$ 
\item (I.4.) If $\mathit{HistoricVerifyData}$ outputs $1$ and there exist a message $m \in C$ 
that has been decided in epoch $i$, then for all $1 \leq j < i$, $(j, \mathbf{Com}(\mathbf{pk_{j+1}}))$ was decided in epoch $j$. 
\item (I.5.) If $\mathit{HistoricVerifyData}$ outputs $1$ and a message $m'$ has been decided in $C$ 
such that \\ $\mathit{IsCommitment}(m') = 1$, then there exist $m, d_m \in C$ with $\mathit{ValidateData}(m, d_m) = 1$, $m$ decided in $C$
and $\mathit{epoch_{id}}(m) = \mathit{epoch_{id}}(m')$ such that 
$$\mathbf{pk_{\mathit{epoch_{id}}(m)+1}} = \mathit{NextEpochKeys}(m, d_m).$$
\end{itemize}
%{\color{blue} I propose changing the order of assumptions to (I.2), (I.1), (I.4), (I.5), (I.3).} \\

\noindent We are now ready to state and prove the security properties of our (accountable) light client systems. 

\begin{theorem}
\label{th:soundness} 
If $\mathit{AS}$ is the secure aggregatable signature scheme defined in instantiation~\ref{insta:bls} and if 
$\mathit{CKS_{\mathcal{R}}}$ is the secure committee key scheme defined in instantiation~\ref{inst:cks}, then, together with 
the assumptions stated at the beginning of Section~\ref{sec:assumptions} and for $\mathcal{R} \in \{ \Rlacom, \Racom \}$, the tuple 
($\mathit{LC.Setup}$, $\mathit{LC.GenerateProof}$, $\mathit{LC.VerifyProof}$) as instantiated in Section~\ref{sec:instantiation} is a light client system.
\end{theorem}


\begin{proof} 
We include the full proof in Section~\ref{supplementary_proof_sec_soundness}.   
\end{proof}

\begin{theorem} 
\label{th:accountability_results}
If $\mathit{AS}$ is the secure aggregatable signature scheme defined in instantiation~\ref{insta:bls} and if 
$\mathit{CKS_{\mathcal{R}}}$ is the secure committee key scheme defined in instantiation~\ref{inst:cks}, then, together 
with the assumptions stated at the beginning of Section~\ref{sec:assumptions} and for $\mathcal{R} \in \{ \Rlacom, \Racom \}$, the tuple 
($\mathit{LC.Setup}$, $\mathit{LC.GenerateProof}$, $\mathit{LC.VerifyProof}$, $\mathit{LC.DetectMisbehaviour}$, 
$\mathit{LC.VerifyMisbehaviour}$) as instantiated in Section~\ref{sec:instantiation} is an accountable light client system.
\end{theorem}
\begin{proof}
\noindent Due to theorem~\ref{th:soundness}, the tuple 
($\mathit{LC.Setup}$, $\mathit{LC.GenerateProof}$, $\mathit{LC.VerifyProof}$, \\ $\mathit{LC.DetectMisbehaviour}$, 
$\mathit{LC.VerifyMisbehaviour}$) as instantiated in Section~\ref{sec:instantiation} is already a light client system. 
It is only left to show that both accountability completeness and accountability soundness also hold and we include the full details of this proof in 
Section~\ref{supplementary_proof_sec_accountability}.   
\end{proof}

\begin{corollary} In an accountable light client system, the number of misbehaving validators output by 
$\mathit{LC.DetectMisbehaviour}$ is $|S|$ and $|S| > 0$.
\end{corollary}
\begin{proof} Due to theorem~\ref{th:accountability_results} and accountability completeness, 
given a valid consensus view $C$, a verifying light client proof $\pi$ for a message $m''$ 
and given the existence in $C$ of a message $m'$ incompatible with $m''$, the number of 
validators that $\mathit{LC.DetectMisbehaviour}$ is able to catch is at least $|S|$. Moreover, due 
to accountability soundness, any public key output by $\mathit{LC.DetectMisbehaviour}$, \ewnp, belongs to a misbehaving validator. 
Finally, using again accountability completeness and, in particular, since $\mathit{LC.VerifyMisbehaviour}$ accepts 
with overwhelming probability the output of an honest party running $\mathit{LC.DetectMisbehaviour}$ and due to assumption (P.1.) it holds that:
$$ |S| = |S_{m'} \cap S_{m''}| = |S_{m'}| + |S_{m''}| - |S_{m'} \cup S_{m''}| \geq t + t - v > 0.$$ 
\end{proof}
