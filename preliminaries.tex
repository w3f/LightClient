%\subsection{Conventions}
%\label{sec:conventions}
%\vspace{-0.03in}
\noindent We assume all algorithms receive an implicit security parameter $\lambda$. 
We use interchangeably ``efficient algorithm" or ``PPT algorithm'' to mean an algorithm that runs in uniform probabilistic polynomial 
time in the length of its input. 
%{\color{blue} Every input to each of our algorithms apart from the message in our BLS (multi)signature 
%has at most polynomial length in the security parameter. If the length of that message would be polynomially bounded we could then just say that 
%that ``efficient algorithm means an algorithm that runs in polynomial time in the security parameter".}
%When we say that $A$ is an efficient adversary we mean that $A$ is a family $\{A_{\lambda}\}_{\lambda \in \mathbb{N}}$ 
%of non-uniform polynomial-size circuits. If the adversary consists of multiple circuit families $A_1, A_2, \ldots$, then we write $A = (A_1, A_2, \ldots)$. 
Wherever necessary, before the run of all the algorithms and protocols, we assume the correct parameters for the 
curves, groups, pairings, the group generators, etc. have been generated and shared with the corresponding parties. 
A function $f(\lambda)$ is negligible in $\lambda$, written as $\mathit{negl}(\lambda)$, if $1/f(\lambda)$ grows faster than 
any polynomial in $\lambda$ and is overwhelming in $\lambda$ if $1-f(\lambda)=\mathit{negl}(\lambda)$. $\mathsf{poly}(\lambda)$ 
is some polynomial in $\lambda$ and \ewnp means except with probability $\mathit{negl}(\lambda)$.
We write $y = A(x; r)$ when algorithm $A$ on input $x$ and randomness $r$, outputs $y$.
We write $y \leftarrow A(x)$ for picking randomness $r$ uniformly at random and setting $y = A(x; r)$. We denote by $|S|$ the cardinality of set $S$. 
Unless otherwise stated, when we write that an event holds with some probability, we implicitly mean 
that the probability is computed over the randomness of all randomised algorithms involved.
%We say a function is negligible in $\lambda$ and denote it by $\mathit{negl}(\lambda)$ if that function vanishes faster than the inverse of any polynomial in $\lambda$. 
%We say that a function is overwhelming in $\lambda$ if it has the form $1- $ some function negligible in $\lambda$. 
%We also use the notation \ewnp to mean except with negligible probability, or, equivalently, with overwhelming probability. We denote by $\mathsf{poly}(\lambda)$ an  
%unspecified function which has a polynomial expression in $\lambda$. 
We generally use boldface font to denote vectors whose components we explicitly make use of in the text and 
we use italic font to denote the rest of the variables. We work over finite fields of large characteristic. 
When we work with polynomials we denote by 
$\mathbb{F}_{<d}[X]$ the set of all polynomials of degree less than $d$ over the field $\mathbb{F}$. For any integer 
$n \geq 1$, we denote by $[n]$ the set $\{1, \ldots, n\}$.
%\vspace{-0.05in}
