\documentclass{llncs}
\pagestyle{plain}

\usepackage{xspace}   
\usepackage{url}   
\usepackage{graphicx} 
\usepackage{amsmath}
\usepackage{amsfonts}
\usepackage{amssymb} %\therefore
\usepackage{hyperref}
\usepackage{cryptocode}
\usepackage{framed} 
% \usepackage[parfill]{parskip} % activate to begin paragraphs with an empty line rather than an indent
    
 %%% packages
\usepackage{booktabs} 
\usepackage{array} 
\usepackage{paralist} 
\usepackage{verbatim} 
\usepackage{subfig}     
\usepackage{mathrsfs}
\usepackage[normalem]{ulem}
\usepackage{soul}

\newcommand{\Avec}{\ensuremath{\mathbf{A}}\xspace}

\newcommand{\negl}{\ensuremath{\mathit{negl}}\xspace}
%\newcommand{\key}[1]{\ensuremath{\mathsf{ck}_{#1}}\xspace}
\newcommand{\einn}{\ensuremath{\mathit{E}_{\mathit{inn}}}\xspace}
\newcommand{\eout}{\ensuremath{\mathit{E}_{\mathit{out}}}\xspace}
\newcommand{\ginn}[1]{\ensuremath{\mathbb{G}_{\mathit{#1,inn}}}\xspace}
\newcommand{\gout}[1]{\ensuremath{\mathbb{G}_{\mathit{#1,out}}}\xspace}
\newcommand{\epinn}{\ensuremath{\mathit{e}_{\mathit{inn}}}\xspace}
\newcommand{\epout}{\ensuremath{\mathit{e}_{\mathit{out}}}\xspace}
\newcommand{\block}{\ensuremath{\mathsf{block}}\xspace}

%%% end article customizations
 \newcommand{\code}[1]{\texttt{#1}}
\newcommand\tstrut{\rule{0pt}{2.6ex}}         % = `top' strut
\newcommand\bstrut{\rule[-0.9ex]{0pt}{0pt}}   % = `bottom' strut
\newcommand{\bgamma}{\boldsymbol{\gamma}}
\newcommand{\bsigma}{\boldsymbol{\sigma}}

\newcommand{\al}[2]{{\uline{#1}}\textcolor{cyan}{(#2)}}
\newcommand{\sergey}[2]{{\uline{#1}}\textcolor{magenta}{(#2)}}


\newcommand{\GoneBLS}{\mathbb{G}_{1}\xspace}
\newcommand{\GtwoBLS}{\mathbb{G}_{2}\xspace}
\newcommand{\GTBLS}{\mathbb{G}_{T}\xspace}

\newcommand{\goneBLS}{\mathit{g}_{1}\xspace}
\newcommand{\gtwoBLS}{\mathit{g}_{2}\xspace}
\newcommand{\gTBLS}{\mathit{g}_{T}\xspace}

\newcommand{\eBLS}{\mathit{e}\xspace}
\newcommand{\HBLS}{\mathit{H}\xspace}
\newcommand{\Hpop}{\mathit{H}_{\mathit{PoP}}\xspace}
\newcommand{\sigpop}{\sigma_{\mathit{PoP}}\xspace}
\newcommand{\pistar}{\pi^{*}_{\mathit{PoP}}\xspace}
\newcommand{\Hpk}{\mathit{H}_{\mathit{DLEQ}, \mathit{pk}}\xspace}
\newcommand{\Hsig}{\mathit{H}_{\mathit{DLEQ},  \mathit{sig}}\xspace}

\newcommand{\Genpk}{\mathit{KeyGen}_{\mathit{DLEQ}, \mathit{pk}}\xspace}
\newcommand{\Provpk}{\mathit{Prove}_{\mathit{DLEQ}, \mathit{pk}}\xspace}
\newcommand{\Veripk}{\mathit{Verify}_{\mathit{DLEQ}, \mathit{pk}}\xspace}

\newcommand{\Gensig}{\mathit{KeyGen}_{\mathit{DLEQ}, \mathit{sig}}\xspace}
\newcommand{\Provsig}{\mathit{Prove}_{\mathit{DLEQ}, \mathit{sig}}\xspace}
\newcommand{\Verisig}{\mathit{Verify}_{\mathit{DLEQ}, \mathit{sig}}\xspace}

\newcommand{\pipk}{\pi_{\mathit{DLEQ}, \mathit{pk}}\xspace}
\newcommand{\pisig}{\pi_{\mathit{DLEQ}, \mathit{sig}}\xspace}

\newcommand{\PSpk}{\mathcal{PS}_{\mathit{DLEQ}, \mathit{pk}}\xspace}
\newcommand{\PSsig}{\mathcal{PS}_{\mathit{DLEQ}, \mathit{sig}}\xspace}


\newcommand{\Rpk}{\mathcal{R}_{\mathit{DLEQ},\mathit{pk}}\xspace}
\newcommand{\Rsig}{\mathcal{R}_{\mathit{DLEQ},\mathit{sig}}\xspace}

\newcommand{\Epk}{\mathcal{E}_{\mathit{DLEQ},\mathit{pk}}\xspace}
\newcommand{\Esig}{\mathcal{E}_{\mathit{DLEQ},\mathit{sig}}\xspace}
\newcommand{\apk}[1]{\ensuremath{\mathit{apk}_{\mathit{#1}}}\xspace}


\newcommand{\PoP}{\mathit{DLEQ}\xspace}

\newcommand{\sk}{\mathit{sk}\xspace}
\newcommand{\pk}{\mathit{pk}\xspace}


\spnewtheorem{construction}[theorem]{Instantiation}{\bfseries}{\itshape}
\spnewtheorem{assumption}[theorem]{Assumption}{\bfseries}{\itshape}
\spnewtheorem{test_claim}[theorem]{Claim}{\bfseries}{\itshape}
\newenvironment{sketch}{{\bfseries}{\itshape}\paragraph{Proof Sketch}}{\hfill$\square$}

\usepackage{dashrule}

\newcommand{\ewnp}{e.w.n.p.\ }


\begin{document}
   

\title{Efficient Aggregatable BLS Signatures \\ with Chaum-Pedersen Proofs}
\author{Jeff Burdges \and Oana Ciobotaru \and Syed Lavasani \and Alistair Stewart}
\institute{Web3 Foundation}
\date{ \today} % activate to display a given date or no date (if empty), otherwise the current date is printed 
%\date{}
 
 \maketitle

\abstract{%
% Aggregate BLS signatures promise simpler logic and better
% performance, but only after aggregation occurs when some distributed
% exports a ``light client proof''.  Individual BLS signatures have
% abysmally slow verification times.  % Not concise enough for an introduction

BLS signatures have fast aggregated signature verification but slow individual signature verification.
We propose a three part optimisation that dramatically reduces CPU time in
large distributed systems using BLS signatures:  
%
First, public keys should be given in both source groups
$\GtwoBLS$ and $\GoneBLS$, with a proof-of-possession~\cite{proofs_of_posession} check for correctness.
%
Second, aggregated BLS signatures should carry their
particular aggregate public key in $\GtwoBLS$, so that verifiers can
do both hash-to-curve and aggregate public key checks in $\GoneBLS$. 
%
Third, individual non-aggregated BLS signatures should carry short
Chaum-Pedersen DLEQ proofs of correctness~\cite{ChaumPedersen}, so that verifying
individual signatures no longer requires pairings, 
which makes their verification much faster. We prove security for these optimisations. 
The proposed scheme is implemented and benchmarked to compare with classical BLS scheme.
}

\section{Introduction} \label{sec:intro}
1Blockchain systems rely on consensus among a number of participants, where the size of this number is important for decentralisation and the foundation of blockchain security. To know that a transaction is valid, one needs to follow the consensus of the blockchain. However, following consensus can become expensive in
terms of bandwidth, storage and computation. Depending on the consensus type, these challenges can be aggravated when the size of participants' set becomes bigger or when the participants' set changes frequently. Light clients (such as SPV clients in Bitcoin \cite{nakamoto2008bitcoin} or inter-blockchain bridge components that support interoperability) are designed to allow resource constrained users to follow consensus of a blockchain with
minimal cost. We are interested in blockchains that use Byzantine agreement type consensus protocols, particularly proof of stake systems
like Polkadot~\cite{polkadot}, Ethereum~\cite{ethereum}
%~\cite{eth2}
or many other systems~\cite{cosmos,tendermint_paper, celo}. These protocols
may have a large number of consensus participants, from 1000s to 100000s, and in such PoS protocols, the set of participants often changes regularly. \\

\vspace{-0.2cm}
\noindent Following the consensus protocols in the examples above entails proving that a large subset of a designated set of participants,
which are called validators, signed the same message (e.g., a block header). Existing approaches have limiting shortcomings as follows:
1) verifying all signatures which has a large communication overhead for large validator sets;
2) verifying a single aggregatable signature, by computing an aggregate public key from the signer's public keys, has the shortcoming that any verifier still needs to know the entire list of public keys and this, again, has expensive communication if the list changes frequently;
3) verifying a threshold signature which has two shortcomings: first, such a signature does not reveal the set of signers impacting the
security of PoS systems; second, it requires an interactive setup which becomes expensive if the validator set is large or changes frequently.

\vspace{-0.2cm}
\paragraph{Our Approach:} We introduce a {\em Committee Key Scheme} to succinctly prove that a subset of signers
signed a message using a commitment to a list of public keys of all signers. Our primitive is an extension of an aggregatable signature scheme and it allows proving the correctness of an aggregate key of a subset of signers, which can be used to verify an aggregate signature.
In more detail, the committee key scheme defines a committee key which is a commitment to all the signers' public keys and the subset of signers are specified using a bitvector.
The committee key  scheme generates a succinct proof that a particular subset of the list of public keys signed a message. The proof can be verified using the committee key.
%Because of the way the aggregatable signature scheme works,
Using the committee key, the proof and the bitvector, one can verify that the corresponding subset of validators signed the message.
Public keys or signatures are usually 100s of bits long and as a result, although the bitvector has length proportional to the number of validators, this scheme reduces the amount of
data required by a factor of 100 times or more. We could instantiate our committee key scheme using any universal SNARK scheme and suitable commitment scheme.
However, to avoid long prover times for large validator sets, we use optimised custom SNARKs
. We have implemented this scheme
(Section~\ref{sec_implementation}) and it gives fast enough proving times for the use cases we consider: a prover with
commodity hardware can generate these custom SNARK proofs in real time, i.e., as fast as the consensus generates instances of this problem.

\vspace{-0.2cm}
\paragraph{Accountable Light Client Application:} %To understand when and how our scheme described above is useful and compare it to other approaches, we return to what a light client is used for.
Light clients allow resource constrained devices such as browsers or phones to follow decentralised consensus protocols. A blockchain is also resource constrained and hence could benefit from a light client system. In this case a light client verifier (e.g, smart contracts on
Ethereum) allows building trustless bridges protocols between blockchains. % \cite{BridgeSOK}.
Currently, computation and storage costs on existing blockchains are much higher than those in a browser on a modern phone. If such a bridge is responsible for securing assets with high total value, then the corresponding light client system which defines such a light client verifier must be secure as well as efficient. Using the primitives and techniques described in this work, one can design a light client system with the following properties: accountability, asynchronous safety and incrementability reviewed below.  \vspace{-0.2cm}
%\begin{itemize}
%\item
\noindent\paragraph{Accountability}Our light client system is accountable, i.e., if the light client verifier is misled and the transcript of its communication is given to the network then one can identify a large
number (e.g., 1/3) of misbehaving consensus participants (e.g., validators in our case). Identifying misbehaving consensus participants
is challenging in the light client system context when we want to send minimal data to the light client verifier. However,
identifying misbehaviour is necessary for any proof of stake protocols including Polkadot and Ethereum whose security relies on identifying and punishing misbehaving consensus participants.
%\item
\vspace{-0.2cm}
\noindent\paragraph{Asynchronous Safety}Our light client system has asynchronous safety i.e., under the consensus' honesty assumptions, our light client verifier cannot be misled even if it has a restricted view of the
network, e.g., only connecting to one node, which may be malicious. This is because our light client system inherits the property of asynchronous safety from the Byzantine agreement
protocol of the blockchain. Such light client systems would not be possible for consensus based on longest chains.
%\item
\vspace{-0.2cm}
\noindent\paragraph{Succinctness}Our light client system is incremental - i.e its succinct state is incrementally updated - it is optimised to make these updates efficiently, which is particularly relevant for the bridge application, as opposed to trying to optimise verifying consensus decisions from the blockchain genesis.
%\end{itemize}
\vspace{-0.2cm}
\subsection{Impact on decentralisation} For a blockchain network, having  a large number of validators  contributes greatly to better decentralization. This leads to better security both in terms of less point of physical failure and being able to distribute control over consensus which makes collusion harder. Some protocols have restricted their validator numbers to make light clients or bridges more efficient, e.g., by being able to run a DKG for threshold signatures (e.g., Dfinity~\cite{dfinity}) or obtaining Byzantine agreement with all validators on every block (e.g., Cosmos~\cite{tendermint_paper}). More efficient light clients for blockchains with large validator sets offer both decentralisation and interoperability (bridging) without compromise.

\vspace{-0.25cm}
\subsection{Relevance to Bridge Security}
%\vspace{-0.2cm}
\noindent In this section we review the impact of our scheme on bridge security. Blockchain bridges are protocols that allow value transfer between blockchains. Bridges have frequently been the target of attacks. We note that \$1.2 billion has been stolen in attacks on insecure bridges during first 8 months of 2022 alone ~\cite{elliptic_harmony,elliptic_nomad}. Of the top 10 crypto thefts of all times, \$1.6bn out of \$3.4bn come from bridge attacks \cite{elliptic_nomad}. These confirm that bridges have frequently been a weaker point, compared to the security of the blockchains themselves and they carry a lot of economical value.

\noindent An ideal bridge would be as secure as the least secure of the two blockchains. The most secure bridges use on-chain light client systems, e.g., Cosmos IBC protocol~\cite{IBC_paper}, to achieve this. Each bridged chain follow the other chain's consensus on-chain. To simplify, we will consider an on-chain light client of chain B on chain A, although B will also have the same for A. If B's consensus and the on-chain logic of A are secure, then adversary cannot convince the logic of A that B decided some event that B's clients do not agree as decided. This translates to the adversary for example not being able to create value on A without having locked any value on B.

\noindent A main reason why bridges might not use this approach is efficiency. Smart contracts and other on-chain logic is an extremely resource constrained environment compared to browsers or phones that light clients might target. One approach for efficiency is to design B's consensus so that the light clients are cheaper, for example by reducing the validator numbers. Cosmos chains currently have 33-175 validators~\cite{CosmosValNYX},\cite{CosmosValHUB}. Many chains have many more, e.g., Ethereum's hundreds of thousands of validators, for more decentralisation and security. Alternatively, the light client can use threshold signature may be used however that means not having the same accountability guarantees and also limits validator numbers in practice, both discussed elsewhere.

% Notaries are obviously bad but lots of money has been stolen from such bridges [TODO: Expand this].

\noindent Another approach to reducing on-chain complexity is optimism. Entities make a claim on chain A that something happened on chain B and this is accepted if no entity makes an on-chain challenge within a certain time, claiming that this is incorrect and triggering a more expensive procedure. A bridge that uses this approach in Optics for bridging Celo to other chains~\cite{CeloOptics}. A less extreme example of this approach is NEAR's Rainbow bridge~\cite{NEARrainbowB}, where signatures are stored but not checked unless the correctness of a signature is challenged. The optimism approach relies on the censorship resistance of blockchain for security. In practice, blockchains may be censored for a period of time by an attacker with enough resources. An example of this was the result of the first round of Fomo3D on Ethereum~\cite{Fomo3DPM}, a smart contract that would pay a jackpot, a large amount (in the end 10,469 ETH), to the last user to pay the contract when no user does so for 30 seconds. The jackpot grew to such a large amount that it was worth a user buying up all the block space for 30 seconds~\cite{Fomo3DPM}. For a claim and challenge protocol, the challenge is itself quite computationally expensive, so it may be sufficient to increase the cost of computation, the gas price on Ethereum, to make such a challenge unprofitable. Security against this attack requires a large reward for challenges or a long challenge period. For example the rainbow bridge has an 8 hour challenge period~\cite{RainbowBridgeFAQ}. Long challenge periods would mean that bridge operations take a long time with consequences for usability.

\noindent Stakers in proof of stake protocol have an incentive for the chain (chain A) using that protocol to keep working, however they may not have stake in a chain (chain B) bridged to their chain. As a result, they may have no particular incentive in the correct functioning of a light client of chain A on chain B and so not to mislead the light client. In the case when the protocol of chain A has slashing, if an accountable light client on chain B is misled, one can prove to chain A, using information that is publicly available on chain B, of validator of chain A misbehaving in a way that will result in those validators being slashed on chain A. This gives the bridge similar economic security to chain A itself.

\vspace{-0.08in}
\subsection{Applicability of Our Scheme}

\noindent Our scheme is applicable to proof of stake blockchains where if something is decided by the chain, then a message is signed by some threshold fraction of a validator set, defined as a set of nodes or their public keys, which changes at well-defined times, those changes being signed by an appropriate threshold of the existing set. As mentioned such chains as Polkadot, the many Cosmos chains, or Ethereum fit this model. Our scheme is not applicable to chains using proof of work or many other proof of X schemes. Nor proof of stake protocols when only random validators or random subsets of validators decide something and the whole set never votes, such as protocols using the longest chain rule without a finality gadget.

\noindent Our scheme might well require a hard fork to be applied to many blockchains, especially those that have not implemented the required cryptography. It should be easily implementable for chains that use BLS signatures for consensus but those using signatures that do not support aggregation (e.g., the many using Ed25519), would need to use SNARKs with much slower prover time (e.g., zkBridge~\cite{zkBridge} for Ed25519)). To naively implement our scheme, we would also want validators to compute and sign the commitment to the next set. We note however that this is not strictly necessary, as the commitment could be computed on chain, maybe in a smart contract, as long as light client proof of the result of this computation can be constructed. This would result in longer proofs that cover validator set changes. For blockchains with expensive on-chain computation, native code support for the cryptography we use e.g., with precompiles for smart contracts might be required. It is planned to make the required changes to Polkadot and implement this scheme for it. We discuss in detail what would be required for a light client of Ethereum in Appendix~\ref{sec:ethereum}.
\vspace{-0.1in}
%The following paragraph should be commented out and changed in case of a conference submission.
\paragraph{Structure} The paper is organised as follows.
In Section~\ref{sec:sketch}, we sketch our proposed protocols and compare
them to existing work. In
Section~\ref{sec_implementation} we give benchmarks for our custom SNARKs
implementation.
In Section~\ref{sec_prelims}, we give cryptographic
preliminaries necessary for later sections. In Section~\ref{sec_apk_proofs}
we describe one of our custom SNARKs and our committee key scheme.
 We conclude in Section~\ref{conclusions}. Our paper includes
an extensive appendix for more details.% as follows.
%\vspace{-0.01in}
\vspace{-0.25cm}
\section{Our Solution}
\label{sec:sketch}
\vspace{-0.1cm}

In this section we present a sketch of our solution for both the committee key scheme and the accountable light client system,
then describe the technical challenges and contributions and finish with an overview of related work.

\vspace{-0.3cm}
\subsection{Sketch of Committee Key Scheme}
\label{sec:lcsketch}
\vspace{-0.1cm}

\noindent Suppose that a prover wants to prove to a verifier that a subset $S$ of some set $T$ of signers with equal stakes have signed a message.
One obvious approach would be using BLS aggregatable signatures with the following steps:

\begin{itemize}
\item[a.] Verifier knows all public keys $\{\mathit{pk}_i\}_{i \in T}$ of signers.% in $T$.

\item[b.] Prover sends the verifier an aggregatable signature $\sigma$ and a representation of the subset $S$.

\item[c.] Verifier computes the aggregate public key $\mathit{apk}=\sum_{i \in S} \mathit{pk}_i$ of the public keys of signers in $S$.
Then it verifies the aggregatable signature $\sigma$ for the aggregate public key $\mathit{apk}$ and it accepts if the verification succeeds.
\end{itemize}

\noindent However, we can represent a subset $S$ of a list of signers compactly using a bitvector $b$:
the $i$th signer in the list is in $S$ if and only if the $i$th bit of $b$ is $1$. Our committee key scheme describes an alternative approach:

\begin{itemize}
\item[a'.]\label{a'} Verifier knows a commitment $C$ to the list of public keys $(pk_i)_{i \in T}$.

\item[b'.]\label{b'} Prover sends the verifier an aggregatable signature $\sigma$, a bitvector $b$ representing $S$, an aggregate public key
$\mathit{apk}$ and $\pi$, a succinct proof that $\mathit{apk}=\sum_i b_i \mathit{pk}_i$ i.e.,
that $\mathit{apk}$ is the aggregate public key for the subset of signers in $S$ given by the bitvector $b$; all of the public keys in $S$ are a subset
of the list of public keys committed to using $C$.

\item[c'.] The verifier using $C$, $\mathit{apk}$ and the bitvector $b$ checks if $\pi$ is valid.
It then verifies $\sigma$ against $\mathit{apk}$ and accepts if both steps succeed.
\end{itemize}

\noindent With the above committee key scheme, if $C$ and $\pi$ are constant size,
the communication cost becomes $O(1)+|T|$ bits instead of $|T|$ public keys.

So far we have implicitly assumed validators have equal stakes.
One can generalise our approach  above to validators with unequal stakes by including at~\ref{a'}, a'.,  a commitment to all stakes
and to~\ref{b'}, b'., a claimed total signing stake that can be proved by including, inside the SNARK, a scalar product between stakes of the signing validators and the bitvector.
The bitvector remains part of the input as it is needed for ensuring accountability.

\subsection{From CKS to Accountable Light Client}
\label{sec_intro_committee}

Below we sketch how a light client verifier uses our committee key scheme. Suppose that a light client verifier wants to know some information $\mathit{info}_n$ about the state of a blockchain at block number $n$ without having to download the entire blockchain. Another entity, a full node, who knows all the data of the blockchain and is following the consensus, should be able to convince the light client verifier using a computational proof that $info_n$ was indeed decided.

We assume that $\mathit{info}_n$ can be proven from a commitment to the state at block number $n$ that is signed by validators, here we assume that this commitment is a block hash $H_n$. To convince the light client verifier that $H_n$  was decided, the full node needs to convince the light client verifier that a threshold number $t$ of validators from the current validator set signed $H_n$, where $t$ depends on the type of consensus. Byzantine fault tolerant based consensus often uses $t$ to be over 2/3 of the total number of validators.

\noindent\paragraph{Keeping Track of the Validator Set:} A light client verifier must be initialised with a committee key $cpk_1$ corresponding to the genesis validator set with keys $\bf{pk}_1$. At the end of each epoch, i.e., the time a validator set needs to be updated, the validators set of epoch $i$, with keys $\bf{pk}_i$ sign a message $(i,cpk_{i+1})$ where $cpk_{i+1}$ is a commitment to the validator set for the next epoch $\bf{pk}_{i+1}$. The light client verifier keeps track of $cpk_i$ for each epoch. A light client proof must include a committee key scheme proof that a bitvector of validators, with a threshold number of 1s, with keys committed to in $cpk_i$ signed $(i,cpk_{i+1})$. To convince a light client verifier knowing only $cpk_1$ of something in block $n$, all such proofs up to the epoch containing block $n$ must be included. For an incremental light client system, such as one on a bridge, these validator set update proofs only need to be given once an epoch.
\vspace{-0.05in}

\noindent\paragraph{Proving the General Claim $\mathit{info}_n$:} Once the light client verifier is convinced of $cpk_{n-1}$ for the epoch $n-1$ and $t$ of the validators in epoch $n-1$ signed $H_n$, it needs a committee key scheme proof for $cpk_n$ and a bitfield with $t$ ones that $t$ validators signed $H_n$. Finally, such a proof needs the opening of the commitment $H_n$ to $\mathit{info}_n$.
\vspace{-0.08in}

\noindent\paragraph{Accountability:} Now suppose that a full node obtains a light client proof for something that contradicts something it sees as decided
by the blockchain. For our bridge use case, all light client proofs will be publicly available on another blockchain. We assume that we can express this
contradiction in terms of a pair of messages that should never be signed by an honest validator, and that any validators doing so can be punished.
These we call incompatible messages. In this example, such pairs of messages should include validator sets commitment to different
commitments $(i,cpk_{i+1})$ and $(i,cpk'_{i+1})$, $cpk_{i+1} \neq cpk'_{i+1}$ and similarly distinct $H_n$ and $H'_n$. If a light client proof
contains a message signed by a committee key which is a commitment to the public keys of a known set of validators which is incompatible with a
message the same set signed on the blockchain, then the signature in the proof is a valid BLS signature with the claimed set of signers and so the full
node should be able to report that public keys that signed both messages misbehaved. If an incompatible message was signed by a committee key
which doesn't correspond to the claimed epoch's validator set, then at some point previously the light client proof must have shown that the committee
key for some correct validator set signed the wrong committee key for the next set which is a message that is incompatible with the correct committee
key that they signed on the real chain. Note that the accountability of our light client system instantiation relies on the accountability of the
underlying consensus protocol. Indeed, our light client is accountable only if signatures on incompatible messages are enough for consensus
accountability e.g., in Casper FFG~\cite{CasperFFG} and it is not directly applicable to consensus protocols where forensics (such as in~\cite{forensics})
are required for accountability, e.g., Polkadot's GRANDPA, Section 4.1~\cite{GRANDPA}. If the consensus protocol is not accountable with signatures,
then the consensus protocol needs to be modified by adding another layer (e.g. ABC\cite{ABC}, Polkadot's BEEFY~\cite{BEEFY}).
\vspace{-0.05in}.

\noindent\paragraph{Efficiency Gain:} If one follows the obvious approach described above using BLS aggregation and aims to convince the
light client verifier that $\mathit{info}_n$ is decided, then one needs to send $O(v)$ public keys for each validator set change, where $v$ is
the upper bound on the size of the validator set. Using our succinct committee key scheme however, one requires only a constant size
proof and $v$ bits for each validator set change to convince the light client that $\mathit{info}_n$ was decided. Since a public key or
signature typically takes 100s of bits, our approach achieves much smaller proof sizes.  More details our achieved efficiency are
available in Section~\ref{sec_implementation}.
\vspace{-0.05in}

\noindent\paragraph{Formalisation:} We give a formal model for the security properties of our accountable light client in Appendix~\ref{sec:LCinstantiation}.

\vspace{-0.1cm}
\subsection{Our Custom SNARKs}
\label{sec_intro_custom_snarks}
%\vspace{-0.1cm}

%\noindent In the following we discuss how we use custom SNARKs with efficient prover time to implement the committee key scheme.
%While we achieved very fast proving times in our implementation, it came at the cost of not using a general purpose SNARK protocol,
%leading to a more involved security model and the necessity of additional security proofs. \\
\noindent Here we discuss how we use custom SNARKs with efficient prover time to implement our committee key scheme.
While we achieved very fast proving time in our SNARKs implementation, this came at the cost of not using a general purpose
SNARK protocol, in turn leading to a more involved security model and the necessity of additional security proofs.  \\

\vspace{-0.05cm}
%\noindent Our SNARKs have inputs an aggregate public key $\mathit{apk}$, a commitment $C$ to a list of public keys $(pk_i)$, and a bitvector $(b_i)$.
%It needs to be a proof that  $apk=\sum_i b_i pk_i$. The SNARK verifies witnesses to the partial sums $kacc_j = h + \sum_{i=0}^j b_i pk_i$
%where $h$ is chosen to avoid the incompleteness of the addition formulae we use.} We list two of the optimisations we used for our custom SNARK below.
\noindent The public inputs for our SNARKs are: an aggregate public key $\mathit{apk}$, a commitment $C$ to the list of
public keys $(pk_i)_{i \in T}$ and a bitvector $(b_i)_{i \in T}$ succinctly representing a subset $S$ of public keys.
Our SNARKs provers output a proof that $apk=\sum_{i \in T} b_i pk_i$ and that $C$ is the commitment to the list of public keys
$(pk_i)_{i \in T}$. However the list itself is a witness for the relations defining our SNARKs and so the verifiers do not need it
and do not have to parse or check anything based on this possibly long list.
We detail below two further optimisations of our custom SNARKs.

%\begin{itemize}
%\item Our scheme is an instance of commit and prove SNARKs (see section~\ref{sec:commit_prove} for more details) that works as follows.
%The verifier takes only a commitment to part of the input of the SNARK, for us $C$ is a commitment to the list of public keys, and the list public
%keys themselves are not used by the verifier. We use the same polynomial commitment scheme for this purpose as is used in th SNARK itself.
%Hence, we do not need to add the decommitment of $C$ to the SNARK constraint system. Since our constraint system is simple adding a decommitment
%to use, e.g. a hash for the commitment, would increase the size of the constraint system and lead to several orders of magnitude increase in prover time.
%The tradeoff is that we cannot use an existing SNARK system or polynomial constraint compiler as a black box, making the proofs in this paper more complicated.}
%\item
\vspace{-0.05in}
\paragraph{Commit and Prove SNARKs:} Our SNARKs are an instance of commit and prove SNARKs (see Section~\ref{sec:commit_prove}).
The underlying commitment scheme used for computing the public input commitment $C$ is the same as the (polynomial) commitment scheme used in the rest of our SNARK(s). Hence, we do not need to add a witness for $C$
to the SNARK constraint system in the same way we would have to if our commitment scheme were, e.g., to use a hash function.
The constraints for checking a hash inside our custom SNARKs would increase the size of the constraint system so much that it would lead to several orders of magnitude increase in our prover time.
The trade-off for our SNARKs design (i.e., with a commitment as part of the public input) is that we cannot use an existing SNARK compiler as a black box.

%\item Our constraint system is of a form where the wiring together of different constraints is trivial enough that we can avoid doing a permutation argument
%or sparse matrix vector product that general proving systems would use to wire together gates. This reduces the proof size and proving time.}
%\item
\vspace{-0.05in}
\paragraph{Constraint System Simplicity: }Our constraint system is simple enough such that our custom SNARKs do not require a permutation argument or a matrix-vector product argument
which general proving systems need to bind together gates. In fact, the underlying circuit for our SNARKs can be described as an affine addition gate with a couple of constraints added to avoid the incompleteness of our addition formulae. This simplification leads to smaller proof sizes  and faster proving times.


%\vspace{-0.2cm}

\vspace{-0.2cm}
\subsection{Related Work}

\subsubsection{Naive Approaches and Their Use in Blockchains}
There are a number of approaches commonly used in practice to verifying that a subset of a large set signed a message.
%However, among these, the approaches that have slow verification limit the size of the validators' set, which in turn limits decentralisation.
\vspace{-0.2cm}
\noindent \paragraph{Verify All Signatures}  One could verify a signature for each signing validator. This is what participants  do in protocols like Polkadot~\cite{polkadot}, with 297 validators
% \cite{PolkaExplorer}
(or Kusama with 1000 validators) %\cite{PolkaExplorer}
and Tendermint~\cite{tendermint_paper}, which is frequently used with 100 validators). %\cite{CosmosExplorer}).
The Tendermint light client system, which is accountable and uses the verification of all individual signatures approach,
is used in bridges in the IBC protocol\cite{IBC_paper}. This approach becomes prohibitively expensive for a light client verifier when there are 1000s or millions of signatures.
\vspace{-0.1in}
\noindent \paragraph{Aggregatable Signatures} One could use an aggregatable signature scheme like BLS~\cite{BLS_signatures,boneh_compact_multisig}  and reduce this to verifying one signature, but that requires calculating an aggregate public key. This aggregate key is different for every subset of signers and needs to be calculated from the public keys. This is what Ethereum
%~\cite{eth2}
does, which currently has 415,278 validators. %\cite{EthExplorer}.
However for a light client verifier, it is expensive to keep a list of 100,000s of public keys updated. As such only full nodes of Ethereum use this approach and instead light clients verifiers of Ethereum~\cite{sync_committee} follow signatures of randomly selected subsets of validators of size 512. This means that the resulting light client system is not accountable because these 512 validators are only backed by a small fraction of the total stake.
\vspace{-0.2cm}
\noindent \paragraph{Threshold Signatures} Alternatively a threshold signature scheme may be used, with one public key for the entire set of validators. This approach was adopted by Dfinity~\cite{GrothDKG}. Threshold signature schemes used in practice use secret sharing for the secret key corresponding to the single public key. This gives the schemes two downsides. Firstly, they require a communication-heavy distributed key generation protocol for the setup which is difficult to scale to large numbers of validators. Indeed, despite recent progress~\cite{AggregatableDKG,GrothDKG,LWEDKG}, it is still challenging to implement setup schemes for threshold signatures across a peer-to-peer network with a large number of participants, which is what many blockchain related use cases require. Moreover, such a setup may need repeating whenever the signer set changes. Secondly, for secret sharing based threshold signature schemes, the signature does not depend on the set of signers and so we cannot tell which subset of the validators signed a signature i.e., they are not accountable. Dfinity~\cite{GrothDKG} uses a re-shareable BLS threshold signature, where the threshold public key remains the same even when the validator set changes. Such a signature scheme
provides the light client verifier with a constant size proof, even over many validator set changes, but means that the proof not only does not identify which of a particular set of validators are misbehaving, but also we cannot say when this misbehaviour happened i.e., which validator set misbehaved. This is because the signature would be the same for any threshold subset of any validator set.

%\vspace{-0.2cm}
\noindent It is worth noting that if a protocol has already implemented aggregatable BLS signatures, our committee key scheme can be used without altering the consensus layer. Indeed it may be easier to alter a protocol that uses individual
signatures to use aggregatable BLS signatures than to implement threshold signatures from scratch because the latter requires waiting for an interactive setup before making validator set changes.

%\vspace{-0.2cm}
\subsubsection{Using SNARKs to Roll up Consensus}
%\vspace{-0.1cm}

\noindent{Celo~\cite{celo} and Mina~\cite{mina} blockchains have associated light client verifiers which allow their resource constrained users
to efficiently and securely sync from the beginning of the blockchain to the latest block.}

%\vspace{-0.2cm}
\noindent \paragraph{Plumo~\cite{plumo}} is the most relevant comparison to our scheme. It also tackles the problem we consider, i.e., that of
proving validator set changes. In more detail, Plumo uses a Groth16 SNARK~\cite{groth16} to prove that enough validators signed
a statement using BLS signatures from a set of the public keys. In Celo~\cite{celo}, the blockchain that designed and plans to use
Plumo, validators may change every epoch which is about a day long and the Plumo's SNARK iteratively proves 120 epochs worth of
validator set changes. Since in Celo there are no more than 100 validators in a validator set at any one time, the respective public
keys are used in plain as public input for Plumo's SNARK, as opposed to a succinct polynomial commitment in the case of our custom SNARKs.
All of the above increase the size of Plumo's prover circuit. Since Plumo is designed to help resource constrained light clients sync from scratch,
it is not an impediment that the Plumo SNARK cannot be efficiently generated, i.e., in real time. In the case of a light client verifier for bridges
(i.e., the most resource constrained application), we expect it to be in sync at all times and, by design, we care only about one validator
set change at a time. Our slimmed down and custom SNARK not only can be generated in real time, but, also due to the use of specialised
commitments schemes for public keys, our validator sets can scale up to much larger sizes as well without impacting the efficiency of our system.

\begin{comment}
\paragraph{Mina} achieves light clients with $O(1)$ sized light client proofs using recursive SNARKs.
This requires some nodes have a large computational overhead to produce these proofs.
%Also because this requires verifying consensus with small circuits, they do not use the consensus paradigm discussed above where a majority of validators sign, and instead use a longest chain rule version of proof of stake~\cite{mina}.
Their protocol is not accountable because, as with Dfinity above, it is not possible to tell from the proof which validators signed off on a fork, nor when this happened.
%Another downside is that because the proof only shows the length of a chain (and its block density), similar to a Bitcoin SPV proof, a light client needs to be connected to an
%honest node to tell if a block is in the longest chain. If the client is connected to a single malicious node, it could be given a proof for a shorter fork and not see any proofs of chains the fork choice rule would preder.
\end{comment}

%\vspace{-0.2cm}
\paragraph{Mina~\cite{mina}} achieves light clients with $O(1)$ sized light client proofs using recursive SNARKs. This requires some nodes have a large computational overhead to produce proofs. Also because this requires verifying consensus with small circuits, they do not use the consensus paradigm
discussed above where a majority of validators sign, and instead use a longest chain rule version of proof of stake~\cite{mina}.
Their protocol is not accountable because, as with Difinity above, it is not possible to tell from the proof which validators signed off on a fork, nor when this happened.
Another downside is that because the proof only shows the length of a chain (and its block density), similar to a Bitcoin SPV proof, a light client needs to be
connected to an honest node to tell if a block is in the longest chain. If the client is connected to a single malicious node, it could be given a proof for a
shorter fork and not see any proofs of chains the fork choice rule would prefer.
\vspace{-0.03cm}
\subsubsection{Commit-and-Prove and Related Approaches}
\label{sec:commit_prove}

\noindent Our custom SNARKs are an instance of the commit-and-prove (CP) paradigm~\cite{KilianPhD,CLOS02,CP_proposal,HP_paper,CP_paper, lunar, eclipse}
which is a generalisation for zero-knowledge proofs/arguments in which the prover proves statements about values that are committed. We employ this paradigm by using a commitment to a large dataset as part of the public
input, in a similar way to \cite{HP_paper,CP_paper}. Other CP SNARK protocols (\cite{CP_paper,interoperability_2, lunar, eclipse} consider having (ZK) SNARKs with portions of their respective witnesses overlapping, a very
different problem to ours. Hash-and-prove~\cite{HP_paper} considers using hashes of a large input dataset as part of the public input for a SNARK. LegoSNARK ~\cite{CP_paper} considers using commitments which work
with the SNARK protocols more directly and don't need an opening to be explicitly verified inside the SNARK. \cite{CP_paper} claims a 5000x speedup in proving time using this technique for Groth16 instead of hashing. Our
custom schemes use a version of this approach, using KZG polynomial commitments~\cite{KZG_10} as partof of the public inputs, and we expect that several orders of magnitude for our proving time improvement over Celo's
Plumo protocol that uses hashes comes from this. Additionally, we also have a very simple constraint system that would have a considerable increase in size if we hashed the input.
Our formalism, which generalised PLONK~\cite{plonk}'s polynomial protocols and compiler to work with polynomial commitments for the input, is different from prior work on CP SNARKs.


\begin{comment}
In practice, commit-and-prove systems (for short, CP) can be used to compress a large data structure and then prove something about its
content (e.g., polynomial commitments~\cite{KZG_10}, vector commitments~\cite{vector_commitment_1}, accumulators~\cite{first_accumulator}).
CP schemes can also be used to decouple the publishing of commitments to some data from the proof generation: each of these actions may be
performed by different parties or entities~\cite{zkp_reference}. Finally, commitments can be used to make different proof systems
interoperable~\cite{CP_paper,interoperability_2}. Our SNARKs have properties from the first two categories, however we could not
have simply re-used an existing argument system: by designing custom circuits and SNARKs, we ensured improved efficiency for our use cases.
\vspace{-0.1in}
\end{comment}


In this context, ECLIPSE~\cite{eclipse} presents a compiler that starts off with popular SNARKs (e.g., Sonic~\cite{sonic}, PLONK~\cite{plonk},
Marlin~\cite{marlin}) and via a new general compilation method generates CP-variants for these SNARKs.
Our proposed compiler uses as a first step the standard PLONK compiler.  As a second step we simply
re-cast the SNARK resulted in the first step as a SNARK for a new relation.
The security of the re-casting holds under mild conditions that deterministically relate some polynomials
processed by the verifier in the ranged polynomial protocol (before applying PLONK compiler) to some
public inputs. To our knowledge, our re-casting conditions are less stringent than the conditions needed
in~\cite{eclipse}.

We cannot use the ECLIPSE compilation technique either in full or in part to compile our custom SNARKs
since the types of NP relations derived after ECLIPSE compilation are simply incompatible with ours. While in the
case of ECLIPSE, the witnesses for the NP relations before compilation remain witnesses also for the relations
after compilation, in our case, some part of the public input before compilation becomes witness after the
re-casting of the SNARK for a new NP relation. Thus, overall, ECLIPSE and the current work solve different problems.
Our compilation method requires only the PLONK compiler without additional computational
steps so it is more efficient than the one in~\cite{eclipse}.

Finally, Lunar~\cite{lunar} obtains CP-SNARKs with a universal and updatable SRS and presents argument systems for
linking committed inputs to the polynomial commitments used in AHP-based arguments. Lunar presents efficiency improvements compared
to LegoSNARK~\cite{CP_paper}, but Lunar and ECLIPSE outperform each other  in different contexts.

%The paragraph below was a comment/was commented out when shortening the paper.
\begin{comment}
\noindent Another paradigm related to commit-and-prove is called hash-and-prove~\cite{HP_paper}: for large data structures or simply data that is expensive to be
handled directly by a computationally constrained verifier, one can hash that data and then create a (succinct) proof for some verifiable computation that uses
the original, large, dataset. The committee key scheme notion that we define in this work has both similarities to but also differences with regard to this
paradigm. The similarities are that, both the way we instantiate our committee key (i.e., using a polynomial commitment
with a trusted universal setup) and the way we instantiate our aggregate public key, can be generalised as some form of (possibly deterministic)
hash function. One difference is that the setup for the polynomial commitment is the same as that from which the proving and verification key for our committee key scheme are
computed; thus our version of the hashes and the keys for the committee key scheme are definitely not independent as in the case of hash-and-commit~\cite{HP_paper}. Finally,
built into our definition of committee key scheme and its security properties, we use a secure aggregatable signature scheme which allows us to design and
prove the security properties of our accountable light client(s). In fact, to add some intuition to the fact that a committee key scheme is more than
just a hash-and-prove instance, we mention that our committee key scheme inherits an unforgeability property from its aggregatable scheme sub-component.
This is one property that as far as we are aware no hash-and-prove scheme has. \\
\vspace{-0.08in}
\end{comment}

\begin{comment}
\noindent When proving the security of our arguments, we use an extension of some of the more commonly employed SNARK definitions which we call a ``a hybrid model SNARK''. This resembles the existing notion of SNARKs with online-offline verifiers as described in~\cite{HP_paper}, where the verifier computation is split into
two parts: during the offline phase some computation (possibly of commitments) happens; this computation takes some public inputs as parameters and, when not
performed by the verifier, it may also be performed (in part) by the prover. The online phase is the main computation performed by the verifier. In the case of our hybrid
model SNARKs, however, the input to the offline counterpart described above (which we call the $\mathit{PartInput}$ algorithm) may even be the witness or
a part of the witness for the respective relation. For our custom SNARKs, $\mathit{PartInput}$ produces part of the public input used by the verifier;
since for our use case, $\mathit{PartInput}$ does handle a portion of the witness, this operation cannot be performed by the verifier for that relation.
Moreover, in our instantiation, $\mathit{PartInput}$ produces computationally binding commitment schemes that are opened by the prover. Both of these properties
are not explicitly part of our general definition for hybrid model SNARKs, but they are crucial and explicitly assumed and used
in proving the security for our compiler's second step (see Appendix~\ref{sec_two_step_compiler}).
\end{comment}
\vspace{-0.1in}
%\begin{comment}
%\subsection{BLS multisignatures} \label{ssec:BLS}

%\noindent {\color{red} We implement and use an efficient BLS multisignature scheme that has both efficient 
verification and efficient key aggregation (for more details, see section~\ref{sec:bls}). 
In general, multisignatures are susceptible to so-called ``rogue-key attacks'' which can be 
mounted whenever the adversary is allowed to choose his public keys arbitrarily. 
In a typical rogue-key attack, the adversary uses a public key that is a function of an 
honest user's key and this allows him to produce forgeries easily. The BLS multisignatures that we define and use 
in this work make no exception. So, in order to protect against rogue-key attacks, 
we enhance them with proofs-of-posession as defined in~\cite{proofs_of_posession}.} \\

\noindent {\color{red}Alternative constructions for BLS multisignatures as well as other defence mechanisms against rogue key attacks exist and 
we briefly review both below. First, a more general case to BLS mltisignatures exists, namely aggregating BLS signatures for different messages 
(e.g., ~\cite{aggregate_BLS_signatures}). In this case, the rouge-key attacks are not a threat anymore, but multisignature verification 
is computationally more expensive than in our case, requiring $O(n)$ parings for $n$ different messages. Second, alternative 
BLS multisignatures exist (e.g.,~\cite{boneh_compact_multisig}) where both the aggregated public key have their size independent 
of the number of signers and the multisignature verification is as fast as in our variant. However, for our application we prefer the scheme 
detailed in section~\ref{sec:bls}: in spite of requiring a one-time verification of proofs-of-posession, its corresponding aggregate key is a 
simple sum of the individual signers' public keys, while, in ~\cite{boneh_compact_multisig} the key aggregation operation involves more 
expensive scalar multiplications every time the key aggregation is performed.}


%TODO: This could be explained somewhere other than the intro if preferred {\color{red}@Alistair from Oana: Can we replace or merge the 4 sentences below on multisignatures with/%into the more detailed description I give above?}

%There are several variants of BLS multisignatures. We will consider the easiest one for which the aggregate public key is simply the sum of the public keys of the signers

%This naive scheme is vulnerable to rogue key attacks, which can be prevented using proofs of possession [cn].  In our case, we can assume that the list of public keys comes from a %trusted source, such as the previous set of validators, who checked the proofs of possession and so the verifier themselves does not need to.

%\subsection{Implementation}
%\label{sec:intro_implementation}

%\noindent {\color{red} Our implementation leverages a pair of pairing-friendly elliptic curves which we call the inner curve and the outer curve respectively, such that the base field of the %inner curve matches the scalar field of the outer curve.} \\

%\noindent {\color{red}The first pair of pairing friendly elliptic curves where the inner curve's base field and the outer curve's scalar field are identical
%but the two curves do not form a cycle has been introduced by ZEXE~\cite{zexe}. The authors call such a pair of elliptic curves a two-chain.
%The ZEXE two-chain curves are BLS12-377 and CP6-782. In this work we build on the two-chain ZEXE instantiation in the following way: we
%keep BLS12-377 as the inner curve and, for efficiency reasons which will be detailed later in the paper, we replace the CP6-782 outer curve with
%BW6-761~\cite{BW6}.\\

%\noindent Both BLS12-377 and BW6-761 are pairing friendly curves and we make use of that as follows. Intuitively we use BLS12-377
%to sign and verify BLS signatures with the public keys being elements of the first source group of the efficient pairing associated with BLS12-377. Hence,
%our BLS signature public keys are natively represented over the base field of BLS12-377. Then we use BW6-761 to prove using succinct proofs
%(i.e., snarks) public key aggregation for public keys signing the same message. Because the base field of BLS12-377 matches the scalar field
%of BW6-761 and since in a snark system the proof is performed in an arithmetic circuit over the scalar field of the curve, so,
%in our case the scalar field of BW6-761, any efficiency loss due to curves mismatch is avoided. \\

%\noindent Note that even if our implementation is instantiated with a specific pairing-friendly two-chain as described above,
%our theoretical results (see Section \ref{sec:snarks}) generalise to any pairing-friendly two-chain and, where possible, we state
%them as such.}
%\end{comment}

% \vspace{-0.1in}
% \section{Preliminaries} \label{sec:conventions}
% \vspace{-0.03in}
\noindent We assume all algorithms receive an implicit security parameter $\lambda$. 
We use interchangeably ``efficient algorithm" or ``PPT algorithm'' to mean an algorithm that runs in uniform probabilistic polynomial 
time in the length of its input. A function $f(\lambda)$ is negligible in $\lambda$, written as $\mathit{negl}(\lambda)$, if $1/f(\lambda)$ 
grows faster than any polynomial in $\lambda$ and is overwhelming in $\lambda$ if $1-f(\lambda)=\mathit{negl}(\lambda)$. 
$\mathsf{poly}(\lambda)$ is some polynomial in $\lambda$ and \ewnp means except with probability $\mathit{negl}(\lambda)$.
We write $y = A(x; r)$ when algorithm $A$ on input $x$ and randomness $r$, outputs $y$. We write $y \leftarrow A(x)$ for picking 
randomness $r$ at uniformly at random and setting $y = A(x; r)$. Finally, for any integer $n \geq 1$, we denote by $[n]$ the set $\{1, \ldots, n\}$.
\vspace{-0.05in}

% \vspace{-0.1in}
% \subsection{Pairings}
\label{sec:pairings}

\noindent If $E$ is an elliptic curve defined over a prime field $\mathbb{F}_{p}$ of large characteristic $p$, 
we denote by $E(\mathbb{F}_{p})$ the abelian group containing all the points $(x, y) \in (\mathbb{F}_{p})^2$ 
that satisfy the elliptic curve equation along with the point at infinity. Let $r$ be a large prime such that $r$ divides 
$|E(\mathbb{F}_{p})|$ and $\mathit{gcd}(p, r) = 1$. The \emph{embedding degree of $E$} is the smallest integer $k$ 
such that $r$ divides $p^{k}-1$. If $k$ is small we say $E$ is \emph{pairing friendly}. 
We call $\mathbb{F}_p$ \emph{the base field of $E$} and $\mathbb{F}_r$ 
(i.e., the prime field of characteristic $r$) \emph{the scalar field of $E$}. \\

\noindent Pairing friendly curves are important to us in this work because they allow us to efficiently construct 
and instantiate aggregatable signatures and SNARKs. For a pairing friendly curve $E$ as above, 
let $\mathbb{G}_1$, $\mathbb{G}_2$ and $\mathbb{G}_T$ be appropriately chosen subgroups of order
$r$ in $E(\mathbb{F}_{p})$, $E(\mathbb{F}_{p^l})$ (for some $l \leq k$)\footnote{$E(\mathbb{F}_{{p}^l})$ is the group of all points 
$(x,y) \in (\mathbb{F}_{p^l})^2$ that satisfy the elliptic curve equation of $E$ along with the point at infinity.} and in the multiplicative group 
$\mathbb{F}_{p^k}^*$ of the extension field $\mathbb{F}_{p^k}$. The types of pairings we are interested in this work are mappings  
$e$ which are secure~\cite{secure_pairings, pairings_for_cryptographers}, efficiently computable, they are defined as 
$e: \mathbb{G}_1 \times \mathbb{G}_2 \rightarrow \mathbb{G}_T$ for 
which \emph{bilinearity} (i.e., $e(a \cdot g_1, b \cdot g_2) = e(g_1, g_2)^{a\cdot b}$, 
$\forall a, b \in \mathbb{Z}_r$, $\forall g_1 \in \mathbb{G}_1$, $\forall g_2 \in  \mathbb{G}_2$) 
and \emph{non-degeneracy} (i.e., if $g_1$ and $g_2$ are generators of $\mathbb{G}_1$ and 
$\mathbb{G}_2$, respectively, then $g_T = e(g_1, g_2)$ is a generator for $\mathbb{G}_T$) hold. \\ 

\noindent Our results in this work hold for a pair of pairing-friendly elliptic curves $\einn$ (\emph{the inner curve}) and  
$\eout$ (\emph{the outer curve}) such that the base field of $\einn$ equals the scalar field of $\eout$. In line with the naming 
from~\cite{zexe}, we call any pair of pairing friendly elliptic curves with such property a \emph{pairing-friendly two-chain}. 
We denote by $\mathbb{F}$ the base field of $\einn$ and we call $p$ its characteristic. We denote by $r$ the characteristic of
the scalar field of $\einn$. We also denote by $\epinn$ and by $\epout$ the efficient, secure pairings over $\einn$ and $\eout$, respectively. \\

\noindent We further denote by $\ginn{1}$, $\ginn{2}$ and $\gtinn$ the two cyclic source groups and the cyclic target group for $\epinn$ 
and $\sginn{1}$, $\sginn{2}$, $\sgtinn$ are uniformly random chosen generators of these three groups. Analogously, $\gout{1}$, 
$\gout{2}$ and $\gtout$ are the two cyclic source groups and the cyclic target group for $\epout$ and $\sgout{1}$, $\sgout{2}$, $\sgtout$ 
are uniformly random chosen generators of these three groups. We consider $\ginn{1}$, $\ginn{2}$, $\gout{1}$, $\gout{2}$ 
with additive notation for their group operation and we consider $\gtinn$ and $\gtout$ with multiplicative notation. 
We additionally write $[x]_{\indexoneinn} = x \cdot \sginn{1}$, $[x]_{\indextwoinn} = x \cdot \sginn{2}$. 
We assume that the curves, groups and fields defined in the last two paragraphs have been 
generated using implicit security parameter $\lambda$. \\

\noindent Finally, we note that in our implementation we instantiate $\einn$ with BLS12-377~\cite{zexe} and $\eout$ with BW6-761~\cite{BW6}.

%\noindent Our implementation leverages a pair of pairing-friendly elliptic curves. 
%For more details see section \ref{sec:intro_implementation}. For brevity we call them \emph{the inner curve} 
%(and by this we refer to the curve {\color{red}BLS12-377}) and \emph{the outer curve} (and we refer to the curve {\color{red}BW6-761}). Moreover, we note 
%that the base field of  {\color{red}BLS12-377} equals the scalar field of {\color{red}BW6-761}. In line with the naming from~\cite{zexe}, 
%we call any pair of pairing friendly elliptic curves with such property a \emph{pairing-friendly two-chain}. \\

%\noindent  We denote by $\mathbb{F}$ the common prime field of {\color{red} BW6-761} and {\color{red}BLS12-377} and we call $p$ its characteristic. 
%The scalar field of {\color{red}BW6-761} is also a prime field. We denote by $r$ its characteristic. We also denote by $e_{\mathit{BLS}}$ and by $e$ the efficient, secure 
%pairings over {\color{red}BLS12-377} and {\color{red}BW6-761}, respectively. $\mathbb{G}_{1,\mathit{BLS}}$, $\mathbb{G}_{2,\mathit{BLS}}$ and 
%$\mathbb{G}_{T,\mathit{BLS}}$ are the two cyclic source groups and the cyclic target group for $e_{\mathit{BLS}}$ 
%and $g_{1,\mathit{BLS}}$, $g_{2,\mathit{BLS}}$, $g_{T,\mathit{BLS}}$ are uniformly random 
%chosen generators of these three groups. Analogously, $\mathbb{G}_1$, $\mathbb{G}_2$ and $\mathbb{G}_T$ are the two cyclic 
%source groups and the cyclic target group for $e$ and $g_1$, $g_2$, $g_T$ are uniformly random chosen generators of these 
%three groups. \\ 

%\noindent  We consider $\mathbb{G}_{1,\mathit{BLS}}$, $\mathbb{G}_{2,\mathit{BLS}}$, $\mathbb{G}_{1}$, $\mathbb{G}_{2}$ 
%with additive notation for their group operation and we consider $\mathbb{G}_{T,\mathit{BLS}}$ and $\mathbb{G}_{T}$ with multiplicative notation. 
%We additionally write $[x]_{1, \mathit{BLS}} = x \cdot g_{1,\mathit{BLS}}$, $[x]_{2,\mathit{BLS}} = x \cdot g_{2,\mathit{BLS}}$,  
%$[x]_{T,\mathit{BLS}} = g_{T,\mathit{BLS}}^x$ and $[x]_1 = x \cdot g_1$, $[x]_2 = x \cdot g_2$, $[x]_T = g_T^x$. \\

%\noindent Our results (see section \ref{sec:snarks}) generalise to any pairing-friendly two-chain and, where possible, 
%we state them as such. For that we denote by $\einn$ and $\eout$ the inner curve and the outer curve, respectively, of any pairing-friendly 
%two-chain and by $\ginn{1}$ we denote the first source group of the pairing over $\einn$. We overload the notation and we denote by $\mathbb{F}$ the 
%common field of any pairing-friendly two-chain. Finally, we assume that the curves, group and field defined in this paragraph have been 
%generated using implicit security parameter $\lambda$. 





\section{Our Aggregatable Signature Scheme} \label{sec:aggregate}
\subsection{Secure Signature Aggregation}
\label{sec:def_aggregate}
An aggregatable signature scheme compresses signatures issued using
possibly different signing keys into one signature. In this work we use
an aggregatable signature scheme making explicit use of proofs-of-possession 
(PoPs)~\cite{proofs_of_posession} in order to protect against rogue-key attacks~\cite{proofs_of_posession}. 
%For our concrete instantiation, we use aggregatable
%BLS signatures with an efficient aggregation procedure, i.e., by adding
%together keys and by multiplying together signatures, and protect against
%rogue key attacks~\cite{proofs_of_posession} using PoPs. This is in contrast to other aggregation
%procedures that do not require PoPs for security but incur a higher computational 
%cost (e.g., due to the use of multi-scalar multiplication~\cite{boneh_compact_multisig}).
%For our concrete use case of accountable light clients systems, our 
%efficient signature aggregation method results in a simple and more efficient
%custom argument scheme (i.e., SNARK, see section~\ref{sec:use_case} for more details), 
%which, in turn, compensates for the cost of having to work with PoPs.

\begin{definition}
\label{def:aggregate}
(Aggregatable Signature Scheme) An aggregatable signature scheme consists of
the following tuple of algorithms ($\mathit{AS.Setup}$, $\mathit{AS.GenerateKeypair}$, $\mathit{AS.VerifyPoP}$, 
$\mathit{AS.Sign}$, $\mathit{AS.AggregateKeys}$, \\ $\mathit{AS.AggregateSignatures}$, $\mathit{AS.Verify}$):
\begin{itemize}

\item $\mathit{pp} \leftarrow  \mathit{AS.Setup}(\mathit{\lambda})$: a setup algorithm that, given a
security parameter $\lambda$, outputs public protocol parameters $\mathit{pp}$.

\item $((\mathit{pk},\mathit{\pi_{PoP}}),\mathit{sk}) \leftarrow \mathit{AS.GenerateKeypair}(\mathit{pp})$:
a key pair generation algorithm that
outputs
a secret key $\mathit{sk}$,
and the corresponding public key $\mathit{pk}$
together with a proof-of-possession $\mathit{\pi_{PoP}}$ of the secret key.

\item $0/1 \leftarrow \mathit{AS.VerifyPoP}(\mathit{pp}, \mathit{pk},\mathit{\pi_{PoP}})$:
a public key verification algorithm that,
given a public key $\mathit{pk}$
and a proof-of-possession $\mathit{\pi_{PoP}}$,
outputs
$1$ if $\mathit{\pi_{PoP}}$ is valid for $\mathit{pk}$ and $0$ otherwise.

\item $\sigma \leftarrow \mathit{AS.Sign}(\mathit{pp}, \mathit{sk}, m)$:
a signing algorithm that,
given a secret key $\mathit{sk}$ and a message $m \in \{0, 1\}^*$, returns a signature $\sigma$.

\item $\mathit{apk} \leftarrow \mathit{AS.AggregateKeys}(\mathit{pp}, (\mathit{pk_i})_{i=1}^{n})$:
a public key aggregation algorithm that,
given a vector of public keys $(\mathit{pk_i})_{i=1}^n$,
returns
an aggregate public key $\mathit{apk}$.

\item $\mathit{asig} \leftarrow \mathit{AS.AggregateSignatures}(\mathit{pp}, (\sigma_i)_{i=1}^n)$:
a signature aggregation algorithm that,
given a vector of signatures $(\sigma_i)_{i=1}^n$,
returns
an aggregate signature $\mathit{asig}$.

\item $0/1 \leftarrow \mathit{AS.Verify}(\mathit{pp}, \mathit{apk}, m, \mathit{asig})$:
a signature verification algorithm that,
given an aggregate public key $\mathit{apk}$, a message $m \in \{0, 1\}^*$, and an aggregate signature $\sigma$,
returns
1 or 0 to indicate if the signature is valid.
\end{itemize}
\noindent We say ($\mathit{AS.Setup}$, $\mathit{AS.GenerateKeypair}$, $\mathit{AS.VerifyPoP}$, 
$\mathit{AS.Sign}$, \\$\mathit{AS.AggregateKeys}$, $\mathit{AS.AggregateSignatures}$, 
$\mathit{AS.Verify}$) is an aggregatable signature scheme if it satisfies \emph{perfect completeness}, \emph{completeness for aggregation}
and \emph{unforgeability} as defined below.
\end{definition}
\noindent \textbf{Perfect Completeness} An aggregatable signature scheme
($\mathit{AS.Setup}$, \\ $\mathit{AS.GenerateKeypair}$, $\mathit{AS.VerifyPoP}$, $\mathit{AS.Sign}$, $\mathit{AS.AggregateKeys}$,\\ 
$\mathit{AS.AggregateSignatures}$, $\mathit{AS.Verify}$) has perfect completeness if for any message $m \in \{0,1\}^*$ and any 
$n\in\mathbb{N}$ it holds that:
\begin{align*}
 \mathit{Pr} & [\mathit{AS.Verify}(\mathit{pp}, \mathit{apk}, m, \mathit{asig})=1 \ \wedge \ \\
 & \wedge \  \forall  i \in [n]\ \mathit{AS.VerifyPoP}(\mathit{pp}, \mathit{pk_i},\mathit{\pi_{\mathit{PoP},i}})=1\ |\\
& \mathit{pp} \leftarrow \mathit{AS.Setup(\lambda)}, \\
& ((pk_{i},\pi_{\mathit{PoP}, i}), sk_{i} ) \leftarrow \mathit{AS.GenerateKeypair}(\mathit{pp}),\ i=1,\ldots,n\\
&  \mathit{apk} \leftarrow \mathit{AggregateKeys}(\mathit{pp}, (\mathit{pk}_{i})_{i=1}^{n}), \\
&  \sigma_i \leftarrow \mathit{AS.Sign}(\mathit{pp}, \mathit{sk_i}, m),\ i=1,\ldots,n, \\
&  \mathit{asig} \leftarrow \mathit{AS.AggregateSignatures(\mathit{pp}, (\sigma_{i})_{i=1}^{n})}] = 1.
\end{align*}
\noindent We note that an aggregatable signature scheme with perfect completeness implies the underlying signature scheme
has perfect completeness when aggregation of keys and aggregation of signatures for $n=1$ are given as the listing of the key and 
the individual signature, and where the verification of the aggregate signature for $n=1$ is the same as the verification of individual signatures. \\

\noindent \textbf{Completeness for Aggregation} An aggregatable signature scheme 
($\mathit{AS.Setup}$, $\mathit{AS.GenerateKeypair}$, $\mathit{AS.VerifyPoP}$, $\mathit{AS.Sign}$,\\
$\mathit{AS.AggregateKeys}$, $\mathit{AS.AggregateSignatures}$, $\mathit{AS.Verify}$)
has completeness for aggregation if, for every adversary $\mathcal{A}$
\begin{align*}
\mathit{Pr} & [\mathit{AS.Verify}(\mathit{pp}, \mathit{apk}, m, \mathit{asig}) = 1 \ (\ast\ast\ast\ast) \ | \ \mathit{pp} \leftarrow \mathit{AS.Setup}(\lambda), \\
& ((\mathit{pk_i}, \pi_{\mathit{PoP},i})_{i=1}^n, m, (\sigma_i)_{i=1}^{n}) \leftarrow \mathcal{A}(\mathit{\mathit{pp})}, \\
& \forall i \in [n], \mathit{AS.VerifyPoP}(\mathit{pp}, \mathit{pk_i}, \pi_{\mathit{PoP},i}) = 1 \ \ \ \ (\ast), \\
& \forall i \in [n], \mathit{AS.Verify}(\mathit{pp}, \mathit{pk_i}, m, \sigma_i) = 1 \ \ \ \ (\ast\ast),\\
& \mathit{apk} \leftarrow \mathit{AS.AggregateKeys}(\mathit{pp},  (\mathit{pk}_{i})_{i=1}^{n}), \\
& \mathit{asig} \leftarrow \mathit{AS.AggregateSignatures}(\mathit{pp}, (\sigma_i)_{i=1}^n) (\ast\ast\ast)] = 1- \negl(\lambda).
\end{align*}

\noindent \textbf{Unforgeable Aggregatable Signature}
For an aggregatable signature scheme ($\mathit{AS.Setup}$, $\mathit{AS.GenerateKeypair}$, $\mathit{AS.VerifyPoP}$, $\mathit{AS.Sign}$,
$\mathit{AS.AggregateKeys}$, $\mathit{AS.AggregateSignatures}$, $\mathit{AS.Verify}$)
the advantage of an adversary against unforgeability is defined by
$$\mathit{Adv}^{\mathit{forge}}_{\mathcal{A}}({\lambda}) = \mathit{Pr}[\mathit{Game}^{\mathit{forge}}_{\mathcal{A}}({\lambda}) =1], \textit{\ where}$$

\begin{align*}
&\mathit{Game}^{\mathit{forge}}_{\mathcal{A}}({\lambda}): \\
& \mathit{pp} \leftarrow \mathit{AS.Setup(\lambda)} \\
& ((\mathit{pk}^*,\pi^*_{\mathit{PoP}}), \mathit{sk}^*) \leftarrow \mathit{AS.GenerateKeypair}(\mathit{pp})\\
& Q \leftarrow \emptyset \\
& ((\mathit{pk_i}, \pi_{\mathit{PoP},i})_{i=1}^{n}, m, \mathit{asig}) \leftarrow \mathcal{A}^{\mathit{OSign}}(\mathit{pp}, (\mathit{pk^*},\pi^*_{\mathit{PoP}})) \\
& \textit{If } \mathit{pk}^* \notin \{\mathit{pk_i}\}_{i=1}^{n} \vee m \in Q, \textit{ then return } 0 \\
& \textit{For } i \in [n] \\
& \ \ \ \ \ \textit{ If } \mathit{AS.VerifyPoP}(\mathit{pp}, \mathit{pk_i}, \pi_{\mathit{PoP},i})=0  \textit{ return } 0 \\
& \mathit{apk} \leftarrow \mathit{AS.AggregateKeys}(\mathit{pp}, (\mathit{pk_i})_{i=1}^{n}) \\
& \textit{Return } \mathit{AS.Verify}(\mathit{pp}, \mathit{apk}, m, \mathit{asig})
\end{align*}
\noindent and
\begin{align*}
& \mathit{OSign}(m_j): \\
& \sigma_j \leftarrow \mathit{AS.Sign}(\mathit{pp}, \mathit{sk}^*, m_j) \\
&  Q \leftarrow Q \cup \{m_j\} \\
& \textit{Return} \ \sigma_j
\end{align*}
\noindent and $\mathcal{A}^{\mathit{OSign}}$ denotes the adversary $\mathcal{A}$ with access to oracle $\mathit{OSign}$. \\
\noindent We say an aggregatable signature scheme is unforgeable if for all efficient adversaries
$\mathcal{A}$ it holds that $\mathit{Adv}^{\mathit{forge}}_{\mathcal{A}}({\lambda}) \leq \mathit{negl}(\lambda)$. 

\subsection{Aggregatable BLS Signatures}
\label{sec:bls}
\noindent In the following, we instantiate the aggregatable signature definition given above with a scheme inspired by the BLS signature
scheme~\cite{BLS_signatures} and its follow-up variants~\cite{proofs_of_posession,boneh_compact_multisig}.
Because, in general, multisignatures are susceptible to so-called ``rogue-key attacks'' which can be
mounted whenever the adversary is allowed to choose his public keys arbitrarily, in order to protect against such
rogue-key attacks, we enhance our multisignature instantiation with proofs-of-possession as defined in~\cite{proofs_of_posession}.
In turn, we instantiate our proofs-of-possession with BLS signatures in the first source group.  
% the non-interactive version of Chaum-Pedersen proofs for showing 
%the equality of discrete logarithm (i.e., the secret key or DLEQ proof) corresponding to a pair of distinct public keys.

\begin{construction}(Aggregatable BLS Signatures) We call aggregatable BLS signatures the following instantiation of aggregatable signatures:
\label{insta_bls}
\begin{itemize}
%\item $(\GoneBLS, \GtwoBLS,  \goneBLS, \gtwoBLS, \GTBLS, \eBLS, \HBLS, \Hpop, \Hpk, \Hsig)$ from $\mathit{pp}$ where 
%$\mathit{pp} \leftarrow  \mathit{AS.Setup}(\lambda)$, 
%where $\GoneBLS$, $\GtwoBLS$,  $\goneBLS$, $\gtwoBLS$, $\GTBLS$, $\eBLS$ are the first and second source groups, their generators and the 
%associated pairing for some BLS elliptic curve $E$, respectively,  %were defined in section~\ref{sec:pairings} and 
%and $\HBLS: \{0,1\}^* \rightarrow \GoneBLS$, $\Hpop: \{0,1\}^* \rightarrow \GoneBLS$, 
%$\Hpk: \{0,1\}^* \rightarrow \mathbb{Z}_r^*$ and $\Hsig: \{0,1\}^* \rightarrow \mathbb{Z}_r^*$ are four hash functions. 

\item $(\GoneBLS, \GtwoBLS,  \goneBLS, \gtwoBLS, \GTBLS, \eBLS, \HBLS, \Hpop, \Hsig)$ from $\mathit{pp}$ where \\ 
$\mathit{pp} \leftarrow  \mathit{AS.Setup}(\lambda)$, 
where $\GoneBLS$, $\GtwoBLS$,  $\goneBLS$, $\gtwoBLS$, $\GTBLS$, $\eBLS$ are the first and second source groups, 
their generators and the associated pairing for some BLS elliptic curve $E$, respectively,  
and $\HBLS: \{0,1\}^* \rightarrow \GoneBLS$, $\Hpop: \{0,1\}^* \rightarrow \GoneBLS$ 
and $\Hsig: \{0,1\}^* \rightarrow \mathbb{Z}_r^*$ are three hash functions, where $r$ is the size of the scalar field of elliptic curve $E$. 

%\item $(\mathit{pk_1}, \mathit{pk_2}, \mathit{sk}, \pipk, \sigpop) \leftarrow \mathit{AS.GenerateKeypair}(\mathit{pp})$, 
%where $\mathit{sk} \xleftarrow{\$} \mathbb{Z}_{r}^{*}$ and $\mathit{pk_1} = \mathit{sk} \cdot \goneBLS \in \GoneBLS$ 
%and $\mathit{pk_2} = \mathit{sk} \cdot \gtwoBLS \in \GtwoBLS$ 
%and \\ $\pipk \leftarrow \Provpk(\goneBLS, \gtwoBLS, \mathit{pk_1}, \mathit{pk_2}, \mathit{sk})$ 
%and $\sigpop = \Hpop(\mathit{pk_2})^{\mathit{sk}}$ and $r$ is the size of the scalar field of elliptic curve $E$. 

\item $(\mathit{pk_1}, \mathit{pk_2}, \mathit{sk}, \sigpop) \leftarrow \mathit{AS.GenerateKeypair}(\mathit{pp})$, 
where $\mathit{sk} \xleftarrow{\$} \mathbb{Z}_{r}^{*}$ and $\mathit{pk_1} = \mathit{sk} \cdot \goneBLS \in \GoneBLS$ 
and $\mathit{pk_2} = \mathit{sk} \cdot \gtwoBLS \in \GtwoBLS$ and 
$\sigpop = \mathit{sk} \cdot \Hpop(\mathit{pk_2})$. 

%\item $0/1 \leftarrow \mathit{AS.VerifyPoP}(\mathit{pp}, \mathit{pk_1}, \mathit{pk_2}, \pipk, \sigpop)$, where $\mathit{AS.VerifyPoP}$ outputs $1$ if 
%$\Veripk(\goneBLS, \gtwoBLS, \mathit{pk_1}, \mathit{pk_2}, \pipk)$ outputs $1$ and if the following equality is fulfilled: 
%$$e(H(m), \mathit{pk}_2) = e(\sigpop, \gtwoBLS).$$
%\noindent In all other cases it outputs $0$.

\item $0/1 \leftarrow \mathit{AS.VerifyPoP}(\mathit{pp}, \mathit{pk_1}, \mathit{pk_2}, \sigpop)$, 
where $\mathit{AS.VerifyPoP}$ outputs $1$ if the following holds:
$$e(\Hpop(\mathit{pk}_2) + t \cdot \goneBLS, \mathit{pk}_2) = e(\sigpop + t \cdot \mathit{pk}_1, \gtwoBLS),$$
\noindent with $t \xleftarrow{\$} \mathbb{Z}_r$. If the verification above does not pass, 
then $\mathit{AS.VerifyPoP}$ outputs $0$.

%\item $(\sigma, \pisig) \leftarrow \mathit{AS.Sign}(\mathit{pp},\mathit{sk}, m)$: 
%where $\sigma = \mathit{sk} \cdot \HBLS(m) \in \GtwoBLS$ and $ \pisig \leftarrow \Provsig(\goneBLS, m, \mathit{pk_1}, \sigma, \mathit{sk})$.

\item $(\sigma, \pisig) \leftarrow \mathit{AS.Sign}(\mathit{pp}, \mathit{sk}, m)$: 
where $\sigma = \mathit{sk} \cdot \HBLS(m) \in \GoneBLS$ and $ \pisig \leftarrow \Provsig(\goneBLS, m, \mathit{pk_1}, \sigma, \mathit{sk})$. We 
give full details for the argument system $\PSsig = (\Gensig, \Provsig, \\ \Verisig)$ at the end of this paragraph.

%\item $\mathit{apk} = (\mathit{apk_1}, \mathit{apk_2}) \leftarrow \mathit{AS.AggregateKeys}(\mathit{pp}, (\mathit{pk_1^i}, \mathit{pk_2^i})_{i=1}^{n})$, 
%where  $\mathit{apk_1} = \sum_{i=1}^{n} \mathit{pk_1^i}$ and $\mathit{apk_2} = \sum_{i=1}^{n} \mathit{pk_2^i}$. 

\item $\mathit{apk}_1 \leftarrow \mathit{AS.AggregateKeys}(\mathit{pp}, (\mathit{pk_1^{(i)}}, \mathit{pk_2^{(i)}})_{i=1}^{n})$, 
where $$\mathit{apk}_1 = \sum_{i=1}^{n} \mathit{pk_1^{(i)}}.$$ 

%\item $\mathit{asig} \leftarrow \mathit{AS.AggregateSignatures}(\mathit{pp}, (\sigma_i, \pi_i)_{i=1}^n)$, 
%where \\ $\mathit{asig} = (\sigma_1, \pi_1)$  if $n=1$ and $\mathit{asig} = (\sum_{i=1}^{n} \sigma_i, \bot)$ if $n>1$. 

\item $\mathit{asig} \leftarrow \mathit{AS.AggregateSignatures}(\mathit{pp}, (\mathit{pk_1^{(i)}}, \mathit{pk_2^{(i)}})_{i=1}^{n}, (\sigma^{(i)}, \pi_i)_{i=1}^n)$, 
where \\ $\mathit{asig} = (\sigma^{(1)}, \pi_1)$  if $n=1$ and $\mathit{asig} = (\sum_{i=1}^{n} \sigma^{(i)}, \mathit{apk_2}, \bot)$ if $n>1$, where 
$$\mathit{apk}_2 = \sum_{i=1}^{n} \mathit{pk_2^{(i)}}.$$

\item $0/1 \leftarrow  \mathit{AS.Verify}(\mathit{pp}, \mathit{apk_1}, m, \mathit{asig})$, where $\mathit{AS.Verify}$ outputs $1$ iff either 
(100) or (200) hold; in all other cases, it outputs $0$. \\
 (100) $\mathit{asig}_3 = \bot \ \wedge \ e(\mathit{asig}_1 + t \cdot \mathit{apk_1}, \gtwoBLS) = e(H(m) + t \cdot \goneBLS, \mathit{asig}_2)$, 
where $$t \xleftarrow{\$} \mathbb{Z}_{r}^{*}.$$ 
(200) There exists no component $\mathit{asig}_3$ and $$\Verisig(\goneBLS, m, \mathit{apk_1}, \mathit{asig}_1, \mathit{asig}_2) = 1.$$ 

\end{itemize}
Above we have used the following argument system %systems 
\begin{comment}
$$ \PSpk = (\Genpk, \Provpk, \Veripk) \ \ \ \ (1)$$ where 
\begin{itemize}
\item $(\GoneBLS, \goneBLS, \GtwoBLS, \gtwoBLS, \Hpk) \leftarrow \Genpk(\lambda)$ as a subprotocol of $\mathit{AS.Setup}(\lambda)$. 
%We can also do proofs of possession by Chaum-Pedersen DLEQ proofs. Now we are proving the discrete log equality of 
%a pair of $G_1$ elements $g_1, pk_1$ and a pair of $G_2$ elements $g_2,pk_2$ (and also that the  dicrete log $sk$ is known).

\item $\pipk = (c,s) \leftarrow \Provpk(\goneBLS, \gtwoBLS, \mathit{pk_1}, \mathit{pk_2}, \mathit{sk})$ where \\ $k \xleftarrow{\$} \mathbb{Z}_{r}^{*}$, $A =k \cdot \goneBLS$, 
$B=k \cdot  \gtwoBLS$, $c = \Hpk (\goneBLS, \gtwoBLS, \mathit{pk_1}, \mathit{pk_2}, A, B)$, \\ $s = k - c \cdot \mathit{sk} \mod r$. 

\item $0/1 \leftarrow \Veripk(\goneBLS, \gtwoBLS, \mathit{pk_1}, \mathit{pk_2}, (c,s))$, 
where $\Veripk$ outputs 1 if  $c= \Hpk (\goneBLS, \gtwoBLS, \mathit{pk_1}, \mathit{pk_2}, A', B')$
where $A' = s \cdot \goneBLS + c \cdot \mathit{pk_1}$ and $B' = s \cdot \gtwoBLS  + c \cdot \mathit{pk_2}$ and it outputs $0$ otherwise.  
\end{itemize}

and 
\end{comment}
$$ \PSsig = (\Gensig, \Provsig, \Verisig) \ \ \ \ (1)$$ 
\begin{itemize}
\item $(\GoneBLS, \goneBLS, \Hsig) \leftarrow \Gensig(\lambda)$ as a subprotocol of $\mathit{AS.Setup}(\lambda)$. 
\item $\pisig = (c,s) \leftarrow \Provsig(\goneBLS, m, \mathit{pk_1}, \sigma, \mathit{sk})$ where \\ $k \xleftarrow{\$} \mathbb{Z}_{r}^{*}$, $A =k \cdot \goneBLS$, 
$B=k \cdot H(m)$, $c = \Hsig (\goneBLS, m, \mathit{pk_1}, \sigma, A, B)$, \\ $s = k - c \cdot \mathit{sk} \mod r$. 
\item $0/1 \leftarrow \Verisig(\goneBLS, m, \mathit{pk_1}, \sigma, (c,s))$, 
where $\Verisig$ outputs 1 if  $c= \Hsig (\goneBLS, m, \mathit{pk_1}, \sigma, A'', B'')$
where $A'' = s \cdot \goneBLS + c \cdot \mathit{pk_1}$ and $B'' = s \cdot H(m)  + c \cdot \sigma$ and it outputs $0$ otherwise.  
\end{itemize}
\end{construction}

%\noindent Note: It is easy to show that $\PSpk$ and $\PSsig$ are zero-knowledge non-interactive arguments of knowledge for relations 
%$\Rpk$ and $\Rsig$, respectively, where 
%$$\Rpk = \{( \GoneBLS, \GtwoBLS, \goneBLS, \gtwoBLS, \mathit{pk_1}, \mathit{pk_2}); \mathit{sk}) : \mathit{pk_1} = \mathit{sk} \cdot \goneBLS, \mathit{pk_2} = \mathit{sk} \cdot \gtwoBLS \},$$
%$$\Rsig = \{( \GoneBLS, \goneBLS, m, \mathit{pk_1}, \sigma); \mathit{sk}) : \mathit{pk_1} = \mathit{sk} \cdot \goneBLS, \sigma = \mathit{sk} \cdot H(m)\},$$
%and $\GoneBLS$, $\GoneBLS$ are generated by $\goneBLS, \gtwoBLS$, respectively.

\noindent Note: It is easy to show that $\PSsig$ is a zero-knowledge non-interactive argument of knowledge for relation $\Rsig$, where 
$$\Rsig = \{( \GoneBLS, \goneBLS, m, \mathit{pk_1}, \sigma); \mathit{sk}) : \mathit{pk_1} = \mathit{sk} \cdot \goneBLS, \sigma = \mathit{sk} \cdot H(m)\},$$
%and $\GoneBLS$, $\GtwoBLS$ are generated by $\goneBLS, \gtwoBLS$, respectively.
and $\GoneBLS$ is generated by $\goneBLS$. 

%Note that this takes the verifier 2 $G_1$ exponentiations and 2 $G_2$ exponentiations. This may not be a lot faster than the two 
%pairings required to verify a BLS signature with a different hash and using the same as the "Fast verification of aggregate public keys with proof" section.

%\begin{theorem} Assuming that co-CDH holds for $e$ (see Appendix for a reminder of this assumption) 
%and $H$, $\Hpop$ $\Hpk$ and $\Hsig$ are modelled as random oracles, then instantiation~\ref{insta_bls} 
%is an aggregatable signature scheme as per definition~\ref{def:aggregate}. 
%\end{theorem}
\begin{theorem} Assuming that co-CDH holds for $e$ (see Appendix~\ref{sec:appendix} for a reminder of this assumption) 
and $H$, $\Hpop$ and $\Hsig$ are modelled as random oracles, then Instantiation~\ref{insta_bls} 
is an aggregatable signature scheme as per Definition~\ref{def:aggregate}. 
\end{theorem}
\begin{proof} 
The \emph{perfect completeness} property is very easy to prove. \\ 
\noindent Regarding \emph{completeness for aggregation}, the non-trivial case is when $n>1$. 
Let $\mathcal{A}_1$ be an efficient adversary trying to break this property. Due to our instantiation, 
we have the following explicit notation 
$(\mathit{pk_i}, \pi_{\mathit{PoP},i})_{i=1}^n = ((\mathit{pk_1^{(i)}}, \mathit{pk_2^{(i)}}), \sigma^{(i)}_{\mathit{PoP}})_{i=1}^n$. 
Since $(\ast)$ holds, then by the Schwartz-Zippel lemma 
$e(\goneBLS, \mathit{pk}_2^{(i)}) = e(\mathit{pk}_1^{(i)}, \gtwoBLS), \forall i \in [n]$, which, due to the bilinearity of pairing $e$ 
and the fact that the associated source groups $\GoneBLS$ and $\GtwoBLS$ are cyclic, it implies that 
$\forall i \in [n]$, $\exists \ (\mathit{sk}_i)_{i=1}^n \in (\mathbb{Z}_r)^{n}$ such that $\mathit{pk}_1^{(i)} = \mathit{sk}_i \cdot \goneBLS$ and 
$\mathit{pk}_2^{(i)} = \mathit{sk}_i \cdot \gtwoBLS$. Since $(\ast\ast)$ holds, then due to the existential soundness of $\PSsig$,  
%there exist efficient extractor $\Esig$ such that, 
except with negligible probability, there exist $(\mathit{sk}_i)_{i=1}^n \in (\mathbb{Z}_r)^{n}$ such that 
$\mathit{pk}_1^{(i)} = \mathit{sk}_i \cdot \goneBLS$ and $\sigma^{(i)} = \mathit{sk}_i \cdot \HBLS(m)$. 
The above properties together with $(\ast\ast\ast)$ in turn, imply that, except with negligible probability, 
$\mathit{apk}_1 = (\sum_{i=1}^n \mathit{sk}_i) \cdot \goneBLS$, $\mathit{asig}_2 = (\sum_{i=1}^n \mathit{sk}_i) \cdot \gtwoBLS$ and 
$\mathit{asig}_1 = (\sum_{i=1}^n \mathit{sk}_i) \cdot \HBLS(m)$. This, in turn, implies that the aggregated signature verification $(\ast\ast\ast\ast)$ 
holds, except with negligible probability. \\ 

\noindent Regarding \emph{unforgeability}, let $\mathcal{A}_2$ be an efficient adversary trying to break this property 
(see Definition~\ref{def:aggregate}). %Without loss of generality, we assume that whenever $\mathcal{A}_2$ outputs 
%a potential forgery, and, in particular a set of public keys and their PoPs, $\mathcal{A}_2$ has already queried the random 
%oracle $\Hsig$ for all the hash values involved in the proofs and in the individual alleged signatures corresponding to the output 
%public keys and message of $\mathcal{A}_2$. Indeed, adding this constraint on the forger, does not decrease its probability of success. 
Next, we follow a similar proof technique as detailed for Theorem 15.2, part b in~\cite{dabo_book} where given a successful adversary 
$\mathcal{A}_2$ against unforgeability one can construct a successful adversary $\mathcal{B}$ against the co-CDH assumption as follows:
%For brevity, below we only state the differences and modifications in our proof: 

Adversary $\mathcal{B}$ is given a tuple $(u_1 = \alpha \cdot \goneBLS, u_2 = \alpha \cdot \gtwoBLS, v_1 = \beta \cdot \goneBLS)$ 
where $\alpha,\beta \xleftarrow{\$} \mathbb{Z}_{r}$, as in the co-CDH attack game (see Appendix~\ref{sec:appendix}); 
$\mathcal{B}$ needs to compute $z_1 = \alpha\beta \cdot \goneBLS = \alpha \cdot v_1$. 
%First, their $g_0$ is replaced in our case by $\goneBLS$ and their $g_1$ by our $\gtwoBLS$.  
First, $\mathcal{B}$ sends the public key $(\mathit{pk}^* = (u_1, u_2), \sigma^{*}_{\mathit{PoP}})$ to the forger 
$\mathcal{A}_2$, where $\sigma^{*}_{\mathit{PoP}} = \delta^* \cdot u_1$ and 
$\delta^* \xleftarrow{\$} \mathbb{Z}_r$, $\Hpop(\mathit{pk_2}^*) = \delta^* \cdot \goneBLS$. 
Hence $\sigma^{*}_{\mathit{PoP}} =  \alpha \cdot \Hpop(\mathit{pk_2}^*)$.  
%with $\pi^* = (c^*, s^*)$ where $(c^*, s^*)$ is computed by choosing $c^*, s^*$ 
%uniformly at random in $\mathbb{Z}_r^*$, setting $A^* = s^* \cdot \goneBLS + c^* \cdot \mathit{u_1}$ and 
%$B^* = s^* \cdot \gtwoBLS  + c^* \cdot \mathit{u_2}$ and finally $\mathcal{B}$ records in the corresponding table 
%the value for the simulated random oracle $\Hpk$ in $(\goneBLS, \gtwoBLS, \mathit{u_1}, \mathit{u_2}, A^*, B^*)$ 
%as equal to $c^*$.\footnote{These are also the steps taken by a zero-knowledge simulator for relation $\Rpk$.} 

Afterwards, $\mathcal{A}_2$ makes a sequence of queries: $Q_{\mathit{ro}}$ hash queries to $H$, $Q_{\mathit{sig}}$ signature queries, 
$Q_{\mathit{pop}}$ queries to $\Hpop$, %{\color{red} TO DO: DO I need explicit queries to the signature defining the PoPs?}
%$Q_{\mathit{DLEQ},\mathit{pk}}$ queries to $\Hpk$ 
and $Q_{\mathit{DLEQ},\mathit{sig}}$ queries to $\Hsig$. To these queries, 
adversary $\mathcal{B}$ responds as follows:

%\noindent The forger $\mathcal{A}_2$ makes a sequence of signatures and hash queries to simulated oracles $H$, $\Hpk$ and $\Hsig$ and 
%$\mathit{OSign}$ maintained by $\mathcal{B}$ as follows: 
\begin{itemize}
\item Overall, hash queries to $H$ are handled %as in $\mathcal{B}$'s simulation, theorem 15.2. part b in~\cite{dabo_book}
by $\mathcal{B}$ by first choosing a random $\omega \in \{1, \ldots, Q_{\mathit{ro}}\}$. 
(Note that among the $Q_{\mathit{ro}}$ hash queries there may be one message that is not part of the signature queries from 
$\mathcal{A}_2$ but is output as an alleged forgery by $\mathcal{A}_2$. Using random value $\omega$, $\mathcal{B}$ is trying to 
guess the index of that precise message queried by $\mathcal{A}_2$.)
Then, for $j = 1,2,..., Q_{\mathit{ro}}$, when $\mathcal{A}_2$ issues hash query number $j$ (i.e., query for $H(m_j)$), 
$\mathcal{B}$ responds by:
\begin{itemize}
\item if $j \neq \omega$ then $\mathcal{B}$ chooses $\rho_j \xleftarrow{\$} \mathbb{Z}_r$ and sets $H(m_j) :=  \rho_j \cdot \goneBLS$,
\item if $j = \omega$ then $\mathcal{B}$ sets $H(m_{\omega}) = v_1$.
\end{itemize}
\item Signing queries $m_j$ are answered as follows: The first component $\sigma^{(j)}$ of the individual 
signature for a message $m_j$ is $\rho_j \cdot u_1 = \alpha \cdot H(m_j)$. This can be answered correctly by $\mathcal{B}$ 
as long as he guesses the correct index $\omega$ defined above and, also, since all the corresponding values $\rho_j$ 
are chosen and known by him. Regarding the second component of the individual signature, i.e., 
$\pi^{(j)}_{\mathit{DLEQ}, \mathit{sig}}$, this can be computed by $\mathcal{B}$ using the fact 
that it can program the oracle for $\Hsig$ and, also, since the non-interactive argument of knowledge $\PSsig$ has zero-knowledge. Indeed, 
$\pi^{(j)}_{\mathit{DLEQ}, \mathit{sig}}$ is computed as $(c_j, s_j)$ where $c_j$, $s_j$ are chosen 
uniformly at random in $\mathbb{Z}_r^*$, setting $A_j = s_j \cdot \goneBLS + c_j \cdot \mathit{u_1}$ and 
$B_j = s_j \cdot H(m_j)  + c_j \cdot \sigma^{(j)}$ and finally $\mathcal{B}$ records in the corresponding table 
the value for the simulated random oracle $\Hsig$ in $(\goneBLS, m_j, u_1, \sigma^{(j)}, A_j, B_j)$ as equal to $c_j$. 
 
\item $\Hsig(\goneBLS, m', \mathit{pk_{1}^{(j)}}, \sigma^{(j)}, A_j^{'}, B_j^{'})$ queries are received, stored in a table 
and retrieved as consistent queries to a simulated random oracle. %with the mention that when a new query for $\Hsig$ is received 
%by $\mathcal{B}$ and this query contains a tuple $(m', \mathit{pk_{1}^{(t)}}, \sigma'^{(t)})$ while a different tuple containing 
%$(m', \mathit{pk_{1}^{(t)}}, \sigma^{(t)})$ with $\sigma'^{(t)} \neq \sigma^{(t)}$
%has already been received by $\mathcal{B}$ as a query for $\Hsig$, then an error message $\bot$ is output to 
%$\mathcal{A}_2$.\footnote{This handling of the simulation of $\Hsig$ is motivated by the fact that in our instantiation our signatures are deterministic.} 
%Analogously, only the first query received by $\mathcal{B}$ for $\Hpk$ with input including $(\goneBLS, \gtwoBLS, \mathit{pk_1^{(t)}}, \mathit{pk_2^{(t)}})$ is answered and 
%recorded. The reason is that for each distinct pair of keys $(\mathit{pk_1^{(t)}}, \mathit{pk_2^{(t)}})$, only one proof-of-possession is needed. 
%As an additional property, the oracle value for $\Hpk(\goneBLS, \gtwoBLS, \mathit{pk_1^{(t)}}, \mathit{pk_2^{(t)}}, A_t^{'}, B_t^{'})$ 
%is computed as $\psi_t \cdot \goneBLS +  v_1$, where $\psi_t \xleftarrow{\$} \mathbb{Z}_r$. 
\item $\Hpop(m_j)$ queries for $m_j \neq \mathit{pk}_2^*$ by choosing $\delta_j \xleftarrow{\$} \mathit{Z}_r$ 
and setting $\Hpop(m_j) = v_1 + \delta_j \cdot \goneBLS$; $\Hpop(\mathit{pk}_2^*)$ has already been defined.  
\end{itemize}

\noindent Eventually, $\mathcal{A}_2$ outputs a valid aggregate forgery 
$$(((\mathit{pk}_1^{(i)}, \mathit{pk}_2^{(i)}), \sigma^{(i)}_{\mathit{PoP}})_{i=1}^n, m, \mathit{asig})$$ where 
%$\forall i \in[n], \Veripk(\goneBLS, \gtwoBLS, \mathit{pk_1^{(i)}}, \mathit{pk_2^{(i)}}, \pi_{\mathit{pk}}^{(i)}) = 1$ and \\ 
$$\forall i \in[n], e(\Hpop(\mathit{pk_2}^{(i)}) + t_i \cdot \goneBLS, \mathit{pk_2}^{(i)}) = 
e(\sigma_{\mathit{PoP}}^{(i)} + t_i \cdot \mathit{pk_1}^{(i)}, \gtwoBLS) \ \ (10), $$ 
with $t_i \xleftarrow{\$} \mathbb{Z}_r, \forall i \in [n]$ and $$\mathit{AS.Verify}(\mathit{pp}, 
\sum_{i=1}^n \mathit{pk}_1^{(i)}, m, \mathit{asig}) = 1 \ \ (20)$$ 
and $\mathit{pk}^* = (u_1, u_2)$ is among the public keys and $m$ was not signed before.
\begin{itemize}
\item Case 1: $n =1$. In this case, the valid forgery has the form 
$$(\mathit{pk}_1^{(1)} = u_1, \mathit{pk}_2^{(1)} = u_2, \sigma^*_{\mathit{PoP}} , m, \sigma^*).$$ 
If index $\omega$ was guessed correctly by $\mathcal{B}$, then by definition of $H$ above $H(m) = v_1$; 
moreover, by the definition of valid forgery, we also have  
$$\Verisig(\goneBLS, m, \mathit{pk_1^*}= u_1 = \alpha \cdot \goneBLS, \sigma^*, \pi^*) =1. \ \ \ \ \ (30)$$ 
Due to knowledge soundness of the argument system $\PSsig$ with respect to simulated oracle $H$, 
$(30)$ implies that, with overwhelming probability,  
$\sigma^* = \alpha \cdot H(m)$. But since $H(m) = v_1$, we conclude the value $z_1 = \alpha \cdot v_1$ 
that $\mathcal{B}$ needs to output in the co-CDH game is $\sigma^*$, so $\mathcal{B}$ can indeed output $z_1$ 
with the help of $\mathcal{A}_2$. This concludes the proof in this case.

\item Case 2: $ n > 1$. To conclude the proof, $\mathcal{B}$ uses a similar technique 
as in Theorem 15.2 part b in~\cite{dabo_book} for computing the first component $\sigma^*$ of signature 
which corresponds to public key $\mathit{pk}^*$ by dividing out of $\mathit{asig_1}$ all the first components of every 
individual signature (corresponding to every public key $\mathit{pk^{(i)}} \neq \mathit{pk^*}$). Note that  
there are two types of signatures: type I corresponding to public key $\mathit{pk^{(i)}} =  \mathit{pk^*} = (u_1, u_2)$ and type II
corresponding to all other public keys $\mathit{pk^{(i)}} \neq (u_1, u_2)$. For type I signature, $\mathcal{B}$
only needs to know how many times the public key (and, consequently, the respective signature) is repeated in the output of $\mathcal{A}_2$. 
For type II signatures, let $\alpha_i \in \mathbb{Z}_r$ be such that $\mathit{pk}_2^{(i)} = \alpha_i \cdot \gtwoBLS$. Due to $(10)$, this implies 
that $\mathit{pk_1}^{(i)} = \alpha_i \cdot \goneBLS$ and 
$\sigma_{\mathit{PoP}}^{(i)} = \alpha_i \cdot \Hpop(\mathit{pk_2}^{(i)})$. $\mathcal{B}$ can compute for a public key $\mathit{pk^{(i)}} \neq (u_1, u_2)$ 
the corresponding signature's first component as $\sigma^{(i)} = \sigma_{\mathit{PoP}}^{(i)} - \delta_i \cdot \mathit{pk_1^{(i)}}$. It is clear that 
$$\sigma^{(i)} = \sigma_{\mathit{PoP}}^{(i)} - \delta_i \cdot \mathit{pk_1^{(i)}} = \alpha_i \cdot v_1 + \alpha_i \delta_i \cdot \goneBLS - 
\alpha_i \delta_i \cdot \goneBLS =  \alpha_i \cdot v_1  = \alpha_i \cdot H(m).$$
\noindent Since $\sigma^{(i)}$ is a correct signature, it verifies $$e(\sigma^{(i)}, g_2) = e(H(m), \mathit{pk_2}^{(i)}) \ \ \ (40)$$ 
Moreover, let $d$ be the number of repetitions of $\mathit{pk}^{*}$ in the output of $\mathcal{A}_2$. 
If $s$ is the sum of all the signatures that $\mathcal{B}$ can compute 
(i.e., for public keys different from $\mathit{pk}^{*}$), then from $(20)$ we obtain 
$e(s,\gtwoBLS) = e(H(m), \sum_{i=1}^n \mathit{pk_2}^{(i)} - d \cdot \mathit{pk_2}^{*})$ and, together with $(40)$, this implies 
$e(d \cdot \sigma^{*}, \gtwoBLS) = e(\mathit{asig} - s, \gtwoBLS) = e(H(m), \mathit{pk_2}^{*})^d$.
Finally, using also the definitions of $H(m)$, $\mathit{pk_2}^{\ast}$ and $z_1$ we have 
\begin{align*}
& e(\mathit{asig}, \gtwoBLS) = e(s, \gtwoBLS) \cdot e(\mathit{asig} - s, \gtwoBLS) = e(s, \gtwoBLS) \cdot e(H(m), \mathit{pk_2}^{*})^d =  \\
& = e(s, \gtwoBLS) \cdot e(z_1, \gtwoBLS)^{d} = e(s, \gtwoBLS) \cdot e(d \cdot z_1, \gtwoBLS).  
\end{align*} 
\noindent Hence, $ d \cdot \sigma^{*} = \mathit{asig} - s = d \cdot z_1$. The value $z_1$ that $\mathcal{B}$ 
needs to output is computable as $d^{-1} \cdot (\mathit{asig} - s)$. 
\end{itemize}
\end{proof}

% \section{Example Use Case} \label{sec:use_case}
% \vspace{-0.01in}
\noindent In the following, we give a short example how one can use the BLS signature scheme from instantiation~\ref{insta_bls} as an 
efficient building block for constructing the accountable light client system for PoS blockchains proposed in~\cite{ourLC}. 

\section{Benchmarks}
\label{sec:benchmarks}

The implementation of the scheme proposed in this paper is using Rust programming language and can be found as part of the Web3 Foundation BLS Library~\cite{web3.bls.2022}. 
The BLS library uses Arkworks Framework~\cite{arkworks.2022} as the backend to perform curve and arithmetic operations. The curve used for benchmarking is the BLS12-377 curve introduced in~\cite{bowe20_zexe}.

The benchmarks have been measured using a `cargo bench` running on a single thread on an `Intel(R) Xeon(R) E5-2440 0 @ 2.40GHz` CPU.

\subsection{Verification using Aggregated $\GoneBLS$ Public Keys}
\label{sec:benchmark_aggregate_verify}

Table~\ref{tab:aggregated_signature_bench} lays out the details of time needed to perform operations related to aggregate the signers' public keys and verifying the aggregated signature signed by 1000 signers or 10000 
signers, respectively. In both scenarios we assume that an aggregated signature in $\GoneBLS$ and a list of signers' public keys is handed to the verifier. \\

In the first case, in accordance with aggregated BLS signature verification, the verifier has the list of signers' public keys in $\GtwoBLS$ and aggregates them to check the validity of the given signature.
In the second case the aggregated public key of the signers in $\GtwoBLS$ is additionally handed to the verifier. The verifier has the list of signers' public keys in $\GoneBLS$ and aggregates them to check the validity of the signature using the scheme proposed in this work.

\begin{table}[h!]
\begin{tabular}{| l | l | l |}

\hline

\textbf {Scheme} & \textbf {1000 Signers} & \textbf{10000 Signers} 	 \\
\hline
Aggregate public keys in $\GtwoBLS$ and verify (standard BLS) & 10.815 ms & 69.062 ms \\
\hline
Aggregate public keys in $\GoneBLS$ and verify (our scheme)  & 6.815 ms & 22.159 ms \\
\hline
\end{tabular}
\caption{Verification time for 1000/10000 aggregated signatures using public keys in $\GoneBLS$ and $\GtwoBLS$, respectively, on BLS12-377 curve.}
\label{tab:aggregated_signature_bench}
\end{table}

The benchmarks in Table~\ref{tab:aggregated_signature_bench} demonstrate that when the number of signers increases, the advantage of performing the aggregation of public keys in $\GoneBLS$ becomes more prominent. 
This result is expected because the scheme presented in this work performs only two extra scalar multiplications in $\GoneBLS$ compared to standard BLS verification with aggregate public key in $\GtwoBLS$. 
%However, when the number of signers increases, so does the number of elliptic curve additions performed in the faster $\GoneBLS$ group compared to operations in the slower $\GtwoBLS$ 
%(performed by standard BLS verification). \
Thus, when the number of signers increases, the benefit from having to perform a large number of elliptic curve additions in the faster $\GoneBLS$ group compared to the same number of 
additions in the slower $\GtwoBLS$ group overcompensates for the overhead cost of two extra scalar multiplications in the case of our scheme. 
%Eventually, the efficiency improvement w.r.t verification for our new scheme outperforms the overhead cost of the two extra scalar multiplications.  

\subsection{Verification of Individual  BLS Signatures via Verification of Chaum-Pedersen Proofs}
\label{sec:benchmark_chaum_pederson_verify}

Table~\ref{tab:individual_signature_bench} illustrates the efficiency of verification when using additional Chaum-Pedersen proofs that accompany the individual BLS signatures and the BLS public keys in the first source group $\GoneBLS$.

\begin{table}[h!]
\begin{tabular}{| l | l |}

\hline

\textbf{Scheme} & \textbf{1000 Signatures} 	 \\
\hline
BLS verification (applying pairings with each individual signature)   & 4534 ms \\
\hline
Chaum-Pedersen verification for BLS signature and key (our scheme)   & 2480 ms \\
\hline
\end{tabular}
\caption{Verification time for 1000 signatures verified individually (non-aggregated) using  BLS verification vs using the Chaum-Pedersen proofs.}
\label{tab:individual_signature_bench}
\end{table}

Note that for the BLS verification, we have applied pairings to each of the 1000 signatures\footnote{In our scheme, the first component of the individual signatures is in $\GoneBLS$ and the signature is over the corresponding signer's $\GtwoBLS$ public key.} rather than aggregating those signatures and applying pairings to the aggregated signature. There are real-world practical applications which require individual verification of BLS signatures, such as when individual signatures are gossiped to an aggregator node and the gossiper is supposed not to gossip invalid signatures. Similarly is the case of identifying the signer(s) accountable for the failure of an aggregated signature.\\

Considering that the pairing costs as much as eight naive $\GoneBLS$ scalar multiplications in the Artworks library, we have benefited from about a 45\% reduction of the cost of the verification of time of individual signatures. We note that there are various avenues to improve the efficiency of the scalar multiplications (such as employing efficient multi-scalar multiplication methods) which have not been accounted for and can make our scheme even more efficient.

\section{Conclusion} \label{sec:conclusion}
\vspace{-0.01in}
\noindent In this paper we define, instantiate and prove the security properties for an aggregatable BLS signature 
scheme such that when used appropriately, it allows the verifier to essentially store or work only with short public keys 
and signatures because they are in the first target group of the associated BLS pairing. We have implemented
our signature scheme and we have provided concrete benchmarks and an evaluation of its efficiency benefits. We are planning to improve the efficiency of the implementation by applying other 
optimisations related to multi-scalar multiplications.



\bibliography{bibliography} 
\bibliographystyle{ieeetr}

\appendix
\section{Appendix} \label{sec:appendix}
\noindent Below we remind the reader the co-CDH assumption, which is a variation of the standard computational Diffie-Hellman assumption (CDH)
for the case when two groups are used. Let $E$ be a pairing friendly elliptic curve and 
let $\GoneBLS$, $\GoneBLS$ and $\GTBLS$ be appropriately chosen subgroups of order $r$ with  $\goneBLS$, $\gtwoBLS$ 
generators for the first two subgroups, respectively. Let $e: \GoneBLS \times \GtwoBLS \rightarrow \GTBLS$ be a secure 
pairing~\cite{secure_pairings,pairings_for_cryptographers}. 
\paragraph{Attack Game co-CDH} For a given adversary $\mathcal{A}$ the attack runs as follows:
\begin{itemize}
\item The challenger computes $\alpha, \beta \xleftarrow{\$} \mathbb{Z}_r$, $u_1 \leftarrow \goneBLS^{\alpha}$, $u_2 \leftarrow \gtwoBLS^{\alpha}$,   
$v_1 \leftarrow \goneBLS^{\beta}$, $z_1 \leftarrow \goneBLS^{\alpha\beta}$ and gives the tuple $(u_1, u_2, v_1)$ to $\mathcal{A}$; $\alpha$ is used twice, 
once in $\GoneBLS$ and in $\GtwoBLS$. 
\item The adversary $\mathcal{A}$ outputs some $\hat{z_1} \in \GoneBLS$.
\end{itemize}
\noindent $\mathcal{A}$'s advantage in solving the co-CDH problem for $e$, denoted by $$\mathit{Adv}^{\mathit{coCDH}}[\mathcal{A},e]$$ is the probability 
that $\hat{z_1} = z_1$.
\paragraph{co-CDH Assumption} We say that the co-CDH assumption holds for the pairing $e$ if for all efficient adversaries $\mathcal{A}$, 
the quantity $\mathit{Adv}^{\mathit{coCDH}}[\mathcal{A},e]$ is negligible.
If $e$ is a symmetric pairing, then $\GoneBLS = \GtwoBLS$ and $\goneBLS = \gtwoBLS$ in which case the co-CDH assumption is identical to the standard CDH assumption. 

\end{document}        
