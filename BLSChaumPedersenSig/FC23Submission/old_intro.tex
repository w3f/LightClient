\begin{comment}
Aggregation of BLS signatures simplifies some distributed systems,
usually by being an exportable proofs of byzantine agreement.
After agreement and aggregation occurs, the aggregate signature saves
foreign aggregate verifiers both compute and bandwidth over checking
numerous faster signature, especially as scales like Ethereum's hundreds of thousands of signers.

We do however handle individual BLS signatures before aggregation occurs.
Individual BLS signatures have extremely slow verification compared to alternatives
like Schnorr or ECDSA signatures dut to using pairings.
So either most nodes incur this high verification costs for every other node's
signatures, or else system designers choose more centralized gossip
flavors, which then harms liveness.
\end{comment}

\paragraph{Problem Statement} Aggregation of BLS signatures simplifies some distributed systems,
usually by being exportable proofs of byzantine agreement. After agreement and aggregation occurs, 
the aggregate signature saves foreign aggregate verifiers both compute and bandwidth over checking
numerous slow signatures, especially at the scale of Ethereum's hundreds of thousands of signers.
Indeed, individual BLS signatures have very slow verification, so either most nodes incur this high verification 
costs for every other node's signatures, or, else, system designers choose more centralised gossip
flavours which may harm liveness.

\begin{comment}
We suggest that gossip messages containing individual BLS signatures carry
a Chaum-Pedersen DLEQ proofs of their correctness, so that verifying
individual signatures no longer requires pairings.
%
We avoid complex bisection logic in aggregation this way too, but increase
signature message size by 32 bytes.
% meh..  We need a second group $G_f$ in which scalar multiplication 
\end{comment}

\paragraph{Our Solution} We suggest that gossip messages containing individual 
BLS signatures carry Chaum-Pedersen~\cite{ChaumPedersen} DLEQ proofs of their correctness, so that 
verifying individual signatures no longer require pairings. Moreover, a classical BLS signature 
places the public key and signature on opposite source groups of the pairing (that is for the type III pairings that 
are preferred for modern cryptographic systems due to efficiency considerations), 
but the choice of which of the groups is used for the keys or signatures creates trade offs. 
We mitigate these trade offs by proposing the following:


\paragraph{Protocol:}

We adopt standard notation for pairing friendly curves throughout.

% The original BLS protocol was written for Type 1 pairings, where there is only one elliptic curve and prime order subgroup.
As a rule, one always does BLS signatures on curves with type III pairings
today, such as BLS12-381, so there are two groups $\GoneBLS$ and $\GtwoBLS$,
with $\GtwoBLS$ having much slower arithmetic and hash-to-curve, due to
being defined over an extension field.  In fact, recent progress
\cite{TNFS} against the discrete log problem in extension fields
could precipitate adopting pairings with an higher embedding degree,
which would slow $\GtwoBLS$ further.

A classical BLS signature places the public key and signature on opposite
curves, but the choice of which creates trade offs.  We mitigate these
trade offs:

First, we provide the public key on both curve,
 like $\pk = (\sk \goneBLS, \sk \gtwoBLS)$.
We enforce this public key structure during the public key validation
phase present in many aggregate BLS protocols e.g. from 
imposing proof-of-possession checks~\cite{proofs_of_posession}.  % or $e(-\pk_1, \pk_2) = \gTBLS$

Second, we create BLS signatures $\sigma = \sk H(m)$ using the much
faster $\GoneBLS$ hash-to-curve $H(m)$, but also provide
 $\pi = \mathit{DLEQ}( \sigma / H(m) = \pk_1 / \goneBLS)$.
Now $\pi$ proves correctness of $\sigma$, by
 assuming public key validation.
% Interestingly, $\pi$ permits signing an extra message besides $m$.

\def\apk{\mathit{apk}}

Third, an aggregator node provides both the aggregate BLS signature,
as well as aggregate signer keys $\apk_1 \in \GoneBLS$ and
$\apk_2 \in \GtwoBLS$, by summing the $\sigma$s, $\pk_1$s, and $\pk_2$s.

At this point, aggregate verifiers check that $\apk_1$ and $\apk_2$ were
identically computed with only two scalar multiplications in $\GoneBLS$.
In the end, aggregate verifiers save time despite these multiplications
because their hash-to-curve runs on $\GoneBLS$, far faster than on $\GtwoBLS$.
Also, verifiers have fast additions when checking if $\apk_1$ represents
a suitable signer set, which speeds up zk proofs of $\apk_1$ like \cite{ourLC} too.



We formalize signature aggregation so that we can prove security for above
construction, including our public key structure, and aggregator nodes
supplying aggregate public key. 

We remark that a natural variation on our protocol exists in which
one finds an even faster non-pairing curve $\mathcal{S}$ having the same
group order as $\GoneBLS$,  publishes triplet public keys like
$\pk = (\sk \goneBLS, \sk \gtwoBLS, \sk S)$, in which $S$ generates
$\mathcal{S}$, and then DLEQ proofs use $\pk_3$ for performance.
We do not discuss this variant because it falls under our security
arguments without additional mathematical complexities,
 aside from choosing curve parameters for $\mathcal{S}$.
 
 
\paragraph{Other Related Work:} Alternative constructions for BLS signatures as well as other defence mechanisms against 
rogue key attacks~\cite{proofs_of_posession}, exist and we briefly review both below. First, aggregation of BLS signatures for different messages have been studied before
(e.g.,~\cite{aggregate_BLS_signatures}). In this case, rouge-key attacks are not a threat anymore (and, hence, PoPs are not a necessity anymore), 
but signature verification is computationally more expensive than in our case, requiring $O(n)$ parings for $n$ different messages. Second, alternative aggregatable 
BLS signatures exist (e.g.,~\cite{boneh_compact_multisig}) where both the aggregated public key have their size independent 
of the number of signers and the signature verification is as fast as in our variant. However, for the blockchain use case we envision 
(see section~\ref{sec:use_case} and~\cite{ourLC}) we prefer our scheme detailed in section~\ref{sec:bls}: its corresponding key aggregation is a simple sum 
of the individual signers' public keys, while, in~\cite{boneh_compact_multisig} the key aggregation operation involves more expensive scalar 
multiplications every time the key aggregation is performed.
