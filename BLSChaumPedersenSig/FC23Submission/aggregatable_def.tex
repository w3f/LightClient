\subsection{Secure Signature Aggregation}
\label{sec:def_aggregate}
An aggregatable signature scheme compresses signatures issued using
possibly different signing keys into one signature. In this work we use
an aggregatable signature scheme making explicit use of proofs-of-possession 
(PoPs)~\cite{proofs_of_posession} in order to protect against rogue-key attacks~\cite{proofs_of_posession}. 
%For our concrete instantiation, we use aggregatable
%BLS signatures with an efficient aggregation procedure, i.e., by adding
%together keys and by multiplying together signatures, and protect against
%rogue key attacks~\cite{proofs_of_posession} using PoPs. This is in contrast to other aggregation
%procedures that do not require PoPs for security but incur a higher computational 
%cost (e.g., due to the use of multi-scalar multiplication~\cite{boneh_compact_multisig}).
%For our concrete use case of accountable light clients systems, our 
%efficient signature aggregation method results in a simple and more efficient
%custom argument scheme (i.e., SNARK, see section~\ref{sec:use_case} for more details), 
%which, in turn, compensates for the cost of having to work with PoPs.

\begin{definition}
\label{def:aggregate}
(Aggregatable Signature Scheme) An aggregatable signature scheme consists of
the following tuple of algorithms ($\mathit{AS.Setup}$, $\mathit{AS.GenerateKeypair}$, $\mathit{AS.VerifyPoP}$, 
$\mathit{AS.Sign}$, $\mathit{AS.AggregateKeys}$, \\ $\mathit{AS.AggregateSignatures}$, $\mathit{AS.Verify}$):
\begin{itemize}

\item $\mathit{pp} \leftarrow  \mathit{AS.Setup}(\mathit{\lambda})$: a setup algorithm that, given a
security parameter $\lambda$, outputs public protocol parameters $\mathit{pp}$.

\item $((\mathit{pk},\mathit{\pi_{PoP}}),\mathit{sk}) \leftarrow \mathit{AS.GenerateKeypair}(\mathit{pp})$:
a key pair generation algorithm that
outputs
a secret key $\mathit{sk}$,
and the corresponding public key $\mathit{pk}$
together with a proof-of-possession $\mathit{\pi_{PoP}}$ of the secret key.

\item $0/1 \leftarrow \mathit{AS.VerifyPoP}(\mathit{pp}, \mathit{pk},\mathit{\pi_{PoP}})$:
a public key verification algorithm that,
given a public key $\mathit{pk}$
and a proof-of-possession $\mathit{\pi_{PoP}}$,
outputs
$1$ if $\mathit{\pi_{PoP}}$ is valid for $\mathit{pk}$ and $0$ otherwise.

\item $\sigma \leftarrow \mathit{AS.Sign}(\mathit{pp}, \mathit{sk}, m)$:
a signing algorithm that,
given a secret key $\mathit{sk}$ and a message $m \in \{0, 1\}^*$, returns a signature $\sigma$.

\item $\mathit{apk} \leftarrow \mathit{AS.AggregateKeys}(\mathit{pp}, (\mathit{pk_i})_{i=1}^{n})$:
a public key aggregation algorithm that,
given a vector of public keys $(\mathit{pk_i})_{i=1}^n$,
returns
an aggregate public key $\mathit{apk}$.

\item $\mathit{asig} \leftarrow \mathit{AS.AggregateSignatures}(\mathit{pp}, (\sigma_i)_{i=1}^n)$:
a signature aggregation algorithm that,
given a vector of signatures $(\sigma_i)_{i=1}^n$,
returns
an aggregate signature $\mathit{asig}$.

\item $0/1 \leftarrow \mathit{AS.Verify}(\mathit{pp}, \mathit{apk}, m, \mathit{asig})$:
a signature verification algorithm that,
given an aggregate public key $\mathit{apk}$, a message $m \in \{0, 1\}^*$, and an aggregate signature $\sigma$,
returns
1 or 0 to indicate if the signature is valid.
\end{itemize}
\noindent We say ($\mathit{AS.Setup}$, $\mathit{AS.GenerateKeypair}$, $\mathit{AS.VerifyPoP}$, 
$\mathit{AS.Sign}$, \\$\mathit{AS.AggregateKeys}$, $\mathit{AS.AggregateSignatures}$, 
$\mathit{AS.Verify}$) is an aggregatable signature scheme if it satisfies \emph{perfect completeness}, \emph{completeness for aggregation}
and \emph{unforgeability} as defined below.
\end{definition}
\noindent \textbf{Perfect Completeness} An aggregatable signature scheme
($\mathit{AS.Setup}$, \\ $\mathit{AS.GenerateKeypair}$, $\mathit{AS.VerifyPoP}$, $\mathit{AS.Sign}$, $\mathit{AS.AggregateKeys}$,\\ 
$\mathit{AS.AggregateSignatures}$, $\mathit{AS.Verify}$) has perfect completeness if for any message $m \in \{0,1\}^*$ and any 
$n\in\mathbb{N}$ it holds that:
\begin{align*}
& \mathit{Pr} [\mathit{AS.Verify}(\mathit{pp}, \mathit{apk}, m, \mathit{asig})=1 \ \wedge \ \forall  i \in [n]\ \mathit{AS.VerifyPoP}(\mathit{pp}, \mathit{pk_i},\mathit{\pi_{\mathit{PoP},i}})=1\ |\\
& \mathit{pp} \leftarrow \mathit{AS.Setup(\lambda)}, \\
& ((pk_{i},\pi_{\mathit{PoP}, i}), sk_{i} ) \leftarrow \mathit{AS.GenerateKeypair}(\mathit{pp}),\ i=1,\ldots,n\\
&\mathit{apk} \leftarrow \mathit{AggregateKeys}(\mathit{pp}, (\mathit{pk}_{i})_{i=1}^{n}), \\
& \sigma_i \leftarrow \mathit{AS.Sign}(\mathit{pp}, \mathit{sk_i}, m),\ i=1,\ldots,n, \\
& \mathit{asig} \leftarrow \mathit{AS.AggregateSignatures(\mathit{pp}, (\sigma_{i})_{i=1}^{n})}] = 1.
\end{align*}
\noindent We note that an aggregatable signature scheme with perfect completeness implies the underlying signature scheme
has perfect completeness. \\

\noindent \textbf{Completeness for Aggregation} An aggregatable signature scheme 
($\mathit{AS.Setup}$, $\mathit{AS.GenerateKeypair}$, $\mathit{AS.Verify}$, $\mathit{AS.Sign}$,
$\mathit{AS.AggregateKeys}$, $\mathit{AS.AggregateSignatures}$, $\mathit{AS.Verify}$)
has completeness for aggregation if, for every adversary $\mathcal{A}$
\begin{align*}
&\mathit{Pr}[\mathit{AS.Verify}(\mathit{pp}, \mathit{apk}, m, \mathit{asig}) = 1 \ (\ast\ast\ast) \ | \ \mathit{pp} \leftarrow \mathit{AS.Setup}(\lambda), \\
& ((\mathit{pk_i}, \pi_{\mathit{PoP},i})_{i=1}^n, m, (\sigma_i)_{i=1}^{n}) \leftarrow \mathcal{A}(\mathit{\mathit{pp})}, \\
& \forall i \in [n], \mathit{AS.VerifyPoP}(\mathit{pp}, \mathit{pk_i}, \pi_{\mathit{PoP},i}) = 1 \ \ \ \ (\ast), \\
& \forall i \in [n], \mathit{AS.Verify}(\mathit{pp}, \mathit{pk_i}, m, \sigma_i) = 1 \ \ \ \ (\ast\ast),\\
& \mathit{apk} \leftarrow \mathit{AS.AggregateKeys}(\mathit{pp},  (\mathit{pk}_{i})_{i=1}^{n}), \\
&  \mathit{asig} \leftarrow \mathit{AS.AggregateSignatures}(\mathit{pp}, (\sigma_i)_{i=1}^n)] = 1- \negl(\lambda).
\end{align*}


\noindent \textbf{Unforgeable Aggregatable Signature}
For an aggregatable signature scheme ($\mathit{AS.Setup}$, $\mathit{AS.GenerateKeypair}$, $\mathit{AS.VerifyPoP}$, $\mathit{AS.Sign}$,
$\mathit{AS.AggregateKeys}$, $\mathit{AS.AggregateSignatures}$, $\mathit{AS.Verify}$)
the advantage of an adversary against unforgeability is defined by
$$\mathit{Adv}^{\mathit{forge}}_{\mathcal{A}}({\lambda}) = \mathit{Pr}[\mathit{Game}^{\mathit{forge}}_{\mathcal{A}}({\lambda}) =1], \textit{\ where}$$
\vspace{-0.25in}
\begin{align*}
&\mathit{Game}^{\mathit{forge}}_{\mathcal{A}}({\lambda}): \\
& \mathit{pp} \leftarrow \mathit{AS.Setup(\lambda)} \\
& ((\mathit{pk}^*,\pi^*_{\mathit{PoP}}), \mathit{sk}^*) \leftarrow \mathit{AS.GenerateKeypair}(\mathit{pp})\\
& Q \leftarrow \emptyset \\
& ((\mathit{pk_i}, \pi_{\mathit{PoP},i})_{i=1}^{n}, m, \mathit{asig}) \leftarrow \mathcal{A}^{\mathit{OSign}}(\mathit{pp}, (\mathit{pk^*},\pi^*_{\mathit{PoP}})) \\
& \textit{If } \mathit{pk}^* \notin \{\mathit{pk_i}\}_{i=1}^{n} \vee m \in Q, \textit{ then return } 0 \\
& \textit{For } i \in [n] \\
& \ \ \ \ \ \textit{ If } \mathit{AS.VerifyPoP}(\mathit{pp}, \mathit{pk_i}, \pi_{\mathit{PoP},i})=0  \textit{ return } 0 \\
& \mathit{apk} \leftarrow \mathit{AS.AggregateKeys}(\mathit{pp}, (\mathit{pk_i})_{i=1}^{n}) \\
& \textit{Return } \mathit{AS.Verify}(\mathit{pp}, \mathit{apk}, m, \mathit{asig})
\end{align*}
\noindent and
\begin{align*}
& \mathit{OSign}(m_j): \\
& \sigma_j \leftarrow \mathit{AS.Sign}(\mathit{pp}, \mathit{sk}^*, m_j) \\
&  Q \leftarrow Q \cup \{m_j\} \\
& \textit{Return} \ \sigma_j
\end{align*}
\noindent and $\mathcal{A}^{\mathit{OSign}}$ denotes the adversary $\mathcal{A}$ with access to oracle $\mathit{OSign}$. \\
\noindent We say an aggregatable signature scheme is unforgeable if for all efficient adversaries
$\mathcal{A}$ it holds that $\mathit{Adv}^{\mathit{forge}}_{\mathcal{A}}({\lambda}) \leq \mathit{negl}(\lambda)$. 

\subsection{Aggregatable BLS Signatures}
\label{sec:bls}
\noindent In the following, we instantiate the aggregatable signature definition given above with a scheme inspired by the BLS signature
scheme~\cite{BLS_signatures} and its follow-up variants~\cite{proofs_of_posession,boneh_compact_multisig}.
Because, in general, multisignatures are susceptible to so-called ``rogue-key attacks'' which can be
mounted whenever the adversary is allowed to choose his public keys arbitrarily, in order to protect against such
rogue-key attacks, we enhance our multisignature instantiation with proofs-of-possessio as defined in~\cite{proofs_of_posession}.
In turn, we instantiate our proofs-of-possession with the non-interactive version of Chaum-Pedersen proofs for showing 
the equality of discrete logarithm (i.e., the secret key or DLEQ proof) corresponding to a pair of distinct public keys.

%\noindent In the following, we instantiate the aggregatable signature definition given above with a scheme inspired by the BLS signature
%scheme~\cite{BLS_signatures} and its follow-up variants~\cite{proofs_of_posession,boneh_compact_multisig}. However, the proofs-of-possession 
%will be instantiated with Schnorr signatures.

\begin{construction}(Aggregatable BLS Signatures) We call aggregatable BLS signatures the following instantiation of aggregatable signatures:
\label{insta_bls}
\begin{itemize}
\item $(\GoneBLS, \GtwoBLS,  \goneBLS, \gtwoBLS, \GTBLS, \eBLS, \HBLS, \Hpk, \Hsig)$ from $\mathit{pp}$ where 
$\mathit{pp} \leftarrow  \mathit{AS.Setup}(\lambda)$, 
where $\GoneBLS$, $\GtwoBLS$,  $\goneBLS$, $\gtwoBLS$, $\GTBLS$, $\eBLS$ are the first and second source groups, their generators and the 
associated pairing for some BLS elliptic curve $E$, respectively,  %were defined in section~\ref{sec:pairings} and 
and $\HBLS: \{0,1\}^* \rightarrow \GtwoBLS$ and $\Hpk: \{0,1\}^* \rightarrow \mathbb{Z}_r^*$ and $\Hsig: \{0,1\}^* \rightarrow \mathbb{Z}_r^*$
are two hash functions. 

\item $(\mathit{pk_1}, \mathit{pk_2}, \mathit{sk}, \pipk) \leftarrow \mathit{AS.GenerateKeypair}(\mathit{pp})$, 
where $\mathit{sk} \xleftarrow{\$} \mathbb{Z}_{r}^{*}$ and $\mathit{pk_1} = \mathit{sk} \cdot \goneBLS \in \GoneBLS$ 
and $\mathit{pk_2} = \mathit{sk} \cdot \gtwoBLS \in \GtwoBLS$ 
and \\ $\pipk \leftarrow \Provpk(\goneBLS, \gtwoBLS, \mathit{pk_1}, \mathit{pk_2}, \mathit{sk})$ 
and $r$ is the size of the scalar field of elliptic curve $E$. 
%was defined in section~\ref{sec:pairings}.

\item $0/1 \leftarrow \mathit{AS.VerifyPoP}(\mathit{pp}, \mathit{pk_1}, \mathit{pk_2}, \pipk)$, where $\mathit{AS.VerifyPoP}$ outputs $1$ if 
$\Veripk(\goneBLS, \gtwoBLS, \mathit{pk_1}, \mathit{pk_2}, \pipk)$ outputs $1$ and $0$ otherwise.

\item $(\sigma, \pisig) \leftarrow \mathit{AS.Sign}(\mathit{pp}, \mathit{sk}, m)$: 
where $\sigma = \mathit{sk} \cdot \HBLS(m) \in \GtwoBLS$ and $ \pisig \leftarrow \Provsig(\goneBLS, m, \mathit{pk_1}, \sigma, \mathit{sk})$

\item $\mathit{apk} = (\mathit{apk_1}, \mathit{apk_2}) \leftarrow \mathit{AS.AggregateKeys}(\mathit{pp}, (\mathit{pk_1^i}, \mathit{pk_2^i})_{i=1}^{n})$, 
where  $\mathit{apk_1} = \sum_{i=1}^{n} \mathit{pk_1^i}$ and $\mathit{apk_2} = \sum_{i=1}^{n} \mathit{pk_2^i}$. 
\item $\mathit{asig} \leftarrow \mathit{AS.AggregateSignatures}(\mathit{pp}, (\sigma_i, \pi_i)_{i=1}^n)$, 
where \\ $\mathit{asig} = (\sigma_1, \pi_1)$  if $n=1$ and $\mathit{asig} = (\sum_{i=1}^{n} \sigma_i, \bot)$ if $n>1$. 
\item $0/1 \leftarrow  \mathit{AS.Verify}(\mathit{pp}, (\mathit{apk_1}, \mathit{apk_2}), m, \mathit{asig})$, where $\mathit{AS.Verify}$ outputs $1$ if either 
$\mathit{asig}_2 = \bot$ and $e(\mathit{asig}_1 + t \cdot \mathit{apk_1}, \gtwoBLS) = e(H(m) + t \cdot \goneBLS, \mathit{apk_2})$, with $t \xleftarrow{\$} \mathbb{Z}_{r}^{*}$ 
or $\mathit{asig}_2 \neq \bot$ and $\Verisig(\goneBLS, m, \mathit{apk_1}, \mathit{asig}_1, \mathit{asig}_2)$; in all other cases, it outputs $0$. 
\end{itemize}
Above we have used the following proof systems 
$$ \PSpk = (\Genpk, \Provpk, \Veripk) \ \ \ \ (1)$$ 
where 
\begin{itemize}
\item $(\GoneBLS, \goneBLS, \GtwoBLS, \gtwoBLS, \Hpk) \leftarrow \Genpk(\lambda)$ as a subprotocol of $\mathit{AS.Setup}(\lambda)$. 
%We can also do proofs of possession by Chaum-Pedersen DLEQ proofs. Now we are proving the discrete log equality of 
%a pair of $G_1$ elements $g_1, pk_1$ and a pair of $G_2$ elements $g_2,pk_2$ (and also that the  dicrete log $sk$ is known).

\item $\pipk = (c,s) \leftarrow \Provpk(\goneBLS, \gtwoBLS, \mathit{pk_1}, \mathit{pk_2}, \mathit{sk})$ where \\ $k \xleftarrow{\$} \mathbb{Z}_{r}^{*}$, $A =k \cdot \goneBLS$, 
$B=k \cdot  \gtwoBLS$, $c = \Hpk (\goneBLS, \gtwoBLS, \mathit{pk_1}, \mathit{pk_2}, A, B)$, \\ $s = k - c \cdot \mathit{sk} \mod r$. 

\item $0/1 \leftarrow \Veripk(\goneBLS, \gtwoBLS, \mathit{pk_1}, \mathit{pk_2}, (c,s))$, 
where $\Veripk$ outputs 1 if  $c= \Hpk (\goneBLS, \gtwoBLS, \mathit{pk_1}, \mathit{pk_2}, A', B')$
where $A' = s \cdot \goneBLS + c \cdot \mathit{pk_1}$ and $B' = s \cdot \gtwoBLS  + c \cdot \mathit{pk_2}$ and it outputs $0$ otherwise.  
\end{itemize}

and 
$$ \PSsig = (\Gensig, \Provsig, \Verisig) \ \ \ \ (2)$$ 
\begin{itemize}
\item $(\GoneBLS, \goneBLS, \Hsig) \leftarrow \Gensig(\lambda)$ as a subprotocol of $\mathit{AS.Setup}(\lambda)$. 
\item $\pisig = (c,s) \leftarrow \Provsig(\goneBLS, m, \mathit{pk_1}, \sigma, \mathit{sk})$ where \\ $k \xleftarrow{\$} \mathbb{Z}_{r}^{*}$, $A =k \cdot \goneBLS$, 
$B=k \cdot H(m)$, $c = \Hpk (\goneBLS, m, \mathit{pk_1}, \sigma, A, B)$, \\ $s = k - c \cdot \mathit{sk} \mod r$. 
\item $0/1 \leftarrow \Verisig(\goneBLS, m, \mathit{pk_1}, \sigma, (c,s))$, 
where $\Verisig$ outputs 1 if  $c= \Hsig (\goneBLS, m, \mathit{pk_1}, \sigma, A', B')$
where $A' = s \cdot \goneBLS + c \cdot \mathit{pk_1}$ and $B' = s \cdot H(m)  + c \cdot \sigma$ and it outputs $0$ otherwise.  
\end{itemize}
\end{construction}
\noindent Note: It is easy to show that $\PSpk$ and $\PSsig$ are zero-knowledge non-interactive arguments of knowledge for relations $\Rpk$ and $\Rsig$, respectively, where 
$$\Rpk = \{( \GoneBLS, \GtwoBLS, \goneBLS, \gtwoBLS, \mathit{pk_1}, \mathit{pk_2}); \mathit{sk}) : \mathit{pk_1} = \mathit{sk} \cdot \goneBLS, \mathit{pk_2} = \mathit{sk} \cdot \gtwoBLS \},$$
$$\Rsig = \{( \GoneBLS, \goneBLS, m, \mathit{pk_1}, \sigma); \mathit{sk}) : \mathit{pk_1} = \mathit{sk} \cdot \goneBLS, \sigma = \mathit{sk} \cdot H(m)\},$$
and $\GoneBLS$, $\GoneBLS$ are generated by $\goneBLS, \gtwoBLS$, respectively.

%Note that this takes the verifier 2 $G_1$ exponentiations and 2 $G_2$ exponentiations. This may not be a lot faster than the two 
%pairings required to verify a BLS signature with a different hash and using the same as the "Fast verification of aggregate public keys with proof" section.

\begin{theorem} Assuming that co-CDH holds for $e$ and $H$, $\Hpk$ and $\Hsig$ are modelled as random oracles, 
then instantiation~\ref{insta_bls} is an aggregatable signature scheme as per definition~\ref{def:aggregate}. 
\end{theorem}
\begin{proof} The \emph{perfect completeness} property is very easy to prove. \\ 

\noindent Regarding \emph{completeness for aggregation}, the non-trivial case is when $n>1$. 
Let $\mathcal{A}_1$ be an efficient adversary trying to break this property. Since $(\ast)$ and $(\ast\ast)$ hold, then by the 
knowledge soundness of $\PSpk$ and $\PSsig$ there exist efficient extractors $\Epk$ and $\Esig$ such that, except with negligible probability can 
extract $(\mathit{sk}_i)_{i=1}^n$ such that $\mathit{pk}_1^i = \mathit{sk}_i \cdot \goneBLS$, $\mathit{pk}_2^i = \mathit{sk}_i \cdot\gtwoBLS$ 
and $\sigma_{i,1} = \mathit{sk}_i \cdot \HBLS(m)$. This, in turn, implies that except with negligible probability, 
$\mathit{apk}_1 = (\sum_{i=1}^n \mathit{sk}_i) \cdot \goneBLS$, $\mathit{apk}_2 = (\sum_{i=1}^n \mathit{sk}_i) \cdot \gtwoBLS$ and 
$\mathit{asig}_1 = (\sum_{i=1}^n \mathit{sk}_i) \cdot \HBLS(m)$. This, in turn, implies that the aggregated signature verification $(\ast\ast\ast)$ holds, except 
with negligible probability. \\ 

\noindent Regarding \emph{unforgeability}, let $\mathcal{A}_2$ be an efficient adversary trying to break this property (see definition~\ref{def:aggregate}). 
Without loss of generality, we assume that whenever $\mathcal{A}_2$ outputs a potential forgery, and, in particular a set of public keys and their PoPs, 
$\mathcal{A}_2$ has already queried the random oracle $\Hsig$ for all the hash values involved in the proofs and individual signatures allegedly corresponding to the
output public keys and message. Indeed, adding this constraint on the forger, does not decrease its probability of success. Next, we follow a similar 
proof technique as detailed for theorem 15.2, part b in~\cite{dabo_book} where given a successful adversary $\mathcal{A}_2$ against unforgeability one 
can construct a successful adversary $\mathcal{B}$ against the co-CDH assumption. For brevity, below we only state the differences and modifications in our proof: 
first, their $g_0$ is replaced in our case by $\goneBLS$ and their $g_1$ by our $\gtwoBLS$. The adversary $\mathcal{B}$ begins by sending the public 
key $\mathit{pk^*}= (u_0, u_1, \pi^*)$ to the forger $\mathcal{A}_2$, where $\pi^* = (c^*, s^*)$ with $(c^*, s^*)$ computed by choosing $c^*, s^*$ uniformly at random in 
$\mathbb{Z}_r^*$, setting $A^* = s^* \cdot \goneBLS + c^* \cdot \mathit{u_0}$ and $B^* = s^* \cdot \gtwoBLS  + c^* \cdot \mathit{u_1}$ and 
finally $\mathcal{B}$ records $\Hpk (\goneBLS, \gtwoBLS, \mathit{u_0}, \mathit{u_1}, A^*, B^*)$ as equal to $c^*$ in a corresponding table.\footnote{These are also the steps taken by a zero-knowledge simulator for relation $\Rpk$.} 

\noindent The forger $\mathcal{A}_2$ makes makes a sequence of signatures and hash queries to simulated oracles $H$, $\Hpk$ and $\Hsig$ and 
$\mathit{OSign}$ maintained by $\mathcal{B}$ as follows: 
\begin{itemize}
\item $H(m')$ queries are handled as in $\mathcal{B}$'s simulation, theorem 15.2. part b~\cite{dabo_book}.   
\item $\Hpk(\goneBLS, \gtwoBLS, \mathit{pk_1^{t}}, \mathit{pk_2^{t}}, A_t^{'}, B_t^{'})$ 
and $\Hsig(\goneBLS, m', \mathit{pk_{t}}, \sigma_{t}, A_t^{''}, B_t^{''})$ queries are received, stored in two tables and retrieved as 
consistent queries to two distinct and simulated random oracles with the mention that when a new query for $\Hsig$ is received 
by $\mathcal{B}$ and this query contains a tuple $(m', \mathit{pk_{t}}, \sigma_{t}^{'})$ while a different tuple $(m', \mathit{pk_{t}}, \sigma_{t})$ 
has already been received by $\mathcal{B}$ as for $\Hsig$, then an error message $\bot$ is output to 
$\mathcal{A}_2$.\footnote{This handling of the simulation of $\Hsig$ is motivated by the fact that in our instantiation our signatures are deterministic.} 
\item Signing queries $m'$ are handled via a correct simulation of $\mathit{OSign}$ oracle performed by $\mathcal{B}$. 
\end{itemize}
\noindent Eventually, $\mathcal{A}_2$ outputs a valid aggregate forgery $$((\mathit{pk}_1^i, \mathit{pk}_2^i, \pi_{\mathit{pk}, i})_{i=1}^n, m, \mathit{asig})$$ where 
$\forall i \in[n], \Verisig(\goneBLS, \gtwoBLS, \mathit{pk_1^i}, \mathit{pk_2^i}, \pi_{\mathit{pk}, i}) = 1$ and \\ 
$\mathit{AS.Verify}(\mathit{pp}, (\sum_{i=1}^n \mathit{pk}_1^i, \sum_{i=1}^n \mathit{pk}_2^i), m, \mathit{asig}) = 1$ and $\mathit{pk^*} = (u_0, u_1)$ 
is among the public keys and $m$ was not signed before. To conclude the proof, $\mathcal{B}$ uses a similar technique as in theorem 15.2 part b~\cite{dabo_book}  
for computing the first component of signature $\sigma^*$ (which corresponds to public key $\mathit{pk}^*$) by dividing out of $\mathit{asig_1}$ 
all the first components of every individual signature (corresponding to every public key $\mathit{pk_t} \neq \mathit{pk^*}$) 
with the following differences: there are only two types of signatures, type I with 
$\mathit{pk_t} = (u_0, u_1) = (\alpha \cdot \goneBLS, \alpha \cdot \gtwoBLS)$ (and handled as per type I in the proof of theorem 15.2 part b~\cite{dabo_book}) 
and the equivalent of type III with $\mathit{pk_t} \neq (u_0, u_1) = (\alpha \cdot \goneBLS, \alpha \cdot \gtwoBLS)$. In order to retrieve the corresponding signature's 
$\sigma_t$ first component such that $\forall t, \sigma_t \neq \sigma^*$, $\mathcal{B}$ looks up the storying table for simulated $\Hsig$ 
and retrieves the entry including both $m$ and $\mathit{pk}_t$. As per our assumption about $\mathit{A}_2$ and as per the way the simulation for 
$\Hsig$ was defined, such an entry exists and is unique. 
\end{proof}