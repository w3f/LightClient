
BLS signatures introduced by Boneh, Lynn and Shacham~\cite{BLS_signatures} are a popular aggregatable signature scheme. 
It has short aggregate signatures that can be efficiently verified. However due to the heavier cryptography involved, i.e., 
pairings on elliptic curves, individual signatures are much slower to verify compared to e.g. Schnorr or ECDSA signatures. \\

Aggregation of BLS signatures~\cite{BLS_signatures} simplifies some distributed systems,
usually by being exportable proofs of byzantine agreement. After agreement and aggregation occur, 
the aggregate signature saves foreign aggregate verifiers both compute and bandwidth over checking
numerous slow signatures, especially at the scale of Ethereum's hundreds of thousands of signers.
Because individual signature verification is very slow, either most nodes incur the high verification 
costs for every other node's signatures, or, else, system designers choose more centralised gossip
flavours which may harm liveness. \\

As a rule, today one always does BLS signatures on curves with type III pairings,
i.e. the pairing takes as input elements of two different groups $\GoneBLS$ and $\GtwoBLS$.
Typically, such as with the standard BLS12-381 curve, these have very different
performance characteristics with $\GtwoBLS$ having much slower arithmetic and hash-to-curve, due to
being defined over an extension field.  In fact, recent progress against the discrete log problem in 
extension fields~\cite{NumberFieldSieveAttack} could precipitate adopting pairings with a higher embedding degree, 
which would slow $\GtwoBLS$ further. We want the parts of our protocol that need to be 
efficient to use all or mostly $\GoneBLS$ operations. \\

\begin{comment}
We suggest that gossip messages containing individual BLS signatures carry
a Chaum-Pedersen DLEQ proofs of their correctness, so that verifying
individual signatures no longer requires pairings.
%
We avoid complex bisection logic in aggregation this way too, but increase
signature message size by 32 bytes.
% meh..  We need a second group $G_f$ in which scalar multiplication 
\end{comment}

In this work we propose individual BLS signatures carry Chaum-Pedersen~\cite{ChaumPedersen} DLEQ proofs of their correctness, done in $\GoneBLS$, so that 
verifying them no longer requires pairings or $\GtwoBLS$ operations. Moreover, a classical BLS signature 
places the public key and signature on opposite groups of the pairing (for type III pairings), 
but the choice of which of the groups is used for the keys or signatures creates trade offs. 
We mitigate these trade offs by proposing the following:

\paragraph{Protocol Sketch:}
\begin{comment}
We adopt standard notation for pairing friendly curves throughout.

% The original BLS protocol was written for Type 1 pairings, where there is only one elliptic curve and prime order subgroup.

A classical BLS signature places the public key and signature on opposite
curves, but the choice of which creates trade offs. We mitigate these
trade offs:
\end{comment}

First, we provide the public key on both source groups%both curves,
like $\pk = (\sk \cdot \goneBLS, \sk \cdot \gtwoBLS)$.
We enforce this public key structure during the public key validation
phase present in many aggregate BLS protocols in order to protect against rogue-key attacks~\cite{proofs_of_posession}
by adding proof-of-possession check~\cite{proofs_of_posession} via a pairing check.
%, which, in our case, are not paring based, thus computationally involved,  
%but, they are more efficient and require verification of Chaum-Pedersen proofs~\cite{ChaumPedersen}. \\

Second, we create BLS signatures $\sigma = \sk \cdot H(m)$ using the much
faster $\GoneBLS$ hash-to-curve $H(m)$, but also provide a Chaum-Pedersen proof
$\pi_{\mathit{sig}}$ that $\log_{H(m)}(\sigma)= \log_{\goneBLS}(\pk_1)$.
Verifying the correctness of $\sigma$ is thus reduced to verifying the correctness of $\pi_{\mathit{sig}}$, 
which avoids any expensive pairing operations. \\
% Interestingly, $\pi$ permits signing an extra message besides $m$. \\

Third, an aggregator node provides both the aggregate BLS signature,
as well as aggregate signer keys %$\apk_1 \in \GoneBLS$ and
$\apk{2} \in \GtwoBLS$, by summing the individual signatures and second source group public keys.
At this point, aggregate verifiers only need to compute $\apk{1}$ as the sum of first source group public keys and 
check that $\apk{2}$ was correctly computed with only two additional scalar multiplications in $\GoneBLS$.
In the end, aggregate verifiers save time despite these multiplications
because their hash-to-curve runs on $\GoneBLS$, far faster than on $\GtwoBLS$.
Even when using alternative methods for computing $\apk{1}$ that involve custom SNARKs such as in~\cite{ourLC}, 
the aggregate verifiers still require operations only in the smaller and faster source group. 
%Also, aggregate verifiers have fast additions when checking if $\apk_1$ represents
%a suitable signer set, which speeds up zk proofs of $\apk_1$ like \cite{ourLC} too. \\

We remark that a natural variation on our protocol exists in which
one finds an even faster non-pairing curve $\mathcal{S}$ having the same
group order as $\GoneBLS$,  publishes triplet public keys like
$\pk = (\sk \cdot \goneBLS, \sk \cdot \gtwoBLS, \sk \cdot S)$, in which $S$ generates
$\mathcal{S}$, and then DLEQ proofs use $\pk_3$ for performance.
We do not discuss this variant because it falls under our security
arguments without additional mathematical complexities,
aside from choosing curve parameters for $\mathcal{S}$. \\

Variants of the improvements described above have been circulating as folklore. 
In this work, we put things on firm formal ground by clearly describing the protocol (which we present in detail in Section~\ref{sec:bls}), but also 
by including a formal definition of signature aggregation in Section~\ref{sec:def_aggregate} and also by providing a security proof for our instantiation in Section~\ref{sec:bls}. 
In Section~\ref{sec:benchmarks} we present the comparison of the efficiency of the proposed scheme with classic BLS scheme. \\ 
 
\paragraph{Related Work:} Alternative constructions for BLS signatures as well as other defence mechanisms against 
rogue key attacks~\cite{proofs_of_posession}, exist and we briefly review both below. First, aggregation of BLS signatures for different messages have been studied before
(e.g.,~\cite{aggregate_BLS_signatures}). In this case, rouge-key attacks are not a threat anymore (and, hence, PoPs are not a necessity anymore), 
but signature verification is computationally more expensive than in our case, requiring $O(n)$ parings for $n$ different messages. Second, alternative aggregatable 
BLS signatures exist (e.g.,~\cite{boneh_compact_multisig}) where both the aggregated public key have their size independent 
of the number of signers and the signature verification is comparable to our variant. However, for the blockchain use cases we envision 
(for example the accountable light client system from~\cite{ourLC}) we prefer our scheme detailed in Section~\ref{sec:bls}: its corresponding key aggregation is a simple sum 
of the individual signers' public keys, while, in~\cite{boneh_compact_multisig} the key aggregation operation involves more expensive scalar 
multiplications every time the key aggregation is performed.
