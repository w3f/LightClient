\noindent Below we remind the reader the co-CDH assumption, which is a variation of the standard computational Diffie-Hellman assumption (CDH)
for the case when two groups are used. Let $E$ be a pairing friendly elliptic curve and 
let $\GoneBLS$, $\GoneBLS$ and $\GTBLS$ be appropriately chosen subgroups of order $r$ with  $\goneBLS$, $\gtwoBLS$ 
generators for the first two subgroups, respectively. Let $e: \GoneBLS \times \GtwoBLS \rightarrow \GTBLS$ be a secure 
pairing~\cite{secure_pairings,pairings_for_cryptographers}. 
\paragraph{Attack Game co-CDH} For a given adversary $\mathcal{A}$ the attack runs as follows:
\begin{itemize}
\item The challenger computes $\alpha, \beta \xleftarrow{\$} \mathbb{Z}_r$, $u_1 \leftarrow \goneBLS^{\alpha}$, $u_2 \leftarrow \gtwoBLS^{\alpha}$,   
$v_1 \leftarrow \goneBLS^{\beta}$, $z_1 \leftarrow \goneBLS^{\alpha\beta}$ and gives the tuple $(u_1, u_2, v_1)$ to $\mathcal{A}$; $\alpha$ is used twice, 
once in $\GoneBLS$ and once in $\GtwoBLS$. 
\item The adversary $\mathcal{A}$ outputs some $\hat{z_1} \in \GoneBLS$.
\end{itemize}
\noindent $\mathcal{A}$'s advantage in solving the co-CDH problem for $e$, denoted by $$\mathit{Adv}^{\mathit{coCDH}}[\mathcal{A},e]$$ is the probability 
that $\hat{z_1} = z_1$.
\paragraph{co-CDH Assumption} We say that the co-CDH assumption holds for the pairing $e$ if for all efficient adversaries $\mathcal{A}$, 
the quantity $\mathit{Adv}^{\mathit{coCDH}}[\mathcal{A},e]$ is negligible.
If $e$ is a symmetric pairing, then $\GoneBLS = \GtwoBLS$ and $\goneBLS = \gtwoBLS$ in which case the co-CDH assumption is identical to the standard CDH assumption.