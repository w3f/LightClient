\section{Is our protocol applicable to Ethereum?} 
\label{sec:ethereum}

We believe that our protocol could not feasibly be directly applied to Ethereum as it is without a hard fork, but it would be easy to apply it with changes that might 
feasibly be implemented, even with changes not specifically designed with our protocol in mind. Ethereum already uses BLS signatures on the BLS12-381 curve 
in consensus. To work with BLS12-381, Our protocol would use KZG commitments on the BW6-767 curve~\cite{bw6767}. Because the base field of BL12-381 is not 
very 2-adic, a prover would need a more complicated FFT algorithm but this is feasible~\cite{bw6767}. We also need an appropriately sized subgroup of the multiplicative 
field to use for our Lagrange basis commitments. An easy calculation gives the small prime factors $2,3^2,11,23,47,10177$ and 859267 for the order of the multiplicative 
group and any product of these larger than the number of validators gives a usable subgroup.
 
Next we consider who constructs the the KZG commitment to the validators public keys. For the shortest light client proofs validators would construct and sign this 
commitment, which would require a change to the consensus logic. As an alternative, a smart contract could compute the commitments on chain. This requires the 
EVM to have access to the active validator's public keys and would also require a precompile for BW6-767 operations to be feasible. We note that there have been 
many suggestions for adding elliptic curve operations for different curves to Ethereum (e.g. EIPs 2537,2538,3026~\cite{EIPs}) but few have been implemented so 
far. We would expect this to be the bottleneck for implementing our protocol on Ethereum.
 
Finally we compare running our scheme on the full validator set to Ethereum's current light client design~\cite{ethlight}. That uses s subset of 1024 public keys 
that changes every epoch (i.e. 64 blocks or 12.8 minutes). It is not accountable because it would take less than 1024 validators misbehaving to deceive a light client. 
We remark that 1024 384-bit public keys is comparable in size to 1 bit for all of Ethereum's over 500000 validators, and as a result our light client scheme can be 
used for an accountable light client with a similar overhead to Ethereum's existing unaccountable scheme.

With epochless Casper FFG~\cite{Gasper}, 64 aggregated signatures are required to represent a single Casper FFG vote, 2 of which are required for a proof. The form 
of accountable safety satisfied by Casper FFG~\cite{CasperFFG} suffices for us: if two forks are finalised, then then 2/3 of the validator set voted on two messages such 
that signing both is punishable. One complication of a Casper FFG light client is that valid votes include two blockhashes, one of which is required to be the ancestor 
of the other. There needs to be an efficient "proof of ancestry" such as introducing a more efficient commitment to previous block hashes, e.g. Merkle Mountain 
Range of blockhashes as suggested in~\cite{flyclient}.
