\section{Compiler for Hybrid Model SNARKs with Mixed Inputs}
\label{sec_two_step_compiler}
%\vspace{-0.05in} 
\subsection{Technical Challenges and Contributions Regarding our Custom SNARKs} 
\label{sec:technical_challenges}
%\vspace{-0.1cm}

In order to define and implement our committee key scheme accountable light client systems and in order to design the custom SNARKs that support our efficiency results, 
we had to tackle some technical challenges and make additional contributions as summarised below.

%\vspace{-0.2cm}
\paragraph{Extending PLONK Compiler to Mixed Commitment and Vectors NP Relations} Firstly, our custom SNARKs takes inspiration from PLONK \cite{plonk} in terms design of the proof system used, and of the circuits and gates. However, our SNARKs also have differences compared to PLONK. PLONK applies to NP relations  that use  vectors of field elements for 
public inputs and witnesses.  However we need SNARKs whose defining NP relations also have  polynomial commitments (in our case, the committee key $C$) as part of their public inputs. Hence, the original PLONK compiler does not suffice; we extend it with a second step in which we show that under certain conditions which our protocol fulfils, the SNARKs obtained using the original PLONK compiler are also SNARKs for a mixed type of NP relation containing both vectors and polynomial commitments. The full details and proofs can be found in Section~\ref{sec_two_step_compiler} 
and we believe this compiler extension to be of independent interest. 

%\vspace{-0.2cm}
\paragraph{Conditional NP Relations for Efficiency} Secondly, we also require NP relations that have a well-defined subpredicate which is verified outside the SNARKs. In a blockchain instantiation, 
any current validator set has to come to a consensus, among other things, on the next validator set, represented by a set of public keys. The validator set 
computes and signs a pair of polynomial commitments to the next set of validators' public keys. Before including a public key in the set, the validators perform several 
checks on the proposed public key, such as being in a particular subgroup of the elliptic curve. This check is not performed by the SNARKs' constraint system, but is 
required for the correctness of the statement the SNARKs prove. This design decision makes our SNARKs more efficient, but it also means we have to extend the 
usual definition of NP relations to conditional NP relations, where in fact, one of the subpredicates that define the conditional relation is checked outside the SNARKs 
or ensured due to a well-defined assumption. We introduce the general notion of conditional NP relation in Section~\ref{sec:conditional_relations} and describe our 
concrete conditional NP relations in Section~\ref{sec:snarks}.   

%\vspace{-0.2cm}
\paragraph{Hybrid Model SNARKs} In line with the two above technical challenges and the solutions we came up with, we extend the existing definitions 
related to SNARKs~\cite{groth16,plonk} by introducing an algorithm which we call $\mathit{PartInput}$. For our use case, this allows us to separate the public input for the NP relations that define our custom SNARKs in two: a part 
that is computed by the current set of validators on the blockchain in question and the rest of the public input plus the corresponding SNARK proof are 
computed by a (possibly malicious) prover interacting with the light client verifier. Our newly introduced notion of hybrid model SNARK (see Section~\ref{sec:snarks_defs}) 
generalises this public input separation concept and its definition is used to prove the security of our custom SNARKs in Section~\ref{sec_two_step_compiler}.

\subsection{SNARK compiler}

\noindent We present a two-steps PLONK-based compilation technique from 
ranged polynomial protocols for conditional NP relations (formal definition in Section~\ref{supplementary_poly_protocols_appendix}) to hybrid 
model SNARKs (Definition~\ref{dfn_snark}) such that the conditional NP relations that define the SNARKs we compile in the 
second step contain both polynomial commitments and vectors of field elements as public inputs. By using just the first step of our 
compiler which is equivalent to the original PLONK compiler~\cite{plonk}, one would 
not be able to obtain SNARKs with mixed public inputs consisting of both vectors of field elements and also poly commitments. 
In turn, this type of NP relations with mixed inputs is crucial for designing accountable light clients via the use of committee key schemes 
(see Section~\ref{sec:inst_committee_key}).\\
\vspace{-0.2in}
%\subsubsection{Our Compiler: Step 1} 
\subsection{Our Compiler: Step 1} 
\label{compiler_step_1}
\vspace{-0.05in}
%\noindent \textbf{(PLONK Compiler - from Polynomial Protocols to SNARKs)} \\

\noindent Our first step applies the PLONK compiler~\cite{plonk} (Lemma 4.7): we compile the information theoretical ranged polynomial protocols $\Pla$ and $\Pa$ 
for relations $\Rla$ and $\Ra$, respectively (see sections \ref{sec_la} and \ref{sec_a}) into 
(hybrid model) SNARKs $\Plastar$, and $\Pastar$, respectively. We can define this compilation step 
for any ranged polynomial protocols for relations (as per Definition~\ref{supplementary_poly_protocols_appendix} in section~\ref{def_ranged_poly_protocol}). 
\begin{comment}
In order to do that we need: 
\begin{itemize}
\item  The batched version of KZG polynomial commitments~\cite{KZG_10} described in section 3 of PLONK~\cite{plonk}.\footnote{In fact, 
one can replace the use of KZG polynomial commitments with any binding polynomial commitment that has knowledge-soundness, including non-homomorphic polynomial commitments, 
such as FRI-based polynomial commitments (e.g., RedShift~\cite{redshift}). If the optimisation gained from PLONK linearisation technique is a goal, 
then, with minimal changes one can use any homomorphic polynomial commitment, e.g., the discrete logarithm based polynomial commitment 
from Halo~\cite{halo}.}
\item A general compilation technique: such a technique has been already defined in Lemma 4.7 of PLONK; combined with Lemma 4.5 
from PLONK this technique can be applied with minor adaptations (this includes the corresponding technical measures) to the notion of ranged 
polynomial protocols.  
\item So far, both the ranged polynomial protocols for relations and the protocols resulted after the first compilation step have been explicitly defined as interactive 
protocols. In order to obtain the non-interactive version of the latter (essentially the N in SNARK) one has to apply the Fiat-Shamir 
transform~\cite{FS_transform}, \cite{FS_transform_with_proof}, \cite{SE_plonk}.
\end{itemize}

\end{comment}
%\begin{comment}
\noindent Let $\mathcal{R}$ be a (conditional) NP relation, let $\mathscr{P}_{\mathcal{R}}$ be a ranged polynomial protocol for 
relation $\mathcal{R}$ and let $\mathscr{P}^*_{\mathcal{R}}$ be the SNARK compiled from $\mathscr{P}_{\mathcal{R}}$ using the PLONK compiler.  
The compilation technique requires the SNARK prover of  $\mathscr{P}^*_{\mathcal{R}}$ to compute 
polynomial commitments to all polynomials that the prover $\mathcal{P}_{poly}$ in $\mathscr{P}_{\mathcal{R}}$ sent to the ideal party $\mathcal{I}$. Analogously, 
it requires the SNARK verifier of $\mathscr{P}^*_{\mathcal{R}}$ to compute polynomial commitments to all pre-processed polynomials\footnote{This is a one-time computation that is 
reused by the SNARK verifier for all SNARK proofs over the same circuit.} as well polynomial commitments to polynomials the verifier $\mathcal{V}_{poly}$ 
in $\mathscr{P}_{\mathcal{R}}$ sent to the ideal party $\mathcal{I}$. Then, the SNARK prover sends the SNARK verifier openings to 
all the polynomial commitments computed by him as well as the polynomial commitments computed by the SNARK verifier. The SNARK 
prover additionally sends the corresponding batched proofs for polynomial commitment openings. In turn, the SNARK verifier accepts or rejects based 
on the result of the verification of the batched polynomial commitment scheme. \\
%\end{comment}

%\noindent A more efficient compilation technique exists which reduces the number of polynomial commitments and alleged polynomial commitments openings 
%(i.e., both group elements and field elements) sent by the SNARK prover to the SNARK verifier; this, in turn, reduces the size of the SNARK proof. 
%This technique is called linearisation and is described, at a high level, after Lemma 4.7 in PLONK. The existing description however covers only the 
%SNARK prover side and it does not detail the SNARK verifier side so in the following we cover that. \\
\noindent PLONK proposes a more efficient compilation technique (i.e., linearisation, see explanation after Lemma 4.7 in PLONK) 
which reduces the SNARK proof size. 
%\noindent By functionality, the vectors that are handled by the the verifier $\mathcal{V}_{poly}$ are 
%of two types: pre-processed vectors and public input vectors. These two types of vectors are used by $\mathcal{V}_{poly}$ 
%to obtain, via interpolation over the range on which the respective range polynomial protocol is defined, pre-processed polynomials 
%(as used in the definition 2 in section 2 of supplementary material, e.g., polynomial $aux(X)$ used in section~\ref{sec_a}) and 
%public-inputs-derived polynomials (e.g., polynomials $pkx(X)$ and $pky(X)$ used in sections~\ref{sec_la} and ~\ref{sec_a} 
%and polynomial $b(X)$ used in section~\ref{sec_la}). The efficient linearisation technique allows the SNARK verifier to reduce the 
%number of polynomial commitments it has to compute compared to the general PLONK compiler in the following way. Instead of 
%having to compute polynomial commitments to all polynomials $\mathcal{V}_{poly}$ sends to $\mathcal{I}$ (including any corresponding 
%pre-processed polynomials), the SNARK verifier computes polynomial evaluations at one or multiple random points (as per the linearisation 
%step specific requirements) for all the polynomials that are either easy to evaluate (e.g., polynomial $aux(X)$ used in section~\ref{sec_a}) or 
%all the polynomials that are obtained from vectors that do not take up a large amount of memory (e.g., polynomial $b(X)$ used in section~\ref{sec_la}). 
%For the rest of the polynomials (e.g., $\mathit{pkx}(X)$ and $\mathit{pky}(X)$), the SNARK verifier computes polynomial commitments as before.\\
For our specific case, this allows the SNARK verifier to reduce the 
number of polynomial commitments it has to compute compared to the general PLONK compiler by computing 
polynomial evaluations at one or multiple random points (as per the linearisation step specific requirements) 
for all the polynomials that are either easy to evaluate (e.g., polynomial $aux(X)$ used in section~\ref{sec_a} and $\Ra$) or 
all the polynomials that are obtained from input vectors that do not take up a large amount of memory (e.g., polynomial $b(X)$ used in section~\ref{sec_la} and $\Rla$).
%\noindent We note we can apply all the techniques mentioned above, including the combined prover-and-verifier-side linearisation 
%to compile our ranged polynomial protocols $\Pla$ and $\Pa$ into the corresponding SNARKs $\Plastar$ and $\Pastar$, respectively.
Finally, we state in Section~\ref{def_ranged_poly_protocol}, Lemma~\ref{le:compilation_step_1} under which conditions and how efficiently 
one can compile ranged polynomial protocols for pure vector-based conditional NP relations 
into hybrid model SNARKs using only the original PLONK compiler and we give a more in-depth explanation of this step in section~\ref{first_step_compiler}.\\
\noindent \textbf{(PLONK Compiler - from Polynomial Protocols to SNARKs)} \\

\noindent We summarise and exemplify below the PLONK-based compilation technique~\cite{plonk} from 
ranged polynomial protocols for conditional NP relations (formal definition in Section~\ref{supplementary_poly_protocols_appendix}) to 
SNARKs for pure vector-based NP relations. This is also the first of our two-steps compiler. Concretely, our first step applies the PLONK compiler~\cite{plonk} (lemma 4.7): 
we compile the information theoretical ranged polynomial protocols $\Pla$ and $\Pa$ for relations $\Rla$ and $\Ra$, respectively (see Sections~\ref{sec_la},\ref{sec_a}) 
into SNARKs $\Plastar$, and $\Pastar$, respectively. We can define this compilation step for any ranged polynomial protocols for relations 
(as per Definition~\ref{supplementary_poly_protocols_appendix} in Section~\ref{def_ranged_poly_protocol}). In order to do that we need: 
\begin{itemize}
\item  The batched version of KZG polynomial commitments~\cite{KZG_10} described in Section 3 of PLONK~\cite{plonk}.\footnote{In fact, 
one can replace the use of KZG polynomial commitments with any binding polynomial commitment that has knowledge-soundness, including non-homomorphic polynomial commitments, 
such as FRI-based polynomial commitments (e.g., RedShift~\cite{redshift}). If the optimisation gained from PLONK linearisation technique is a goal, 
then, with minimal changes one can use any homomorphic polynomial commitment, e.g., the discrete logarithm based polynomial commitment 
from Halo~\cite{halo}.}
\item A general compilation technique: such a technique has been already defined in lemma 4.7 of PLONK; combined with lemma 4.5 
from PLONK this technique can be applied with minor adaptations (this includes the corresponding technical measures) to the notion of ranged 
polynomial protocols.  
\item So far, both the ranged polynomial protocols for relations and the protocols resulted after the first compilation step have been explicitly defined as interactive 
protocols. In order to obtain the non-interactive version of the latter (essentially the N in SNARK) one has to apply the Fiat-Shamir 
transform~\cite{FS_transform}, \cite{FS_transform_with_proof}, \cite{SE_plonk}.
\end{itemize}

\noindent Let $\mathcal{R}$ be a (conditional) NP relation, let $\mathscr{P}_{\mathcal{R}}$ be a ranged polynomial protocol for 
relation $\mathcal{R}$ and let $\mathscr{P}^*_{\mathcal{R}}$ be the SNARK compiled from $\mathscr{P}_{\mathcal{R}}$ using the PLONK compiler.  
The compilation technique requires the SNARK prover of  $\mathscr{P}^*_{\mathcal{R}}$ to compute 
polynomial commitments to all polynomials that the prover $\mathcal{P}_{poly}$ in $\mathscr{P}_{\mathcal{R}}$ sent to the ideal party $\mathcal{I}$. Analogously, 
it requires the SNARK verifier of $\mathscr{P}^*_{\mathcal{R}}$ to compute polynomial commitments to all pre-processed polynomials\footnote{This is a one-time computation that is 
reused by the SNARK verifier for all SNARK proofs over the same circuit.} as well polynomial commitments to polynomials the verifier $\mathcal{V}_{poly}$ 
in $\mathscr{P}_{\mathcal{R}}$ sent to the ideal party $\mathcal{I}$. Then, the SNARK prover sends the SNARK verifier openings to 
all the polynomial commitments computed by him as well as the polynomial commitments computed by the SNARK verifier. The SNARK 
prover additionally sends the corresponding batched proofs for polynomial commitment openings. In turn, the SNARK verifier accepts or rejects based 
on the result of the verification of the batched polynomial commitment scheme. \\

\noindent A more efficient compilation technique exists which reduces the number of polynomial commitments and alleged polynomial commitments openings 
(i.e., both group elements and field elements) sent by the SNARK prover to the SNARK verifier; this, in turn, reduces the size of the SNARK proof. 
This technique is called linearisation and is described, at a high level, after Lemma 4.7 in PLONK. The existing description however covers only the 
SNARK prover side and it does not detail the SNARK verifier side so in the following we cover that. \\

\noindent By functionality, the vectors that are handled by the the verifier $\mathcal{V}_{poly}$ are 
of two types: pre-processed vectors and public input vectors. These two types of vectors are used by $\mathcal{V}_{poly}$ 
to obtain, via interpolation over the range on which the respective range polynomial protocol is defined, pre-processed polynomials 
(as used in the Definition~\ref{supplementary_poly_protocols_appendix} in Section~\ref{def_ranged_poly_protocol}, e.g., polynomial $aux(X)$ used in Section~\ref{sec_la}) and 
public-inputs-derived polynomials (e.g., polynomials $pkx(X)$ and $pky(X)$ used in Sections~\ref{sec_la},\ref{sec_a})
and polynomial $b(X)$ used in Section~\ref{sec_la}). The efficient linearisation technique allows the SNARK verifier to reduce the 
number of polynomial commitments it has to compute compared to the general PLONK compiler in the following way. Instead of 
having to compute polynomial commitments to all polynomials $\mathcal{V}_{poly}$ sends to $\mathcal{I}$ (including any corresponding 
pre-processed polynomials), the SNARK verifier computes polynomial evaluations at one or multiple random points (as per the linearisation 
step specific requirements) for all the polynomials that are either easy to evaluate (e.g., polynomial $aux(X)$ used in Section~\ref{sec_a}) or 
all the polynomials that are obtained from vectors that do not take up a large amount of memory (e.g., polynomial $b(X)$ used in Section~\ref{sec_la}). 
For the rest of the polynomials (e.g., $\mathit{pkx}(X)$ and $\mathit{pky}(X)$), the SNARK verifier computes polynomial commitments as before.\\

\noindent We note we can apply all the techniques mentioned above, including the combined prover-and-verifier-side linearisation 
to compile our ranged polynomial protocols $\Pla$ and $\Pa$ into the corresponding SNARKs $\Plastar$ and $\Pastar$, respectively. 
To conclude this step, we formally state in Section~\ref{def_ranged_poly_protocol}, Lemma~\ref{le:compilation_step_1} under which condition and how efficiently 
one can compile ranged polynomial protocols for conditional NP relations (where the public inputs are interpreted as vector of field elements) 
into hybrid model SNARKs by using only the original PLONK compiler. \\

\vspace{-0.2in}
%\subsubsection{Our Compiler: Step 2}
\subsection{Our Compiler: Step 2}
\label{compiler_step_2}
\vspace{-0.05in}
\noindent \textbf{(Mixed Vector and Commitments based NP Relations and Associated SNARKs)} \\

\noindent The type of NP relations we have worked with so far as well as the more general PLONK NP relation 
(\cite{plonk}, Section 8.2) have vector of field elements as public inputs. Next we show that SNARKs 
compiled using Step 1 can become, under certain assumption, SNARKs for a new type of NP relation 
that specifically contains polynomial commitments as part of the input. Interpreting 
our already compiled SNARKs as SNARKs for this new type of NP relation is essential for designing 
accountable light client systems via committee key schemes (see Instantiation~\ref{inst:cks} 
in Section~\ref{sec:inst_committee_key}).  

\noindent Let conditional NP relation $\mathcal{R}_{\mathit{vec}}^c$  be:
\begin{align*}
\mathcal{R}_{\mathit{vec}}^c = \{&(\mathbf{input_1} \in \mathbf{\mathcal{D}_1}, \mathbf{input_2} \in\mathbf{\mathcal{D}_2}; \mathbf{witness_1}): \\  
&p_1(\mathbf{input_1}, \mathbf{input_2}, \mathbf{witness_1}) = 1 \ | \ c(\mathbf{input_1}) = 1 \ \wedge\ \\
&\wedge \ p_2(\mathbf{input_1}, \mathbf{input_2}, \mathbf{witness_1}) = 1 \},
\end{align*}
\noindent where $\mathbf{input_1}$, $\mathbf{input_2}$ are two sets of public input vectors 
belonging domains  $\mathcal{D}_1$, $\mathcal{D}_2$. $\mathbf{witness_1}$ is a set of witness vectors and $c$, $p_1$, $p_2$ are predicates. 
Let $\mathscr{P}_{\mathit{vec}}$ be a ranged polynomial protocol for relation $\mathcal{R}_{\mathit{vec}}^c$. Note that since $c$ applies 
only to a part of the public input for relation $\mathcal{R}_{\mathit{vec}}^c$ 
(i.e., $\mathbf{input_1}$), we can apply Lemma~\ref{le:compilation_step_1} of Section~\ref{supplementary_poly_protocols_appendix} and Step 1 
of our compiler to polynomial protocol $\mathscr{P}_{\mathit{vec}}$. \\

\noindent We make the following hybrid model assumptions:
\begin{itemize}
\item (HMA.1.) $\mathcal{V}_{poly}$ in $\mathscr{P}_{\mathit{vec}}$ computes 
$\mathit{Q_{1,\mathbf{input_1}}}(X), \ldots, \mathit{Q_{m, \mathbf{input_1}}}(X)$ which depend deterministically on $\mathbf{input_1}$ and sends them to $\mathcal{I}$. 
\item (HMA.2.) $\mathcal{V}_{poly}$ in $\mathscr{P}_{\mathit{vec}}$ does not use $\mathbf{input_1}$ in any further computation of 
any other polynomials or values its sends to $\mathcal{I}$.
\item (HMA.3.) By evaluating $\mathit{Q_{1,\mathbf{input_1}}}(X), \ldots, \mathit{Q_{m, \mathbf{input_1}}}(X)$ over the range on which 
$\mathscr{P}_{\mathit{vec}}$ is defined one obtains (using some efficiently computable and deterministic transformations) the set of vectors $\mathbf{input_1}$. 
\end{itemize} 
We denote by $\mathscr{P}^*_{\mathit{vec}}$ the hybrid model SNARK obtained after compiling $\mathscr{P}_{\mathit{vec}}$ using compilation Step 1. 
Due to (HMA.1.) and according to Step 1, the SNARK verifier in 
$\mathscr{P}^*_{\mathit{vec}}$ computes $$\mathit{Com_1} = \mathit{Com}(\mathit{Q_{1,\mathbf{input_1}}}), \ldots, \mathit{Com_m} = \mathit{Com}(\mathit{Q_{m,\mathbf{input_1}}})$$ 
which are KZG poly commitments to $\mathit{Q_{1,\mathbf{input_1}}}(X), \ldots, \mathit{Q_{m, \mathbf{input_1}}}(X)$. We denote vector
$(\mathit{Com_1}, \ldots, \mathit{Com_m})$ by $\mathbf{Com}(\mathbf{input_1})$ and we denote 
by $\mathcal{C}$ the set of all $\mathit{KZG}$ poly commitments or vectors of such poly commitments. We also define the relation: 
\begin{align*}
\mathcal{R}_{\mathit{vec}, \mathit{com}}^c = \{& \mathbf{C} \in \mathcal{C}, \mathbf{input_2} \in \mathbf{\mathcal{D}_2}; \mathbf{witness_1}, \mathbf{witness_2}):  \\
& p_1(\mathbf{witness_2}, \mathbf{input_2}, \mathbf{witness_1}) =1  \ |\ c(\mathbf{witness_2}) = 1  \ \wedge\  \\
& \wedge\ p_2(\mathbf{witness_2}, \mathbf{input_2}, \mathbf{witness_1}) = 1\ \wedge \\
& \wedge\ \mathbf{C} = \mathbf{Com}(\mathbf{witness_2})\}
\end{align*}
\noindent Finally, for $\mathbf{input_1}$ part of $\mathit{state_1}$, we define $\mathit{SNARK.PartInput}$:
\begin{align*} 
&\mathit{SNARK.PartInput}(\mathit{srs}, \mathit{state_1},\mathcal{R}_{\mathit{vec}, \mathit{com}}^c) \\  
& \mathit{If \ }  c(\mathbf{input_1}) = 0 \textit{ then} \ \mathit{Return} \\
& \mathit{Else} \\
& \ \ \ \ \textit{Compute via interpolation on } \mathscr{P}_{\mathit{vec}}  \textit{ range } 
\mathit{Q_{1,\mathbf{input_1}}}(X), \ldots, \mathit{Q_{m, \mathbf{input_1}}}(X).\\
& \ \ \ \ \mathbf{C} = (\mathit{Com}(Q_{1,\mathbf{input_1}}(X)), \ldots, \mathit{Com}(Q_{m,\mathbf{input_1}}(X))) \\
& \ \ \ \ \mathit{state_2} =  \mathit{state_1} \cup \{ \mathbf{C} \} \textit{ then} \ \mathit{Return} (\mathit{state_2,  \mathbf{C}})
\end{align*}

\noindent With the above notation, \textbf{our compiler's Step 2 is:} \\
\noindent The alleged hybrid model SNARK $\mathscr{P}_{\mathit{vec}}^{h}$ for relation $\mathcal{R}_{\mathit{vec}, \mathit{com}}^c$ is:
\begin{itemize}
\item $\mathit{SNARK.Setup}$ and $\mathit{SNARK.KeyGen}$ are as for relation $\mathcal{R}^{c}_{\mathit{vec}}$.
\item $\mathit{SNARK.PartInput}$ for relation $\mathcal{R}^{c}_{\mathit{vec}}$ 
(see Lemma~\ref{le:compilation_step_1} in Section~\ref{supplementary_poly_protocols_appendix}) 
is replaced with $\mathit{SNARK.PartInput}$ for relation $\mathcal{R}_{\mathit{vec}, \mathit{com}}^c$.
\item $\mathit{SNARK.Prover}$ for relation $\mathcal{R}_{\mathit{vec}, \mathit{com}}^c$ is identical with 
$\mathit{SNARK.Prover}$ for relation $\mathcal{R}^{c}_{\mathit{vec}}$ (as compiled using Step 1) with the appropriate 
re-interpretation of the public inputs and witness.
\item $\mathit{SNARK.Verifier}$ for relation $\mathcal{R}_{\mathit{vec}, \mathit{com}}^c$ is identical with 
$\mathit{SNARK.Verifier}$ for relation $\mathcal{R}^{c}_{\mathit{vec}}$ (as compiled using Step 1) with the appropriate 
re-interpretation of the public inputs and such that $\mathit{SNARK.Verifier}$ for $\mathcal{R}_{\mathit{vec}, \mathit{com}}^c$ does
not compute the polynomial commitments to the polynomials defined by assumption (HMA.1.).
\end{itemize}
%\vspace{-0.2in}
\noindent \begin{lemma} 
\label{sergey_type_relations} 
Let $\mathscr{P}_{\mathit{vec}}$ be a ranged polynomial protocol for relation $\mathcal{R}^c_{\mathit{vec}}$ defined above and let 
$\mathscr{P}_{\mathit{vec}}^{*}$ be the hybrid model SNARK for relation $\mathcal{R}^c_{\mathit{vec}}$ secure in the AGM obtained 
by compiling $\mathscr{P}_{\mathit{vec}}$ using our compiler's Step 1. If the hybrid model assumptions (HMA.1.) - (HMA.3.) hold w.r.t. 
protocol $\mathscr{P}_{\mathit{vec}}$ and $\mathit{State}_{\mathcal{R}_{\mathit{vec}, \mathit{com}}} \neq \emptyset $ then 
$\mathscr{P}_{\mathit{vec}}^{h}$ as compiled using our compiler's Step 2 is a hybrid model SNARK for relation 
$\mathcal{R}_{\mathit{vec}, \mathit{com}}^c$ secure also in the AGM.
\end{lemma}

\begin{proof} Let $\mathcal{E}_{\mathit{KZG}}$ and $\mathcal{E}$ be the extractors from the knowledge-soundness definitions for the 
$\mathit{KZG}$ batch polynomial commitment scheme (as in Definition 3.1, Section 3 in~\cite{plonk}) and the hybrid model 
SNARK $\mathscr{P}^*_{\mathcal{R}}$ for relation $\mathcal{R}^c_{\mathit{vec}}$ (as per Definition~\ref{dfn_snark}), respectively. 
Let $\mathcal{A}$ be an adversary against knowledge soundness in the hybrid model w.r.t. 
$\mathscr{P}_{\mathit{vec}}^{h}$ and relation $\mathcal{R}_{\mathit{vec}, \mathit{com}}^c$ and let $\mathit{aux}_{\mathit{SNARK}} \in \mathcal{D}$ 
and let $\mathit{state_1} \in \mathit{State}_{\mathcal{R}_{\mathit{vec}, \mathit{com}}}$; let 
$(\mathbf{C},\mathit{state_2}) = \mathit{SNARK.PartInput}(\mathit{srs}, \mathit{state_1}, \mathcal{R}_{\mathit{vec}, \mathit{com}}^c)$. 
By the definition of $\mathit{SNARK.PartInput}$ for $\mathscr{P}_{\mathit{vec}}^{h}$, there exists 
$\mathbf{input_1}$ such that $\mathbf{C} = \mathbf{Com}(\mathbf{input_1})$ and $c(\mathbf{input_1}) = 1$. 
We denote by $(\mathbf{input_2}, \pi)$ the output of $\mathcal{A}(\mathit{srs}, \mathit{state_2},  \mathcal{R}_{\mathit{vec}, \mathit{com}}^c)$ 
and let $\mathcal{A}_1$ be the part of $\mathcal{A}$ that sends openings and batched proofs for the polynomial commitments in 
$\mathbf{C}$. \\

\noindent On the one hand, if $\mathit{SNARK.Verifier}(\mathit{srs}_{\mathit{vk}}, (\mathbf{C},\mathbf{input_2}),\pi,\mathcal{R}_{\mathit{vec}, \mathit{com}}^c)$ 
in $\mathscr{P}_{\mathit{vec}}^{h}$ accepts, then the $\mathit{KZG}$ verifier corresponding to 
$\mathcal{A}_1$ also accepts. When such an event takes place, then, \ewnp $\mathcal{E}_{\mathit{KZG}}$ extracts polynomials 
$Q'_1(X), \ldots, Q'_m(X)$ that represent witnesses for the vector $\mathbf{C}$ of commitments and the alleged openings of $\mathcal{A}_1$. 
Because the $\mathit{KZG}$ polynomial commitment scheme is binding and by the definition of 
$\mathit{SNARK.PartInput}$ for $\mathscr{P}_{\mathit{vec}}^{h}$, we obtain that $Q'_1(X) = Q_1(X), \ldots, Q'_m(X) = Q_m(X).$ 
Since per (HMA.3.), the set $\{Q_1(X), \ldots, Q_m(X)\}$ evaluates to $\mathbf{input_1}$ over the range over which $\mathscr{P}_{\mathit{vec}}$ 
was defined, \ewnp the witness polynomials extracted by $\mathcal{E}_{\mathit{KZG}}$ evaluate to $\mathbf{input_1}$. \\

\noindent On the other hand, if $\mathit{SNARK.Verifier}(\mathit{srs}_{\mathit{vk}}, (\mathbf{C},\mathbf{input_2}),\pi,\mathcal{R}_{\mathit{vec}, \mathit{com}}^c)$ 
in $\mathscr{P}_{\mathit{vec}}^{h}$ accepts, then \\
$\mathit{SNARK.Verifier}(\mathit{srs}_{\mathit{vk}}, (\mathbf{input_1},\mathbf{input_2}),\pi,\mathcal{R}_{\mathit{vec}}^c)$ 
in $\mathscr{P}_{\mathit{vec}}^{*}$ also accepts. In turn, this acceptance together with the fact that $\mathscr{P}_{\mathit{vec}}^{*}$ 
has knowledge-soundness as per Definition~\ref{dfn_snark}, it implies $\mathcal{E}$ \ewnp extracts $\mathbf{witness_1}$ 
such that $(\mathbf{input_1}, \mathbf{input_2}, \mathbf{witness_1}) \in \mathcal{R}_{\mathit{vec}}^{c} \ (\#).$ 

\noindent By the definition of $\mathit{SNARK.PartInput}$ for $\mathscr{P}_{\mathit{vec}}^{h}$ and the way $\mathbf{input_1}$ was defined, 
it holds that $c(\mathbf{input_1}) = 1$. Due to $(\#)$ and by the definition of relation $\mathcal{R}_{\mathit{vec}}^{c}$, 
the predicates: $p_1$($\mathbf{input_1}$, $\mathbf{input_2}$, $\mathbf{witness_1}$) $= 1$ and 
$p_2(\mathbf{input_1}, \mathbf{input_2}, \mathbf{witness_1}) = 1$ hold. If we let 
$\mathbf{witness_2} = \mathbf{input_1}$, then 
$(\mathbf{C} = \mathbf{Com}(\mathbf{input_1}), \mathbf{input_2}, \mathbf{witness_1}, \mathbf{input_1}) \in \mathcal{R}_{\mathit{vec}, \mathit{com}}^c,$ so 
using $\mathcal{E}_{\mathit{KZG}}$ and $\mathcal{E}$ we can build an extractor for any knowledge-soundness adversary $\mathcal{A}$ for alleged 
hybrid model SNARK $\mathscr{P}_{\mathit{vec}}^{h}$ for relation $\mathcal{R}_{\mathit{vec}, \mathit{com}}^c$, which concludes the proof.
\end{proof}

\noindent It is straightforward to apply the technique described above to our SNARKs $\Plah$ and $\Pah$ 
compiled in Step 2 and obtain relations $\Rlacom$ and $\Racom$ as described below such that they fulfil Lemma~\ref{sergey_type_relations}.\footnote{Due to our specific application to proof-of-stake blockchain context in which we make use of our custom SNARKs, 
the assumption/requirement that  $\mathit{State}_{\mathcal{R}_{\mathit{vec}, \mathit{com}}} \neq \emptyset$ for 
$\mathcal{R}_{\mathit{vec}, \mathit{com}} \in \{\Rlacom, \Racom \}$ is fulfilled.}
%\vspace{-0.1in}
\begin{align*}
 {\Rlacom} = \{ & (\mathbf{C} \in \mathcal{C}, \mathbf{bit} \in \mathbb{B}^n, \mathit{apk} \in \mathbb{F}^2; \mathbf{pk}) : \mathit{apk} = \sum_{i=0}^{n-2} [\mathit{bit_i}] \cdot \mathit{pk_i} \ | \\ 
& \mathbf{pk} \in \ginn{1}^{n-1} \ \wedge \  \mathbf{C} = \mathbf{Com}(\mathbf{pk}) \} 
\end{align*}
%\vspace*{-0.75cm}
\begin{align*}
 {\Racom}  = \{ & (\mathbf{C} \in \mathcal{C}, \mathbf{b'} \in \mathbb{F}_{|\block|}^{\frac{n}{\block}}, \mathit{apk} \in \mathbb{F}^2;\mathbf{pk}, \mathbf{bit}) : \mathit{apk} = \sum_{i=0}^{n-2} [\mathit{bit_i}] \cdot \mathit{pk_i} \ | \\ 
& \mathbf{pk} \in \ginn{1}^{n-1} \ \wedge \ \mathbf{bit} \in \mathbb{B}^n  \wedge b'_{j} = \sum_{i=0}^{\block -1}2^i \cdot \mathit{bit_{\block \cdot j + i}}, \forall j < \frac{n}{\block}  \wedge \  \mathbf{C} = \mathbf{Com}(\mathbf{pk}) \} 
\end{align*}
For completeness, we also include the full rolled out SNARK $\Pah$ for relation $\Racom$ in Section~\ref{sec:rolled_out} and we provide a comparison between PLONK universal 
SNARK and our custom SNARKs in Section~\ref{suplementary_plonk_comparison}.  
%\vspace*{-0.75cm}